\documentclass{article}
\begin{document}
\title{Lectures Notes for Probability}
\author{Emulie Chhor}
\maketitle

\section*{Introduction}

Le premier cours de probabilité comporte 4 chapitres

\begin{enumerate}
    \item Introduction aux Probabilités
    \item Variables Aléatoires
    \item Espérance de Variables Aléatoires
    \item Fonctions Génératrices et Théorème Limite Central
\end{enumerate}

\section{Introduction à la Probabilité}

\subsection{Définition Classique de Probabilité}

\subsection{Règles de dénombrement}

\begin{enumerate}
    \item Règle de multiplication
    \item Règle d'addition
\end{enumerate}

\subsection{Règles de dénombrement II}

\begin{enumerate}
    \item Permutations
    \item Combinaisons
    \item Boules et Urnes
\end{enumerate}

\subsection{Règles de dénombrement III}

\begin{enumerate}
    \item Formule de Pascal
    \item Formule du Binome
    \item Coefficients Binomiaux
\end{enumerate}

\subsection{Espaces de Probabilité}

\begin{enumerate}
    \item Espace Fondamental
    \item Formule du Binôme
\end{enumerate}

\subsection{Probabilité Conditionnelle}

\begin{enumerate}
    \item Définition
    \item Formule de Bayes
    \item Conditionnement
\end{enumerate}

\subsection{Indépendance}

\begin{enumerate}
    \item Pairwise Independance
    \item Indépendance
    \item Conditionnement
\end{enumerate}

\section{Variables aléatoires}

\subsection{Variables aléatoires discrètes}

\subsection{Définition}

\subsection{Loi de Variables aléatoires discrètes}

\begin{enumerate}
    \item Loi Binomiale
    \item Loi Géométrique
    \item Loi Hypergéométrique
    \item Loi Binomiale Négative
    \item Loi de Poisson
    \item Processus de Poisson
\end{enumerate}

\subsection{Variables aléatoires continues}

\subsection{Définition}

\begin{enumerate}
    \item Fonction de Densité
    \item Fonction de Répartition
\end{enumerate}

\subsection{Loi de Variables aléatoires continues}

\begin{enumerate}
    \item Loi Uniforme
    \item Loi Exponentielle
    \item Loi Normale
    \item Loi Log-Normale
    \item Loi Gamma
    \item Loi Chi-Deux
    \item Loi Cauchy
\end{enumerate}

\subsection{Variables aléatoires simultanées}

\subsection{Définitions v.a. discrètes}

\begin{enumerate}
    \item Fonction de Masse Conjointe
    \item Fonction de Masse Marginale
    \item Fonction de Masse Conditionnelle
\end{enumerate}

\subsection{Définitions v.a. continues}

\begin{enumerate}
    \item Fonction de Densité Conjointe
    \item Fonction de Densité Marginale
    \item Fonction de Densité Conditionnelle
\end{enumerate}

\subsection{Variables aléatoires Indépendates}

\subsection{Somme de Variables Aléatoires}
\subsection{Transformation de Variables Aléatoires}

\section{Espérance de Variables Aléatoires}

\section{Fonction Génératrice et Théorème Limite Central}

\subsection{Fonction Génératrice}

\subsection{Théorèmes Limites}

\subsection{Loi des Grands Nombres}
\subsection{Théorème Limite Central}
\subsection{Inégalité de Bienyamé-Tchebychev}

\end{document}
\end{article}
