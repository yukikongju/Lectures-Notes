\documentclass{article}
\usepackage{hyperref}
\begin{document}
\title{Lecture Notes for Data Structures}
\author{Emulie Chhor}
\maketitle

\section{Overview of the field of Data Structures and Algorithms}

One of the most important field of computer science is about Data Structures and
algorithms, which is learn via 4 courses:

\begin{enumerate}
    \item Discrete Maths
    \item Data Structures
    \item Algorithms
    \item Theory of Computation
    \item Modèle de Recherche Opérationnelle (optional)
\end{enumerate}

\newtheorem{definition}{Definition}[subsection]
\newtheorem{theorem}{Theorem}[subsection]
\newtheorem{proposition}{Proposition}[section]
\newtheorem{property}{Propriété}[subsection]
\newtheorem*{remark}{Remarque}
\newtheorem*{problem}{Problème}
\newtheorem*{intuition}{Intuition}

\section{Why study Data Structures}

\section{How to study Data Structures}%
\label{sec:How to study Data Structures}

Lorsqu'on étudie les structures de données, on doit comprendre que
chaque structure de donnée possède un invariant qui lui donne ses propriétés,
et on doit préserver ces propriétés lors des opérations:
\begin{enumerate}
    \item Propriétés
    \item Invariant
    \item Helper Methods
    \item Opérations: insertion, suppresion, recherche
    \item Complexité des opérations: worse, best, average
    \item Implementation
\end{enumerate}

\section{Introduction}

The study of data structures can be divided in the following chapters:

\begin{enumerate}
    \item Basics Data Structures: Dynamic Arrays and Linked List
    \item Basics ADT: Bag, Stack and Queue
    \item Sorting Algorithms
    \item Heap
    \item Treap (optional)
    \item Binary Search Trees
    \item Splay (optional)
    \item Balanced Binary Trees
    \item Hash Table
    \item Graph Theory
\end{enumerate}

\section{Basic Data Structures}

\subsection{Overview}

Le premier chapitre en structure des données introduit les notions de
dynamic array et de linked list. En gros, ce sont des structures de
données qui sont utilisées pour implémenter des structures abstraites.\\

Ces structures de données supportent certaines opérations
\begin{enumerate}
    \item insertion
    \item deletion
    \item recherche
\end{enumerate}

\subsection{Dynamic Arrays}

Le dynamic array est utilisé surtout pour storer des éléments dont on
connait la taille et l'index parce qu'il requiert d'être resized

\begin{enumerate}
    \item insertion: si la capacité est atteinte, on double la capacité
	et on copie les données dans le nouvel array
    \item deletion: si la capacité est le quart de la taille du array,
	on copie les éléments dans un array qui est la moitié de la
	taille initiale
    \item recherche: recherche linéaire si éléments non-ordonnées,
	recherche binaire si les éléments sont ordonnées
\end{enumerate}

\textbf{Complexité du resize: Preuve Crédit-Débit}

On peut montrer que le resize prend un temps constant amorti par une
preuve crédit-débit. Soit 1\$, le cout pour push et pop.
On sait qu'à chaque puissance de n éléments, on va devoir insérer n éléments
et resize 1 fois. Ainsi, on va devoir:
\begin{enumerate}
    \item ajouter les éléments une première fois (on n'a pas atteint la
	capacité): n fois
    \item resize: coute 2n, car on doit pop n éléments et push n éléments
	dans le nouvel array
    \item nombre d'éléments: n
\end{enumerate}

Ainsi, le cout amorti est de $ \frac{n+2n}{n} = 3 $

\subsection{Linked List}

Il existe 3 types de linked list:
\begin{enumerate}
    \item Singly Linked List: Peuvent seulement être traversée du head
	au tail, car on n'a pas de pointeur au précédant
    \item Doubly Linked List: Can be traversed backward because we
	store the previous node, but takes twice the space
    \item Circular Linked List: the head and tail can reach each other
\end{enumerate}

Le linked list supporte les opérations suivantes:
\begin{enumerate}
    \item insertion: On traverse la liste séquentiellement, et on ajoute
	le nouveau noeud entre le previous et le prochain en sauvegardant
	le pointeur du prochain dans une variable temporaire
    \item deletion: On change le pointeur du previous et du prochain en
	sauvegardant le pointeur du prochain avant d'enlever le noeud
    \item recherche: On doit faire de la recherche séquentielle parce
	qu'on n'a pas accès aux indices
\end{enumerate}

\subsection{Dynamic Arrays vs Linked List}

\section{Basic Abstract Data Types}

\subsection{Overview}

Le bag, stack et queue sont les DS les plus basiques. On peut les
implémenter avec des arrays, mais il est préférable de le faire avec des
arraylist, puisqu'on ne connait pas leur taille d'avance (et on ne
travaille qu'avec le head/tail)

\subsection{Bag}

Le bag ne supporte qu'une seule opérations: insertion. Pour la recherche,
on ne peut faire qu'une recherche séquentielle, car les éléments n'ont
pas d'ordre.

\subsection{Stack}

Le stack, communément appelé FILO (first-in, last-out), est un type
abstrait qui est utilisé pour le cache et l'OS. Elle supporte les 3
opérations de base:
\begin{enumerate}
    \item push (insertion): ajoute le nouvel élément au top
    \item pop (suppression): si non vide, enleve l'élément eu top
    \item recherche: pas fait pour ça, mais on peut peek
\end{enumerate}

\subsection{Queue}

La queue, communément appelée FIFO (first-in, first-out), supporte
les 3 opérations de base
\begin{enumerate}
    \item enqueue (insertion): insère l'élément à la fin
    \item dequeue (suppression): si non vide, enlève l'élément en
	tête de la liste
    \item recherche: pas fait pour ça, mais on peut peek
\end{enumerate}

\section{Sorting Algorithms}

\subsection{Overview}

Idéalement, le meilleur algorithme de tri devrait être linéarithmique
et ne pas utiliser d'espace auxiliaire. Cependant, il n'existe pas de
sorting algorithms satisfaisant ces 2 propriétés. Ainsi, on peut
discerner quelques sorting algorithms:

\begin{enumerate}
    \item Selection Sort
    \item Insertion Sort
    \item Bubble Sort
    \item Mergesort
    \item Quicksort
    \item Heapsort
    \item Raddix Sort
\end{enumerate}

Notons qu'on apprend les sorting algorithms, non pas pour les implémenter,
mais plutôt parce que leur structure et temps de complexité sont de bons
introductions à l'analyse de la complexité. On veut donc être en mesure
de connaitre le input qui va donner un temps optimal, temps moyen, temps
pire temps.

\subsection{Selection Sort}
\subsection{Insertion Sort}
\subsection{Bubble Sort}
\subsection{Mergesort}
\subsection{Quicksort}
\subsection{Heapsort}
\subsection{Raddix Sort}

\section{Heap}

\subsection{Invariant}%
\label{sub:Invariant}

Un heap est une structure de données qui implémente des files de propriété.
Il existe 2 types de heap:
\begin{enumerate}
    \item MinHeap: $ parent < child $
    \item MaxHeap: $ parent > child $
\end{enumerate}

Remarquons qu'on a des inégalités strictes, car on ne veut pas ajouter
les mêmes valeurs au heap. De plus, l'implémentation est similaire,
la seule chose qui change est la comparaison qu'on fait: less vs more

\subsection{Helper Methods}%
\label{sub:Helper Methods}

Le heap possède 2 helpers methods: swim et sink

\textbf{Swim}

\begin{enumerate}
    \item Trouver le parent
    \item Échanger avec le parent jusqu'à l'invariant soit satisfait
	ou qu'on aille atteint le root
\end{enumerate}

\textbf{Sink}

\begin{enumerate}
    \item Trouver le plus grand/petit enfant
    \item Échanger avec l'enfant jusqu'à l'invariant soit satisfait
	ou qu'on atteigne un NIL
\end{enumerate}

\subsection{Operations}%
\label{sub:Operations}

\begin{enumerate}
    \item Insertion
	\begin{itemize}
	    \item ajouter l'élément à la dernière position
	    \item swim() pour retrouver sa bonne position
	\end{itemize}
    \item Suppression
    \begin{itemize}
        \item Échanger le root avec le dernier élément
	\item supprimer le dernier élément
	\item sink le root pour qu'il soit à la bonne position
    \end{itemize}
    \item Recherche: Puisqu'il n'y a pas de relations entre les
	2 enfants, on doit faire une recherche linéaire
\end{enumerate}

\subsection{Complexité des Opérations}%
\label{sub:Complexité des Opérations}


\section{Binary Search Trees}

\subsection{Overview}%
\label{sub:Overview}

Le BST est une structure de donnée utilisée, car elle maintient un ordre.
Son invariant: $ node.left < node < node.right $. Comme avec le heap,
on n'accepte pas de valeurs duppliquées

\subsection{Helper Methods}%
\label{sub:Helper Methods}

Les helpers methods sont:
\begin{enumerate}
    \item FindMin: trouver la plus petite valeur dans le sous-arbre en
	parcourant l'arbre en allant toujours vers la gauche
    \item FindMax: trouver la plus grande valeur dans le sous-arbre en
	parcourant l'arbre en allant toujours vers la droite
\end{enumerate}

Ces helpers methods sont utiles, car elles nous permet de choisir le
successeur/prédécesseur lors de la suppression

\subsection{Operations}%
\label{sub:Operations}

\begin{enumerate}
    \item Insertion: Parcourir l'arbre
	\begin{itemize}
	    \item Si element < node.val: allez à gauche
	    \item Si element > node.val: allez à droite
	    \item Si element = node.val: ne pas ajouter de duppliqué
	\end{itemize}
    \item Suppression
	\begin{itemize}
	    \item Parcourir le tree pour trouver l'élément à enlever
	    \item Choisir le prédécesseur/successeur si le noeud
		supprimé possède un enfant (ou plus)
	\end{itemize}
    \item Recherhe: Puisqu'on maintient une structure, on peut effectuer
	une recherche binaire
\end{enumerate}

\subsection{Complexité}%
\label{sub:Complexité}

\subsection{Implementation}%
\label{sub:Implementation}

Pour l'implémentation, on déclare une classe noeud qui a les paramètres
suivants:
\begin{enumerate}
    \item value
    \item left child
    \item right child
    \item (optional) parent
    \item (optional) height
\end{enumerate}

\subsubsection{Operations in Java}%
\label{ssub:Operations in Java}

TODO

\section{Balanced Binary Search Trees}

\subsection{Overview}%
\label{sub:Overview}

Les balanced Binary Trees sont la version 2.0 des BST. Essentiellement,
on a vu que les arbres BST ne garantissent pas une hauteur logarithmique,
alors le temps de complexité de ses opérations peut aller jusqu'à O(n),
ce qui n'est pas bon. Les BBST sont des arbres, qui, par leurs invariants,
permettent de garantir une hauteur logarithmique et ainsi un temps
de complexité O(log n) au pire.\\

On discerne 3 BBST:
\begin{enumerate}
    \item AVL Trees: rotations et balance factor
    \item Red-Black Trees: rotation et couleur
    \item B-Trees: noeuds avec plusieurs éléments
\end{enumerate}

Notons que la recherche pour le AVL et le Red-Black Tree se font avec
un binary search, car ce sont des BST balancés.

\subsection{AVL Trees}

\subsubsection{Overview}%
\label{ssub:Overview}

Les arbres AVL qui utilisent la rotation et un balance factor pour
garantir un temps logarithmique au pire. Pour ce faire, on doit
respecter l'invariant suivant: la différence de hauteur entre 2
sous-arbres doit être d'au plus 1.

\subsubsection{Helper Methods: Rotations}%
\label{ssub:Helper Methods}

\begin{enumerate}
    \item Rotations: left and right
    \item FindHeight: Calculer balance factor
\end{enumerate}

\textbf{Rotations}

Pour un left rotation, on a un arbre right heavy et on veut le rendre
left heavy:
\begin{enumerate}
    \item right child becomes the root
    \item root becomes left child of new root
    \item rigt child.left becomes right child of old root
\end{enumerate}

De la même façon, pour un right rotation, on a un arbre left heavy et
on veut le rendre right heavy

\textbf{FindHeight}

On veut trouver la hauteur du subtree, car cela nous permet de calculer le
balance factor.
$$ return 1 + max(findHeight(node.left), findHeight(node.right)) $$

\subsubsection{Operations}%
\label{ssub:Operations}

\textbf{Operations}

\begin{enumerate}
    \item Insertion
	\begin{itemize}
	    \item Perform BST insertion
	    \item Update balance factor
	    \item Perform Rotation if invariant violated
	\end{itemize}
    \item Suppression
	\begin{itemize}
	    \item Perform BST deletion
	    \item Update balance factor
	    \item Perform Rotation if invariant violated
	\end{itemize}
\end{enumerate}

\textbf{Updating Balance Factor}

\begin{enumerate}
    \item recalculer hauteur à partir du root
\end{enumerate}

\begin{remark}[Balance Factor]
    $$ \text{ return } height(x.left) - height(x.right) $$
\end{remark}

\textbf{Balance an AVL Tree}

Lorsque l'invariant est violé (bf >1), on doit rotater les branches. En
considérant que zig: left heavy et zag: right heavy. On
doit considérer 4 cas:
\begin{enumerate}
    \item zig-zig: rotateRight
    \item zag-zag: rotateRight
    \item zig-zag: rotateLeft pour avoir zig-zig
    \item zag-zig: rotateRight pour avoir zag-zag
\end{enumerate}

\textbf{Associer le balance factor à chaque cas}

TODO

\subsubsection{Implementation}%
\label{ssub:Implementation}

Princeton Implementation:
\url{https://algs4.cs.princeton.edu/code/edu/princeton/cs/algs4/AVLTreeST.java.html}

\subsection{Red-Black Trees}

\subsubsection{Overview}%
\label{ssub:Overview}

Les red-black trees utilisent la rotation et la couleur pour garantir
une hauteur logarithmique. L'invariant est donné p/r à la hauteur noire
de l'arbre: les noeud de couleur rouge = 0, alors que les noeuds de
couleur noire comptent pour 1. Les red0black trees respectent les propriétés
suivantes:
\begin{enumerate}
    \item Node is red at insertion
    \item Root node and NIL nodes are black
    \item Child of red nodes are black
    \item Black height from root to NIL is the same for all paths
\end{enumerate}

Ils possèdent les caractéristiques suivantes:
\begin{enumerate}
    \item Shortest Path: alternating red-black trees
    \item Longest Path: only black node
\end{enumerate}

\subsubsection{Operations}%
\label{ssub:Operations}

\begin{enumerate}
    \item Insertion
	\begin{itemize}
	    \item
	\end{itemize}
    \item Suppression
\end{enumerate}

\subsubsection{Implementation}%
\label{ssub:Implementation}

Princeton Red-Black Tree Implementation:
\url{https://algs4.cs.princeton.edu/33balanced/RedBlackBST.java.html}

\subsection{B-Trees}

\section{Hash Table}

\subsection{Collisions}

\subsection{Load Factor}

\subsection{Dealing with Collision}

\subsubsection{Separate Chaining}
\subsubsection{Open Adressing}

\textbf{Linear Probing}
\textbf{Double Hashing}

\section{Graph Theory}

\subsection{Terminology}
\subsection{Graph Representation}

\begin{enumerate}
    \item Edges List
    \item Adjacency List
    \item Adjacency Matrix
\end{enumerate}

\subsection{Problems in Graph Theory}

\end{document}
\end{article}
