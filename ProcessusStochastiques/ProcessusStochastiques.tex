\documentclass{article}
\usepackage{amsmath}
\usepackage{amsfonts}
\usepackage{amsthm}
\usepackage{parskip}
\usepackage{textgreek}
\begin{document}
\title{Notes de Cours - Processus Stochastiques}
\author{Emulie Chhor}
\maketitle

\section*{Introduction}

Le cours de Processus Stochastiques comporte 6 chapitres:

    \begin{enumerate}
	\item Généralités sur les Processus
	\item Chaîne de Markov à Temps Discret
	\item Processus de Poisson
	\item Chaîne de Markov à Temps Continu
	\item Martingales à Temps Discret
	\item Mouvement Brownien
    \end{enumerate}

\newtheorem{definition}{Definition}[subsection]
\newtheorem{theorem}{Theorem}[subsection]
\newtheorem{corollary}{Corollary}[subsection]
\newtheorem{lemma}[theorem]{Lemma}
\newtheorem{proposition}{Proposition}[section]
\newtheorem{axiom}{Axiome}
\newtheorem{property}{Propriété}[subsection]
\newtheorem*{remark}{Remarque}
\newtheorem*{problem}{Problème}
\newtheorem*{intuition}{Intuition}

\subsection{Pourquoi étudier les Processus Stochastique?}

\pagebreak
\section{Genéralitées sur les processus}

\subsection{Overview}

\subsection{Processus Stochastiques}

\begin{definition}[Processus Stochastiques à temps discret]
\end{definition}

\begin{definition}[Processus Stochastiques à temps continu]
\end{definition}

\subsection{Marche aléatoire}



\subsection{Processus de Branchement}
\subsection{Processus de Comptage}




\section{Chaîne de Markov a temps discret}
\subsection{définition, exemples, probabilitées/matrice de transition, classification d’états}
\subsection{distribution stationnaire, théorème érgodique et applications}
\pagebreak

\section{Processus de Poisson}
\subsection{Overview}
\subsection{définitions et propriétés}
\subsection{processus de Poisson composée et applications à l’assurance.}
\pagebreak

\section{Chaîne de Markov a temps continu}
\subsection{Overview}
\subsection{définition, exemples, matrice d’intensité, classification d’états}
\subsection{files d’attente et processus de naissance-mort}
\subsection{distribution stationnaire et comportement asymptotique, applications}
\pagebreak

\section{Martingales a temps discret}
\subsection{Overview}
\subsection{définitions and Propriétés}
\subsection{théorème d’arrêt et applications.}
\pagebreak

\section{Mouvement Brownian}
\subsection{Overview}
\subsection{définitions et construction}
\subsection{brève introduction aux martingales à temps continu et thérème d’arrêt}
\subsection{mouvement brownien géométrique and applications à la finance.}

\pagebreak

\end{document}
\end{article}
