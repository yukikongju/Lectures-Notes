\documentclass{article}
\usepackage{amsmath}
\usepackage{amsfonts}
\usepackage{amsthm}
\usepackage{parskip}
\usepackage{textgreek}
\begin{document}
\title{Résumé des Cours - Processus Stochastiques}
\author{Emulie Chhor}
\maketitle

\section*{Introduction}

Ce document est un résumé des idées importantes présentées lors de chaque cours
de MAT2717

\newtheorem{definition}{Definition}[subsection]
\newtheorem{theorem}{Theorem}[subsection]
\newtheorem{corollary}{Corollary}[subsection]
\newtheorem{lemma}[theorem]{Lemma}
\newtheorem{proposition}{Proposition}[section]
\newtheorem{axiom}{Axiome}
\newtheorem{property}{Propriété}[subsection]
\newtheorem*{remark}{Remarque}
\newtheorem*{problem}{Problème}
\newtheorem*{intuition}{Intuition}

% \renewcommand*{idea}{Big Big Idea}
% \renewcommand*{outline}{Outline}

\pagebreak
\section{Semaine 1}
\subsection{Théorie 1}

\subsection*{Big Idea}

Dans le premier cours de probabilité, on s'est penché sur la notion de variable
aléatoire. Dans ce cours-ci, on voit qu'on doit ajouter un autre paramètre à
ces variables aléatoires: le temps. On ne parle donc plus de variables aléaoires,
mais de processus stochastiques, qui sont des variables aléatoires prenant en
considération le temps. Ainsi, on parle donc de processus stochastique à temps
discrete et à temps continu, dépendemment si on considère le temps discret ou
continu.

\subsection*{Outline}
\begin{enumerate}
    \item Processus Stochastiques
    \item Marche Aléatoire
\end{enumerate}

\subsection{Processus Stochastiques}

Un processus stochastique est une v.a aléatoire qui est dépendante du temps.
Elle peut être discrète ou continue (dépendemment à quelle fréquence on
observe la valeur de la v.a)\\

On dit que le temps est déterministe. Why?\\

On peut écrire un p.s. $X( \omega, t )$, avec omega: les scenarios
possibles et t, le temps. Si on fixe le temps, on parle de variable
aléatoire. Si on fixe l'évènement et on laisse le temps libre, on
parle de trajectoire.

\subsection{Marche Aléatoire}

La marche aléatoire représente la valeur d'un p.s. discret à un temps donnée.
On peut la modéliser de deux façons:
\begin{enumerate}
    \item Conditionnement: valeur à la dernière position $\pm$ 1
    \item Somme de variables indicatrices: $X_n = Y_1 + ... +Y_n$
\end{enumerate}

Il est à noter qu'on préfère représenter une marche aléatoire par une
somme de variable indicatrice puisque ça rend le calcul plus facile.

Espérance: $\mathbb{E} (X_n) = n (2p-1)$
Variance: $ Var(X_n)= 4np(1-p)$

\subsection{D'ou vient la formule de l'esperance et de la variance d'une
marche aleatoire}

Il s'agit de l'esperance et de la variance pour une somme de variable
indicatrice.

$E(x) = n(1 * p + (-1) (1-p))$ : nombre de pas x esperance succes/echec au
i-ème pas
$V(X)$ = somme des variances = $ n [ E(Y^2) - E(Y)^2 ]$

\subsection{Théorie 2}
\subsection*{Big Idea}

Ce cours introduisait les chaines de Markov. Comme on l'a vu au dernier
cours, les processus stochastiques sont des v.a. qui sont en fonction du
temps. Certaine PS ont la propriété de Markov, c-à-d qu'on n'a que besoin
de la dernière valeur pour déterminer la prochaine position et non tout
l'historique. On introduit la notion de matrice de transition pour modéliser
la probabilité de transitionner d'un état à l'autre, et la notion d'état
absorbant, qui est un état duquel on ne peut se sortir une fois atteinte.

\subsection*{Outline}

\begin{enumerate}
    \item Rappel: Processus Stochastiques
    \item Marche Aléatoire
    \item Processus de Branchement
    \item Chaines de Markov: Proba de transition et État absorbant
\end{enumerate}

\subsubsection{Marche Aléatoire}%
\label{ssub:marche_aléatoire}

Il faut se rappeler que la marche aléatoire sert à modéliser un p.s discret.
On a vu qu'il y avait deux façons de l'interpréter:
\begin{enumerate}
    \item Conditionnement: previous +/- 1 dépendemment du succès/échec
    \item Somme de variables indicatrices (bernouilli)
\end{enumerate}

Essentiellement, la probabilité est déterminer par une Bernouilli, puisqu'on
considère que chaque coin flip est iid. On peut aussi visualiser la
marche aléatoire comme un arbre de décision.\\

Exemple: t=2, coin flip: +1 si Head, -1 si Tail
\begin{enumerate}
    \item $x_2 = 2$ avec proba $p^2$ : {HH}
    \item $x_2 = 0$ avec proba $ 2p (1-p)$ : {HT, TH}
    \item $x_2 = -2$ avec proba $ (1-p)^2$ : {TT}
\end{enumerate}

Notons qu'après 2 coins flip, on ne peut pas avoir 1 ou -1, car apres le
premier coin flip, on est à 1 ou -1, et on ne peut que se déplacer de 1
vers le haut ou vers le bas

On a aussi vu qu'on peut tracer la trajectoire de la marche aléatoire, qui
représente la valeur de $x_n$ selon le temps.

\subsubsection{Processus de Branchement}%
\label{ssub:processus_de_branchement}

Le processus de Banchament est une façon de modéliser la générations
d'évènement. Par exemple, en reinforcement learning, ça représente la
transition d'état. Visuellement, on a un arbre et chaque branche mène
à un autre état, associé à une certaine probabilité.

\subsubsection{Chaine de Markov}%
\label{ssub:chaine_de_markov}

Comme mentionné, une chaine de Markov est une p.s qui satisfait à la
propriété de Markov, c-à-d qu'on n'a que besoin de l'état précédant pour
déterminer le futur et non tout l'historique. Dans la vraie vie, une
chaine de markov ne représente pas nécessairement la situation, mais
on fait cette hypothèse afin de simplifier les calculs et éviter de
trainer trop d'information. On écrit:
$$ \mathbb{P} (X_{n+1} = j) = (X_{n+1} = j | X_n + ... + X_0) =
(X_{n+1} = j | X_n) $$

Pour représenter la probabilité de transition d'un état à un autre, on
utilise une matrice de transition.

Finalement, il est important de comprendre la notion d'état absorbant.
Une fois qu'on entre dans cet état, on ne peut plus en sortir. L'exemple
donné en classe concernait le casino: une fois qu'on a plus d'argent à
gamble, on ne peut plus gagner d'argent et on reste à zéro.

Notons que si une chaine de Markov à un état absorbant, il est fort
probable qu'il y ait plusieurs états absorbant et qu'on puisse observer
un pattern.

\subsection{TP}

Cette semaine, le TP se penchait sur 2 sujets:
\begin{enumerate}
    \item Rappel sur les concepts en probabilité
    \item Intro aux Processus Stochastiques
\end{enumerate}

\subsubsection{{Rappel sur les concepts en probabilité}}

\begin{enumerate}
    \item Principe Inclusion-Exclusion
    \item Probilités Totales pour généraliser le nombre de tirage avec et sans
	remise: $ P(A_1|B) P(B) + P(A_2|B)P(B) $
    \item Trouver la densité marginale d'une v.a: si continues, on doit essayer
	de retrouver une fonction de densité pour éviter d'avoir à résoudre
	l'intégrale
\end{enumerate}

\subsubsection{Intro aux Processus Stochastiques}

\begin{enumerate}
    \item Déterminer si une p.s. est Markov ou non
    \item Trouver/Lire une Matrice de Transition
    \item Déterminer la probabilité de transition d'un état à un autre
    \item Tracer le Graphe
\end{enumerate}

\subsection{Intro aux Processus Stochastiques}

\pagebreak

\section{Semaine 2}
\subsection{Théorie 1: Probabilité de Transition à plusieurs étapes}
\subsection*{Big Idea}

Ce cours-ci nous présentait certaines méthodologies pour déterminer
la probabilité de transition à plusieurs étapes pour des P.S discrèts
à temps homogène satisfaisant la propriété de Markov

\subsection*{Outline}
\begin{enumerate}
    \item Modèles: Marche aléatoire, Processus de Branchement, Modèle de
	diffusion de Gaz
    \item Probabilité de transition avec probabilité conditionnelle
    \item Probabilité de transition avec Relation Chapman-Kolmogrov
\end{enumerate}

\subsubsection{Modèles: Marche aléatoire, Processus de Branchement, Modèle de    diffusion de Gaz}

Dans un premier temps, on a réviser les notions de marche aléatoire,
processus de branchement et  modèle de diffusion de gaz. Essentiellement,
on voulait s'apercevoir que
\begin{enumerate}
    \item La valeur de $X_n$ ne peut que se déplacer de 1 (dans la plupart
	du temps)
    \item Conditionnement: on peut trouver la probabilité de transition
	d'un état à l'autre par conditionnement p/r à l'étape précédante
\end{enumerate}

\subsubsection{Probabilité de transition avec probabilité conditionnelle}

Dans les cours précédants, on nous donnait la matrice de transition pour
déterminer la probabilité de se déplacer d'un état à l'autre. Ici, on
voit 2 résultats importants:
\begin{enumerate}
    \item si P.S Markov et homogène (même proba peu import le temps),
	alors $$ P[X_2 = 3 , X_1 = 2 | X_0 = 1] = P[X_{n+2} = 3, X_{n+1} = 2
	| X_n =1] $$, ce qui veut dire que la proba de transition d'état ne
	dépend pas d'un temps quelconque, mais que de la "distance" (nombres
	de pas) entre 2 états, si P.S Markov est homogène
    \item On peut utiliser la probabilité conditionnelle pour déterminer la
	proba de transition entre 2 états
\end{enumerate}

\textbf{Preuve pour 1}

Puisqu'on a une chaine de Markov, les les 2 états sont indépendants et on peut
utiliser la règle de multiplication:

$$ P[X_2 = 3 , X_1 = 2 | X_0 = 1] $$
$$ = \frac{P(X_2 = 3, X_1 =2, X_0 = 1)}{P(X_0 = 1)}  $$
$$ = (\frac{P(X_2 = 3, X_1 =2, X_0 = 1)}{P(X_0 = 1)})
(\frac{P(X_1 = 3, X_1 = 2)}{P(X_0=1)} ) $$
$$ ... $$

\textbf{Probabilité Conditionnelle pour trouver la probabilité de transition}

Plus généralement, pour trouver la probabilité de transition à plusieurs
étapes, on doit conditionner par les étapes intermédiaires:

\begin{enumerate}
    \item 2 étapes: $ P(X_2 = 1 | X_0 =1) = \sum^{n}_{i=1} P(X_2 = 2, X_1 = k | X_0=1) $
    \item 3 étapes: $ P(X_3 = 1, X_2 = 1 | X_0 =1) = P(X_3 = 1, X_2 = 1)
	P(X_2 = 1 | X_1 =1) P(X_1 = 1 | X_0 =2)$
    \item m étapes: $ P(X_{n+m} = j | X_n = i) = P(n=j |X_0 = i) $
\end{enumerate}

\subsubsection{Probabilité de transition avec Relation Chapman-Kolmogrov}

La relation de Chapman-Kolmogrov est pratique, car elle nous permet de généraliser
le calcul pour la probabilité de transition en m étapes. Au lieu de multiplier
individuellement les intermédiaires à chaque étapes, on peut utiliser la
multiplication matricielle pour obtenir la matrice de transition après m
étapes. On note $ P^(m) = P^m $, P: matrice de transition.\\

Par exemple, si on veut la proba de transition à 3 étapes, on n'a qu'à faire
$$ P \times P \times P $$

\subsection{Théorie 2}
\subsection*{Big Idea}
\subsection*{Outline}
\begin{enumerate}
    \item Preuve de la Relation Chapman-Kolmogrov
    \item Classes: Communication entre états
\end{enumerate}

\subsubsection{Preuve de la Relation Chapman-Kolmogrov}

La relation de Chapman-Kolmogov nous dit que si on a une chaine de
Markov à temps homogène, on peut multiplier les matrices de transition
pour obtenir la probabilité de passer de l'état n à l'état n+k
\begin{enumerate}
    \item $ P^{m}=P^m$
    \item $ P^{m+n}=P^m P^n$
\end{enumerate}

La preuve de cette relation se fait par induction, et on montre
l'étape d'induction avec la propriété des chaines de Markov et la
probabilité conditionnelle. Essentiellement, on veut montrer qu'en
multipliant chaque entrées entre elles, on obtient la proba à l'étape
n x la proba à l'étape 1 et qu'en multipliant ces matrices, on obtient
$ P^{n+1}$

\textbf{Cas de base}

\begin{enumerate}
    \item t=0: matrice identité, car on ne change pas d'état
    \item t=1: $P^1 = P$
\end{enumerate}

\textbf{Étape d'induction}

\begin{enumerate}
    \item Puisque CM est homogène dans le temps, alors
	$ P^{n+1} \Longrightarrow P_{ij} ^{n+1} = P[X_{n+1} = j | X_n = i]
	= P[X_{n+1} =j | X_0 = i] $
    \item En considèrant les états intermédiaires, on obtient
	$ P[X_{n+1}=j | X_0 = i] = \sum^{n}_{i=1} P[X_{n+1} = j | X_n = k]$
    \item Puisque c'est une chaine de Markov, on a que
	$$ \sum^{n}_{i=1} P[X_{n+1} = j, X_n = k | X_0 = i]$$
    \item Par probabilité conditionnelle, on a que
	$$ \sum^{n}_{i=1} \frac{P[X_{n+1} = j, X_n = k, X_0 = i]}
	{P[X_0=i]} \frac{P[X_n = k, X_0=i]}{P[X_n = k, X_0=i]} $$
    \item En réarrangeant les termes, on obtient la proba de transition
	à l'étape n et en une étape
	$$ \sum^{n}_{i=1} \frac{P[X_{n+1} = j, X_n = k, X_0 = i]}
	{P[X_n = k, X_0=i]} \frac{P[X_n = k, X_0=i]}{P[X_0=i]} $$
    \item et on applique la définition de la multiplication matricielle
\end{enumerate}

\subsubsection{Classes: Communication entre états}

La deuxième partie du cours mettait l'emphase sur la séparation des
états par des classes. Essentiellement, on veut être capable de
différentier les différentes classes dans une matrice de transition
afin de déterminer si la matrice est irréductible ou non.

\textbf{Qu'est-ce qu'une classe}

Tous les états qui communiquent entre eux forment une classe. Pour que
2 états communiquent entre entre elles, il faut qu'elles satisfassent
les conditions suivantes:
\begin{enumerate}
    \item Réflexive: $ i \leftrightarrow i$
    \item Symétrique: $ i \leftrightarrow j \iif j \leftrightarrow i $
    \item Transitive: $ i \to j, j \tok \Longrightarrow i \to k $
\end{enumerate}

\textbf{Quelles sont les types de classes}

\begin{enumerate}
    \item État absorbant: On ne peut jamais quitter cet état après y
	être entré
    \item État de non-retour: une fois qu'on n'a quitté cet état,
	on ne peut jamais revenir
\end{enumerate}

\textbf{Qu'est-ce qu'une matrice irréductible}

On dit qu'une matrice est irréductible si tous les états communiquent
entre elles. Autrement dit, tous les états forment une seule classe.

\subsection{TP}

TODO

\pagebreak
\section{Semaine 3}
\subsection{Théorie 1}
\subsection*{Big Idea}
\subsection*{Outline}
\begin{enumerate}
    \item
    \item
    \item
\end{enumerate}
\subsection{Théorie 2}
\subsection*{Big Idea}
\subsection*{Outline}
\begin{enumerate}
    \item
    \item
    \item
\end{enumerate}
\subsection{TP}

\pagebreak
\section{Semaine 4}
\subsection{Théorie 1}
\subsection*{Big Idea}
\subsection*{Outline}
\begin{enumerate}
    \item
    \item
    \item
\end{enumerate}
\subsection{Théorie 2}
\subsection*{Big Idea}
\subsection*{Outline}
\begin{enumerate}
    \item
    \item
    \item
\end{enumerate}
\subsection{TP}

\pagebreak
\section{Semaine 5}
\subsection{Théorie 1}
\subsection*{Big Idea}
\subsection*{Outline}
\begin{enumerate}
    \item
    \item
    \item
\end{enumerate}
\subsection{Théorie 2}
\subsection*{Big Idea}
\subsection*{Outline}
\begin{enumerate}
    \item
    \item
    \item
\end{enumerate}
\subsection{TP}

\pagebreak
\section{Semaine 6}
\subsection{Théorie 1}
\subsection*{Big Idea}
\subsection*{Outline}
\begin{enumerate}
    \item
    \item
    \item
\end{enumerate}
\subsection{Théorie 2}
\subsection*{Big Idea}
\subsection*{Outline}
\begin{enumerate}
    \item
    \item
    \item
\end{enumerate}
\subsection{TP}

\pagebreak
\section{Semaine 7}
\subsection{Théorie 1}
\subsection*{Big Idea}
\subsection*{Outline}
\begin{enumerate}
    \item
    \item
    \item
\end{enumerate}
\subsection{Théorie 2}
\subsection*{Big Idea}
\subsection*{Outline}
\begin{enumerate}
    \item
    \item
    \item
\end{enumerate}
\subsection{TP}

\pagebreak
\section{Semaine 8}
\subsection{Théorie 1}
\subsection*{Big Idea}
\subsection*{Outline}
\begin{enumerate}
    \item
    \item
    \item
\end{enumerate}
\subsection{Théorie 2}
\subsection*{Big Idea}
\subsection*{Outline}
\begin{enumerate}
    \item
    \item
    \item
\end{enumerate}
\subsection{TP}

\pagebreak
\section{Semaine 9}
\subsection{Théorie 1}
\subsection*{Big Idea}
\subsection*{Outline}
\begin{enumerate}
    \item
    \item
    \item
\end{enumerate}
\subsection{Théorie 2}
\subsection*{Big Idea}
\subsection*{Outline}
\begin{enumerate}
    \item
    \item
    \item
\end{enumerate}
\subsection{TP}

\pagebreak
\section{Semaine 10}
\subsection{Théorie 1}
\subsection*{Big Idea}
\subsection*{Outline}
\begin{enumerate}
    \item
    \item
    \item
\end{enumerate}
\subsection{Théorie 2}
\subsection*{Big Idea}
\subsection*{Outline}
\begin{enumerate}
    \item
    \item
    \item
\end{enumerate}
\subsection{TP}

\pagebreak
\section{Semaine 11}
\subsection{Théorie 1}
\subsection*{Big Idea}
\subsection*{Outline}
\begin{enumerate}
    \item
    \item
    \item
\end{enumerate}
\subsection{Théorie 2}
\subsection*{Big Idea}
\subsection*{Outline}
\begin{enumerate}
    \item
    \item
    \item
\end{enumerate}
\subsection{TP}

\pagebreak
\section{Semaine 12}
\subsection{Théorie 1}
\subsection*{Big Idea}
\subsection*{Outline}
\begin{enumerate}
    \item
    \item
    \item
\end{enumerate}
\subsection{Théorie 2}
\subsection*{Big Idea}
\subsection*{Outline}
\begin{enumerate}
    \item
    \item
    \item
\end{enumerate}
\subsection{TP}

\pagebreak
\section{Semaine 13}
\subsection{Théorie 1}
\subsection*{Big Idea}
\subsection*{Outline}
\begin{enumerate}
    \item
    \item
    \item
\end{enumerate}
\subsection{Théorie 2}
\subsection*{Big Idea}
\subsection*{Outline}
\begin{enumerate}
    \item
    \item
    \item
\end{enumerate}
\subsection{TP}

\pagebreak
\section{Semaine 14}
\subsection{Théorie 1}
\subsection*{Big Idea}
\subsection*{Outline}
\begin{enumerate}
    \item
    \item
    \item
\end{enumerate}
\subsection{Théorie 2}
\subsection*{Big Idea}
\subsection*{Outline}
\begin{enumerate}
    \item
    \item
    \item
\end{enumerate}
\subsection{TP}

\pagebreak
\section{Semaine 15}
\subsection{Théorie 1}
\subsection*{Big Idea}
\subsection*{Outline}
\begin{enumerate}
    \item
    \item
    \item
\end{enumerate}
\subsection{Théorie 2}
\subsection*{Big Idea}
\subsection*{Outline}
\begin{enumerate}
    \item
    \item
    \item
\end{enumerate}
\subsection{TP}

\pagebreak
\section{Semaine 16}
\subsection{Théorie 1}
\subsection*{Big Idea}
\subsection*{Outline}
\begin{enumerate}
    \item
    \item
    \item
\end{enumerate}
\subsection{Théorie 2}
\subsection*{Big Idea}
\subsection*{Outline}
\begin{enumerate}
    \item
    \item
    \item
\end{enumerate}
\subsection{TP}

\pagebreak
\end{document}
\end{article}
