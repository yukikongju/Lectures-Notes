\documentclass{article}
\usepackage{amsmath}
\usepackage{amsfonts}
\usepackage{amsthm}
\usepackage{parskip}
\usepackage{textgreek}
\begin{document}
\title{Résumé des Cours - Processus Stochastiques}
\author{Emulie Chhor}
\maketitle

\section*{Introduction}

Ce document est un résumé des idées importantes présentées lors de chaque cours
de MAT2717

\newtheorem{definition}{Definition}[subsection]
\newtheorem{theorem}{Theorem}[subsection]
\newtheorem{corollary}{Corollary}[subsection]
\newtheorem{lemma}[theorem]{Lemma}
\newtheorem{proposition}{Proposition}[section]
\newtheorem{axiom}{Axiome}
\newtheorem{property}{Propriété}[subsection]
\newtheorem*{remark}{Remarque}
\newtheorem*{problem}{Problème}
\newtheorem*{intuition}{Intuition}

% \renewcommand*{idea}{Big Big Idea}
% \renewcommand*{outline}{Outline}

\pagebreak
\section{Semaine 1}
\subsection{Théorie 1}

\subsection*{Big Idea}

Dans le premier cours de probabilité, on s'est penché sur la notion de variable
aléatoire. Dans ce cours-ci, on voit qu'on doit ajouter un autre paramètre à
ces variables aléatoires: le temps. On ne parle donc plus de variables aléaoires,
mais de processus stochastiques, qui sont des variables aléatoires prenant en
considération le temps. Ainsi, on parle donc de processus stochastique à temps
discrete et à temps continu, dépendemment si on considère le temps discret ou
continu.

\subsection*{Outline}
\begin{enumerate}
    \item Processus Stochastiques
    \item Marche Aléatoire
\end{enumerate}

\subsection{Processus Stochastiques}

Un processus stochastique est une v.a aléatoire qui est dépendante du temps.
Elle peut être discrète ou continue (dépendemment à quelle fréquence on
observe la valeur de la v.a)\\

On dit que le temps est déterministe. Why?\\

On peut écrire un p.s. $X( \omega, t )$, avec omega: les scenarios
possibles et t, le temps. Si on fixe le temps, on parle de variable
aléatoire. Si on fixe l'évènement et on laisse le temps libre, on
parle de trajectoire.

\subsection{Marche Aléatoire}

La marche aléatoire représente la valeur d'un p.s. discret à un temps donnée.
On peut la modéliser de deux façons:
\begin{enumerate}
    \item Conditionnement: valeur à la dernière position $\pm$ 1
    \item Somme de variables indicatrices: $X_n = Y_1 + ... +Y_n$
\end{enumerate}

Il est à noter qu'on préfère représenter une marche aléatoire par une
somme de variable indicatrice puisque ça rend le calcul plus facile.

Espérance: $\mathbb{E} (X_n) = n (2p-1)$
Variance: $ Var(X_n)= 4np(1-p)$

\subsection{D'ou vient la formule de l'esperance et de la variance d'une
marche aleatoire}

Il s'agit de l'esperance et de la variance pour une somme de variable
indicatrice.

$E(x) = n(1 * p + (-1) (1-p))$ : nombre de pas x esperance succes/echec au
i-ème pas
$V(X)$ = somme des variances = $ n [ E(Y^2) - E(Y)^2 ]$

\subsection{Théorie 2}
\subsection*{Big Idea}
\subsection*{Outline}
\begin{enumerate}
    \item
    \item
    \item
\end{enumerate}
\subsection{TP}

\pagebreak

\section{Semaine 2}
\subsection{Théorie 1}
\subsection*{Big Idea}
\subsection*{Outline}
\begin{enumerate}
    \item
    \item
    \item
\end{enumerate}
\subsection{Théorie 2}
\subsection*{Big Idea}
\subsection*{Outline}
\begin{enumerate}
    \item
    \item
    \item
\end{enumerate}
\subsection{TP}

\pagebreak
\section{Semaine 3}
\subsection{Théorie 1}
\subsection*{Big Idea}
\subsection*{Outline}
\begin{enumerate}
    \item
    \item
    \item
\end{enumerate}
\subsection{Théorie 2}
\subsection*{Big Idea}
\subsection*{Outline}
\begin{enumerate}
    \item
    \item
    \item
\end{enumerate}
\subsection{TP}

\pagebreak
\section{Semaine 4}
\subsection{Théorie 1}
\subsection*{Big Idea}
\subsection*{Outline}
\begin{enumerate}
    \item
    \item
    \item
\end{enumerate}
\subsection{Théorie 2}
\subsection*{Big Idea}
\subsection*{Outline}
\begin{enumerate}
    \item
    \item
    \item
\end{enumerate}
\subsection{TP}

\pagebreak
\section{Semaine 5}
\subsection{Théorie 1}
\subsection*{Big Idea}
\subsection*{Outline}
\begin{enumerate}
    \item
    \item
    \item
\end{enumerate}
\subsection{Théorie 2}
\subsection*{Big Idea}
\subsection*{Outline}
\begin{enumerate}
    \item
    \item
    \item
\end{enumerate}
\subsection{TP}

\pagebreak
\section{Semaine 6}
\subsection{Théorie 1}
\subsection*{Big Idea}
\subsection*{Outline}
\begin{enumerate}
    \item
    \item
    \item
\end{enumerate}
\subsection{Théorie 2}
\subsection*{Big Idea}
\subsection*{Outline}
\begin{enumerate}
    \item
    \item
    \item
\end{enumerate}
\subsection{TP}

\pagebreak
\section{Semaine 7}
\subsection{Théorie 1}
\subsection*{Big Idea}
\subsection*{Outline}
\begin{enumerate}
    \item
    \item
    \item
\end{enumerate}
\subsection{Théorie 2}
\subsection*{Big Idea}
\subsection*{Outline}
\begin{enumerate}
    \item
    \item
    \item
\end{enumerate}
\subsection{TP}

\pagebreak
\section{Semaine 8}
\subsection{Théorie 1}
\subsection*{Big Idea}
\subsection*{Outline}
\begin{enumerate}
    \item
    \item
    \item
\end{enumerate}
\subsection{Théorie 2}
\subsection*{Big Idea}
\subsection*{Outline}
\begin{enumerate}
    \item
    \item
    \item
\end{enumerate}
\subsection{TP}

\pagebreak
\section{Semaine 9}
\subsection{Théorie 1}
\subsection*{Big Idea}
\subsection*{Outline}
\begin{enumerate}
    \item
    \item
    \item
\end{enumerate}
\subsection{Théorie 2}
\subsection*{Big Idea}
\subsection*{Outline}
\begin{enumerate}
    \item
    \item
    \item
\end{enumerate}
\subsection{TP}

\pagebreak
\section{Semaine 10}
\subsection{Théorie 1}
\subsection*{Big Idea}
\subsection*{Outline}
\begin{enumerate}
    \item
    \item
    \item
\end{enumerate}
\subsection{Théorie 2}
\subsection*{Big Idea}
\subsection*{Outline}
\begin{enumerate}
    \item
    \item
    \item
\end{enumerate}
\subsection{TP}

\pagebreak
\section{Semaine 11}
\subsection{Théorie 1}
\subsection*{Big Idea}
\subsection*{Outline}
\begin{enumerate}
    \item
    \item
    \item
\end{enumerate}
\subsection{Théorie 2}
\subsection*{Big Idea}
\subsection*{Outline}
\begin{enumerate}
    \item
    \item
    \item
\end{enumerate}
\subsection{TP}

\pagebreak
\section{Semaine 12}
\subsection{Théorie 1}
\subsection*{Big Idea}
\subsection*{Outline}
\begin{enumerate}
    \item
    \item
    \item
\end{enumerate}
\subsection{Théorie 2}
\subsection*{Big Idea}
\subsection*{Outline}
\begin{enumerate}
    \item
    \item
    \item
\end{enumerate}
\subsection{TP}

\pagebreak
\section{Semaine 13}
\subsection{Théorie 1}
\subsection*{Big Idea}
\subsection*{Outline}
\begin{enumerate}
    \item
    \item
    \item
\end{enumerate}
\subsection{Théorie 2}
\subsection*{Big Idea}
\subsection*{Outline}
\begin{enumerate}
    \item
    \item
    \item
\end{enumerate}
\subsection{TP}

\pagebreak
\section{Semaine 14}
\subsection{Théorie 1}
\subsection*{Big Idea}
\subsection*{Outline}
\begin{enumerate}
    \item
    \item
    \item
\end{enumerate}
\subsection{Théorie 2}
\subsection*{Big Idea}
\subsection*{Outline}
\begin{enumerate}
    \item
    \item
    \item
\end{enumerate}
\subsection{TP}

\pagebreak
\section{Semaine 15}
\subsection{Théorie 1}
\subsection*{Big Idea}
\subsection*{Outline}
\begin{enumerate}
    \item
    \item
    \item
\end{enumerate}
\subsection{Théorie 2}
\subsection*{Big Idea}
\subsection*{Outline}
\begin{enumerate}
    \item
    \item
    \item
\end{enumerate}
\subsection{TP}

\pagebreak
\section{Semaine 16}
\subsection{Théorie 1}
\subsection*{Big Idea}
\subsection*{Outline}
\begin{enumerate}
    \item
    \item
    \item
\end{enumerate}
\subsection{Théorie 2}
\subsection*{Big Idea}
\subsection*{Outline}
\begin{enumerate}
    \item
    \item
    \item
\end{enumerate}
\subsection{TP}

\pagebreak
\end{document}
\end{article}
