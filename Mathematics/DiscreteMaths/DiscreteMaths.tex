\documentclass{article}
\begin{document}
\title{Lecture Notes for Discrete Maths}
\author{Emulie Chhor}
\maketitle
\section{Introduction}

The Study of Discrete Maths can be divided in a few sections:\\

    \begin{enumerate}
	\item Logic and Quantifiers
	\item Set Theory
	\item Proofs
	\item Algorithms
	\item Number Theory and Cryptography
	\item Induction and Recursion
	\item Counting and Combinatorics
	\item Discrete Probability
	\item Functions and Relations
	\item Boolean Algebra
	\item Graph Theory and Trees
    \end{enumerate}

Le cours de Maths discrète est spécial, car les chapitres ne sont pas
vraiment successifs. Ils forment plutôt une collection des concepts de
base importants pour les branches mathématiques suivantes:
\begin{enumerate}
    \item Logique
    \item Set Theory
    \item Number Theory
    \item Combinatorics
    \item Algorithms
    \item Graph Theory
\end{enumerate}

\section{Logic and Quantifiers}

\subsection{Why do we care about logic?}

Learning about logic is useful for several reasons. First of all, studying logic
is the backbone of all mathematical proofs done in advance maths. Furthermore,
logic can also be applied in other areas. For instance, in computer science, we
can use propositional logic to simplify digital circuits. In philosophy, we use
logic to make sound argument.

\subsection{Meta for Logic}

The chapter of logic is structured as follows:

    \begin{enumerate}
	\item Define some definition/properties
	\item Translating Statement to Mathematical Statement using these symbols
	\item More Problems using these logical operators
    \end{enumerate}

\subsection{Propositional Logic}

The goal of this section is to determine wether a statement is true or false
using truth table or propositional logic.\\

We also want to translate sentence into a propositional statement or with its
quantifiers.

\subsubsection{What is a Statement?}

A statement is a proposition that can either be true or false, but not both.
It links propositional variables with logical operators. A statement can be
atomic or compound depending on the number of ideas it connects.

\subsubsection{Logical Operators}

As stated earlier, there are a few logical operators that can be used to link
propositional variables together. Each of these operators are associated with
a truth table that tells us the truth value of a proposition based on the value
of its propositional variables.\\

Here is a list of the logical Operators:

\begin{enumerate}
    \item Conjonction
    \item Disjunction (inclusive or)
    \item Exclusive Or
    \item Negation
    \item Implication
    \item Biconditional
\end{enumerate}

TODO: Add Truth Table for logical Operators

\subsubsection{Propositional Equivalence}

When we search for propositional equivalence, we want to determine wether or not
two or more statement have the same propositional value, regardless of the value
of the propositional variables. To determiner propositional equivalence, we use
truth table and determine if there is a tautology ie if both truth table are the
same.

\subsubsection{Translating Sentences into Propositional Statement}

\subsection{Quantifiers}

Another important concept in logic is quantifiers. Quantifiers tells us wether
a proposition holds true for all elements or some elements.\\

The most used quantifiers are the following:

    \begin{enumerate}
	\item Universal Quantifier
	\item Existential Quantifier
	\item Uniqueness Quantifier
    \end{enumerate}

\subsubsection{Negating a quantifier}

\subsubsection{Translating Statement in Statement with Quantifiers}

\subsection{Inference}

The third concept in logic is the notion of inference. Using some premisses,
we want to determine wether a conclusion holds true or not using inference rules.

There are a few inference rules:

    \begin{enumerate}
	\item Modus Ponens
	\item Modus Tollens
	\item Hypothetical Syllogism
	\item Disjunctive Syllogism
	\item Addition
	\item Simplification
	\item Conjunction
	\item Resolution
    \end{enumerate}


\subsection{Proofs Techniques}

When we want to prove a statement, there is a few proofs techniques we can use.
Depending on the structure of the theorems, we may want to use one method over
another.

Here is a list of the proofs techniques:

    \begin{enumerate}
	\item Direct Proofs
	\item Contraposition
	\item Contradiction
	\item Induction
    \end{enumerate}

\subsubsection{More Tips}

Althought most of the proofs follows the structure mentionned above, we can use
additional tricks to make our proofs a little easier

    \begin{enumerate}
	\item Exhaustive Proofs and Proofs by Cases
	\item Without Loss Of Generality
	\item Counter-Example
	\item Existence Proofs
    \end{enumerate}


\section{Set Theory}

\section{Number Theory and Cryptography}

\section{Induction and Recursion}

\section{Counting and Combinatorics}

\subsection{Overview}%
\label{sub:Overview}

Les preuves des formules combinatoires se font par argument de comptage,
ce qui diffère du type de preuve qu'on faiit habituellement. Souvent,
la preuve et l'intuition de la formule ne font qu'un.

\begin{enumerate}
    \item Binome de Newton
    \item
    \item Inégalité de Pascal
    \item Nombre de sous-ensembles
    \item Identité de Vandermonde
\end{enumerate}

\subsection{Theorems in Combinatorics}%
\label{sub:Theorems in Combinatorics}

\subsubsection{Binome de Newton}%
\label{ssub:Binome de Newton}

\subsubsection{Égalité des combinaisons}%
\label{ssub:Égalité des combinaisons}

\begin{theorem}
    $ {n \choose k} = {n \choose n-k} $
\end{theorem}

\begin{intuition}
    k parmi n est similaire à dénombrer non k parmi n
\end{intuition}

\subsubsection{Inégalité de Pascal}%
\label{ssub:Inégalité de Pascal}

\begin{theorem}[Pascal's Inequality]
    $ {n+1 \choose k} = {n \choose k} + {n \choose k-1} $
\end{theorem}

\begin{intuition}
    On considère un ensemble de n+1 éléments duquel on lui enlève un
    élément. On doit donc enlever toutes les combinaison qu'on peut former
    avec l'élément enlevé.
\end{intuition}

\subsubsection{Nombre de sous-ensembles}%
\label{ssub:Nombre de sous-ensembles}

\begin{theorem}[Number of subsets]
    Le nombre de sous-ensembles qu'on peut former avec un ensemble de
    taille n est $$ \sum^{n}_{i=0} {n \choose k} = 2^n $$
\end{theorem}

\begin{intuition}
    Pour compter le nombre de sous-ensemble de taille n, on doit considérer
    la combinaison. Si on veut déterminer l'ensemble des sous-ensembles, on
    n'a qu'à additionner le nombre de sous-ensembles pour chacune des
    tailles. Finalement, on peut utiliser le binôme de Newton pour montrer
    que $ \sum^{n}_{i=0} {n \choose k} = 2^n $
\end{intuition}

\subsubsection{Vandermonde's Identity}%
\label{ssub:Vandermonde's Identity}

\begin{theorem}[Vandermonde's Identity]
    $$ \sum^{n}_{i=0} {m \choose k}{n \choose r-k}= {m+n \choose r} $$
\end{theorem}

\begin{intuition}
    TODO
\end{intuition}

\begin{corollary}
    $ \sum^{n}_{i=0} {n \choose k} ^2 = {2n \choose n} $
\end{corollary}

\subsubsection{Triangle de Pascal}%
\label{ssub:Triangle de Pascal}


\section{Functions and Relations}%
\label{sec:Functions and Relations}

\subsection{Functions}%
\label{sub:Functions}

\begin{defintion}[Functions]
\end{defintion}

\section{Graph Theory and Trees}

\section{Ressources}%
\label{sec:Ressources}

\begin{enumerate}
    \item Maths Sorcerer - Set Theory, Functions
    \item Kimberly Brehm
\end{enumerate}

\end{document}
\end{article}
