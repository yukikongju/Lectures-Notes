\documentclass{article}
\usepackage{amsmath}
\usepackage{amsfonts}
\usepackage{amsthm}
\usepackage{hyperref}
\usepackage{parskip}
\usepackage{textgreek}
\begin{document}
\title{Lecture Notes for Epistodic History of Mathematics - }
\author{Emulie Chhor}
\maketitle

\section*{Introduction}

\begin{enumerate}
    \item The Ancient Greeks
    \item Zeno's Paradox and the Concept of Limit
    \item The Mystical Mathematics of Hypatia
    \item The Arabcs and the Development of Algebra
    \item Cardano, Abel, Galois, and the Solving of Equations
    \item René Descartes and the Idea of Coordinates
    \item The Invention of Differential Calculus
    \item Complex Numbers and Polynomial
    \item Sophie Germain and Fermat's Last Problem
    \item Cauchy and the Foundations of Analysis
    \item The Prime Numbers
    \item Dirichlet and How to Count
    \item Riemann and the Geometry of Surfaces
    \item Georg Cantor and the Orders of Infinity
    \item The Number Systems
    \item Henri Poincaré, Child Prodigy
    \item Sonya Kovaleskaya and machanics
    \item Emmy Noether and Algebra
    \item Methods of Proofs
    \item Alan Turing and Cryptography
\end{enumerate}


\newtheorem{definition}{Definition}[subsection]
\newtheorem{theorem}{Theorem}[subsection]
\newtheorem{corollary}{Corollary}[subsection]
\newtheorem{lemma}[theorem]{Lemma}
\newtheorem{proposition}{Proposition}[section]
\newtheorem{axiom}{Axiome}
\newtheorem{property}{Propriété}[subsection]
\newtheorem*{remark}{Remarque}
\newtheorem*{problem}{Problème}
\newtheorem*{intuition}{Intuition}


\section{The Ancient Greeks}
\subsection{Overview}%
\label{sub:Overview}

The chapters focusses on the three most importants mathematicians from the
greek era and their main works
\begin{enumerate}
    \item Pythagoras: Pythagoras Theorem and the existence of incommensurable
	numbers, Pythagoras triples
    \item Euclid: the 5 axioms of geometry, Euclidean algorithm for long
	division, Euclidean Geometry
    \item Archimedes: Polygones areas, Area of a circle, Eureka/masse volumique/
	Poussée d'Archimedes
\end{enumerate}

\section{Zeno's Paradox and the Concept of Limit}
\subsection{Overview}%
\label{sub:Overview}
\section{The Mystical Mathematics of Hypatia}
\subsection{Overview}%
\label{sub:Overview}
\section{The Arabics and the Development of Algebra}
\subsection{Overview}%
\label{sub:Overview}
\section{Cardano, Abel, Galois, and the Solving of Equations}
\subsection{Overview}%
\label{sub:Overview}
\section{René Descartes and the Idea of Coordinates}
\subsection{Overview}%
\label{sub:Overview}
\section{The Invention of Differential Calculus}
\subsection{Overview}%
\label{sub:Overview}
\section{Complex Numbers and Polynomial}
\subsection{Overview}%
\label{sub:Overview}
\section{Sophie Germain and Fermat's Last Problem}
\subsection{Overview}%
\label{sub:Overview}
\section{Cauchy and the Foundations of Analysis}
\subsection{Overview}%
\label{sub:Overview}
\section{The Prime Numbers}
\subsection{Overview}%
\label{sub:Overview}
\section{Dirichlet and How to Count}
\subsection{Overview}%
\label{sub:Overview}
\section{Riemann and the Geometry of Surfaces}
\subsection{Overview}%
\label{sub:Overview}
\section{Georg Cantor and the Orders of Infinity}
\subsection{Overview}%
\label{sub:Overview}
\section{The Number Systems}
\subsection{Overview}%
\label{sub:Overview}
\section{Henri Poincaré, Child Prodigy}
\subsection{Overview}%
\label{sub:Overview}
\section{Sonya Kovaleskaya and machanics}
\subsection{Overview}%
\label{sub:Overview}
\section{Emmy Noether and Algebra}
\subsection{Overview}%
\label{sub:Overview}
\section{Methods of Proofs}
\subsection{Overview}%
\label{sub:Overview}
\section{Alan Turing and Cryptography}
\subsection{Overview}%
\label{sub:Overview}

\section{Ressources}%
\label{sec:Ressources}

\subsection{Books}%
\label{sub:Books}

- An Epistodic History of Mathematics by Steven Krangtz


\end{document}
\end{article}

