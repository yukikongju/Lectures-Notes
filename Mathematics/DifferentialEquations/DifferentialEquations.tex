\documentclass{article}
\usepackage{amsmath}
\usepackage{amsfonts}
\usepackage{amsthm}
\usepackage{hyperref}
\usepackage{parskip}
\usepackage{textgreek}
\begin{document}
\title{Lecture Notes for Differential Equations}
\author{Emulie Chhor}
\maketitle

\section*{Introduction}

\begin{enumerate}
    \item Introduction
    \item First Order Differential Equations
    \item Second Order Differential Equations
\end{enumerate}

\newtheorem{definition}{Definition}[subsection]
\newtheorem{theorem}{Theorem}[subsection]
\newtheorem{corollary}{Corollary}[subsection]
\newtheorem{lemma}[theorem]{Lemma}
\newtheorem{proposition}{Proposition}[section]
\newtheorem{axiom}{Axiome}
\newtheorem{property}{Propriété}[subsection]
\newtheorem*{remark}{Remarque}
\newtheorem*{problem}{Problème}
\newtheorem*{intuition}{Intuition}

\section{Introduction}

\subsection{Overview}%
\label{sub:Overview}

\begin{enumerate}
    \item Why study differential Equations
    \item Slope Field and Isoclines
\end{enumerate}


\subsection{Why study differential equations}%
\label{sub:Why study differential equations}

On étudie les équations différentielles parce qu'elles nous permet de
résoudre des problèmes où il est plus facile de décrire la situation
en terme de changement plutôt qu'en terme absolue. Plus généralement,
résoudre une équation différentielle, c'est de trouver une fonction qui
satisfait la relation donnée par cette fonction et ses dérivées.

Il est important de se souvenir qu'on veut modéliser des phénomènes
réels. Ainsi, pour trouver la "formule" à résoudre, on s'inspire de
la physique,chimie,économie pour avoir une assez bonne interprétation
du changement.

\subsection{Différents Types d'équations différentielles}%
\label{sub:Différents Types d'équations différentielles}

Il existe plusieurs façons de caractériser les équations différentielles:
\begin{enumerate}
    \item Ordinary Differential Equations vs Partial Differential Equations
    \item Linear vs Nonlinear Differential Equations
    \item Homogeneous vs inhomogeneous Differential Equations
    \item Standard form vs implicit form
    \item General vs particular solution
\end{enumerate}

Il est important de savoir faire la différence entre ces types d'équations,
car les statrégies utilisées pour les résoudre vont différer. Un peu comme
avec les différentes stratégies qu'il existe pour résoudre des intégrales, ici,
on veut déterminer la bonne forme.

\subsection{Slope Field and Isoclines}%
\label{sub:}

TODO

\section{First Order Differential Equations}

\subsection{Overview}%
\label{sub:Overview}

\begin{enumerate}
    \item Existence and Uniqueness Theorem
    \item Solving Linear First Order Equations
\end{enumerate}

\subsection{Existence and Uniquenes Theorem}%
\label{sub:Existence and Uniquenes Theorem}

\subsubsection{Overview}%
\label{sub:Overview}

Avant de se lancer dans des calculs d'intrégrale pour résoudre l'équation
différentielle, il est important de déterminer si une solution existe,
et si oui, si il en existe plusieurs.

\subsection{Solving Linear First Order Equations}%
\label{sub:Solving Linear First Order Equations}

\subsubsection{Overview}%
\label{ssub:Overview}

On veut identifier la forme de l'équation différentielle et montrer
comment résoudre cette équation.

\begin{enumerate}
    \item Separating Variables
    \item Integrating Factors
    \item Logisitc Growth
    \item Exact Differential Equation
    \item Bernouilli Differential Equations
\end{enumerate}




\section{Second Order Differential Equations}

\section{Ressources}%
\label{sec:Ressources}

\subsection{Books}%
\label{sub:Books}

\subsection{Courses}%
\label{sub:Courses}

\subsection{Exercices}%
\label{sub:Exercices}

\end{document}
\end{article}

