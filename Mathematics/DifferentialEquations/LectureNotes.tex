\documentclass{article}
\usepackage{amsmath}
\usepackage{amsfonts}
\usepackage{amsthm}
\usepackage{parskip}
\usepackage{textgreek}
\begin{document}
\title{Lecture Notes - Differential Equations by Professor Macauley}
\author{Emulie Chhor}
\maketitle

\section*{Introduction}

This document is a summary of concepts I have learned from Professor
Macauley's Differential Equations Course.

Essentiellement, le cours d'équation différentielle se distingue
en 2 cours:
\begin{enumerate}
    \item Ordinary Differential Equations
    \item Partial Differential Equations
\end{enumerate}

\pagebreak

\part{Ordinary Differential Equations} % (fold)%
\label{prt:Differential Equations}

\section{Overview}%
\label{sec:Overview}

% part Differential Equations (end)

The first course is separated into the following chapter:
\begin{enumerate}
    \item Introduction to ODE
    \item First Order Differential Equations
    \item Second Order Differential Equations
    \item Systems of Differential Equations
    \item Laplace Transforms
    \item Fourier Series and Boundary Value Problems
    \item Partial Differential Equations
\end{enumerate}

On distingue les équations linéaires par:
\begin{enumerate}
    \item Type: ODE vs PDE
    \item Order: Normal Form vs
    \item Linearity:
\end{enumerate}

\newtheorem{definition}{Definition}[subsection]
\newtheorem{theorem}{Theorem}[subsection]
\newtheorem{corollary}{Corollary}[subsection]
\newtheorem{lemma}[theorem]{Lemma}
\newtheorem{proposition}{Proposition}[section]
\newtheorem{axiom}{Axiome}
\newtheorem{property}{Propriété}[subsection]
\newtheorem*{remark}{Remarque}
\newtheorem*{problem}{Problème}
\newtheorem*{intuition}{Intuition}

\pagebreak

\section{Introduction to ODE}
\subsection*{Overview}

Le premier chapitre introduit la notion d'Ordinary Differential Equations,
qui sont des equations differentielles à une seule variable. On
n'apprend pas encore comment les résoudre, mais on désire les dessiner
puisque ça nous donne une bonne idée de l'allure de la famille de solutions.

Notons qu'il existe d'autres tyes d'équations différentielles (ex:
PDEs, qui sont des équa diff avec 2 variables), mais on se penche sur les
ODEs en premier.

On verra 4 méthodes pour résoudre les ODEs:
\begin{enumerate}
    \item Separating Variables: isolate dy/y and integrate both sides
    \item Integrating Factor: multiplier par une constante d'intégration
	pour pouvoir intégrer
    \item Variables Parameters
    \item indetermined coefficient: $ y(t) = y_h(t) + y_p(t) $

\end{enumerate}

\subsection{What is a differential equation? }

\begin{definition}[Linear Differential Equation]
    On dit qu'une transformation est linéaire si les coefficients de
    l'équation différentielle sont de degré 0 et que y et ses dérivées
    sont exposant 1
    \begin{enumerate}
	\item Forme Différentielle: $ M(x,y) dx + N(x,y) dy = 0$
	\item Forme Normale: $ \frac{dy}{dx} = \frac{N(x,y)}{M(x,y)} $
	    derivative isolated
    \end{enumerate}
    Une équation est linéaire si ses coefficients sont exprimés qu'en
    fonction de sa variable indépendantes
\end{definition}

\begin{definition}[Order of Derivative]
    L'ordre d'une équation différentielle est la plus grande dérivée.
    $$ g(x) = a_n (x) \frac{d^n y}{dx^n} + a_{n+1} \frac{d^{n-1}}
    {d x^{n-1}}  + ... + a_1(x) \frac{dy}{dx}  + a_0 (x) y $$
\end{definition}

\begin{definition}[Solution générale et particulière]
    \begin{enumerate}
        \item Particular Solutions: no arbitrary parameters $ y = 3x$
	\item One-Parameter family of solutions $ y=3x+c$
	\item Trivial Solution: y=0
    \end{enumerate}
    Une solution explicite est une équation qui n'isole pas y. Une solution
    implicite est une équation avec y qui n'est pas isolée.
    Plus généralement, si on a une constante, on parle de solutions
    générale, sinon, on parle de solution particulière.
\end{definition}

Une équation différentielle est une équation définie par ses fonctions
et ses dérivées. On l'utilise pour modéliser:
\begin{enumerate}
    \item Growth : $ P'(t) = r P(t)$
    \item Decay : $ P'(t) = r (1- \frac{P(t)}{M})$
    \item Logistic Equation: $ P'(t) = r (1- \frac{P(t)}{M})P(t)$
\end{enumerate}

Éventuellement, le graphe devrait converger vers une valeur quelconque.\\

Il est à noter qu'on cherche la famille de fonctions qui vérifie l'équation.
 On pourra choisir la solution parmi cette famille de solutions plus tard.

 \begin{problem}
     \begin{enumerate}
         \item Écrire l'équation sous sa forme normale ou différentielle
	 \item Classifier l'équation différentielle par son ordre et
	     sa linéarité
	    \item Verify if function is a solution to differential
		equation
     \end{enumerate}
 \end{problem}

\subsection{Plotting solutions to differential equations. }

Présentement, on n'est pas encore capable de résoudre une équation
différentielle, mais on peut utiliser les outils du calcul pour tracer
le graphe pour avoir une idée de l'allure de la famille de solutions.

On distingue 2 types de solutions:
\begin{enumerate}
    \item Isocline: y' est une constante: $ y' = 0$, with c
    \item Autonomous ODEs: y' est une fonction quelconque: $ y' = f(y)$
\end{enumerate}

\subsubsection{Slope Field}%
\label{ssub:Slope Field}

On distingue 3 types de slope field
\begin{enumerate}
    \item Attractor: toutes les courbes convergent vers une valeur. On
	dit qu'il y a un équilibre et que les courbes sont stables
    \item Repeller: Les courbes divergent et on dit qu'elles sont
	instables
    \item Semi-Stable: Les courbes sont de type "attractor" d'un bord
	et repeller de l'autre
\end{enumerate}

\subsubsection{Comment tracer les solutions}

TODO

\subsection{Approximating solutions to differential equations. }

Présentement, on n'a pas enore les outils pour résoudre une équa diff.
Par contre, on peut toujours se servir des concepts vu en calcul pour
approximer les solutions à une équations différentielles. Il existe
plusieurs méthodes:
\begin{enumerate}
    \item Euler's method: Approximating using stepwise Linear Approximations
    \item Runge-Kutta's Method: skipped
\end{enumerate}

\subsubsection{Euler's Method}

La méthode d'Euler conciste à tracer la fonction à l'aide d'approximation
linéaire. On doit d'abord choisir un "step", qui représente la distance
sur lequel on trace notre droite. Plus le step est petit, plus notre graphe
aura l'air d'une courbe.

Plus généralement, si on a $ y' = f(t,y)$ et $ y(t_0)$ avec un stepsize
de h, on a:
$ y_{n+1} = y_n + h \cdot f(x_n, y_n)$
% $$ (t_{k+1}, y_{k+1}) = (t_k + h, y_k + f(t_k, y_k) \cdot h) $$
Il s'agit donc d'une méthode récursive pour tracer des graphes:
on détermine la pente entre deux steps en considérant le triangle
rectangle qui forme le point précédent et le point suivant.

\subsubsection{Improved Euler's Method}%
\label{ssub:Improved Euler's Method}

La méthode de Euler a tendance à sous-estimer la valeur de la pente
puisqu'on considère que la pente est la même à un certain point donné.
Ce qu'on veut faire, c'est de calculer une nouvelle pente
$$ m = f(x_0 + h, y_0 + h \cdot f(x_0, y_0)) $$
Cependant, cette méthode a tendance à surestimer la pente. Ainsi, on veut
trouver un pente moyenne en considérant la pente calculée à partir de
l'ancienne méthode et de la nouvelle méthode.
$ y_{n+1} = y_n + h \cdot [f(x_n, y_n) + f(x_n+h, y_n + h \cdot
f(x_n, y_n))]$

\pagebreak
\section{First Order Differential Equations}
\subsection*{Overview}

\subsection{Separation of variables}

Cette setion met l'emphase sur la séparation de variables. En gros, on
veut réécrire notre équation différentielle sous la notation de Leibniz
afin d'isoler dy/y et dx/x et d'avoir une équation p/r au temps. On voit
qu'en intégrant, on obtient les équations suivantes:
\begin{enumerate}
    \item Exponential Growth: $ y' = ky \Longleftrightarrow y(t) = C e^{kt} $
    \item Exponential Decay: $ y' = -ky \Longleftrightarrow y(t) = C e^{-kt} $
    \item Decay to Initial Value: $ y' = -k (y-M) \Longleftrightarrow
	y(t) = C e^{-kt} + M $
\end{enumerate}

Il faut aussi se souvenir que le k, représentant le rate of change, nous
dit à quel point le decay/growth se fait rapidement.

\begin{remark}[Stratégies de résolution]
    Pour résoudre des intégrales, il est parfois utile d'utiliser
    \begin{enumerate}
        \item intégration par fraction partielle
	\item Rational zero theorem: trouver les facteurs de p et q, et
	    résoudre par division euclidienne une foit qu'on a trouvé le
	    premier zéro
    \end{enumerate}
\end{remark}

\subsubsection{Models with Separable Method}%
\label{ssub:Models with Separable Method}

\textbf{Newton's Law of Cooling}

Newton's Law of cooling is used when we want to model an object whose
cooling point is at some degree. The temperature of the object start at
$T_0$ and converge toward a value T. The rate of change is given by
$$ \frac{dT}{dt}  = k(T-T_m)$$
where k is a constant and $(T - T_m)$ is the difference between the
object's temperature and the room in which it resides

Using the separable variables method, we get $$ T = c e^{kt} + T_m
= (T_0 - T_m) e^{kt} + T_m$$

\textbf{Exponential Growth}

The exponential growth is used when the rate of change grows without
restrictions. The rate of change for an exponential growth is given
by $$ \frac{dy}{dt} = ky $$

Using the separable variables method, we get
$$ A = A_0 e^kt $$

\begin{remark}
    \begin{enumerate}
        \item If k<0, we have a decay
	\item If k>0, we have a growth
    \end{enumerate}
\end{remark}

\textbf{Logistic Growth}

The logistic growth correspond to the exponential growth, but we have
a restrictions that forbids us to grow/decay after a given point.
The rate of change is given by $$ \frac{dP}{dt}  = r (\frac{k-P}{k}) P $$
where r is a rate of growth, P is the size of the current population, and
$(\frac{k-P}{k} )$ is the carrying capacity

When solving this equation using separable variables, we get
$$ P = \frac{k}{1+ce^{-rt}} $$

\subsection{Initial value problems}

Cette section focussait sur la résolution de ODEs. Tout d'abord, on
introduisait la notion de valeur initiale. Auparavant, on voulait
trouver la solution générale du ODEs. Ici, on nous donne des valeurs
intiales $ y(t_0) = y_0 $. On veut trouver une solution particulière.

\begin{intuition}
    Graphiquement, c'est comme si on calculait la famille de solutions
    générale, et on choisissait la courbe qui passait par le point initial
\end{intuition}

Pour ce faire, on doit trouver:
\begin{enumerate}
    \item C: initial rate of change that can be found by solving
	initial value (t=0)
    \item k: constant rate of change that can be found by using
	initial and end time (t=0 and t=5 for exemple)
    \item t: time an which we measure value
    \item P(t): value of function given using for all parameters
\end{enumerate}

Dépendemment du problème, on doit solve for une des 4 variables
\begin{enumerate}
    \item Find Initial Rate of Change : solve for C
    \item Find how much x is worth at time t: use C, k, t to find P(t)
    \item Find half live: use P(t), P(0), C, k to find t, the time when
	P has value of P(t)
\end{enumerate}

\begin{remark}
    On utilise souvent les logs pour abaisser l'exposant. On préfère
    travailler avec des fractions positives
\end{remark}

\begin{theorem}[Existence and Uniqueness Theorem for first order IVP]
    Let $ \frac{dy}{dx}  = f(x,y), y(x_0)=y_0$. If f(x,y) and
    $ \frac{\partial f}{\partial y} $ are both continuous in a
    neighborhood of $(x_0, y_0)$, then there is a unique solution
    definied on an interval containing $x_0$
\end{theorem}

\begin{intuition}
    Before integrating to find a solution to a differential equation,
    we want to determine wether:
    \begin{enumerate}
        \item Is there a solution?
	\item Are there more solutions?
    \end{enumerate}
\end{intuition}


\begin{problem}
    \begin{enumerate}
        \item Determine if the IVP has a unique solution
    \end{enumerate}
\end{problem}

\subsection{Falling objects with air resistance: Newton's 2nd Law}

Dans cette section, on voit que la 2e loi de Newton F=ma qui considère
la résistance de l'air, peut être modéliser avec une ODEs: exponential
decay to value. ON voit aussi pourquoi on voudrait intégrer une ODEs

\textbf{Comment trouver modéliser v(t) avec une ODE}

La deuxième loi de Newton nous dit que
\begin{enumerate}
    \item Sans résistance de l'air: $ F=ma=-mg $
    \item Avec résistance de l'air: $ F=-mg - R(v) $, R(v): résistance
	de l'air
\end{enumerate}

De plus, on peut considérer que la résistance à l'air est proportionnelle
à la vitesse (dans le sens inverse): $$ R(v) = -rv $$

Aussi, il faut se souvenir que
\begin{enumerate}
    \item Vitesse: $ v(t) = d'(t) $, d: distance
    \item Accélération: $ a(t) = v'(t) $, v: vitesse
\end{enumerate}

Ainsi, on a que la 2nd loi de Newton qui considère la résistance de l'air
est $$ v' = -g - \frac{r}{m} v $$, qui peut être remodelée pour obtenir
un exponential decay to a value
$$ v' = -g - \frac{r}{m} = \frac{r}{m} (\frac{-mg}{r} - v),
k = \frac{r}{m}, A = \frac{-mg}{r}$$, avec v: vitesse limitante,
A: terminal velocity

Ainsi, on a que $$ V(t) = -\frac{mg}{r} + C e^{\frac{-r}{m} t} $$

\textbf{Problèmes}

Dépendemment du problème, on nous demande de trouver:
\begin{enumerate}
    \item terminal velocity: $ A = \frac{-mg}{r} $
    \item Limiting velocity: $ v' = 0 $
    \item C: initial value
    \item m: masse
    \item g: constante de gravité
    \item v(t): velocity after t time
    \item rate of change r: use terminal velocity and solve for r
    \item distance d(t): on n'a qu'à intégrer la fonction de vitesse
	$$ \int_{{a}}^{{b}} {v(t)} \: d{t} $$
\end{enumerate}

Bref
\begin{enumerate}
    \item Comment on a trouver la ODE pour modéliser l'accélération
	en considérant le frottement de l'air
    \item Résoudre des problèmes en utilisant la modélisation de F=ma
	en ODE
\end{enumerate}

\subsection{Solving 1st order inhomogeneous ODEs.}

Depuis le début du cours, on a vu comment résoudre des équations homogènes
à l'aide de la méthode par séparation. Cependant, cette stratégie ne peut pas
être utilisée pour résoudre des équations inhomogènes

Rappel: la différence entre une équation homogène et inhomogène est que f(t) = 0
\begin{enumerate}
    \item Homogenous Equation: $ y' + a(t)y(t) = 0 $
    \item Inhomogenous Equation: $ y' + a(t)y(t) = f(t)$
\end{enumerate}

On discerne deux méthodes pour résoudre des ODEs inhomogènes:
\begin{enumerate}
    \item Integrating Factor
    \item Variation of Parameters
\end{enumerate}

Notons que les 2 méthodes sont équivalentes, mais il est préférable d'utiliser
la deuxième, puisqu'elle possède un "built-in correction" nous permettant de
voir si on a fait des erreur

\subsubsection{Integrating Factor}

La première méthode consiste à multiplier par un facteur d'intégration afin de
pouvoir intégrer. Pour choisir ce facteur d'intégration, on doit regarder a(t),
le coefficient qui multiplie y(t). Cette méthode nous permet de simuler
une équation différentielle homogène et utiliser la séparation de variable.

Plus généralement, les étapes sont les suivantes:
\begin{enumerate}
    \item Identifier a(t) afin de trouver le facteur d'intégration
    \item Calculer le facteur d'intégration: $ e^( \int_{{}}^{{}} {A(t)} \:
	d{t} {}) $
    \item Multiplier l'équation inhomogène par le facteur d'intégration des
	deux côtés: on obtient "l'inverse du produit de la dérivée"
    \item Écrire le produit de la dérivée comme une dérivée
    \item Intégrer des deux bords, et isoler y pour trouver la solution
	générale
\end{enumerate}

Notons pour que cette stratégie fonctionne, on doit savoir intégrer la partie
de droite

\begin{problem}
    \begin{enumerate}
	\item Mixing Problems: $ \frac{dA}{dt} = (rate in) - (rate out)$
	\item RL Ciruit:
    \end{enumerate}
\end{problem}

\subsubsection{Variation of Parameters}

La deuxième méthode consiste à trouver l'équation homogène en ignorant la
constante f(t), puis à plugger le guess dans l'équation inhomogène

Les étapes sont les suivantes:
\begin{enumerate}
    \item Trouver la solution à l'équation homogène en ignorant f(t)
    \item Guesser la solution générale pour trouver
	y et y': $ y(t) = v(t) y_h (t) = v e^t $
    \item Solve for v by isolating v' and integrating both sides
    \item Plugger y et y' dans l'équation inhomogène: il devrait y avoir des
	termes qui s'annulent
    \item Substitutionner v dans l'équation générale
	$ y(t) = v(t) y_h (t) = v e^t $
\end{enumerate}

En d'autres mots, on devrait retrouver la constante f(t) dans le v

TO REVIEW

\subsection{Linear differential equations}

Cette section nous présente 2 résultas cachés des équations différentielles:
\begin{enumerate}
    \item Superposition: Les solutions des équations différentielles homogènes
	sont linéairement indépendants
    \item Inhomogenous ODEs: on peut trouver la solution générale d'une
	ODEs inhomogènes en additionnant son équation homogène et une
	équation particulière
\end{enumerate}

\subsubsection{Superposition}

La superposition nous dit que si une ODE homogène $y'+a(t)y(t) = 0$ a comme
solution $y_1(t)$ et $y_2(t)$, alors $ C_1 y_1(t) + C_2 y_2(t) $ est une
solution $ \forall c_1, c_2$

\textbf{Pourquoi c'est vrai?}

Si on plug $ C_1 y_1(t) + C_2 y_2(t) $ dans l'équation homogène initiale
$y'+a(t)y(t) = 0$, on obtient que $ C_1 \cdot 0 + C_2 \cdot 0$

\subsubsection{Quick Trick to solve Inhomogenous ODEs}

On peut trouver la solution générale d'une ODE inhomogène en additionnant
son équation homogène et une équation particulière:
$$ y(t) = y_h (t) + y_p (t) $$

Les étapes sont les suivantes:
\begin{enumerate}
    \item Trouver la solution Homogène $y_h$
    \item Trouver la solution particulière $y_p$ (choisir 0)
    \item Trouver la solution générale: $$ y(t) = y_h (t) + y_p (t) $$
\end{enumerate}

\textbf{Pourquoi ça marche}

Si y est la solution générale $y'+a(t)y(t) = f(t)$ et la solution particulière
$y_p'+a(t)y_p(t) = f(t)$, alors en les soustrayant, on obtient
$$ (y-y_p)' +a(t) (y-y_p) = 0$$, et $ (y - y_p) $ est une solution à
l'équation homogène

\subsection{Basic mixing problems.}

Dans un problème de mixing problem, on veut déterminer la concentration
de soluté dans un solvant à un temps donné. Le problème le plus simple
est le cas suivant: on ajoute de l'eau à la même concentration qu'il
n'en sort. On caractérise le volume à un temps donné par
$$ Concentration(t) = \frac{x(t)}{Vol(t)} $$

De plus, on sait que le rate of change à un temps t est donné par
$$ x'(t) = (rate in) - (rate out) $$

Ainsi, on doit calculer
\begin{enumerate}
    \item Rate In := (volume rate) x (concentration) := (volume qui sort
	dans un intervalle de temps) x (concentration du solvant)
    \item Rate Out := (volume rate) x (concentration) :=
	(volume qui sort) x (concentration du solvant dans le tank) :=
	(volume) x $(\frac{x(t)}{Vol(t)} )$
    \item Solve for x and for C using one of the 4 methods: on cherche
	x pour pouvoir trouver C
\end{enumerate}

\subsection{Advanced mixing problems.}
TODO
\subsection{The logistic equation.}

\pagebreak

\subsection{Substitution}%
\label{sub:Substitution}

\subsubsection{Overview}%
\label{ssub:Overview}

When we have a first order differential equations that is nonlinear and
cannot be solved by separating variables, we use substitution. There
are two methods:
\begin{enumerate}
    \item Homogeneous equations: if differential equation is homogeneous,
	substitute by y=vx to get turn it into separating variables
    \item Bernouilli Equations
\end{enumerate}

\textbf{Homogeneous Equations}

If a differential equation of first order is a homogeneous equations,
then we can use the following substitution to turn the ODE into a
separable equation

\begin{enumerate}
    \item Determine if the equations is homogeneous: f(tx,ty) = f(x,y)
    \item If it is homogeneous, substitute equation by y=vx
    \item Solve
\end{enumerate}

\textbf{Bernouilli Equations}

ODEs that are nonlinear and cannot be solved by separating variables are
Bernouilli if they take the following form
$$ \frac{dy}{dx} + f(x)y = g(x) y^n$$
If the ODE is Bernouilli, then we can substitute with $ v = y^{1-n}$,
which will make the equation linear. Finally, we solve using Integrating
Factor (or another method to solve linear first order ODEs)

\begin{remark}
    \begin{enumerate}
        \item If n=0: not an ODEs
	\item If n=1: Separable Variables
    \end{enumerate}
\end{remark}

\subsection{Exact Equations}

Les équations exactes sont des équations qui prennent la forme de
$$ M(x,y) dx + N(x,y) dy =0$$

Pour les résoudre, on doit:
\begin{enumerate}
    \item Écrire l'équation sous sa forme normale:
	$ M(x,y) dx + N(x,y) dy =0$
    \item Déterminer si l'équation différentielle est exacte:
	$ \frac{M(x,y)}{\partial y} = \frac{N(x,y)}{\partial x} $
    \item Si l'équationest exacte, on trouve la solution en intégrant
	$ \int M(x,y) dy, \int N(x,y) dx$
    \item Additionner les termes différents
\end{enumerate}

\section{Second Order Differential Equations}
\subsection*{Overview}

\subsection{Second order linear ODEs.}

\subsubsection{Overview}%
\label{ssub:Overview}

Il existe plusieurs stratégies pour résoudre des équations différentielles
de 2e ordre
\begin{enumerate}
    \item Reduction of Order: substitution pour réduire rendre le ODE 2nd
	à un 1rst ODE
    \item The method of undetermined coefficients.
\end{enumerate}

Il faut aussi se souvenir du concept de combinaison linéaire: on peut
générer une famille de solutions avec la combinaison linéaire des
solutions générales trouvées.

\subsubsection{Reduction of Order}%
\label{ssub:Reduction of Order}

On peut utiliser la méthode de reduction of order si on connait une
des solutions. Pour ce faire, on doit
\begin{enumerate}
    \item Trouver la substitution: $ y= u y_1$ et trouver la nouvelle
	équation en terme de cette substitution
    \item Réduire l'ordre en posant une autre substitution: $ v=u'$
    \item Résoudre pour v=u'
    \item Intégrer v pour obtenir u pour ensuite substituer dans
	l'équation originale $y= u y_1$
\end{enumerate}

\begin{remark}
    Si l'équation est
    \begin{enumerate}
        \item Homogène: devient first order et se résout avec separating
	    variables
	\item Nonhomogène: devient first order et se résout avec
	    integrating factor
    \end{enumerate}
\end{remark}

\subsection{Equations with constant coefficients.}
\subsubsection{Overview}%
\label{ssub:Overview}

Lorsqu'on a une ODE de deuxième ordre qui prend la forme
$ ay'' + by' + cy = 0$, on dit que'on a une équation linéaire avec
des coefficients constants. Intuitivement, pour résoudre cette ODE,
on veut que la fonction y soit une exponentielle, car ses dérivées
existent et peuvent être exprimées comme une combinaison linéaire des
précédantes. On a donc $ y=e^{mx}, y'=me^{mx}, y''=m^2 e^{mx}$. Ainsi,
on obtient
$$ ay'' + by' + cy = 0$$
$$ a(m^2 e^{mx}) + b(me^{mx})+c(e^{mx})=0$$
$$ e^{mx} (am^2+bm+c)=0$$

Puisque $e^{mx} \neq 0$ (sinon f n'existe pas), il faut que le polynome
caractéristique $(am^2+bm+c)=0$ et donc trouver ses racines.

On distingue 3 types de solutions pour le polynome caractéristique:
\begin{enumerate}
    \item Distinct Real Root
    \item Repeated Real Root
    \item Complex Roots
\end{enumerate}

\subsubsection{Distinct Real Root}

Si on obtient des racines réelles distinctes (avec formule quadratique
ou en factorisant), on n'a qu'à substituer $m_1$ et $m_2$ dans
$$ y= c_1 e^{m_1x}+c_2 e^{m_2x} $$

\subsubsection{Repeated Real Root}

Si la racine trouvée est unique, on dit que la racine est de multiplicité
2. Cependant, on ne peut pas utiliser la formule
$ y= c_1 e^{m_1x}+c_2 e^{m_2x} $, car les solutions générales
${e^{mx}, e^{mx}}$ ne sont pas linéairement indépendantes, et on ne
peut pas écrire notre combinaison linéaire.

À la place, on utilise la formule suivante:
$$ y= c_1 e^{mx}+c_2 xe^{mx} $$

Pour s'en convaincre, on n'a qu'à vérifier que $y=xe^{mx}$ est une
solution

\subsubsection{Complex Roots}

Lorsqu'on a une racine négative, on doit faire appel à des racines
complexes. Cependant, on ne veut pas trainer des nombres complexes.
il est donc pratique d'utiliser la formule suivante:
si $ m = \alpha \pm \beta i$, alors la solution générale est
$$ y = e^ {\alpha x} (c_1 cos(\beta x) + c_2 sin(\beta x)) $$

\subsection{The method of undetermined coefficients.}

\subsubsection{Overview}%
\label{ssub:Overview}

La méthode undertermined coefficients nous permet de résoudre des équations
différentielles non-homogène en additionnant la solution complémentaire et
la solution particulière. On veut résoudre $ ay''+by'+cy=g(x)$

\subsubsection{Étapes de résolution}%
\label{ssub:Étapes de résolution}

Notons que la solution générale est donnée par $ y= y_c + y_p$, où $y_c$ est
la solution complémentaire trouvée en résolvant l'équation homogène et
$y_p$ est l'équation particulière trouvée en résolvant TODO. Notons que
l'ensemble de solutions fondamental doit être linéairement indépendant.

\begin{enumerate}
    \item Trouver la solution complémentaire: $y_p = c_1 e^{m_1 x} + c_2 e^{m_2 x}$ (ou autre dépendemment des valeurs de m)
	\begin{itemize}
	    \item Résoudre le polynôme caractéristique: $ am^2+bm+c=0$
		(donné par la solution homogène: $ ay''+by'+cy=g(x)$)
		pour obtenir $m_1$ et $m_2$
	    \item Remplacer $m_1, m_2$ pour obtenir $y_c$
	\end{itemize}
    \item Trouver la solution particulière en s'assurant de poser $y_p$
	linéairement indépendant
	\begin{itemize}
	    \item Poser $y_p$ et trouver ses dérivées première et deuxième
	    \item Remplacer dans l'équation initiale $ ay''+by'+cy=g(x)$
	    \item Résoudre le système pour obtenir la valeur de A
	\end{itemize}
\end{enumerate}

\subsubsection{Types de problèmes}%
\label{ssub:Types de problèmes}

Comme pour la méthode à coefficients déterminé, il y a 3 types de solutions:
\begin{enumerate}
    \item Distinct Real Root
    \item Repeated Real Root
    \item Complex Roots
\end{enumerate}

\subsection{Simple harmonic motion.}
\subsection{Damped and driven harmonic motion.}
\subsection{Variation of parameters.}

\subsubsection{Overview}%
\label{ssub:Overview}

La méthode de variation de paramètres est utilisée pour résoudre des
équations non-homogène $ ay''+by'+cy=g(x)$. Essentiellement, c'est une
méthode qui combine à la fois la règle de Cramer et la méthode d'ordre
de réduction en utilisant les "Wronskian" pour évaluer les déterminants

\subsubsection{Étapes de Résolution}%
\label{ssub:Étapes de Résolution}

Encore une fois, on trouve la solution générale en additionnant la
solution complémentaire et la solution particulière: $y=y_c+y_p$

\begin{enumerate}
    \item Trouver la solution complémentaire en résolvant l'équation
	homogène: $ay''+by'+cy=0 \Longrightarrow am^2+bm+c=0$
	\begin{itemize}
	    \item Résoudre le polynôme caractéristique: $ay''+by'+cy=0 \Longrightarrow am^2+bm+c=0$
	    \item $y_c = c_1 (...) + c_2 (...)$ (dépendemment du type de
		solution homogène)
	\end{itemize}
    \item Trouver la solution particulière: $ y_p = u_1 y_1 + u_2 y_2$
	\begin{itemize}
	    \item Trouver $y_1$ et $y_2$ à partir de la solution
		complémentaire
	    \item Construire la Wronskian : $w, w_1, w_2$ à l'aide de la
		règle de Cramer
		$$ w = \begin{vmatrix}
		    y_1 & y_2\\
		    y_1' & y_2'\\
		\end{vmatrix}$$
		$$ w_1 = \begin{vmatrix}
		    0 & y_2\\
		    g(x) & y_2'\\
		\end{vmatrix}$$
		$$ w_2 = \begin{vmatrix}
		    y_1 & 0\\
		    y_1' & g(x)\\
		\end{vmatrix}$$
	    \item Calculer $u_1 ' et u_2 '$: $u_1'=\frac{w_1}{w}$,
		$u_2'=\frac{w_2}{w}$
	    \item Intégrer $u_1$ et $u_2$ pour obtenir $u_1$ et $u_2$
	\end{itemize}
    \item La solution générale: $y=y_c+y_p$
\end{enumerate}

\begin{remark}
    Si $a \neq 1$, on n'a qu'à diviser $\frac{g(x)}{a}$
\end{remark}

\subsection{Cauchy-Euler equations.}

\subsubsection{Overview}%
\label{ssub:Overview}

Les équations de Cauchy-Euler sont des équations différentielles qui
peuvent être homogène ou non-homogène. La méthode de résolution dépend
de la forme de l'équation différentielle:
\begin{enumerate}
    \item homogène: poser $y=m^x$
    \item inhomogène: utiliser la wronskian
\end{enumerate}

\subsubsection{Homogeneous Cauchy-Euler Equations}%
\label{ssub:Cauchy-Euler Equations}

On peut résoudre une équation Cauchy-Euler homogène en posant $y=x^m$,
$y'=m x^{m-1}$, $y''=m (m-1) x^{m-2}$. En remplacant dans l'équation
homogène $a^2xy''+bxy'+cy=0$, on obtient
$$ax^2 [m(m-1)x^{m-2}]+bx[mx^{m-1}]+cx^m=0$$
$$x^m (am^2+(b-a)m+c)=0$$
et on n'a qu'à résoudre le polynôme caractéristique $ (am^2+(b-a)m+c)$
pour obtenir la solution $y=c_1 x^{m_1}+c_2 x^{m_2}$

Types de solutions:
\begin{enumerate}
    \item Solutions réelles différentes: $y=c_1 x^{m_1}+c_2 x^{m_2}$
    \item Solution réelle unique: $y = c_1 x^m + c_2 x^m ln|x|$
    \item Solution complexe: $y= c_1 x^\alpha cos(\beta ln|x|)
	+ c_2 x^\alpha sin(\beta ln|x|)$
\end{enumerate}

\subsubsection{Inhomogeneous Cauchy-Euler Equations}%
\label{ssub:Cauchy-Euler Equations}

Pour résoudre des équations de Cauchy-Euler non-homogène
$a^2xy''+bxy'+cy=g(x)$, on utilise les Wronskian
$$ w = \begin{vmatrix}
    y_1 & y_2\\
    y_1' & y_2'\\
\end{vmatrix}$$
$$ w_1 = \begin{vmatrix}
    0 & y_2\\
    \frac{g(x)}{ax^2}  & y_2'\\
\end{vmatrix}$$
$$ w_2 = \begin{vmatrix}
    y_1 & 0\\
    y_1' & \frac{g(x)}{ax^2} \\
\end{vmatrix}$$

\subsection{Power series solutions.}
\subsection{The method of Frobenius.}

\pagebreak

\section{Systems of Differential Equations}
\subsection*{Overview}

\subsection{Basic matrix algebra.}
\subsection{Eigenvalues and eigenvectors.}
\subsection{Mixing with two tanks.}
\subsection{Solving a 2x2 system of ODEs.}
\subsection{Phase portraits with real eigenvalues.}
\subsection{Phase portraits with complex eigenvalues.}
\subsection{Phase portraits with repeated eigenvalues.}
\subsection{Stability of phase portraits.}
\subsection{of parameters for systems.}

\pagebreak
\section{Laplace Transforms}
\subsection*{Overview}

\subsection{What is a Laplace transform?}
\subsection{Properties and applications of Laplace transforms}
\subsection{Discontinuous forcing terms}
\subsection{Periodic forcing terms}
\subsection{Impulse functions}
\subsection{Convolution}

\pagebreak
\section{Fourier Series and Boundary Value Problems}

\subsection{Introduction to Fourier series}
\subsection{Computing Fourier series}
\subsection{Fourier sine and cosine series}
\subsection{Complex Fourier series}
\subsection{Applications of Fourier series}
\subsection{Boundary value problems}

\subsection*{Overview}

\pagebreak
\section{Partial Differential Equations}
\subsection*{Overview}

\subsection{The heat equation}
\subsection{Different boundary conditions}
\subsection{The transport equation}
\subsection{The wave equation}
\subsection{Harmonic functions}
\subsection{Laplace's equation}
\subsection{The 2D heat equation}
\subsection{2D wave equation}

\pagebreak

\section{Systems of Nonlinear Differential Equations}

\subsection*{Overview}

\subsection{Modeling with nonlinear systems}
\subsection{Linearization and steady-state analysis}
\subsection{Predator-prey models}

\part{Partial Differential Equations} % (fold)%
\label{prt:Partial Differential Equations}
% part Partial Differential Equations (end)

\section{Overview}%
\label{sec:Overview}

\begin{enumerate}
    \item Some Linear Algebra
    \item Linear Differential Equations
    \item Fourier Series
    \item Boundary Value Problems and Sturm-Liouville Theory
    \item Partial Differential Equations (PDE) on bounded domains
    \item Partial Differential Equations (PDE) on unbounded domains
    \item Higher-Dimensional PDEs
\end{enumerate}

\pagebreak

\section{Some linear algebra}
\subsection*{Overview}
\subsection{Vector spaces}
\subsection{Linear independence and spanning sets}
\subsection{Linear maps.}
\subsection{Inner products and orthogonality}
\pagebreak
\section{Linear differential equations}
\subsection*{Overview}
\subsection{The fundamental theorem of linear ODEs}
\subsection{Linear independence and the Wronskian}
\subsection{Inhomogeneous ODEs and affine spaces}
\subsection{Undetermined coefficients}
\subsection{Power series solutions to ODEs}
\subsection{Singular points and the Frobenius method}
\subsection{Bessel's equation}
\pagebreak
\section{Fourier series}
\subsection*{Overview}
\subsection{Fourier series and orthogonality.}
\subsection{Computing Fourier series and exploiting symmetry.}
\subsection{Solving ODEs with Fourier series}
\subsection{Fourier sine and cosine series.}
\subsection{Complex inner products and Fourier series.}
\subsection{Real vs. complex Fourier series.}
\subsection{Fourier transforms.}
\subsection{Pythagoras, Parseval, and Plancherel.}
\pagebreak
\section{Boundary value problems and Sturm-Liouville theory}
\subsection*{Overview}
\subsection{Boundary value problems.}
\subsection{Symmetric and Hermitian matrices.}
\subsection{Self-adjoint linear operators.}
\subsection{Sturm-Liouville theory}
\subsection{Generalized Fourier series}
\subsection{Some special orthogonal functions}
\pagebreak
\section{Partial differential equations (PDEs) on bounded domains}
\subsection*{Overview}
\subsection{Fourier's law and the diffusion equation}
\subsection{Boundary conditions for the heat equation}
\subsection{The transport and wave equations}
\subsection{The Schrödinger equation}
\pagebreak
\section{PDEs on unbounded domains}
\subsection*{Overview}
\subsection{The heat and wave equations on the real line.}
\subsection{Semi-infinite domains and the reflection method.}
\subsection{Solving PDEs with Laplace transforms.}
\subsection{Solving PDEs with Fourier transforms.}
\pagebreak
\section{Higher-dimensional PDEs}
\subsection*{Overview}
\subsection{Harmonic functions and Laplace's equation.}
\subsection{Eigenfunctions of the Laplacian.}
\subsection{The heat and wave equations in higher dimensions.}
\subsection{The Laplacian in polar coordinates.}
\subsection{Three PDEs on a disk}

\pagebreak

\section{Ressources}%
\label{sec:Ressources}

\subsection{Books}%
\label{sub:Books}

\begin{enumerate}
    \item A First Course in Differential Equations by Dennis G. Zill
    \item Ordinary and Partial Differential Equations by John Cain (intuition for harder ODEs)
\end{enumerate}

\subsection{Courses}%
\label{sub:Courses}

\begin{enumerate}
    \item Edgar Haley Math 24 - Differential Equations
    \item Professor Macauley - Differential Equations
    \item Professor Macauley - Advanced Maths for Engineering: A course
	on Partial Differential Equations
    \item Houston Math Prep - Differential Equations
\end{enumerate}

\subsection{Exercices}%
\label{sub:Exercices}

\begin{enumerate}
    \item Math Sorcerer - Differential Equations Playlist: great for easier
	problems, but seems like drilling rather than actually learning
    \item Elementary Differential Equations (+solutions) by William Trench:
	lots of exercices (recommended)
    \item Timothy Norflok's Math2080 - Differential Equations: Worksheet
	with detailed solutions (recommended)
    \item Schaum's Differential Equations
\end{enumerate}

\pagebreak
\end{document}
\end{article}

