\documentclass{article}
\usepackage{amsmath}
\usepackage{amsfonts}
\usepackage{amsthm}
\usepackage{hyperref}
\usepackage{parskip}
\usepackage{textgreek}
\begin{document}
\title{Lecture Notes for Famous Math Problem - NJ Wildberger}
\author{Emulie Chhor}
\maketitle

\section*{Introduction}

\newtheorem{definition}{Definition}[subsection]
\newtheorem{theorem}{Theorem}[subsection]
\newtheorem{corollary}{Corollary}[subsection]
\newtheorem{lemma}[theorem]{Lemma}
\newtheorem{proposition}{Proposition}[section]
\newtheorem{axiom}{Axiome}
\newtheorem{property}{Propriété}[subsection]
\newtheorem*{remark}{Remarque}
\newtheorem*{problem}{Problème}
\newtheorem*{intuition}{Intuition}

\section{Factoring large numbers into primes}

\subsection{Overview}%
\label{sub:Overview}

Un problème important de la théorie des nombres provient de Euler,
qui a montré que tout entier peut s'exprimer comme un produit de
nombres premiers. C'est ce qu'on appelle le théorème fondamental du
calcul. Par contre, on n'est pas capable de trouver tous les facteurs
de nimporte quel nombre. Plus tard, Gauss a essayé de résoudre cette
question à l'aide de l'arithmétique modulaire, et 2 résultats en découlent:
\begin{enumerate}
    \item Modular Arithmetic
    \item Fermat's Little Theorem
    \item Euler Theorem
\end{enumerate}

De plus, on apprend que si on travaille avec des nombres très très grand,
ce nombre n'existe pas, car l'univers n'est pas assez grand pour écrire
ce nombre.

\subsection{Théorème Fondamental du Calcul}

\begin{theorem}[Théorème Fondamental du Calcul]
    Tout entier positif s'exprime comme le produit de nombre premiers
\end{theorem}

\subsection{Modular Arithmetic}

\begin{theorem}[Modular Arithmetic]
    Développé par Gauss, l'arithmétique modulaire dit que
    $$ a \equiv b \, (mod \, m) $$ $$ \Longleftrightarrow
    \text{ a, b have the same remainder when divided by m } $$
    $$ \Longleftrightarrow m divides a-b $$
    $$ \Longleftrightarrow m|(a-b)$$
\end{theorem}

\subsection{Fermat's Little Theorem}

\begin{theorem}[Fermat's Little Theorem]
    If a and p are relatively prime, that is $ (a,p) = 1$, then
    $$ a_{p-1} \equiv 1 \, (mod \, p)$$
\end{theorem}

\subsection{Euler Theorem}

\begin{theorem}[Euler Theorem]
    If (a,m)=1, then $$ a^{\Phi (m)} \equiv \, (mod \, m) $$
    où $\Phi(m)$ est le nombre de chiffre relativement prime de 1 jusqu'à
    m-1
\end{theorem}

\begin{problem}
    \begin{enumerate}
        \item
    \end{enumerate}
\end{problem}


\section{The Collatz conjecture (3n+1 problem)}
\subsection{Overview}%
\label{sub:Overview}
\section{Apollonius' circle construction problems}
\subsection{Overview}%
\label{sub:Overview}
\section{The Graceful Tree Conjecture}
\subsection{Overview}%
\label{sub:Overview}
\section{Omar Khayyam and the Binomial Theorem}
\subsection{Overview}%
\label{sub:Overview}
\section{Archimedes' squaring of a parabola}
\subsection{Overview}%
\label{sub:Overview}
\section{Newcomb's paradox}
\subsection{Overview}%
\label{sub:Overview}
\section{Euler's triangulation of a polygon}
\subsection{Overview}%
\label{sub:Overview}
\section{Distances to the sun and moon}
\subsection{Overview}%
\label{sub:Overview}
\section{The integral of $x^n$ (a)}
\subsection{Overview}%
\label{sub:Overview}
\section{The integral of $x^n$ (b)}
\subsection{Overview}%
\label{sub:Overview}
\section{Steiner's regions of space problem}
\subsection{Overview}%
\label{sub:Overview}
\section{Euclid's construction problems I}
\subsection{Overview}%
\label{sub:Overview}
\section{The rotation problem and Hamilton's discovery of quaternions I}
\subsection{Overview}%
\label{sub:Overview}
\section{The rotation problem and Hamilton's discovery of quaternions IV}
\subsection{Overview}%
\label{sub:Overview}
\section{Japanese Temple Problems I Famous Math Problems 14}
\subsection{Overview}%
\label{sub:Overview}
\section{Euler's relation between vertices, edges and faces of the Platonic solids 15}
\subsection{Overview}%
\label{sub:Overview}
\section{The area of a triangle and Archimedes' formula}
\subsection{Overview}%
\label{sub:Overview}
\section{Are all true mathematical statements provable?}
\subsection{Overview}%
\label{sub:Overview}
\section{The most fundamental and important problem in mathematics}
\subsection{Overview}%
\label{sub:Overview}
\section{The rational number line and irrationalities (b)}
\subsection{Overview}%
\label{sub:Overview}
\section{Stevin numbers, infinitesimals and complex numbers}
\subsection{Overview}%
\label{sub:Overview}
\section{Dedekind cuts and computational difficulties with real numbers}
\subsection{Overview}%
\label{sub:Overview}
\section{The perspective image of a square I}
\subsection{Overview}%
\label{sub:Overview}
\section{The perspective image of a square IV: the number theoretic side}
\subsection{Overview}%
\label{sub:Overview}
\section{Computing cyclotomic polynumbers}
\subsection{Overview}%
\label{sub:Overview}
\section{Irreducibility and the Schoenemann-Eisenstein criterion}
\subsection{Overview}%
\label{sub:Overview}
\section{Logical difficulties with cyclotomic fields}
\subsection{Overview}%
\label{sub:Overview}
\section{Tips and tricks for computing cyclotomic polynumbers}
\subsection{Overview}%
\label{sub:Overview}
\section{How to construct the (true) complex numbers I}
\subsection{Overview}%
\label{sub:Overview}
\section{The remarkable Dihedron algebra}
\subsection{Overview}%
\label{sub:Overview}
\section{The geometry of the Dihedrons (and Quaternions)}
\subsection{Overview}%
\label{sub:Overview}
\section{The true algebra of complex numbers - via Dihedrons!}
\subsection{Overview}%
\label{sub:Overview}
\section{How to develop a proper theory of infinitesimals I}
\subsection{Overview}%
\label{sub:Overview}
\section{Dual complex numbers and Leibniz's differentiation rules}
\subsection{Overview}%
\label{sub:Overview}
\section{An algebraic infinitesimal approach to product and chain rules}
\subsection{Overview}%
\label{sub:Overview}
\section{Infinitesimal Calculus with Finite Fields}
\subsection{Overview}%
\label{sub:Overview}
\section{How to set up fractions I}
\subsection{Overview}%
\label{sub:Overview}
\section{How to set up fractions II -- the computational challenge}
\subsection{Overview}%
\label{sub:Overview}

\begin{enumerate}
    \item Insights into Mathematics - Famous Maths Problems playlist
\end{enumerate}

\end{document}
\end{article}

