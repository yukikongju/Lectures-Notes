\documentclass{article}
\usepackage{amsmath}
\usepackage{amsfonts}
\usepackage{amsthm}
\usepackage{hyperref}
\usepackage{parskip}
\usepackage{textgreek}
\begin{document}
\title{Lecture Notes for Number Theory}
\author{Emulie Chhor}
\maketitle

\section*{Introduction}

\newtheorem{definition}{Definition}[subsection]
\newtheorem{theorem}{Theorem}[subsection]
\newtheorem{corollary}{Corollary}[subsection]
\newtheorem{lemma}[theorem]{Lemma}
\newtheorem{proposition}{Proposition}[section]
\newtheorem{axiom}{Axiome}
\newtheorem{property}{Propriété}[subsection]
\newtheorem*{remark}{Remarque}
\newtheorem*{problem}{Problème}
\newtheorem*{intuition}{Intuition}

\section{Ressources}%
\label{sec:Ressources}

\subsection{Books}%
\label{sub:Books}

- Elementary

\subsection{Courses}%
\label{sub:Courses}

- Number Theory Playlist by Glenn Olson

\subsection{Exercices}%
\label{sub:Exercices}

\end{document}
\end{article}

