\documentclass{article}
\usepackage{amsmath}
\usepackage{amsfonts}
\usepackage{amsthm}
\usepackage{parskip}
\usepackage{textgreek}
\begin{document}
\title{Lecture Notes for Theory Of Numbers - Borcheds}
\author{Emulie Chhor}
\maketitle

\section*{Introduction}

This is my lectures notes from Field Winner's Richard Borcheds Theory of
Numbers

\newtheorem{definition}{Definition}[subsection]
\newtheorem{theorem}{Theorem}[subsection]
\newtheorem{corollary}{Corollary}[subsection]
\newtheorem{lemma}[theorem]{Lemma}
\newtheorem{proposition}{Proposition}[section]
\newtheorem{axiom}{Axiome}
\newtheorem{property}{Propriété}[subsection]
\newtheorem*{remark}{Remarque}
\newtheorem*{problem}{Problème}
\newtheorem*{intuition}{Intuition}

\section{Introduction}

This lecture gives an overview of the subject that will be discussed
throughout the course. Essentially, number theory is the study of
primes.
\begin{enumerate}
    \item
\end{enumerate}

\section{Euclid's theorem}
\section{Euclid's algorithm}
\section{Linear Diophantine equations}
\section{Fundamental theorem of arithmetic}
\section{Multiplicative functions}
\section{Dirichlet series}
\section{Congruences: Introduction}
\section{Fermat's theorem}
\section{Prime tests}
\section{Congruences: Euler's theorem}
\section{Congruences: Chinese remainder theorem}
\section{Congruences: Primitive roots}
\section{Wilson's theorem}
\section{RSA cryptography}
\section{Chevalley-Warning theorem}
\section{Quadratic residues}
\section{Gauss's lemma}
\section{Quadratic reciprocity}
\section{Jacobi symbol}

\end{document}
\end{article}
