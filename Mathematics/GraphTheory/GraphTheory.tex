\documentclass{article}
\usepackage{amsmath}
\usepackage{amsfonts}
\usepackage{amsthm}
\usepackage{hyperref}
\usepackage{parskip}
\usepackage{textgreek}
\begin{document}
\title{Lecture Notes for Graph Theory}
\author{Emulie Chhor}
\maketitle

\section*{Introduction}

\begin{enumerate}
    \item Fundamentals Concepts: Types of Graphs, Path/Cycle, Degrees
    \item Trees
    \item Connectivity
    \item Optimization
    \item Shortest Path: Trails, Circuit, Path and Cycles
    \item Planar Graphs
    \item Flow
    \item Coloring
    \item Matching
    \item Ramsey Theory
\end{enumerate}

\newtheorem{definition}{Definition}[subsection]
\newtheorem{theorem}{Theorem}[subsection]
\newtheorem{corollary}{Corollary}[subsection]
\newtheorem{lemma}[theorem]{Lemma}
\newtheorem{proposition}{Proposition}[section]
\newtheorem{axiom}{Axiome}
\newtheorem{property}{Propriété}[subsection]
\newtheorem*{remark}{Remarque}
\newtheorem*{problem}{Problème}
\newtheorem*{intuition}{Intuition}

\section{Fundamentals Concepts: Graphs, Digraphs, Degrees}

\subsection{Why study Graph Theory}%
\label{sub:Why study Graph Theory}

TODO

\subsection{Overview}%
\label{sub:Overview}

\begin{enumerate}
    \item What is a graph
    \item Terminology: walk, trail, path, circuit, cycle
    \item Graph Cycle
    \item Connected Vertices and Connected Graphs
    \item Types of Graphs: Path Graph, Cycle Graph, Complete Graph,
	Complement of a graph, Bipartite Graph, Complete Bipartite Graph
    \item Directed Graphs
    \item Degree of a Graph
\end{enumerate}

\subsection{What is a Graph}%
\label{sub:What is a Graph}

Un graphe est un "ordered pair" composé de deux éléments:
\begin{enumerate}
    \item Vertex: ensemble des "noeuds" composants le graphe
    \item Edges: ensemble de sous-ensembles qui nous dit quels "noeuds"
	sont reliés
\end{enumerate}

Notre but est de différentier les différents types de graphes et de
définir la terminlogie pour parler d'un graphe
\begin{enumerate}
    \item Undirected Graph vs Directed Graph
    \item Simple Graph
    \item Order, Size
    \item Adjacence
\end{enumerate}

\begin{definition}[Graph]
    A graph G is an ordered pair G=(V,E) where V is a finite set of
    elements and E is a set of 2 subsets of V
\end{definition}

\begin{definition}[Undirected Graph]
    An undirected graph is a graph whose edge subsets are not ordered.
    In other word, if two nodes are connected, then we can reach a to b
    and b from a.
\end{definition}

\begin{definition}[Directed Graph]
    A directed graph, also called digraph, is a graph that has a
    direction associated with its edges. In other words, the subsets
    in the Edge set are ordered. The edges are called arcs.
    \begin{enumerate}
        \item Out Degrees: Number of vertices comming out from x noted
	    $ od_G(x)$
	\item In Degrees: Number of vertices comming in to x noted
	    $id_G(x)$
    \end{enumerate}
\end{definition}

\begin{definition}[Multigraph and Pseudographs]
    A multigraph is a graph G=(V,E) is an undirected graph where the
    edges set is a multiset, which means that there can be multiple edges
    between two vertices. The number of distinct edge is called the
    multiplicity
\end{definition}

\begin{definition}[Order and Size]
    \begin{enumerate}
	\item Order |V| : number of vertex in the graph
	\item Size |E|: number of edges in the graph
    \end{enumerate}
\end{definition}

\begin{definition}[Simple Graph]
    \begin{enumerate}
	\item No loop
	\item No multiples edges
    \end{enumerate}
\end{definition}

\begin{definition}[Adjacence]
    On peut parler d'adjacence pour les vertex et les edges.
    \begin{enumerate}
	\item Vertex Adjacence: 2 vertex are adjacents if they are
	    connected by an edge
	\item Edge Adjacence: 2 edges are adjacent if they have a
	    vertex in between them
    \end{enumerate}
\end{definition}

\begin{theorem}
    The sum of the degree of all vertices is an even number
    $$ \sum deg(v)$$
    Plus généralement, the sum of the degrees of all vertices is
    twice the number of edges
    $$ \sum deg(v) = 2 |E| $$
\end{theorem}

\subsection{Terminology}%
\label{ssub:Terminology}

\begin{definition}[Walk]
    \begin{enumerate}
	\item Walk: Sequence of adjacent vertices. We can go back on our
	    steps: we can traverse edges and vertices several times.
	    We say the vertices lie on the walk.
	\item Length: Number of "steps" we make (even though we may go
	    back and forth).
	\item Open walk: the final vertex is not the same as where we
	    started
	\item Closed Walk: the end vertex is the same where we started
    \end{enumerate}
\end{definition}

\begin{remark}
    On peut utiliser les définitions suivantes pour les trail et autres
    aussi:
    \begin{enumerate}
	\item open/closed
	\item endpoints
	\item length
    \end{enumerate}
\end{remark}

\begin{definition}[Trail]
    A sequence of adjacent vertices without traversing the same edge
    more than once
\end{definition}

\begin{definition}[Path]
    A path is a sequence of adjactent vertices, but we cannot traverse
    the same vertices more than once (which also means we can't
    traverse the same edge). Can be defined as
    \begin{enumerate}
	\item List of vertices: $ P=(v_1, v_2, ..., v_8)$
	\item List of alternating vertices and edges: $ P=(v_1, v_1v_2,
	    ..., v_8) $
    \end{enumerate}
    Habituellement, on préfère définir un chemin par une liste de vertices
\end{definition}

\begin{definition}[Circuit]
    Closed trail of length 3 or more
\end{definition}

\begin{definition}[Cycle]
    Closed path that has a length greater than or equal to 3. Sometimes,
    the definition differ and we may traverse vertex several times (but
    can't cross edges).
\end{definition}

\begin{definition}[Path and Cycle]
    \begin{enumerate}
	\item A Path $P_n$ is a graph whose vertices can be arranged
	    in a sequence such that the edge set is
	    $ E = {v_i v_{i+1} | i = 1,2,...,n-1} $
	\item A Cycle $C_n$ is a graph whose vertices can be arranged in
	    a cyclic sequence such that the edge set is
	    $ E = {v_i v_{i+1} | i = 1,2,...,n-1} \cup{v_1v_n}$
    \end{enumerate}
\end{definition}

\begin{definition}[Degree of Path and Cycle]
    The degree of a path and a cycle is the number of vertex it has.
\end{definition}

\begin{definition}[Girth]
    Smallest Cycle in the graph
\end{definition}

\begin{definition}[Distance and Diameter between vertices]
    Soit deux noeud u et v.
    \begin{enumerate}
	\item Distance entre u et v: plus court chemin entre u et v
	\item Diameter entre u et v: plus long chemin entre u et v
    \end{enumerate}
\end{definition}

\begin{theorem}[Properties of Degrees in Path and Cycle]
    \begin{enumerate}
	\item A path of degree n has n nodes and (n-1) edges
	\item A cycle of degree n has n nodes and n edges
    \end{enumerate}
\end{theorem}

\begin{proposition}
    Every graph G contains a path of length n and a cycle of length
    at least n+1
\end{proposition}

\subsection{Connected and Disconnected Graphs}%
\label{sub:Connected Graphs}

\begin{definition}[Connected Graph]
    A graph is connected if for every pair of disinct vertices $ u,v \in
    V(G)$, there is a path from u to v in G. Otherwise, we say the
    graph is disconnected
\end{definition}

\begin{definition}[Connected Vertices]
\end{definition}

\begin{definition}[Open and Closed Neighborhood]
    TODO
\end{definition}

\subsection{Families of Graph and Special Graph}%
\label{sub:Families of Graph and Special Graph}

\begin{enumerate}
    \item Complete Graph $K_n$: simple graph with an edge between every
	pair of vertices
    \item Empty graph: Graph with no edges
    \item Bipartite Graph: a graph whose vertex can be partitionned into
	two sets $V_1$ and $V_2$ such that every edges $ u,v \in E$ has
	$u \in V_1$ and $v in V_2$
    \item Complete Bipartite Graph: every node can reach all nodes in
	the other subset (end)
    \item Star
    \item k-regular graph: each vertex is degree k
    \item Cubic Graph: 3-regular graph (ex: Petersen Graph)
    \item Irregular graph:
    \item Path Graph:
    \item Cycle Graph:
    \item Hypercube Graph
\end{enumerate}

\subsubsection{Bipartite Graphs}%
\label{sub:Bipartite Graphs}

\begin{definition}[Bipartite Graph]
    A graph is bipartite if we can split that graph in two sets such that
    all vertices in A maps to B
\end{definition}

\begin{definition}[Complete Bipartite Graph]
    A complete bipartite graph is a bipartite graph where all vertices in
    A maps to all vertices in B
\end{definition}

\begin{theorem}[Bipartite graph and odd cycle]
    A graph is bipartite $\Longleftrightarrow$ it has no odd cycle
\end{theorem}

\begin{proof}[If a graph is bipartite, then it has no odd cycle]
    The proof is done by contradiction
    \begin{enumerate}
	\item Let G be a bipartite graph and c be an odd cycle such that
	    $ c=(v_1, v_2, ..., v_n, v_1)$
	\item Because G is bipartite, then we can partition odd vertices
	    into set X and even vertices into set Y (because adjacent
	    vertices cannot be in the same component)
	\item Since c is an odd cycle, then $v_n$ is odd.
	\item This contradicts the fact that G is bipartite, because two
	    adjacent vertices are in the same component $(v_1 and v_n)$
	    are adjacents and both odd.
    \end{enumerate}
\end{proof}

\begin{proof}[If a graph has no odd cycle, then it is bipartite]
    The proof is also done by contradiction
    \begin{enumerate}
	\item Let G be a graph with no odd cycle.
	\item Let's partition the vertices of the cycles by its parity
	    such that $$ X = {v \in V(G) | d(v,w) \text{ is even }} $$
	    $$ Y = {v \in V(G) | d(v,w) \text{ is odd }} $$
	    $$ X \cap Y = {} \text{ (distance is unique) }$$
	    $$ X \cup Y = V(G) \text{ (connected graph) }$$
	\item SFC, there are two adjacents vertices that are in the same
	    set: $a,b \in X$ or $a,b \in Y$ such that $ab \in E(V) $
	\item Suppose a=w, then d(a,w)=0 (the distance is even). Thus, d(b,w)
	    is even and d(a,b) is even. However, since a,b are adjacents
	    then d(a,b)=1. Therefore, $a \neq b \neq w$
	\item Consider the shortest path from aw denoted by P, the
	    shortest path from bw denoted by Q, and m be the last
	    common vertex of P and Q. Let $P_1$ and $Q_1$ be the path from
	    a to m and b to m respectively and $P_2$ and $Q_2$ both be the
	    path from m to w
	\item Then $|Q_1| = |P_1|, |P|=|P_1|+|P_2|, |Q|=|Q_1|+|Q_2|$. Since
	    a,b are in the same set, then d(a,w) and d(b,w) must have the
	    same parity. Since $|P_1|=|Q_1|$ by construction, then
	    $|P_2|$ and $|Q_2$ must have the same partity by parity of
	    integers
	\item If we construct a cycle from M to A to B using d(a,b),
	    $P_2$. $Q_2$, we have an odd cycle: $|P_2|+|Q_2|+1=2k+1 \forall
	    k \in \mathbb{Z}$, which contradicts the fact that G has no
	    odd cycle
    \end{enumerate}
\end{proof}

\subsubsection{Complete Graph}%
\label{ssub:Complete Graph}

\begin{theorem}
    Let G=(V,E) be a graph with m vertex. Alors, la somme de tous les
    degrés d'un graphe est le double du nombre de edges.
    $$ \sum deg(v) = 2 |E| = 2m$$
\end{theorem}

\begin{remark}
    La preuve se fait par induction. On suppose qu'on a un graphe de m+1
    edges et qu'on lui enlève un dege arbitraire.
\end{remark}

\begin{theorem}[Handshaking Theorem]
    The Number of Edges in a Complete Graph is $ |E| = \frac{N(N-1)}{2} $
    Proof:
    \begin{enumerate}
	\item N vertex in graph
	\item The degree of each vertex is N-1 by definition of a complete
	    graph (each node is connected to all the other)
	\item Sum of all vertex degrees is $\sum d(v_i) = N(N-1)$
	\item Number of edges is $ |E| = \frac{N(N-1)}{2} $ because we
	    counted all edges twice
    \end{enumerate}
\end{theorem}

\subsubsection{Complement of a Graph}%
\label{ssub:Complement of a Graph}

\begin{definition}[Complement of a Graph]
    Let G be a graph. The complement of G, noted $\bar G$. uv is an edge of
    $\bar G \Longleftrightarrow $ uv is not an edge of G
\end{definition}

\begin{definition}[Self Complementary Graph]
    A graph that is isometric to its complement is self complementary
\end{definition}

\begin{theorem}[Connectivity of the complement of a graph]
    A graph or its complement must be connected $\Longleftrightarrow$
    if a graph is disconnected, then its complement must be connected
\end{theorem}

\begin{proof}[Connectivity of the complement of a graph]
    \begin{enumerate}
	    Let's work with the second proposition
	\item $(\Longrightarrow )$ If G is a disconnected graph and u,v be two
	    vertices in G such that u and v are in different components. Then,
	    by definition, uv must be in $\bar G$. Therefore, uv $\in \bar G$, a
	    connected graph.
	\item $(\Longleftarrow)$ If uv $\in V(G)$, such that they are in the same
	    components. Then u,v are adjactents. If uv are not in the same
	    component, there exist w such that $ uw \in V(G)$ and $vw \in V(G)$,
	    therefore, there exist a uv path in $\bar G$
    \end{enumerate}
\end{proof}

\begin{corollary}[There is not disconnected self complement graph]
    If there were such a graph, then G and $\bar G$ must be disconnected,
    which contradicts the fact that G or $\bar G$ must be connected.
\end{corollary}

\subsection{Degrees of a Graph}%
\label{sub:Degrees of a Graph}

\begin{definition}
    \begin{enumerate}
	\item minimum degree: $\delta(G)$
	\item maximum degree: $\Delta(G)$
	\item Isolated Vertex: deg(G)=0
	\item End Vertex (leaf): deg(G)=1
    \end{enumerate}
\end{definition}

\begin{theorem}[Every Graph has an even number of odd Degree vertices]
    Proof by contradiction by using the fact that sum of odd number is even
    and Handshaking
    \begin{enumerate}
	\item Let G be a graph with odd number of odd degree vertices.
	    Let X and Y be the sets of even and odd vertices respectively
	    such that $ X={v \in G(V)| \text{ deg(v) is even }} $,
	    $ Y={v \in G(V)| \text{ deg(v) is odd}} $
	\item Remark that
	    \begin{itemize}
		\item $ \sum^{}_{v \in G(V)} = 2m$ because of Handshaking Theorem
	    \item $\sum^{}_{v \in X}  = 2k$ (even) because the sum of even
		number is even
	    \item $\sum^{}_{v \in X}  = 2k$ (even) because the sum of odd
		number is even (where our contradiction lies)
	    \end{itemize}
	\item However, $ \sum^{}_{v \in Y} = \sum^{}_{v \in G(V)} -
	\sum^{}_{v \in X} = 2(k-l) $ is even, which contradicts the facts that
	    $\sum^{}_{v \in Y}$ is odd.
    \end{enumerate}
\end{theorem}

\begin{theorem}[Degree sum condition for connected graph]
    Let G be a graph of order n. If $deg(u)+deg(v) \geq n-1$, then G is
    connected and $diam(G) \leq 2$
\end{theorem}

\begin{proof}
    Pour montrer que G est connexe, on veut montrer qu'il existe un
    chemin entre d'un vertex u à v. Si u et v sont adjacent, alors c'est
    trivial. Si u et v ne sont pas adjactents, on utilise l'hypothèse
    $deg(u)+deg(v) \geq n-1 \geq n-2$ nous dit qu'il existe n-1 edges qu'il
    existe un vertex intermédiaire w par lequel on peut passer pour se
    rendre à v, et donc qu'il existe un chemin (u,w,v) de deg(2).
\end{proof}

\begin{theorem}[Minimum Degree Condition for Connected Graph]
    If G is a graph of order n with $\delta(G) \geq \frac{n-1}{2} $,
    then G is connected. Note: $\delta(G)$ is the minimum degree of
    a graph
\end{theorem}

\begin{proof}
    On veut utiliser le degree sum condition for connected graph.
    \begin{enumerate}
	\item $deg(u)+deg(v) \geq \delta(G)+\delta(G)$
	\item $deg(u)+deg(v) \geq \frac{n+1}{2} + \frac{n+1}{2} $
	    par hypothèse
	\item $deg(u)+deg(v) \geq n-1$, which is always true by degree
	    sum condition
    \end{enumerate}
\end{proof}

\begin{remark}[Necessary vs Sufficient condition]
    The minimum degree condition for connected graph is sufficient to say
    if a graph is connected, but not necessary.
\end{remark}

\subsection{Isomorphic Graph}%
\label{sub:Isomorphic Graph}

\begin{definition}[Isomorphic Graph]

\end{definition}


\begin{definition}[Degree of Sequence]
    The degree of a sequence is the ordered set of degree(v)
    \begin{enumerate}
	\item non-increasing: $ d_n \geq d_{n+1}$
    \end{enumerate}
\end{definition}

\begin{theorem}
    If two graph  are isomorphic, then they have the same degree sequence
\end{theorem}



\section{Connectivity}

\subsection{Overview}%
\label{sub:Overview}

\begin{enumerate}
    \item Vertex Cuts and Connectivity
    \item Edges Cuts and Connectivity
    \item Minimum Spanning Trees
    \item Menger Theorem
    \item Eulerian and Hamiltonians path/cycles
\end{enumerate}

\subsection{Fundementals of Connectivity}

\begin{definition}[Connected Vertices]
    Two vertices are connected if there exist a path between them
\end{definition}

\begin{definition}[Connected Graph]
    A graph is connected if, for every vertices in the graph, we can
    reach any other node. If a graph is not connected, we say it is
    disconnected made of components.
\end{definition}

\begin{remark}
    A connected graph has one single component
\end{remark}

\begin{definition}[Components of a graph]
    A component of a graph is a maximal connected subgraph, which means
    that
    \begin{enumerate}
	\item Connected: All the nodes in the subgraph can be reached
	    from one to another
	\item Maximal: there no is node or vertex that we can add without
	    violating the connected property
    \end{enumerate}
\end{definition}

\begin{theorem}[Connected Graph contains Two Non-cut Vertices]
    WHAT
\end{theorem}


\section{Trees}%
\label{sec:Trees}

\subsection{Overview}%
\label{sub:Overview}

\begin{enumerate}
    \item Properties
\end{enumerate}



\section{Optimization}
\section{Shortest Path}
\section{Planar Graphs}

\subsection{Overview}%
\label{sub:Overview}

\begin{enumerate}
    \item What is a Planar Graph
\end{enumerate}

\section{Coloring}
\section{Flow}

\section{Ressources}%
\label{sec:Ressources}

\subsection{Books}%
\label{sub:Books}

- Reinhard Diestel: Graph Theory
- Discrete Structure by Michiel Sidt (recommended)

\subsection{Courses}%
\label{sub:Courses}

- Wrath of Math: Graph Theory Playlist
- Sarada Herke: Graph Theory
- Lecture Notes from JL Martin Math 105 - Topics in Mathematics:
\url{https://jlmartin.ku.edu/courses/math105-F11/}

\subsection{Exercices}%
\label{sub:Exercices}

- Introduction to Graph Theory by Douglas B. West: Proofs-based book on
graph theory
- Combinatorics and Graph Theory - Vasudev

\end{document}
\end{article}
