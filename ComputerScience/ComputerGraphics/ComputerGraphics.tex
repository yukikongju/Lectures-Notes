\documentclass{article}
\usepackage{amsmath}
\usepackage{amsfonts}
\usepackage{amsthm}
\usepackage{parskip}
\usepackage{textgreek}
\begin{document}
\title{Lecture Notes for CMU 15-462: Computer Graphics}
\author{Emulie Chhor}
\maketitle

\section*{Introduction}

Courses Notes for CMU 15-462: Computer Graphics by Keenan Crane

\section{Course Overview}

\textbf{Overview}

\begin{enumerate}
    \item What is Computer Graphics
    \item What do we learn in Computer Graphics
    \item What is Modelization
    \item What is Rendering
    \item Projects throughout the course
\end{enumerate}

\subsection{What is Computer Graphics}

The goal of computer graphics is to simulate sensory information by
turning digital information into sensory stimuli (visual, sound, texture).
Its "inverse" would be the field of computer vision, whose goal is to
translate sensory information in our environment into digital information.

\subsection{What do we learn in Computer Graphics}

There are two phases to learning computer graphics:
\begin{enumerate}
    \item Theory: What are the techniques used to simulate sensory stimuli
    \item Systems: How can we simulate those stimuli efficiently
\end{enumerate}

\textbf{Theory}

\begin{enumerate}
    \item Representation: How do we encode shape and motion?
    \item Sampling and Aliasing: How can we acquire and reproduce signal
    \item Numerical Methods: How can we modify signal
    \item Radiometry and Light Transport: How can we reproduce light
	behavior
    \item Perception: What can we learn from psychology and physiology
	to enhance human perception
\end{enumerate}

\textbf{Systems}

\begin{enumerate}
    \item Parallel and Heterogenous Processing
    \item Graphics-Specific Programming Languages
\end{enumerate}

\subsection{What is Modelization}

The first concept to understand in computer graphics is that we can
produce simuli in two steps:
\begin{enumerate}
    \item Modelization: How can we describe object to a computer
    \item Rasterization: How will the computer translate our model into
	sensory stimuli
\end{enumerate}

Usually, modelization is done by
\begin{enumerate}
    \item Listing a bunch of points
    \item Linking these points with a list of edges
\end{enumerate}

However, we often have to perform some transformations to accurately
represent the object. For instance, we need to find a way to represent
3D object onto a 2D shape.

\textbf{How to convert 3D points into 2D shapes}

One solution is to use perspective to our advantage. We know that object
further from us are smaller, so we can use the "pinhole camera" from
optical physics to realize that we have rectangular triangle, which
allow us to use pythagoras to form a ration between the original points
onto our 2D shape. If we have a 3D point (x,y,z) that we want to translate
onto a 2D shape (u,v), we
\begin{enumerate}
    \item substract the camera from the 3D point
    \item Divide (x,y) by z to get our 2D point (u,v)
\end{enumerate}

\subsection{What is Rendering}

To display our points on the computer, we first have to understand how
the computer screen works. A computer screen is made of pixels, which
are little square that contains a value which tells us how much green,
red and blue should be in the square.

Often, we talk about raster display because we take a continuous object
and represent it with a discrete mapping (pixels grid)

\textbf{Which Rasterization Techniques to use: diamond rule}

Since our shapes aren't alway rectangular, we must decide which pixels
to turn on and off. There exist a bunch of rasterization techniques,
but we often make up our own based on our needs:
\begin{enumerate}
    \item Turn on all pixels that the vector touches
    \item Diamond Rules: Turn pixel only if it passes through the diamond
	shape inside the pixel
\end{enumerate}

\textbf{How to color the pixel}

Like with rasterization, there are several ways to color each pixel,
but we often like to color it using the ratio of the vector inside the
pixel: the more a vector is inside a pixel, the more we make the color
thicker

\textbf{How does the computer find vector: incremental rasterization}

A naive solution to find which pixels to color would be to check each
pixel individually, which takes $O(n^2)$ for O(n) pixels. A better method
would be to use incremental rasterization. We

\begin{enumerate}
    \item Calculate the slope of the line and add it to the y-value to
	find the pixels inside the x-range of the vector
    \item Compute the pixel value
    \item Draw
\end{enumerate}

\subsection{Projects throughout the course}

The course consists of building our own library
\begin{enumerate}
    \item Rasterization
    \item Geometric Modeling
    \item Photorealistic Rendering
    \item Motion and Animation
\end{enumerate}

\section{Linear Algebra}

\textbf{Overview}

\begin{enumerate}
    \item What is Linear Algebra
    \item Geometric and algebraic interpretation of Vector Spaces
    \item Defining what is a linear map
    \item Span, Basis and Orthonormal Basis
    \item System of Linear Equations
    \item Matrices in Linear Algebra
\end{enumerate}

\subsection{What is Linear Algebra}

L'algèbre linéaire est la branche de mathématique qui étudie les espaces
vectoriel. Il est important de maitriser ces notions puisqu'elles
permettent d'abstraire certains concepts géométriques algébriquement,
ce qui nous permet de simplifier nos calculations.

\subsection{Geometric and algebraic interpretation of Vector Spaces}

\textbf{Qu'est-ce qu'un vecteur}

Géométriquement, on peut visualiser un vecteur comme une flèche ayant
une direction et une longuer. Ce vecteur se situe dans un espace qu'on
nomme espace vectoriel. Dans cet espace, ce vecteur possède certaines
propriétés qu'on nomme axiomes.

\textbf{Pourquoi étudier les espaces vectoriels}

Habituellement, l'interprétation géométrique et algébrique des espaces
vectoriels se fait assez bien. On veut donc identifier ces propriétés afin
de pouvoir généraliser les espaces vectoriels non pas uniquement sur des
vecteurs, mais sur des structures plus compliquées tel que des polynômes,
fonctions, matrices, etc.\\

Ainsi, puisqu'on sait qu'un vecteur est caractérisé par sa norme et sa
direction, on veut pouvoir généraliser ces notions et interpréter
géométriquement ces objets.

\subsubsection{Axiomes d'un espace vectoriel}

Un espace vectoriel possèdent les 10 propriétés suivantes (axiomes):
\begin{enumerate}
    \item
    \item
    \item
    \item
    \item
    \item
    \item
    \item
    \item
\end{enumerate}

Dans des cours de maths plus avancés, on verra que les espaces vectoriels
possèdent ces caractéristiques puisque ce sont des fields/ring

De plus, il est utile d'interpréter ces propriétés géométriquement
et algébriquement. C'est donc un bon exercices de voir pourquoi, par
exemple, l'addition de vecteurs est commutatif géométriquement et
algébriquement.

\subsubsection{Interprétation géométrique et algébrique de la norme}

Cette sous section focusse sur les propriétés de la norme d'un objet dans
un espace vectoriel. Ici, on se penchera sur la norme d'un vecteur et
d'une fonction, norme qu'on nomme $L^2$. Notre but sera donc de
\begin{enumerate}
    \item Calculer la norme d'une vecteur et d'une fonction
    \item Interpréter géométriquement et algébriquement les propriétés
	de la norme d'un vecteur et d'une fonction
\end{enumerate}

\textbf{Calcul de la Norme}

La norme d'un vecteur est sa longueur, alors que la norme $L^2$ d'une
fonction est l'aire sous sa courbe. Notons qu'on peut interpréter une
intégrale comme une somme (penser à Riemann)

\begin{enumerate}
    \item Norme d'un vecteur: $ |u| = \sqrt(\sum^{n}_{i=1} u_i ^2) $
    \item Norme d'une fonction $(L^2)$ norm: $ || f ||
	:= \sqrt(\int_{{0}}^{{1}} {f(x) ^2} \: d{x} {}	) $
\end{enumerate}

\subsubsection{Interprétation géométrique et algébrique des propriétés de la norme}

\begin{enumerate}
    \item $ | u| \geq 0 $
    \item $ | u| = 0 \Longleftrightarrow u=0 $
    \item $ |cu| = |c| |u| $
    \item
    \item Inégalité du triangle: $ |u| + |v| \geq |u+v| $
\end{enumerate}

\subsection{Interprétation géométrique et algébrique du produit scalaire}

Le produit scalaire mesure si 2 vecteurs (ou fonction) se dirige vers
la même direction (leur alignement). Encore une fois, on veut
\begin{enumerate}
    \item Calculer le produit scalaire de deux vecteurs/fonctions
    \item Interpréter géométriquement et algébriquement les propriétés
	du produit scalaire
\end{enumerate}

\subsubsection{Calcul du produit scalaire}

\begin{enumerate}
    \item Produit Scalaire d'un vecteur: $ <u,v> := \sum^{n}_{i=1} u_i v_i $
    \item Produit Scalaire d'une fonction: $ <<f,g>> :=
	\int_{{0}}^{{1}} {f(x)g(x)} \: d{x} {} $
\end{enumerate}

\subsubsection{Interprétation géométrique et algébrique des propriétés
du produit scalaire}

\begin{enumerate}
    \item Symmétrique: $ <u,v> = <v,u>$
    \item Projection and scaling
    \item If we scale the vector, the scalar product is also scaled:
	$ <cu, v> = c <u,v> $
    \item a vector is aligned with itself: $ <u,u> = 1$ et $ <u,u> \geq 0$
\end{enumerate}

\textbf{Edges Detection using Derivatives}

Parfois, la norme et le produit vectoriel d'une image ou d'un singal
peuvent être trompeur. Par exemple, si on interpréte la norme d'une image
comme sa "brillance", alors il se pourrait que notre algorithme nous
indique qu'il y ait plus d'information dans une image d'un ciel que dans
une photo d'un chien. Pour ce faire, on peut redéfinir notre définition
de la norme comme étant la dérivée, et ainsi détecter les edges.

\subsection{Defining what is a linear map}

\subsubsection{Pourquoi travailler avec des transformations linéaires}

On veut travailler avec des transormations linéaires parce que
\begin{enumerate}
    \item travailler avec des transformations non-linéaires est un
	problème NP, très difficile à computer
    \item Les transformations usuelles
	(rotations, translation, ..) sont des transformations linéaires
    \item Puisque l'addition et la multiplication sont définies de la
	même façon avant et après transformations, on peut faire des
	approximations de Taylor: on peut approximer des fonctions
	transcendantes avec des fonctions polynômiales.
\end{enumerate}

Plus tard, on verra qu'on peut interpréter les matrices comme des
transformations linéaires.

\textbf{Is it a linear function}

On peut déterminer si une transformation est linéaire de 2 façons:
algébriquement et géométriquement.

Géométriquement:
\begin{enumerate}
    \item L'origine part à la même place
    \item Les lignes restent des lignes
\end{enumerate}

Algébriquement:
\begin{enumerate}
    \item l'addition est définie de la même façon avant et après
	transformation:
    \item la multiplication est définie de la même façon avant et après
	transformation:
\end{enumerate}

Ainsi, la fonction $f(x) = ax+b$ n'est pas une transformation linéaire,
puique géométriquement, elle ne mappe pas les vecteurs à la même origine,
et algébriquement, l'addition n'est pas définie de la même façon avant
et après transformation. Intuitivement, c'est comme si on disait que
l'ordre dans lequel on performe nos transformation change le output,
ce qui n'a pas de sens.

\subsection{Span, Basis and Orthonormal Basis}

\textbf{Span (Espace générateur)}

On peut interpréter un espace générateur de la façon suivante: c'est
l'ensemble de vecteurs qu'on peut obtenir en faisant la combinaison
linéaire de vecteurs d'un ensemble.\\

La notion de span est importante, car on veut pouvoir déterminer si on
notre ensemble de vecteurs est assez grand pour générer les vecteurs
de l'espace

\textbf{Basis}

La base d'un espace vectoriel est simplement l'ensemble des vecteurs
nécessaires pour construire tous les vecteurs d'un espace. On veut
pouvoir générer tous les vecteurs, sans en avoir trop. La base doit être:
\begin{enumerate}
    \item Linéairement indépendant: on a un nombre minimal de vecteurs
	pouvant générer ie il existe une seule façon de construire
	le vecteur
    \item Générateur: on peut faire une combinaison linéaire
\end{enumerate}

Il faut noter que lorqu'on additionne, multiplie des vecteurs, il faut
toujours s'assurer qu'on travaille dans la même base.

\textbf{Orthonormal Basis}

En computer graphics, on veut travailler avec des bases orthonormales,
car la longueur reste la même

\begin{enumerate}
    \item Comment déterminer si la base est orthogonale? les paires des
	vecteurs sont orthogonales entre elles
    \item Comment transformer une base quelqconque en base orthogonale?
	Gram-Schmidt ou QR Decomposition
\end{enumerate}

\textbf{Fourier Transform}

Les fouriers transformations et les fourier decomposition sont
utilisés pour décomposer les signaux sur des bases sinusoidales
$ cos(nx), sin(mx) $ en projettant les fonctions (se répètent
sur des période 2 pi) sur différentes fréquences.

\subsection{System of Linear Equations}

Une autre utilisation de l'algèbre linéaire est p/r à la résolution de
système linéaire. On veut résoudre les systèmes linéaire, car ils nous
disent si on a des intersections (points, droites, plans, ...)

\textbf{Types de solutions}

Encore une fois, on peut interpréter les solutions à un SEL algébriquement
et géométriquement.

\begin{enumerate}
    \item Aucune solution
	\begin{itemize}
	    \item Algébriquement: contradiction [ 0 0 .. 0  | k]
	    \item Géométriquement: il n'y a pas d'intersections
	\end{itemize}
    \item Solution Unique
	\begin{itemize}
	    \item Algébriquement: toutes les variables sont associées à un
		pivot
	    \item Géométriquement: il y a une seule intersection point/droite
	\end{itemize}
    \item Infinité de solutions
	\begin{itemize}
	    \item Algébriquement: variable libre
	    \item Géométriquement: superposition des droites/plans
	\end{itemize}
\end{enumerate}

Notons que les SEL sont aussi associés à l'équation Ax=b, qui nous
permet de trouver la préimage x de b, c-à-d le vecteur original avant
d'avoir appliquer la matrice de transformation A sur le vecteur b

\subsection{Matrices in Linear Algebra}

On a peu parler des matrices, car elles ont plusieurs interprétations
géométriques, ce qui peut nous mélanger dans la compréhension.

TODO

\section{Vector Calculus}
NEXT
\section{Drawing a Triangle and an Intro to Sampling}
\section{Spatial Transformations}
\section{3D Rotations and Complex Representations}
\section{Perspective Projection and Texture Mapping}
\section{Depth and Transparency}
\section{Introduction to Geometry}
\section{Meshes and Manifolds}
\section{Digital Geometry Processing}
\section{Geometric Queries}
\section{Spatial Data Structures}
\section{Color}
\section{Radiometry}
\section{The Rendering Equation}
\section{Numerical Integration}
\section{Monte Carlo Rendering}
\section{Variance Reduction}
\section{Introduction to Animation}
\section{Dynamics and Time Integration}
\section{Optimization}
\section{Physically Based Animation and PDEs}

\end{document}
\end{article}
