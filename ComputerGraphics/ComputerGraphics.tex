\documentclass{article}
\usepackage{amsmath}
\usepackage{amsfonts}
\usepackage{amsthm}
\usepackage{parskip}
\usepackage{textgreek}
\begin{document}
\title{Lecture Notes for CMU 15-462: Computer Graphics}
\author{Emulie Chhor}
\maketitle

\section{Introduction}

    \begin{enumerate}
	\item Item1
    \end{enumerate}

\newtheorem{definition}{Definition}[subsection]
\newtheorem{theorem}{Theorem}[subsection]
\newtheorem{corollary}{Corollary}[subsection]
\newtheorem{lemma}[theorem]{Lemma}
\newtheorem{proposition}{Proposition}[section]
\newtheorem{axiom}{Axiome}
\newtheorem{property}{Propriété}[subsection]
\newtheorem*{remark}{Remarque}
\newtheorem*{problem}{Problème}
\newtheorem*{intuition}{Intuition}

\section{Course Overview}
\section{Linear Algebra}
\section{Vector Calculus}
\section{Drawing a Triangle and an Intro to Sampling}
\section{Spatial Transformations}
\section{3D Rotations and Complex Representations}
\section{Perspective Projection and Texture Mapping}
\section{Depth and Transparency}
\section{Introduction to Geometry}
\section{Meshes and Manifolds}
\section{Digital Geometry Processing}
\section{Geometric Queries}
\section{Spatial Data Structures}
\section{Color}
\section{Radiometry}
\section{The Rendering Equation}
\section{Numerical Integration}
\section{Monte Carlo Rendering}
\section{Variance Reduction}
\section{Introduction to Animation}
\section{Dynamics and Time Integration}
\section{Optimization}
\section{Physically Based Animation and PDEs}

\end{document}
\end{article}
