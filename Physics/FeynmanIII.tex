\documentclass{article}
\usepackage{amsmath}
\usepackage{amsfonts}
\usepackage{amsthm}
\usepackage{hyperref}
\usepackage{parskip}
\usepackage{textgreek}
\begin{document}
\title{Lecture Notes for Feynman Lectures on Physics - Volume III}
\author{Emulie Chhor}
\maketitle

\section*{Introduction}

\newtheorem{definition}{Definition}[subsection]
\newtheorem{theorem}{Theorem}[subsection]
\newtheorem{corollary}{Corollary}[subsection]
\newtheorem{lemma}[theorem]{Lemma}
\newtheorem{proposition}{Proposition}[section]
\newtheorem{axiom}{Axiome}
\newtheorem{property}{Propriété}[subsection]
\newtheorem*{remark}{Remarque}
\newtheorem*{problem}{Problème}
\newtheorem*{intuition}{Intuition}

\section{Chapter 1.Quantum Behavior}
\subsection{1-1. Atomic mechanics}
\subsection{1-2. An experiment with bullets}
\subsection{1-3. An experiment with waves}
\subsection{1-4. An experiment with electrons}
\subsection{1-5. The interference of electron waves}
\subsection{1-6. Watching the electrons}
\subsection{1-7. First principles of quantum mechanics}
\subsection{1-8. The uncertainty principle}
\section{Chapter 2.The Relation of Wave and Particle Viewpoints}
\subsection{2-1. Probability wave amplitudes}
\subsection{2-2. Measurement of position and momentum}
\subsection{2-3. Crystal diffraction}
\subsection{2-4. The size of an atom}
\subsection{2-5. Energy levels}
\subsection{2-6. Philosophical implications}
\section{Chapter 3.Probability Amplitudes}
\subsection{3-1. The laws for combining amplitudes}
\subsection{3-2. The two-slit interference pattern}
\subsection{3-3. Scattering from a crystal}
\subsection{3-4. Identical particles}
\section{Chapter 4.Identical Particles}
\subsection{4-1. Bose particles and Fermi particles}
\subsection{4-2. States with two Bose particles}
\subsection{4-3. States with n Bose particles}
\subsection{4-4. Emission and absorption of photons}
\subsection{4-5. The blackbody spectrum}
\subsection{4-6. Liquid helium}
\subsection{4-7. The exclusion principle}
\section{Chapter 5.Spin One}
\subsection{5-1. Filtering atoms with a Stern-Gerlach apparatus}
\subsection{5-2. Experiments with filtered atoms}
\subsection{5-3. Stern-Gerlach filters in series}
\subsection{5-4. Base states}
\subsection{5-5. Interfering amplitudes}
\subsection{5-6. The machinery of quantum mechanics}
\subsection{5-7. Transforming to a different base}
\subsection{5-8. Other situations}
\section{Chapter 6.Spin One-Half}
\subsection{6-1. Transforming amplitudes}
\subsection{6-2. Transforming to a rotated coordinate system}
\subsection{6-3. Rotations about the z-axis}
\subsection{6-4. Rotations of 180 and 90 about y}
\subsection{6-5. Rotations about x}
\subsection{6-6Arbitrary rotations}
\section{Chapter 7.The Dependence of Amplitudes on Time}
\subsection{7-1. Atoms at rest; stationary states}
\subsection{7-2. Uniform motion}
\subsection{7-3. Potential energy; energy conservation}
\subsection{7-4. Forces; the classical limit}
\subsection{7-5. The “precession” of a spin one-half particle}
\section{Chapter 8.The Hamiltonian Matrix}
\subsection{8-1. Amplitudes and vectors}
\subsection{8-2. Resolving state vectors}
\subsection{8-3. What are the base states of the world?}
\subsection{8-4. How states change with time}
\subsection{8-5. The Hamiltonian matrix}
\subsection{8-6. The ammonia molecule}
\section{Chapter 9.The Ammonia Maser}
\subsection{9-1. The states of an ammonia molecule}
\subsection{9-2. The molecule in a static electric field}
\subsection{9-3. Transitions in a time-dependent field}
\subsection{9-4. Transitions at resonance}
\subsection{9-5. Transitions off resonance}
\subsection{9-6. The absorption of light}
\section{Chapter 10.Other Two-State Systems}
\subsection{10-1. The hydrogen molecular ion}
\subsection{10-2. Nuclear forces}
\subsection{10-3. The hydrogen molecule}
\subsection{10-4. The benzene molecule}
\subsection{10-5. Dyes}
\subsection{10-6. The Hamiltonian of a spin one-half particle in a magnetic field}
\subsection{10-7. The spinning electron in a magnetic field}
\section{Chapter 11.More Two-State Systems}
\subsection{11-1. The Pauli spin matrices}
\subsection{11-2. The spin matrices as operators}
\subsection{11-3. The solution of the two-state equations}
\subsection{11-4. The polarization states of the photon}
\subsection{11-5. The neutral K-meson}
\subsection{11-6. Generalization to N-state systems}
\section{Chapter 12.The Hyperfine Splitting in Hydrogen}
\subsection{12-1. Base states for a system with two spin one-half particles}
\subsection{12-2. The Hamiltonian for the ground state of hydrogen}
\subsection{12-3. The energy levels}
\subsection{12-4. The Zeeman splitting}
\subsection{12-5. The states in a magnetic field}
\subsection{12-6. The projection matrix for spin one}
\section{Chapter 13.Propagation in a Crystal Lattice}
\subsection{13-1. States for an electron in a one-dimensional lattice}
\subsection{13-2. States of definite energy}
\subsection{13-3. Time-dependent states}
\subsection{13-4. An electron in a three-dimensional lattice}
\subsection{13-5. Other states in a lattice}
\subsection{13-6. Scattering from imperfections in the lattice}
\subsection{13-7. Trapping by a lattice imperfection}
\subsection{13-8. Scattering amplitudes and bound states}
\section{Chapter 14.Semiconductors}
\subsection{14-1. Electrons and holes in semiconductors}
\subsection{14-2. Impure semiconductors}
\subsection{14-3. The Hall effect}
\subsection{14-4. Semiconductor junctions}
\subsection{14-5. Rectification at a semiconductor junction}
\subsection{14-6. The transistor}
\section{Chapter 15.The Independent Particle Approximation}
\subsection{15-1. Spin waves}
\subsection{15-2. Two spin waves}
\subsection{15-3. Independent particles}
\subsection{15-4. The benzene molecule}
\subsection{15-5. More organic chemistry}
\subsection{15-6. Other uses of the approximation}
\section{Chapter 16.The Dependence of Amplitudes on Position}
\subsection{16-1. Amplitudes on a line}
\subsection{16-2. The wave function}
\subsection{16-3. States of definite momentum}
\subsection{16-4. Normalization of the states in x}
\subsection{16-5. The Schrödinger equation}
\subsection{16-6. Quantized energy levels}
\section{Chapter 17.Symmetry and Conservation Laws}
\subsection{17-1. Symmetry}
\subsection{17-2. Symmetry and conservation}
\subsection{17-3. The conservation laws}
\subsection{17-4. Polarized light}
\subsection{17-5. The disintegration of the $Λ0$}
\subsection{17-6. Summary of the rotation matrices}
\section{Chapter 18.Angular Momentum}
\subsection{18-1. Electric dipole radiation}
\subsection{18-2. Light scattering}
\subsection{18-3. The annihilation of positronium}
\subsection{18-4. Rotation matrix for any spin}
\subsection{18-5. Measuring a nuclear spin}
\subsection{18-6. Composition of angular momentum}
\subsection{18-7. Added Note 1: Derivation of the rotation matrix}
\subsection{18-8. Added Note 2: Conservation of parity in photon emission}
\section{Chapter 19.The Hydrogen Atom and The Periodic Table}
\subsection{19-1. Schrödinger’s equation for the hydrogen atom}
\subsection{19-2. Spherically symmetric solutions}
\subsection{19-3. States with an angular dependence}
\subsection{19-4. The general solution for hydrogen}
\subsection{19-5. The hydrogen wave functions}
\subsection{19-6. The periodic table}
\section{Chapter 20.Operators}
\subsection{20-1. Operations and operators}
\subsection{20-2. Average energies}
\subsection{20-3. The average energy of an atom}
\subsection{20-4. The position operator}
\subsection{20-5. The momentum operator}
\subsection{20-6. Angular momentum}
\subsection{20-7. The change of averages with time}
\section{Chapter 21.The Schrödinger Equation in a Classical Context: A Seminar on Superconductivity}
\subsection{21-1. Schrödinger’s equation in a magnetic field}
\subsection{21-2. The equation of continuity for probabilities}
\subsection{21-3. Two kinds of momentum}
\subsection{21-4. The meaning of the wave function}
\subsection{21-5. Superconductivity}
\subsection{21-6. The Meissner effect}
\subsection{21-7. Flux quantization}
\subsection{21-8. The dynamics of superconductivity}
\subsection{21-9. The Josephson junction}

\section{Ressources}%
\label{sec:Ressources}

\subsection{Books}%
\label{sub:Books}

\subsection{Courses}%
\label{sub:Courses}

\subsection{Exercices}%
\label{sub:Exercices}

\end{document}
\end{article}


