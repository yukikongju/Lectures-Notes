\documentclass{article}
\usepackage{amsmath}
\usepackage{amsfonts}
\usepackage{amsthm}
\usepackage{hyperref}
\usepackage{parskip}
\usepackage{textgreek}
\begin{document}
\title{Lecture Notes for Feynman Lectures on Physics - Volume II }
\author{Emulie Chhor}
\maketitle

\section*{Introduction}

\newtheorem{definition}{Definition}[subsection]
\newtheorem{theorem}{Theorem}[subsection]
\newtheorem{corollary}{Corollary}[subsection]
\newtheorem{lemma}[theorem]{Lemma}
\newtheorem{proposition}{Proposition}[section]
\newtheorem{axiom}{Axiome}
\newtheorem{property}{Propriété}[subsection]
\newtheorem*{remark}{Remarque}
\newtheorem*{problem}{Problème}
\newtheorem*{intuition}{Intuition}

\section{Chapter 1.Electromagnetism}
\subsection{1-1. Electrical forces}
\subsection{1-2. Electric and magnetic fields}
\subsection{1-3. Characteristics of vector fields}
\subsection{1-4. The laws of electromagnetism}
\subsection{1-5. What are the fields?}
\subsection{1-6. Electromagnetism in science and technology}
\section{Chapter 2.Differential Calculus of Vector Fields}
\subsection{2-1. Understanding physics}
\subsection{2-2. Scalar and vector fields—T and h}
\subsection{2-3. Derivatives of fields—the gradient}
\subsection{2-4. The operator $\nabla$}
\subsection{2-5. Operations with $\nabla$}
\subsection{2-6. The differential equation of heat flow}
\subsection{2-7. Second derivatives of vector fields}
\subsection{2-8. Pitfalls}
\section{Chapter 3.Vector Integral Calculus}
\subsection{3-1. Vector integrals; the line integral of $\nabla \psi$}
\subsection{3-2. The flux of a vector field}
\subsection{3-3. The flux from a cube; Gauss’ theorem}
\subsection{3-4. Heat conduction; the diffusion equation}
\subsection{3-5. The circulation of a vector field}
\subsection{3-6. The circulation around a square; Stokes’ theorem}
\subsection{3-7. Curl-free and divergence-free fields}
\subsection{3-8. Summary}
\section{Chapter 4.Electrostatics}
\subsection{4-1. Statics}
\subsection{4-2. Coulomb’s law; superposition}
\subsection{4-3. Electric potential}
\subsection{4-4. $E=- \nabla \varphi$}
\subsection{4-5. The flux of E}
\subsection{4-6. Gauss’ law; the divergence of E}
\subsection{4-7. Field of a sphere of charge}
\subsection{4-8. Field lines; equipotential surfaces}
\section{Chapter 5.Application of Gauss’ Law}
\subsection{5-1. Electrostatics is Gauss’ law plus . . .}
\subsection{5-2. Equilibrium in an electrostatic field}
\subsection{5-3. Equilibrium with conductors}
\subsection{5-4. Stability of atoms}
\subsection{5-5. The field of a line charge}
\subsection{5-6. A sheet of charge; two sheets}
\subsection{5-7. A sphere of charge; a spherical shell}
\subsection{5-8. Is the field of a point charge exactly $(1/r)^2$?}
\subsection{5-9. The fields of a conductor}
\subsection{5-10. The field in a cavity of a conductor}
\section{Chapter 6.The Electric Field in Various Circumstances}
\subsection{6-1. Equations of the electrostatic potential}
\subsection{6-2. The electric dipole}
\subsection{6-3. Remarks on vector equations}
\subsection{6-4. The dipole potential as a gradient}
\subsection{6-5. The dipole approximation for an arbitrary distribution}
\subsection{6-6. The fields of charged conductors}
\subsection{6-7. The method of images}
\subsection{6-8. A point charge near a conducting plane}
\subsection{6-9. A point charge near a conducting sphere}
\subsection{6-10. Condensers; parallel plates}
\subsection{6-11. High-voltage breakdown}
\subsection{6-12. The field-emission microscope}
\section{Chapter 7.The Electric Field in Various Circumstances (Continued)}
\subsection{7-1. Methods for finding the electrostatic field}
\subsection{7-2. Two-dimensional fields; functions of the complex variable}
\subsection{7-3. Plasma oscillations}
\subsection{7-4. Colloidal particles in an electrolyte}
\subsection{7-5. The electrostatic field of a grid}
\section{Chapter 8. Electrostatic Energy}
\subsection{8-1. The electrostatic energy of charges. A uniform sphere}
\subsection{8-2. The energy of a condenser. Forces on charged conductors}
\subsection{8-3. The electrostatic energy of an ionic crystal}
\subsection{8-4. Electrostatic energy in nuclei}
\subsection{8-5. Energy in the electrostatic field}
\subsection{8-6. The energy of a point charge}
\section{Chapter 9.Electricity in the Atmosphere}
\subsection{9-1. The electric potential gradient of the atmosphere}
\subsection{9-2. Electric currents in the atmosphere}
\subsection{9-3. Origin of the atmospheric currents}
\subsection{9-4. Thunderstorms}
\subsection{9-5. The mechanism of charge separation}
\subsection{9-6. Lightning}
\section{Chapter 10.. Dielectrics}
\subsection{10-1. The dielectric constant}
\subsection{10-2. The polarization vector P}
\subsection{10-3. Polarization charges}
\subsection{10-4. The electrostatic equations with dielectrics}
\subsection{10-5. Fields and forces with dielectrics}
\section{Chapter 11.Inside Dielectrics}
\subsection{11-1. Molecular dipoles}
\subsection{11-2. Electronic polarization}
\subsection{11-3. Polar molecules; orientation polarization}
\subsection{11-4. Electric fields in cavities of a dielectric}
\subsection{11-5. The dielectric constant of liquids; the Clausius-Mossotti equation}
\subsection{11-6. Solid dielectrics}
\subsection{11-7. Ferroelectricity; BaTiO3}
\section{Chapter 12. Electrostatic Analogs}
\subsection{12-1. The same equations have the same solutions}
\subsection{12-2. The flow of heat; a point source near an infinite plane boundary}
\subsection{12-3. The stretched membrane}
\subsection{12-4. The diffusion of neutrons; a uniform spherical source in a homogeneous medium}
\subsection{12-5. Irrotational fluid flow; the flow past a sphere}
\subsection{12-6. Illumination; the uniform lighting of a plane}
\subsection{12-7. The “underlying unity” of nature}
\section{Chapter 13.Magnetostatics}
\subsection{13-1. The magnetic field}
\subsection{13-2. Electric current; the conservation of charge}
\subsection{13-3. The magnetic force on a current}
\subsection{13-4. The magnetic field of steady currents; Ampère’s law}
\subsection{13-5. The magnetic field of a straight wire and of a solenoid; atomic currents}
\subsection{13-6. The relativity of magnetic and electric fields}
\subsection{13-7. The transformation of currents and charges}
\subsection{13-8. Superposition; the right-hand rule}
\section{Chapter 14.The Magnetic Field in Various Situations}
\subsection{14-1. The vector potential}
\subsection{14-2. The vector potential of known currents}
\subsection{14-3. A straight wire}
\subsection{14-4. A long solenoid}
\subsection{14-5. The field of a small loop; the magnetic dipole}
\subsection{14-6. The vector potential of a circuit}
\subsection{14-7. The law of Biot and Savart}
\section{Chapter 15.. The Vector Potential}
\subsection{15-1. The forces on a current loop; energy of a dipole}
\subsection{15-2. Mechanical and electrical energies}
\subsection{15-3. The energy of steady currents}
\subsection{15-4. B versus A}
\subsection{15-5. The vector potential and quantum mechanics}
\subsection{15-6. What is true for statics is false for dynamics}
\section{Chapter 16.Induced Currents}
\subsection{16-1. Motors and generators}
\subsection{16-2. Transformers and inductances}
\subsection{16-3. Forces on induced currents}
\subsection{16-4. Electrical technology}
\section{Chapter 17.The Laws of Induction}
\subsection{17-1. The physics of induction}
\subsection{17-2. Exceptions to the “flux rule”}
\subsection{17-3. Particle acceleration by an induced electric field; the betatron}
\subsection{17-4. A paradox}
\subsection{17-5. Alternating-current generator}
\subsection{17-6. Mutual inductance}
\subsection{17-7. Self-inductance}
\subsection{17-8. Inductance and magnetic energy}
\section{Chapter 18.The Maxwell Equations}
\subsection{18-1. Maxwell’s equations}
\subsection{18-2. How the new term works}
\subsection{18-3. All of classical physics}
\subsection{18-4. A travelling field}
\subsection{18-5. The speed of light}
\subsection{18-6. Solving Maxwell’s equations; the potentials and the wave equation}
\section{Chapter 19.The Principle of Least Action}
\subsection{19-1. A special lecture—almost verbatim}
\subsection{19-2. A note added after the lecture}
\section{Chapter 20.Solutions of Maxwell’s Equations in Free Space}
\subsection{20-1. Waves in free space; plane waves}
\subsection{20-2. Three-dimensional waves}
\subsection{20-3. Scientific imagination}
\subsection{20-4. Spherical waves}
\section{Chapter 21.Solutions of Maxwell’s Equations with Currents and Charges}
\subsection{21-1. Light and electromagnetic waves}
\subsection{21-2. Spherical waves from a point source}
\subsection{21-3. The general solution of Maxwell’s equations}
\subsection{21-4. The fields of an oscillating dipole}
\subsection{21-5. The potentials of a moving charge; the general solution of Liénard and Wiechert}
\subsection{21-6. The potentials for a charge moving with constant velocity; the Lorentz formula}
\section{Chapter 22.AC Circuits}
\subsection{22-1. Impedances}
\subsection{22-2. Generators}
\subsection{22-3. Networks of ideal elements; Kirchhoff’s rules}
\subsection{22-4. Equivalent circuits}
\subsection{22-5. Energy}
\subsection{22-6. A ladder network}
\subsection{22-7. Filters}
\subsection{22-8. Other circuit elements}
\section{Chapter 23.Cavity Resonators}
\subsection{23-1. Real circuit elements}
\subsection{23-2. A capacitor at high frequencies}
\subsection{23-3. A resonant cavity}
\subsection{23-4. Cavity modes}
\subsection{23-5. Cavities and resonant circuits}
\section{Chapter 24.Waveguides}
\subsection{24-1. The transmission line}
\subsection{24-2. The rectangular waveguide}
\subsection{24-3. The cutoff frequency}
\subsection{24-4. The speed of the guided waves}
\subsection{24-5. Observing guided waves}
\subsection{24-6. Waveguide plumbing}
\subsection{24-7. Waveguide modes}
\subsection{24-8. Another way of looking at the guided waves}
\section{Chapter 25.Electrodynamics in Relativistic Notation}
\subsection{25-1. Four-vectors}
\subsection{25-2. The scalar product}
\subsection{25-3. The four-dimensional gradient}
\subsection{25-4. Electrodynamics in four-dimensional notation}
\subsection{25-5. The four-potential of a moving charge}
\subsection{25-6. The invariance of the equations of electrodynamics}
\section{Chapter 26.Lorentz Transformations of the Fields}
\subsection{26-1. The four-potential of a moving charge}
\subsection{26-2. The fields of a point charge with a constant velocity}
\subsection{26-3. Relativistic transformation of the fields}
\subsection{26-4. The equations of motion in relativistic notation}
\section{Chapter 27.Field Energy and Field Momentum}
\subsection{27-1. Local conservation}
\subsection{27-2. Energy conservation and electromagnetism}
\subsection{27-3. Energy density and energy flow in the electromagnetic field}
\subsection{27-4. The ambiguity of the field energy}
\subsection{27-5. Examples of energy flow}
\subsection{27-6. Field momentum}
\section{Chapter 28.Electromagnetic Mass}
\subsection{28-1. The field energy of a point charge}
\subsection{28-2. The field momentum of a moving charge}
\subsection{28-3. Electromagnetic mass}
\subsection{28-4. The force of an electron on itself}
\subsection{28-5. Attempts to modify the Maxwell theory}
\subsection{28-6. The nuclear force field}
\section{Chapter 29.The Motion of Charges in Electric and Magnetic Fields}
\subsection{29-1. Motion in a uniform electric or magnetic field}
\subsection{29-2. Momentum analysis}
\subsection{29-3. An electrostatic lens}
\subsection{29-4. A magnetic lens}
\subsection{29-5. The electron microscope}
\subsection{29-6. Accelerator guide fields}
\subsection{29-7. Alternating-gradient focusing}
\subsection{29-8. Motion in crossed electric and magnetic fields}
\section{Chapter 30.The Internal Geometry of Crystals}
\subsection{30-1. The internal geometry of crystals}
\subsection{30-2. Chemical bonds in crystals}
\subsection{30-3. The growth of crystals}
\subsection{30-4. Crystal lattices}
\subsection{30-5. Symmetries in two dimensions}
\subsection{30-6. Symmetries in three dimensions}
\subsection{30-7. The strength of metals}
\subsection{30-8. Dislocations and crystal growth}
\subsection{30-9. The Bragg-Nye crystal model}
\section{Chapter 31.Tensors}
\subsection{31-1. The tensor of polarizability}
\subsection{31-2. Transforming the tensor components}
\subsection{31-3. The energy ellipsoid}
\subsection{31-4. Other tensors; the tensor of inertia}
\subsection{31-5. The cross product}
\subsection{31-6. The tensor of stress}
\subsection{31-7. Tensors of higher rank}
\subsection{31-8. The four-tensor of electromagnetic momentum}
\section{Chapter 32.Refractive Index of Dense Materials}
\subsection{32-1. Polarization of matter}
\subsection{32-2. Maxwell’s equations in a dielectric}
\subsection{32-3. Waves in a dielectric}
\subsection{32-4. The complex index of refraction}
\subsection{32-5. The index of a mixture}
\subsection{32-6. Waves in metals}
\subsection{32-7. Low-frequency and high-frequency approximations; the skin depth and the plasma frequency}
\section{Chapter 33.Reflection from Surfaces}
\subsection{33-1. Reflection and refraction of light}
\subsection{33-2. Waves in dense materials}
\subsection{33-3. The boundary conditions}
\subsection{33-4. The reflected and transmitted waves}
\subsection{33-5. Reflection from metals}
\subsection{33-6. Total internal reflection}
\section{Chapter 34.The Magnetism of Matter}
\subsection{34-1. Diamagnetism and paramagnetism}
\subsection{34-2. Magnetic moments and angular momentum}
\subsection{34-3. The precession of atomic magnets}
\subsection{34-4. Diamagnetism}
\subsection{34-5. Larmor’s theorem}
\subsection{34-6. Classical physics gives neither diamagnetism nor paramagnetism}
\subsection{34-7. Angular momentum in quantum mechanics}
\subsection{34-8. The magnetic energy of atoms}
\section{Chapter 35.Paramagnetism and Magnetic Resonance}
\subsection{35-1. Quantized magnetic states}
\subsection{35-2. The Stern-Gerlach experiment}
\subsection{35-3. The Rabi molecular-beam method}
\subsection{35-4. The paramagnetism of bulk materials}
\subsection{35-5. Cooling by adiabatic demagnetization}
\subsection{35-6. Nuclear magnetic resonance}
\section{Chapter 36.Ferromagnetism}
\subsection{36-1. Magnetization currents}
\subsection{36-2. The field H}
\subsection{36-3. The magnetization curve}
\subsection{36-4. Iron-core inductances}
\subsection{36-5. Electromagnets}
\subsection{36-6. Spontaneous magnetization}
\section{Chapter 37.Magnetic Materials}
\subsection{37-1. Understanding ferromagnetism}
\subsection{37-2. Thermodynamic properties}
\subsection{37-3. The hysteresis curve}
\subsection{37-4. Ferromagnetic materials}
\subsection{37-5. Extraordinary magnetic materials}
\section{Chapter 38.Elasticity}
\subsection{38-1. Hooke’s law}
\subsection{38-2. Uniform strains}
\subsection{38-3. The torsion bar; shear waves}
\subsection{38-4. The bent beam}
\subsection{38-5. Buckling}
\section{Chapter 39.Elastic Materials}
\subsection{39-1. The tensor of strain}
\subsection{39-2. The tensor of elasticity}
\subsection{39-3. The motions in an elastic body}
\subsection{39-4. Nonelastic behavior}
\subsection{39-5. Calculating the elastic constants}
\section{Chapter 40.The Flow of Dry Water}
\subsection{40-1. Hydrostatics}
\subsection{40-2. The equations of motion}
\subsection{40-3. Steady flow—Bernoulli’s theorem}
\subsection{40-4. Circulation}
\subsection{40-5. Vortex lines}
\section{Chapter 41.The Flow of Wet Water}
\subsection{41-1. Viscosity}
\subsection{41-2. Viscous flow}
\subsection{41-3. The Reynolds number}
\subsection{41-4. Flow past a circular cylinder}
\subsection{41-5. The limit of zero viscosity}
\subsection{41-6. Couette flow}
\section{Chapter 42.Curved Space}
\subsection{42-1. Curved spaces with two dimensions}
\subsection{42-2. Curvature in three-dimensional space}
\subsection{42-3. Our space is curved}
\subsection{42-4. Geometry in space-time}
\subsection{42-5. Gravity and the principle of equivalence}
\subsection{42-6. The speed of clocks in a gravitational field}
\subsection{42-7. The curvature of space-time}
\subsection{42-8. Motion in curved space-time}
\subsection{42-9. Einstein’s theory of gravitation}

\section{Ressources}%
\label{sec:Ressources}


\end{document}
\end{article}


