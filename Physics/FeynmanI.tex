\documentclass{article}
\usepackage{amsmath}
\usepackage{amsfonts}
\usepackage{amsthm}
\usepackage{hyperref}
\usepackage{parskip}
\usepackage{textgreek}
\begin{document}
\title{Lecture Notes for Feynman Lectures on Physics - Volume I}
\author{Emulie Chhor}
\maketitle

\section*{Introduction}

\newtheorem{definition}{Definition}[subsection]
\newtheorem{theorem}{Theorem}[subsection]
\newtheorem{corollary}{Corollary}[subsection]
\newtheorem{lemma}[theorem]{Lemma}
\newtheorem{proposition}{Proposition}[section]
\newtheorem{axiom}{Axiome}
\newtheorem{property}{Propriété}[subsection]
\newtheorem*{remark}{Remarque}
\newtheorem*{problem}{Problème}
\newtheorem*{intuition}{Intuition}

\section{Chapter 1.Atoms in Motion}
\subsection{1-1. Introduction}
\subsection{1-2. Matter is made of atoms}
\subsection{1-3. Atomic processes}
\subsection{1-4. Chemical reactions}
\section{Chapter 2.Basic Physics}
\subsection{2-1. Introduction}
\subsection{2-2. Physics before 1920}
\subsection{2-3. Quantum physics}
\subsection{2-4. Nuclei and particles}
\section{Chapter 3.The Relation of Physics to Other Sciences}
\subsection{3-1. Introduction}
\subsection{3-2. Chemistry}
\subsection{3-3. Biology}
\subsection{3-4. Astronomy}
\subsection{3-5. Geology}
\subsection{3-6. Psychology}
\subsection{3-7. How did it get that way?}
\section{Chapter 4.Conservation of Energy}
\subsection{4-1. What is energy?}
\subsection{4-2. Gravitational potential energy}
\subsection{4-3. Kinetic energy}
\subsection{4-4. Other forms of energy}
\section{Chapter 5.Time and Distance}
\subsection{5-1. Motion}
\subsection{5-2. Time}
\subsection{5-3. Short times}
\subsection{5-4. Long times}
\subsection{5-5. Units and standards of time}
\subsection{5-6. Large distances}
\subsection{5-7. Short distances}
\section{Chapter 6.Probability}
\subsection{6-1. Chance and likelihood}
\subsection{6-2. Fluctuations}
\subsection{6-3. The random walk}
\subsection{6-4. A probability distribution}
\subsection{6-5. The uncertainty principle}
\section{Chapter 7.The Theory of Gravitation}
\subsection{7-1. Planetary motions}
\subsection{7-2. Kepler’s laws}
\subsection{7-3. Development of dynamics}
\subsection{7-4. Newton’s law of gravitation}
\subsection{7-5. Universal gravitation}
\subsection{7-6. Cavendish’s experiment}
\subsection{7-7. What is gravity?}
\subsection{7-8. Gravity and relativity}
\section{Chapter 8.Motion}
\subsection{8-1. Description of motion}
\subsection{8-2. Speed}
\subsection{8-3. Speed as a derivative}
\subsection{8-4. Distance as an integral}
\subsection{8-5. Acceleration}
\section{Chapter 9.Newton’s Laws of Dynamics}
\subsection{9-1. Momentum and force}
\subsection{9-2. Speed and velocity}
\subsection{9-3. Components of velocity, acceleration, and force}
\subsection{9-4. What is the force?}
\subsection{9-5. Meaning of the dynamical equations}
\subsection{9-6. Numerical solution of the equations}
\subsection{9-7. Planetary motions}
\section{Chapter 10.Conservation of Momentum}
\subsection{10-1. Newton’s Third Law}
\subsection{10-2. Conservation of momentum}
\subsection{10-3. Momentum is conserved!}
\subsection{10-4. Momentum and energy}
\subsection{10-5. Relativistic momentum}
\section{Chapter 11.Vectors}
\subsection{11-1. Symmetry in physics}
\subsection{11-2. Translations}
\subsection{11-3. Rotations}
\subsection{11-4. Vectors}
\subsection{11-5. Vector algebra}
\subsection{11-6. Newton’s laws in vector notation}
\subsection{11-7. Scalar product of vectors}
\section{Chapter 12.Characteristics of Force}
\subsection{12-1. What is a force?}
\subsection{12-2. Friction}
\subsection{12-3. Molecular forces}
\subsection{12-4. Fundamental forces. Fields}
\subsection{12-5. Pseudo forces}
\subsection{12-6. Nuclear forces}
\section{Chapter 13.Work and Potential Energy (A)}
\subsection{13-1. Energy of a falling body}
\subsection{13-2. Work done by gravity}
\subsection{13-3. Summation of energy}
\subsection{13-4. Gravitational field of large objects}
\section{Chapter 14.Work and Potential Energy (conclusion)}
\subsection{14-1. Work}
\subsection{14-2. Constrained motion}
\subsection{14-3. Conservative forces}
\subsection{14-4. Nonconservative forces}
\subsection{14-5. Potentials and fields}
\section{Chapter 15.The Special Theory of Relativity}
\subsection{15-1. The principle of relativity}
\subsection{15-2. The Lorentz transformation}
\subsection{15-3. The Michelson-Morley experiment}
\subsection{15-4. Transformation of time}
\subsection{15-5. The Lorentz contraction}
\subsection{15-6. Simultaneity}
\subsection{15-7. Four-vectors}
\subsection{15-8. Relativistic dynamics}
\subsection{15-9. Equivalence of mass and energy}
\section{Chapter 16.Relativistic Energy and Momentum}
\subsection{16-1. Relativity and the philosophers}
\subsection{16-2. The twin paradox}
\subsection{16-3. Transformation of velocities}
\subsection{16-4. Relativistic mass}
\subsection{16-5. Relativistic energy}
\section{Chapter 17.Space-Time}
\subsection{17-1. The geometry of space-time}
\subsection{17-2. Space-time intervals}
\subsection{17-3. Past, present, and future}
\subsection{17-4. More about four-vectors}
\subsection{17-5. Four-vector algebra}
\section{Chapter 18.Rotation in Two Dimensions}
\subsection{18-1. The center of mass}
\subsection{18-2. Rotation of a rigid body}
\subsection{18-3. Angular momentum}
\subsection{18-4. Conservation of angular momentum}
\section{Chapter 19.Center of Mass; Moment of Inertia}
\subsection{19-1. Properties of the center of mass}
\subsection{19-2. Locating the center of mass}
\subsection{19-3. Finding the moment of inertia}
\subsection{19-4. Rotational kinetic energy}
\section{Chapter 20.Rotation in space}
\subsection{20-1. Torques in three dimensions}
\subsection{20-2. The rotation equations using cross products}
\subsection{20-3. The gyroscope}
\subsection{20-4. Angular momentum of a solid body}
\section{Chapter 21.The Harmonic Oscillator}
\subsection{21-1. Linear differential equations}
\subsection{21-2. The harmonic oscillator}
\subsection{21-3. Harmonic motion and circular motion}
\subsection{21-4. Initial conditions}
\subsection{21-5. Forced oscillations}
\section{Chapter 22.Algebra}
\subsection{22-1. Addition and multiplication}
\subsection{22-2. The inverse operations}
\subsection{22-3. Abstraction and generalization}
\subsection{22-4. Approximating irrational numbers}
\subsection{22-5. Complex numbers}
\subsection{22-6. Imaginary exponents}
\section{Chapter 23.Resonance}
\subsection{23-1. Complex numbers and harmonic motion}
\subsection{23-2. The forced oscillator with damping}
\subsection{23-3. Electrical resonance}
\subsection{23-4. Resonance in nature}
\section{Chapter 24.Transients}
\subsection{24-1. The energy of an oscillator}
\subsection{24-2. Damped oscillations}
\subsection{24-3. Electrical transients}
\section{Chapter 25.Linear Systems and Review}
\subsection{25-1. Linear differential equations}
\subsection{25-2. Superposition of solutions}
\subsection{25-3. Oscillations in linear systems}
\subsection{25-4. Analogs in physics}
\subsection{25-5. Series and parallel impedances}
\section{Chapter 26.Optics: The Principle of Least Time}
\subsection{26-1. Light}
\subsection{26-2. Reflection and refraction}
\subsection{26-3. Fermat’s principle of least time}
\subsection{26-4. Applications of Fermat’s principle}
\subsection{26-5. A more precise statement of Fermat’s principle}
\subsection{26-6. How it works}
\section{Chapter 27.Geometrical Optics}
\subsection{27-1. Introduction}
\subsection{27-2. The focal length of a spherical surface}
\subsection{27-3. The focal length of a lens}
\subsection{27-4. Magnification}
\subsection{27-5. Compound lenses}
\subsection{27-6. Aberrations}
\subsection{27-7. Resolving power}
\section{Chapter 28.Electromagnetic Radiation}
\subsection{28-1. Electromagnetism}
\subsection{28-2. Radiation}
\subsection{28-3. The dipole radiator}
\subsection{28-4. Interference}
\section{Chapter 29.Interference}
\subsection{29-1. Electromagnetic waves}
\subsection{29-2. Energy of radiation}
\subsection{29-3. Sinusoidal waves}
\subsection{29-4. Two dipole radiators}
\subsection{29-5. The mathematics of interference}
\section{Chapter 30.Diffraction}
\subsection{30-1. The resultant amplitude due to n equal oscillators}
\subsection{30-2. The diffraction grating}
\subsection{30-3. Resolving power of a grating}
\subsection{30-4. The parabolic antenna}
\subsection{30-5. Colored films; crystals}
\subsection{30-6. Diffraction by opaque screens}
\subsection{30-7. The field of a plane of oscillating charges}
\section{Chapter 31.The Origin of the Refractive Index}
\subsection{31-1. The index of refraction}
\subsection{31-2. The field due to the material}
\subsection{31-3. Dispersion}
\subsection{31-4. Absorption}
\subsection{31-5. The energy carried by an electric wave}
\subsection{31-6. Diffraction of light by a screen}
\section{Chapter 32.Radiation Damping. Light Scattering}
\subsection{32-1. Radiation resistance}
\subsection{32-2. The rate of radiation of energy}
\subsection{32-3. Radiation damping}
\subsection{32-4. Independent sources}
\subsection{32-5. Scattering of light}
\section{Chapter 33.Polarization}
\subsection{33-1. The electric vector of light}
\subsection{33-2. Polarization of scattered light}
\subsection{33-3. Birefringence}
\subsection{33-4. Polarizers}
\subsection{33-5. Optical activity}
\subsection{33-6. The intensity of reflected light}
\subsection{33-7. Anomalous refraction}
\section{Chapter 34.Relativistic Effects in Radiation}
\subsection{34-1. Moving sources}
\subsection{34-2. Finding the “apparent” motion}
\subsection{34-3. Synchrotron radiation}
\subsection{34-4. Cosmic synchrotron radiation}
\subsection{34-5. Bremsstrahlung}
\subsection{34-6. The Doppler effect}
\subsection{34-7. The $\omega, \kappa$ four-vector}
\subsection{34-8. Aberration}
\subsection{34-9. The momentum of light}
\section{Chapter 35.Color Vision}
\subsection{35-1. The human eye}
\subsection{35-2. Color depends on intensity}
\subsection{35-3. Measuring the color sensation}
\subsection{35-4. The chromaticity diagram}
\subsection{35-5. The mechanism of color vision}
\subsection{35-6. Physiochemistry of color vision}
\section{Chapter 36.Mechanisms of Seeing}
\subsection{36-1. The sensation of color}
\subsection{36-2. The physiology of the eye}
\subsection{36-3. The rod cells}
\subsection{36-4. The compound (insect) eye}
\subsection{36-5. Other eyes}
\subsection{36-6. Neurology of vision}
\section{Chapter 37.Quantum Behavior}
\subsection{37-1. Atomic mechanics}
\subsection{37-2. An experiment with bullets}
\subsection{37-3. An experiment with waves}
\subsection{37-4. An experiment with electrons}
\subsection{37-5. The interference of electron waves}
\subsection{37-6. Watching the electrons}
\subsection{37-7. First principles of quantum mechanics}
\subsection{37-8. The uncertainty principle}
\section{Chapter 38.The Relation of Wave and Particle Viewpoints}
\subsection{38-1. Probability wave amplitudes}
\subsection{38-2. Measurement of position and momentum}
\subsection{38-3. Crystal diffraction}
\subsection{38-4. The size of an atom}
\subsection{38-5. Energy levels}
\subsection{38-6. Philosophical implications}
\section{Chapter 39.The Kinetic Theory of Gases}
\subsection{39-1. Properties of matter}
\subsection{39-2. The pressure of a gas}
\subsection{39-3. Compressibility of radiation}
\subsection{39-4. Temperature and kinetic energy}
\subsection{39-5. The ideal gas law}
\section{Chapter 40.The Principles of Statistical Mechanics}
\subsection{40-1. The exponential atmosphere}
\subsection{40-2. The Boltzmann law}
\subsection{40-3. Evaporation of a liquid}
\subsection{40-4. The distribution of molecular speeds}
\subsection{40-5. The specific heats of gases}
\subsection{40-6. The failure of classical physics}
\section{Chapter 41.The Brownian Movement}
\subsection{41-1. Equipartition of energy}
\subsection{41-2. Thermal equilibrium of radiation}
\subsection{41-3. Equipartition and the quantum oscillator}
\subsection{41-4. The random walk}
\section{Chapter 42.Applications of Kinetic Theory}
\subsection{42-1. Evaporation}
\subsection{42-2. Thermionic emission}
\subsection{42-3. Thermal ionization}
\subsection{42-4. Chemical kinetics}
\subsection{42-5. Einstein’s laws of radiation}
\section{Chapter 43.Diffusion}
\subsection{43-1. Collisions between molecules}
\subsection{43-2. The mean free path}
\subsection{43-3. The drift speed}
\subsection{43-4. Ionic conductivity}
\subsection{43-5. Molecular diffusion}
\subsection{43-6. Thermal conductivity}
\section{Chapter 44.The Laws of Thermodynamics}
\subsection{44-1. Heat engines; the first law}
\subsection{44-2. The second law}
\subsection{44-3. Reversible engines}
\subsection{44-4. The efficiency of an ideal engine}
\subsection{44-5. The thermodynamic temperature}
\subsection{44-6. Entropy}
\section{Chapter 45.Illustrations of Thermodynamics}
\subsection{45-1. Internal energy}
\subsection{45-2. Applications}
\subsection{45-3. The Clausius-Clapeyron equation}
\section{Chapter 46.. Ratchet and pawl}
\subsection{46-1. How a ratchet works}
\subsection{46-2. The ratchet as an engine}
\subsection{46-3. Reversibility in mechanics}
\subsection{46-4. Irreversibility}
\subsection{46-5. Order and entropy}
\section{Chapter 47.Sound. The wave equation}
\subsection{47-1. Waves}
\subsection{47-2. The propagation of sound}
\subsection{47-3. The wave equation}
\subsection{47-4. Solutions of the wave equation}
\subsection{47-5. The speed of sound}
\section{hapter 48.Beats}
\subsection{48-1. Adding two waves}
\subsection{48-2. Beat notes and modulation}
\subsection{48-3. Side bands}
\subsection{48-4. Localized wave trains}
\subsection{48-5. Probability amplitudes for particles}
\subsection{48-6. Waves in three dimensions}
\subsection{48-7. Normal modes}
\section{Chapter 49.Modes}
\subsection{49-1. The reflection of waves}
\subsection{49-2. Confined waves, with natural frequencies}
\subsection{49-3. Modes in two dimensions}
\subsection{49-4. Coupled pendulums}
\subsection{49-5. Linear systems}
\section{Chapter 50.Harmonics}
\subsection{50-1. Musical tones}
\subsection{50-2. The Fourier series}
\subsection{50-3. Quality and consonance}
\subsection{50-4. The Fourier coefficients}
\subsection{50-5. The energy theorem}
\subsection{50-6. Nonlinear responses}
\section{Chapter 51.Waves}
\subsection{51-1. Bow waves}
\subsection{51-2. Shock waves}
\subsection{51-3. Waves in solids}
\subsection{51-4. Surface waves}
\section{Chapter 52.Symmetry in Physical Laws}
\subsection{52-1. Symmetry operations}
\subsection{52-2. Symmetry in space and time}
\subsection{52-3. Symmetry and conservation laws}
\subsection{52-4. Mirror reflections}
\subsection{52-5. Polar and axial vectors}
\subsection{52-6. Which hand is right?}
\subsection{52-7. Parity is not conserved!}
\subsection{52-8. Antimatter}
\subsection{52-9. Broken symmetries}

\section{Ressources}%
\label{sec:Ressources}

\subsection{Books}%
\label{sub:Books}

\subsection{Courses}%
\label{sub:Courses}

\subsection{Exercices}%
\label{sub:Exercices}

\end{document}
\end{article}

