\documentclass{article}
\usepackage{amsmath}
\usepackage{amsfonts}
\usepackage{amsthm}
\usepackage{parskip}
\usepackage{textgreek}
\begin{document}
\title{Lectures Notes for Analysis}
\author{Emulie Chhor}
\maketitle

\section*{Introduction}

Le premier cours d'analyse porte sur la même matière que le calcul différentiel
et intégral, mais est plus rigoureux. La majorité du cours met l'emphase sur la
démonstration des preuves et théorèmes et très peu sur le calcul.\\

MAT1000 porte sur les chapitres suivants:\\

    \begin{enumerate}
	\item Nombres Rationnels et Nombres Réels
	\item Inégalités et Valeur Absolue
	\item Suprémum et Infimum
	\item Axiomes de Complétude
	\item Dénombrabilité
	\item Suites
	\item Continuité et Continuité Uniforme
	\item Dérivabilité
	\item Séries
    \end{enumerate}

\newtheorem{definition}{Definition}[subsection]
\newtheorem{theorem}{Theorem}[subsection]
\newtheorem{corollary}{Corollary}[subsection]
\newtheorem{lemma}[theorem]{Lemma}
\newtheorem{proposition}{Proposition}[section]
\newtheorem{axiom}{Axiome}
\newtheorem{property}{Propriété}[subsection]
\newtheorem*{remark}{Remarque}
\newtheorem*{problem}{Problème}
\newtheorem*{intuition}{Intuition}

\newcommand{\euler}{e}
\newcommand{\ramuno}{i}
\pagebreak

\section{Nombres Rationnels et Nombres Réels}

\subsection{Overview}

Malgré que le cours ne met pas l'emphase sur la topologie des réels, il est
important de comprendre que les réels possèdent certaines propriétés qui nous
permettent de manipuler les termes algébriquement. On définit la notion de
corps ordonné qui est TODO

\subsection{Topologie des Réels}

\begin{proposition}
    $\forall a \in \mathbb{Q}, a^2 \neq 2$
\end{proposition}

\begin{lemma}
    Soit $b \in \mathbb{Z}$. Si $b^2$ est pair, alors b est pair
\end{lemma}

\begin{axiom}
    Il existe un corps totalement ordonné de $\mathbb{R}$ tel que:\\
    \begin{enumerate}
	\item $\mathbb{Q} \subseteq \mathbb{R} \text{ et } \mathbb{Q}$ est un sous-corps
	    de $\mathbb{R}$
	\item $\mathbb{R}$ possède la propriété de Complétude
    \end{enumerate}
\end{axiom}

\begin{proposition}
    Soit $a,b \in \mathbb{Q}$ avec a<b. Alors il existe $c \in \mathbb{Q}$ tel que
    $a<c<b$
\end{proposition}

\begin{theorem}[Propriétés d'un corps]
    Un corps possède les propriétés suivantes:\\
    \begin{enumerate}
	\item Commutativité
	\item Associativité
	\item Distributivité
	\item Existence d'éléments neutres
	\item Existence de l'inverse additif et multiplicatif
	\item Lien entre les relations d'ordre et opérations
    \end{enumerate}
\end{theorem}

\begin{problem}[Montrer qu'un nombre est irrationnel]
\end{problem}

\pagebreak

\section{Inégalités et Valeur Absolue}

\subsection{Overview}

TODO

\subsection{Définition de la Valeur Absolue}

\begin{definition}[Valeur Absolue]
\end{definition}

\subsection{Propriétés des Valeurs Absolues}

\begin{remark}
    On peut interpréter la valeur absolue comme la distance entre 2 termes
\end{remark}

\subsection{Propriétés des Inégalités}

\begin{problem}[Montrer l'inégalité]
    Lorsqu'on veut montrer une inégalité, on utilise les astuces suivantes:\\

    \begin{enumerate}
	\item Ajouter et Enlever le même terme
	\item $x^2 > 0$
	\item Utiliser la transitivité pour comparer avec un autre terme
    \end{enumerate}

\end{problem}

\begin{problem}[Résoudre des inégalités]
    Lorsqu'on nous demande de résoudre des inégalités, on doit trouver les
    valeurs qui satisfont l'inégalité. Pour ce faire, il est pratique d'utiliser
    un tableau pour tester les intervalles de cohérence.
\end{problem}

\subsection{Induction}

% \begin{intuition}[Induction]
    % L'induction est une technique de preuve qui se fait en 3 étapes:
    % \begin{enumerate}
	% \item Cas de base
	% \item Hypothèse d'induction: on suppose que l'étape n est vraie
	% \item Étape d'induction: on montre que l'hypothèse est vraie en utilisant l'hypothèse d'induction
    % \end{enumerate}
% \end{intuition}

\subsection{Inégalité de Bernouilli}

\begin{theorem}[Inégalité de Bernouilli]
    $ \forall n \in \mathbb{N}, \forall t > -1, (1+t)^n \geq 1+nt$
\end{theorem}

\pagebreak

\section{Suprémum et Infimum}

\subsection{Overview}

\subsection{Majorant et Minorant}

\begin{definition}[Majorant]
    Soit $E \subseteq \mathbb{R}$. E est majoré si $\exists c \in \mathbb{R}, \forall
    x \in E, x \leq c$. On dit que c est un majorant de E
\end{definition}

\begin{definition}[Minorant]
    Soit $E \subseteq \mathbb{R}$. E est minoré si $\exists c \in \mathbb{R}, \forall
    x \in E, x \geq c$. On dit que c en un minorant de E
\end{definition}

\begin{definition}[Ensemble Borné]
    Soit $E \subseteq \mathbb{R}$. E est borné si E est majoré et minoré
\end{definition}

\begin{proposition}
    E est borné $ \to \exists c \in \mathbb{R}, \forall x \in E, |x|<c $
\end{proposition}

\begin{remark}
    Un majorant et un minorant sont des valeurs que l'ensemble de peut jamais
    atteindre. Il peut en exister plusieurs
\end{remark}

\subsection{Suprémum et Infimum}

\begin{definition}[Supremum]
    Soit $E \subseteq \mathbb{R}$. On dit que $c \in \mathbb{R}$ est le supremum
    de E si c'est le plus petit majorant de E
\end{definition}

\begin{definition}[Infimum]
    Soit $E \subseteq \mathbb{R}$. On dit que $c \in \mathbb{R}$ est l'infimum
    de E si c'est le plus grand minorant de E
\end{definition}

\begin{proposition}
    Soit $ A \subseteq \mathbb{R}, B \subseteq \mathbb{R}$ tel que Sup A et Sup B
    existent. Alors, $sup(A+B)= sup A + sup B$
\end{proposition}

\begin{proposition}
    Le suprémum et l'infimum sont uniques.
\end{proposition}

\begin{problem}[Montrer que la suite possède un suprémum/infimum]
    Pour montrer que la suite possède un supremum/infimum, il faut montrer que:\\
    \begin{enumerate}
	\item c est un majorant/minorant
	\item c est le plus petit majorant/plus grand minorant
    \end{enumerate}

    Pour montrer la première étape, il suffit d'utiliser la définition. Pour
    montrer la deuxième étape, on procède par contradiction. On suppose qu'il
    existe un plus petit majorant/plus grand minorant et on montre que cela
    contredit l'hypothèse que S' était un majorant/minorant

\end{problem}

\begin{problem}[Montrer que la suite ne possède pas un suprémum/infimum]
    Une suite de possède pas de suprémum ou d'infimum si elle n'est pas bornée
    (l'ensemble est infini). On peut supposer qu'elle est bornée et utiliser
    l'axiome de complétude pour montrer qu'elle est majorée/minorée, puis on
    montre qu'il s'agit d'une contradiction avec la propriété Archimédienne.
\end{problem}

\begin{problem}[Montrer que le suprémum/infimum est atteint]
\end{problem}

\pagebreak
\section{Axiomes de Complétude}
\subsection{Overview}

\subsection{Propriété Archimédienne}

\begin{theorem}[Propriété Archimédienne]
    Pour tout $x,y \in \mathbb{R} \text{ avec } x>0$, il existe $ n \in \mathbb{N}
    \text{ tel que } nx>y$
\end{theorem}

\begin{remark}
    La propriété Archimédienne nous dit que les réels ne sont pas bornés. Ainsi,
    si on prend un nombre arbitraire, on peut en trouver un autre qui est plus
    petit ou plus grand.
\end{remark}

\subsection{Axiome de Complétude}

\begin{proposition}[Axiome de Complétude]
    Si $E \in \mathbb{R}$ est non-vide et majorée, alors sup E existe dans
    $\mathbb{R}$
\end{proposition}

\begin{corollary}[Corollaire de l'Axiome de Complétude]
    \begin{enumerate}
	\item $ \forall y \in \mathbb{R}, \exists n \in \mathbb{N}, n>y$
	\item $ \forall x \in \mathbb{R}, \exists! n \in \mathbb{R}$ tel que
	    $n \leq x \neq n+1$ (fonction plancher)
	\item (Densité de $\mathbb{Q}$ dans $\mathbb{Q}$): Soit $x,y \in \mathbb{R}$,
	    alors $\exists r \in \mathbb{Q}, x<r<y$
	\item (Densité de $\mathbb{R} \ \mathbb{Q}$ dans $\mathbb{Q}$): Soit $x,y \in \mathbb{R}$,
	    alors $\exists r \in \mathbb{R} \ \mathbb{Q}, x<r<y$
    \end{enumerate}
\end{corollary}

\begin{theorem}
    L'axiome de complétude est faux dans $\mathbb{Q}$
\end{theorem}

\subsection{Densité des Rationnels}

\begin{theorem}[Densité de $\mathbb{Q}$ dans $\mathbb{R}$]
    Soit $x,y \in \mathbb{R}$, alors $\exists r \in \mathbb{Q}, x<r<y$
\end{theorem}

\begin{theorem}[Densité de $\mathbb{Q}'$ dans $\mathbb{R}$]
    Soit $x,y \in \mathbb{R}$, alors $\exists r \in \mathbb{Q} ', x<r<y$
\end{theorem}

\begin{remark}
    Le densité des rationnels et des irrationnels découle de la propriété
    d'Archimède. Puisque les réels ne sont pas bornés, alors il existe une
    infinité de nombre dans un intervalle, alors il existe nécessairement un
    rationnel et un irrationnel dans cet intervalle.
\end{remark}

\begin{intuition}
    La preuve de la densité se fait en construisant un rationnel à partir de la
    distance entre x et y
\end{intuition}

\subsection{Racine n-ième}

\begin{definition}[Definition Racine n-ième]
    % Soit $q \in \mathbb{Q}, m \in \mathbb{Z}, n \in \mathbb{N} \text{ tel que }
    % q= \fract{m}{n}$. Pour $x \geq 0$, on pose:
    \begin{enumerate}
	% \item $x^q := ^n\sqrt(x^m)$ si $q \geq 0$
	% \item $x^q := (\fract{1}{x}) ^ (-q)$ si $q<0, x \neq 0$
    \end{enumerate}
\end{definition}

\begin{theorem}[Racine n-ième]
    $ \forall n \in \mathbb{N} , \forall x \geq 0, \exists! y \geq 0, y^n =x \,
    (^n\sqrt{x} = y)$
\end{theorem}

\pagebreak

\section{Dénombrabilité}
\subsection{Overview}

\subsection{Injection, Surjection, Bijection}

\begin{definition}[Injection]
\end{definition}

\begin{definition}[Surjection]
\end{definition}

\begin{theorem}[Bijection]
\end{theorem}

\subsection{Cardinalité}

\begin{definition}[Cardinalité]
    Deux ensembles A et B ont le même cardinal s'il existe une bijection $\Phi:A \to B$
\end{definition}

\begin{corollary}
    Si $\Phi:A \to B$ est une bijection, alors $\Phi ^(-1): B \to A$ existe et est
    bijective
\end{corollary}

\begin{definition}[Dénombrabilité]
    Un ensemble A est dénombrable s'il a le même cardinal que $\mathbb{N}$
\end{definition}

\begin{remark}
\item $\mathbb{N}$ est dénombrable
\item Un ensemble dénombrable est infini
\item Un ensemble est dénombrable $\Longleftrightarrow$ on peut former une suite
    infinie $a_1, a_2, ...$ contenant une et une seule fois chaque élément de A
\end{remark}

\begin{theorem}
    Soit $ f:A \to B$, une fonction:\\
    \begin{enumerate}
	\item si f est surjective, alors $|A| \geq |B|$
	\item si f est injective, alors $|A| \leq |B|$
	\item si f est bijective, alors $|A| = |B|$
    \end{enumerate}
\end{theorem}

\begin{proposition}
    \begin{enumerate}
	\item $\mathbb{Z}$ est dénombrable
    	\item $\mathbb{Q}$ est dénombrable
    	\item $\mathbb{R}$ n'est pas dénombrable
    \end{enumerate}
\end{proposition}

\begin{remark}[Diagonale de Cantor]
\end{remark}

\pagebreak
\section{Suites}

\subsection{Overview}

Dans cette section, on essaie de déterminer si une suite converge et si oui,
on veut calculer sa limite

\subsection{Convergence d'une suite}


\begin{definition}[Suite Borné]
    La suite $(a_n)$,\\
    \begin{enumerate}
	\item majorée si $\exists M \in \mathbb{R}, \forall n \in \mathbb{N},
	    (a_n) \leq M$
	\item minorée si $\exists M \in \mathbb{R}, \forall n \in \mathbb{N},
	    (a_n) \geq M$
	\item bornée si $\exists M \in \mathbb{R}, \forall n \in \mathbb{N},
	    |(a_n)| \leq M$
    \end{enumerate}
\end{definition}

\begin{definition}[Croissance et Décroissance d'une Suite]
    La suite $(a_n)$ est\\
    \begin{enumerate}
	\item strictement croissante si $a_(n+1) > a_n, \forall n \in \mathbb{N}$
	\item croissante si $a_(n+1) \geq a_n, \forall n \in \mathbb{N}$
	\item strictement décroissante si $a_(n+1) < a_n, \forall n \in \mathbb{N}$
	\item décroissante si $a_(n+1) \leq a_n, \forall n \in \mathbb{N}$
    \end{enumerate}
\end{definition}

\begin{definition}[Suite Monotone]
    La suite est monotone si la suite est croissante ou décroissante
\end{definition}

\begin{definition}[Convergence d'une Suite]
    Une suite $(a_n)$ est convergente s'il existe un nombre $L \in \mathbb{R}$
    tel que pour $\varepsilon > 0$, il existe $N \in \mathbb{N}$ pour lequel
    $|a_n - L| < \varepsilon$ lorsque $n \geq \mathbb{N}$.\\
    On dit que L est la limite de $(a_n)$, et on écrit
    $$ \lim_{n \to \infty} a_n = L$$
\end{definition}

\begin{definition}[Divergence d'une suite]
    Si $(a_n)$ n'est pas convergente, on dit qu'elle est divergente
    $$ \lim_(n \to \infty) a_n \neq L \Longleftrightarrow \exists \varepsilon 0,
    \forall N \in \mathbb{N}, \exists n \geq N, |a_n - L | \geq \varepsilon$$
\end{definition}

\begin{problem}[Montrer que la suite converge]
    Pour montrer que la suite converge, il faut trouver un N qui satisfait
    l'inégalité $|a_n - L| < \varepsilon$. Pour ce faire, il est pratique de
    partir avec $|a_n - L|$ et de poser N en fonction de $\varepsilon$.\\
    Comme en différentielle, on peut évaluer la limite d'une suite en calculant
    sa limite lorsqu'elle tend vers l'infini
\end{problem}

\begin{intuition}
    On s'imaginer que les valeurs de la suite se trouvent à l'intérieur d'un
    tube $(L-\varepsilon, L+\varepsilon)$ à partir d'un certain N.
\end{intuition}

\begin{theorem}[La limite est unique]
    Si $(a_n)$ est convergente, alors la limite est unique
\end{theorem}

\begin{theorem}[La limite est bornée]
Si $(a_n)$ est convergente, alors $a_n | n \in \mathbb{N} $ est bornée
\end{theorem}

\begin{corollary}
    Si $(a_n)$ n'est pas bornée, alors elle diverge
\end{corollary}

\begin{remark}
    Si $(a_n)$ converge, alors elle est bornée, mais le contraire n'est pas vrai
\end{remark}

\begin{theorem}[Théorème des suites monotones]
\item Si $(a_n)$ est croissante et majorée , alors $\lim_{n \to \infty} a_n =
    sup {a_n | n \in \mathbb{N} =: L}$. On écrit $a_n \nearrow L$
\item Si $(a_n)$ est décroissante et minorée, alors $\lim_{n \to \infty} a_n =
    inf{a_n | n \in \mathbb{N} =: L}$. On écrit $a_n \searrow L$
\end{theorem}

\begin{remark}
    On peut tracer le graphique pour voir que la limite tend vers sup/inf.
    De plus, le premier terme de la suite correspond à l'inf/sup respectivement
\end{remark}

\begin{theorem}
    Pour tout $M \in \mathbb{N}$, on a $\lim_{n \to \infty} a_n =
    \lim_{n \to \infty} a_(n+M-1)$
\end{theorem}

\subsection{Limites suppérieures et Limites Inférieures}

\begin{definition}
    Soit $(a_n)$, une suite
    \begin{enumerate}
	\item $ \sup{m \geq n} a_m := \sup{a_m | m \in \mathbb{N}, m \geq n}$
	\item $ \inf{m \geq n} a_m := \inf{a_m | m \in \mathbb{N}, m \geq n}$
    \end{enumerate}
\end{definition}

\begin{proposition}
    Soit $(a_n)$, une suite bornée. On pose $b_n := \sup_{m \geq n} a_m$ et
    $c_n := \inf_{m \geq n} a_m$, alors\\
    \begin{enumerate}
	\item $(b_n)$ est décroissante et convergente et
	    $ \lim_{n \to \infty} b_n = \inf_{n \geq 1} \sup_{m \geq n} a_m$
	\item $(b_n)$ est croissante et convergente et
	    $ \lim_{n \to \infty} c_n = \sup{n \geq 1} \inf{m \geq n} a_m$
    \end{enumerate}
\end{proposition}

\subsection{Limites Infinies}

\begin{definition}[Limite à l'infinie]
    On dit qu'une suite $(a_n)$ tend (diverge) vers l'infinie si
    $\forall M \mathbb{R}, \exists N \in \mathbb{N}, \forall n \geq N, a_n \geq M$
\end{definition}

\begin{remark}
    Si $a_n$ tend vers $-\infty$, alors $a_n \leq M$
\end{remark}

\subsection{Opérations Élémentaire des Limites}

\begin{theorem}
    Supposons que $\lim_{n \to \infty} a_n$ et $\lim_{n \to \infty} b_n =b,
    a,b, \in \mathbb{R}$, alors\\
    \begin{enumerate}
	\item $\lim_{n \to \infty} |a_n| = |a|$
	\item $\lim_{n \to \infty} (a_n + b_n) = a+b$
	\item $\lim_{n \to \infty} (a_n \cdot b_n) = ab$
	\item Si $b_n \neq 0, b \neq 0, \forall n > N$, alors
	    $\lim_{n \to \infty} \frac{a_n}{b_n} = \frac{a}{b}$
	\item Si $a_n \leq b_n \forall n \geq N$, alors $a \leq b$
	\item Si $a_n \geq 0$, alors $\lim_{n \to \infty} ^p\sqrt(a_n) = ^p\sqrt(a)
	    \forall p \in \mathbb{N}$
    \end{enumerate}
\end{theorem}

\begin{corollary}
    \begin{enumerate}
	\item $\lim_{n \to \infty} (k a_n) =ka , \forall k \in \mathbb{R}$
	\item $\lim_{n \to \infty} (a_n - b_n) = a-b$
    \end{enumerate}
\end{corollary}

\subsection{Critère de Comparaison}

\begin{theorem}[Théorème des 2 Gendarmes]
    Soit $(a_n), (b_n), (c_n)$, des suites tel que $\lim_{n \to \infty} a_n =
    \lim_{n \to \infty} c_n = L$ et $a_n \leq b_n \leq c_n$, alors
    $\lim_{n \to \infty} b_n = L$
\end{theorem}

\begin{corollary}
    Si $|a_n| \rightarrow 0$, alors $a_n \rightarrow 0$
\end{corollary}

\begin{theorem}
    Si $a_n \rightarrow \infty$ et $ b_n \geq a_n$, alors $b_n \rightarrow \infty,
    \forall n \geq N$
\end{theorem}

\subsection{Progression Géométrique}

\begin{definition}[Progression Géométrique]
    Une progession géométrique de raison $q \neq 1$ est une suite de la forme
    $(a, aq, aq^2, ...) = (aq^(n-1))_{n \in \mathbb{N}}$
\end{definition}

\begin{theorem}
    \begin{enumerate}
	\item Si $|q| >1$, alors $|q|^n \rightarrow \infty$
	\item Si $|q| <1$, alors $|q|^n \rightarrow 0$
    \end{enumerate}
\end{theorem}

\begin{theorem}
    Soit $q \neq 1$. On pose $S_n = 1 + q + q^2 + ... + q^(n-1)$\\
    \begin{enumerate}
	\item $S_n = \frac{1 - q^n}{1-q}$
	\item $|q|<1$, alors $S_n \rightarrow \frac{1}{1-q}$
	\item $|q|>1$, alors $S_n$ diverge et $|S_n| \rightarrow \infty$
    \end{enumerate}
\end{theorem}

\subsection{Racine n-ième}

\begin{theorem}
    \begin{enumerate}
	\item $n^(1/n) \rightarrow 1 \, (n \rightarrow \infty)$
	\item Si $a>0$, alors $a^(\frac{1}{n}) \rightarrow 1 $
    \end{enumerate}
\end{theorem}

\subsection{Sous-suites}

\begin{definition}[Sous-suites]
\end{definition}

\begin{remark}
    Lorsque $k \rightarrow \infty$, alors $n_k \rightarrow \infty$, car
    $n_{k-1} < n_k$
\end{remark}

\begin{theorem}
    $a_n \rightarrow L \Longleftrightarrow$ toutes sous-suites de $(a_n)$
    convergent vers L
\end{theorem}

\begin{corollary}
    \begin{enumerate}
	\item Si une sous-suite diverge, alors $(a_n)$ diverge
	% \item Si $a_{n_k} \rightarrow L_{1}$ et $a_{m_k} \rightarrow L_{2} \enq L_{1}$,
	    % alors $(a_n)$ diverge
	\item Si $(a_n)$ converge et $(a_{n_k}) \rightarrow L$, alors $a_n
	    \rightarrow L$
	\item Si $(a_n)$ est monotone et $(a_{n_k}) \rightarrow L$, alors $a_n
	    \rightarrow L$
    \end{enumerate}
\end{corollary}

\begin{theorem}[Théorème de Bolzano-Weistrass]
    Soit $(a_n)$, une suite. Si $(a_n)$ est bornée, alros $(a_n)$ possède
    une sous-suite convergente
\end{theorem}

\begin{remark}[Méthode de Dichotomie]
    La preuve du théorème de Bolzano se fait avec la méthode de la dichotomie.
    On prend l'intervalle fermé et on le sépare en 2 jusqu'à temps d'avoir un
    intervalle tellement petit qu'on peut le considérer comme un point, et
    on dit qu'un point converge vers la même valeur.
\end{remark}

\subsection{Suites de Cauchy}

\begin{definition}[Suite de Cauchy]
    On dit que $(a_n)$ est une suite de Cauchy si
    $$ \forall \varepsilon > 0, \exists N \in \mathbb{N}, \forall n,m \geq N,
    |a_{n+k} - a_n| < \varepsilon$$
\end{definition}

\begin{remark}[Définition alternative de la suite de Cauchy]
    $$ \forall \varepsilon > 0, \exists N \in \mathbb{N}, \forall n,m \geq N,
    |a_m - a_n| < \varepsilon$$
\end{remark}

\begin{theorem}
    $(a_n)$ est une suite de Cauchy $\Longleftrightarrow (a_n)$ est une suite
    convergente
\end{theorem}

\begin{intuition}
    Intuitivement, la suite de Cauchy mesure la distance entre 2 termes
    consécutifs. Pour que la suite converge, on veut que la distance deviennent
    de plus en plus petite et converge vers 0.
\end{intuition}

\subsection{Suites définies par récurrence}

\begin{proposition}[Nombre d'Euler]
    % \begin{enumerate}
	% \item On pose $S_n = \frac{1}{0!} + \frac{1}{1!} + ... + \frac{1}{n!}$
	    % alors $(S_n)$ converge et on pose $e := \lim_{n \to \infty} S_n$
	% \item On pose $t_n = (1+\frac{1}{n})^n$, alors $(t_n)$ est strictement
	    % croissante s et \lim_{n \to \infty} t_n = e$
    % \end{enumerate}
\end{proposition}

\begin{problem}[Calculer la limite d'une suite définie par récurrence]
\end{problem}

\pagebreak

\section{Continuité et Continuité Uniforme}

\subsection{Overview}

\subsection{Definition de la Limite d'une fonction}

\begin{definition}[Domaine et Image]
\end{definition}

\begin{definition}[Intervalle Ouvert et Fermé]
\end{definition}

\begin{definition}[Limite d'une fonction]
    Soit $f:D \rightarrow \mathbb{R}$ et $a \in \bar{D}$. On dit que
    f(x) tend vers L lorsque x tend vers a si
    $$ \forall \varepsilon > 0 , \exists \delta > 0, \forall x \in D,
    0 < |x-a|< \delta \Longrightarrow |f(x) -L| < \varepsilon$$
    On écrit $lim_{x \to a} f(x) = L$
\end{definition}

\begin{remark}[Interprétation Graphique]
    La définition de la limite d'une fonction nous dit que pour nimporte
    quel epsilon, on peut trouver delta à l'intérieur de l'intervalle
    $(L-\varepsilon, L+\varepsilon)$
\end{remark}

\begin{remark}
    Si $f(x) \not\to L, \exists \varepsilon > 0, \forall \delta > 0,
    \exists x \in D, 0 < |x-a|< \delta \Longrightarrow |f(x)-L| \geq
    \varepsilon$
\end{remark}

\begin{problem}[Montrer que la fonction converge]
    Pour montrer que la fonction converge, on doit reverse engineer et
    trouver $\delta (\varepsilon)$. On part avec $|f(x)-L|$ et on essait
    d'isoler $|x-a|$ pour pouvoir poser $\delta$ en fonction de epsilon.
\end{problem}

\begin{problem}[Montrer que la fonction ne converge pas]
\end{problem}

\begin{theorem}
    Soit $f:D \rightarrow \mathbb{R}$ et $a \in \bar{D}$. Les énoncés
    suivants sont équivalents:
    \begin{enumerate}
	\item $lim_{x \to \infty} f(x)=L$
	\item Pour toute suite $(x_n) \subseteq D \ {a}$ tel que
	    $x_n \to a$, on a $lim_{x \to \infty} f(x_n)=L$
    \end{enumerate}
\end{theorem}

\begin{intuition}
    Essentiellement, le théorème ci-haut nous dit que la définition des
    limites est équivalente à la définition de la suite. Intuitivement,
    Si la suite $x_n$ converge, alors on peut travailler à l'intérieur
    de cet intervalle et trouver la fonction qui converge aussi (on
    fixe $|x_n -a|<\varepsilon$)
\end{intuition}

\begin{definition}[Limite à gauche et à droite]
    $$ \forall \varepsilon > 0, \exists \delta > 0, \forall x \in D,
    0 < x-a < \delta \Longrightarrow |f(x)-L|< \varepsilon (resp
    -\delta < x-a <0)$$
\end{definition}

\begin{definition}[Limite infinie]
    $$ \forall M > 0, \exists \delta > 0, \forall x \in D,
    0 < |x-a| < \delta \Longrightarrow f(x)-L > M (resp <M)$$
\end{definition}

\begin{definition}[Limite à l'infini]
    $$ lim_{x \to \infty} f(x)=L \Longleftrightarrow \forall \varepsilon >0,
    \exists M \in \mathbb{R}, \forall x \in D, x >M \Longrightarrow
    |f(x) -L|< \varepsilon (resp <M)$$
\end{definition}

\subsection{Propriétés des limites}

\begin{theorem}[Propriétés des limites de fonctions continues]
    Soit $f,g:D \to \mathbb{R}$. Si $f(x) \to L$ et $g(x) \to M$, alors
    \begin{enumerate}
	\item $ |f(x)| \to L$
	\item $ f(x) + g(x) \to L+M$
	\item $ f(x) g(x) \to LM$
	\item $ \frac{f(x)}{g(x)} \to \frac{L}{M}$, si $M \neq 0$
	\item $ f(x) \leq g(x)$, alors $L \leq M$
	\item $ \forall n \in \mathbb{N}, ^n\sqrt(L)$ si $f(x) \geq
	    L, \forall x \in D$
    \end{enumerate}
\end{theorem}

\begin{corollary}
    Soient P et Q, des polynomes et $Q(a) \neq 0$. Alors
    $ lim_{x \to a} \frac{P(a)}{Q(a)}$
\end{corollary}

\subsection{Definition de la Limite d'une fonction}

\begin{definition}[Fonction Continue en un point]
    f est continue en $a \in D$ si
    $$ \forall \varepsilon > 0 , \exists \delta > 0, \forall x \in D,
    0 < |x-a|< \delta \Longrightarrow |f(x) -L| < \varepsilon$$
\end{definition}

\begin{definition}[Fonction Discontinue en un point]
    f est discontinue en $a \in D$ si elle n'est pas continue en a
\end{definition}

\begin{definition}[Fonction Continue]
    f est continue sur D si $\forall a \in D$, f est continue en a.
    On dit que f est continue
\end{definition}

\begin{proposition}
    Les énoncés suivants sont équivalents:
    \begin{enumerate}
	\item f est continue en a
	\item $ lim_{x \to a} f(x) = f(a)$
	\item $\forall (x_n) \subseteq D, x_n \to a \Longrightarrow
	    f(x_n) \to f(a)$
    \end{enumerate}
\end{proposition}

\begin{intuition}
    La proposition nous dit que la définition de la continuité implique
    la définition de la limite et vice-versa
\end{intuition}

\subsection{Propriétés de fonctions continues}

\begin{proposition}
    Soit $f,g:D \to \mathbb{R}$ sont continues, $a \in D$, alors $|f|,
    f+g, fg, f/g (si g(a) \neq 0), ^n\sqrt(f) (si f \geq 0)$ sont
    continues en a
\end{proposition}

\begin{proposition}
    Les fonctions transcendantes suivantes sont continues en chaque point
    de leur domaine: $e^x, log x, sin(x), cos(x), tan(x), arcsin(x),
    arccos(x), arctan(x)$
\end{proposition}

\begin{proposition}
    La composition de fonctions continues est aussi continues
\end{proposition}

\begin{proposition}
    \begin{enumerate}
	\item $lim_{x \to -\infty} e^x = 0$
	\item $lim_{x \to \infty} e^x = \infty$
	\item $lim_{x \to 0^(+)} log(x) = -\infty$
	\item $lim_{x \to \infty} log(x) = \infty$
    \end{enumerate}
\end{proposition}


\subsection{Théorème des Valeurs Intermédiaires}

\begin{theorem}[Théorème des Valeurs Intermédiaires]
    Soit $f:[a,b] \to \mathbb{R}$ continues sur [a,b] et tel que
    $f(a)<f(b)$. Alors, $\forall c \text{ tel que } f(a)<c<f(b)$,
    il existe $x \in (a,b)$ tel que f(x)=c
\end{theorem}

\begin{intuition}
    Le TVI nous dit que si la fonction est continue dans un intervalle,
    alors on peut atteindre toutes les images à l'intérieur de l'intervalle
    donné par f(a) et f(b)
\end{intuition}

\begin{remark}[Algorithme de Bissection]
    La preuve du TVI se fait avec l'algorithme de bissection, qui marche
    de façon similaire à l'algorithme de dichotomie pour le théorème
    de Bolzano-Weistrass.
\end{remark}

\begin{problem}[Point Fixe]
    Lorsqu'on résout des problèmes de points fixes, on veut transformer
    le problème en problème de zéro. Par exemple, si on a $f(x)=x$, alors
    on veut travailler avec $g(x) = f(x)-x$ et montrer qu'il y a un
    changement de signe dans cet intervalle pour que g possède un zéro
\end{problem}

\begin{problem}[Trouver les racines de la fonction]
    On peut utiliser le théorème des valeurs intermédiaires pour montrer
    qu'il y a une racine dans l'intervalle. Essentiellement, on veut
    montrer que s'il y a un changement de signe dans un intervalle, et
    que la fonction est continue, nécessairement, on passe par l'axe
    des x et la fonction possède un zéro dans cet intervalle.\\
    Si on nous demande de trouver la racine à au plus (erreur), on n'a
    qu'à définir un intervalle dont la moitié est moins que ce terme
    d'erreur
\end{problem}

\subsection{Continuité Uniforme}

\begin{definition}[Continuité Uniforme]
    Soit $f:D \to \mathbb{R}$, une fonction. On dit qu'elle est
    uniformément continue si $\forall \varepsilon > 0, \exists \delta >0,
    \forall x,y \in D, |x-y| \Longrightarrow |f(x) - f(y)|< \varepsilon$
\end{definition}

\begin{theorem}
    Soit $f:[a,b] \to \mathbb{R}$. Si f est continue, alors f est
    uniformément continue. En d'autres mots, si f est une fonction
    continue définie sur un compact (intervalle fermé et borné), alors
    elle est uniformément continue sur cet intervalle.
\end{theorem}

\begin{theorem}[Une Fonction Lipschitz est uniformément continue]
    Soit $f:D \to \mathbb{R}$. S'il existe une constante K tel que
    $K >)$ telle que $$ |f(x) - f(y)| < K |x-y|$$ alors elle est
    uniformément continue
\end{theorem}

\begin{remark}
    Une fonction Lipschitz est bornée, et on sait que toute fonction
    continue qui est fermé et bornée dans un intervalle est convergente,
    et donc continuement uniforme
\end{remark}

\begin{problem}[Montrer que la fonction n'est pas uniformément continue]
    Une fonction est uniformément continue si on peut fermer les extrémités. En
    d'autres mots, il faut que la limite aux extrémités existent et soient
    égale à f(a)
\end{problem}

\pagebreak

\section{Dérivabilité}

\subsection{Overview}

\subsection{Définition de la dérivabilité}

\begin{definition}[Dérivée]
    Soit I =(a,b), un intervalle. Soit $F:I \to \mathbb{R}$, une fonction
    et soit $ c \in I$. ALors f est dérivable en c si
    $$ \lim_{h \to 0} \frac{f(c+h) -f(c)}{h} \text{ existe } $$
    La valeur de la limite s'appelle a dérivée de f en a, notée $f'(A)$
\end{definition}

\begin{remark}
    La définition de la dérivée nous dit que si la dérivée est équivalent
    à prendre deux points et de les rapprocher à une distance très petite,
    qui tend vers 0. Une définition équivalente serait la suivante:
    $$ \lim_{x \to x_0} \frac{f(x) -f(x_0)}{x - x_0} \text{ existe } $$
\end{remark}

\begin{remark}[Interprétation Analytique]
    On sait que la dérivée représente la pente de la fonction en un
    point donné, mais on peut réécrire la fonction de la façon suivante:
    $$ f(c+h) = f(c) + f'(c)h + \epsilon (h)h, \epsilon (h)h \to 0
    \text{ lorsque } h \to 0$$
    Ainsi, puisque f(c+h) = constante + pente + erreur, on peut considérer
    que f ressemble à une fonction affine localement
\end{remark}

\begin{theorem}[Règles de Calcul]
    Soit I, un intervalle ouvert et $f,g:I \to \mathbb{R}$. Si f et g
    sont dérivables en $a \in I$, alors
    \begin{enumerate}
	\item Somme:
	\item Produit:
	\item Quotient:
    \end{enumerate}
\end{theorem}

\begin{theorem}[Règle de Dérivation en Chaine]
    Si f et g sont dérivables en b = f(a), alors $g(f(x))$ est dérivable
    en a et $$ (g \cdot \f)'(a) = g'(f(a)) f'(a) = g'(b) f'(a) $$
\end{theorem}

\subsection{Extremums Relatifs et Absolues}

\begin{definition}[Extremums Locaux]
    Soit $f:I \to \mathbb{R}$, une fonction et soit $a \in I$. On dit
    que a est un maximum local s'il existe un intervalle
    $$(c,d) \subseteq I \text{ tel que } a \in \(c,d) \text{ et }
    f(x) \leq f(a)$$
    On dit que a est un minimum local s'il existe un intervalle
    $$(c,d) \subseteq I \text{ tel que } a \in \(c,d) \text{ et }
    f(x) \geq f(a)$$
\end{definition}

\begin{remark}
    Pour que le maximum/minimum existent, il faut que le maximum/minimum
    soit atteint
\end{remark}

\begin{definition}[Points Critiques]
    Les points critiques de f sont les points x lorsque f(x)=0
\end{definition}

\begin{theorem}
    Supposons $f:I \to \mathbb{R}$ atteint un minimum ou un maximum
    en un point $a \in I$ et que a est un point intérieur de I. Supposons
    aussi que f est dérivable dans l'intervalle. On a que $f'(a)=0$
\end{theorem}

\begin{remark}
    La pente d'un maximum et d'un minimum est nulle
\end{remark}

\begin{remark}
    La fonction f peut atteindre un min/max au bout d'un intervalle sans
    avoir f'(a)=0, car si a est au bout, il n'est pas inclu dans
    l'intervalle
\end{remark}

\begin{theorem}
    f dérivable en a $\Longrightarrow$ f est continue en a, mais la
    contraposée est fausse.
\end{theorem}

\subsection{Propriété des fonctions dérivables}

\subsubsection{Théorème de Rolle}

\begin{definition}[Théorème de Rolle]
    Soit $f:[a,b] \to \mathbb{R}$, une fonction telle que
    \begin{enumerate}
	\item f est continue sur [a,b]
	\item f est dérivable sur (a,b)
	\item f(a)=f(b)
    \end{enumerate}
    alors $\exists c \in (a,b) \text{ tel que } f'(c)=0$
\end{definition}

\begin{remark}
    Le Théorème de Rolle nous dit que si deux points ont la même
    image et que la fonction est continue et dérivable, alors il y
    a un extremums local. On sait que ce ne sera pas un point
    d'inflexion, car la fonction ne peut pas avoir d'asymptote
    verticale dans l'intrervalle si elle est continue.
\end{remark}

\begin{corollary}[Corollaire du Théorème de Rolle]
    Soit $f:[a,b] \to \mathbb{R}$, une fonction continue sur [a,b] et
    n fois dérivable sur (a,b). Supposons que $f^(n) \neq 0, \forall
    x \in (a,b)$. Alors l'équation f(x)=0 possède au plus n solutions
    dans l'intervalle [a,b]
\end{corollary}

\begin{remark}
    Le corollaire nous dit qu'un polynôme a au plus n zéros
\end{remark}

\begin{problem}[Montrer que f possède au plus n racines réelles]
    Pour trouver les zéros, on peut utiliser le théorème des
    valeurs intermédiaires pour construire les intervalles avec changement
    de signe. Pour montrer qu'il n'y a pas un deuxième zéro dans cet
    intervalle, on peut utiliser le théorème de Rolle pour montrer qu'il
    y a une contradiction. On montre que la racine trouvée est à
    l'extérieur de l'intervalle définie.
\end{problem}

\subsubsection{Théorème des Accroissements Finis}

\begin{theorem}[Théorème des Accroissements Finis]
    Soit $f:[a,b] \to \mathbb{R}$, une fonction telle que
    \begin{enumerate}
	\item f est continue sur [a,b]
	\item f est dérivable sur (a,b)
    \end{enumerate}
    alors $\exists c \in (a,b) \text{ tel que } f'(c)=\frac{f(b) - f(a)}
    {b-a}$
\end{theorem}

\begin{intuition}
    Le théorème des accroissements finis nous dit que si on a une
    fonction continue et dérivable et qu'on relie les 2 extrémités
    de l'intervalle, on peut trouver un point intérieur c qui possède
    la même pente que la droite formée par a et b
\end{intuition}

\begin{remark}
    Le théorème des accroissements finis est le théorème de Rolle
    généralisé. C'est comme si on rotationnait notre graphique
\end{remark}

\begin{remark}
    Le théorème des accroissements finis est pratique, car il nous
    permet de rendre notre fonction Lipschitienne:
    $$ |f(b) - f(a)| \leq M |b-a|, \forall x \in \mathbb{R},
    |f'(x) \leq M$$
    Ainsi, si on peut appliquer le théorème des accroissements finis
    sur une fonction, la fonction est uniformément continue, car elle
    est Lipschitienne
\end{remark}

\begin{theorem}[Fonction Constante]
    Soit $f:I \to \mathbb{R}$, une fonction dérivable avec I, un
    intervalle ouvert. Alors, f est constante $\Longleftrightarrow
    f'(x)=0 \forall x \in I$
\end{theorem}


\begin{intuition}
    La preuve se fait avec le théorème de Rolle
\end{intuition}

\begin{problem}[Calculer approximativement $\sqrt(24)$]
    On peut utiliser le théorème des accroissements finis pour
    approximer une racine en posant $f(x)=\sqrt(x)$ et en travaillant
    avec l'intervalle [16,25], car on sait calculer $\sqrt(16)$ et
    $\sqrt(25)$
\end{problem}

\begin{corollary}[Formule de Cauchy]
    Soit f et g, deux fonctions continues sur [a,b] et différentiables
    sur (a,b). Si $g'(x) \neq 0 \forall x \in (a,b)$, il existe au moins
    un point $c \in (a,b)$ tel que
    $$ \frac{f(b)- f(a)}{g(b)-g(a)} = \frac{f'(c)}{g'(c)}$$
\end{corollary}

\begin{remark}
    On peut retrouver la formule de Cauchy à partir du théorème des
    accroissements finis ou de la règle de l'Hospital
\end{remark}

\subsection{Croissance et Décroissance}

\begin{definition}[Fonction Croissante et Décroissante]
    Une fonction $f:I \to \mathbb{R}$ est dite
    \begin{enumerate}
	\item croissante si $a \leq b \Longrightarrow f(a) \leq f(b)$
	\item strictement croissante si
	    $a < b \Longrightarrow f(a) < f(b)$
	\item décroissante si $a \geq b \Longrightarrow f(a) \geq f(b)$
	\item strictement décroissante si
	    $a > b \Longrightarrow f(a) > f(b)$
    \end{enumerate}
\end{definition}

\begin{theorem}
    Une fonction $f:I \to \mathbb{R}$, dérivable sur un intervalle I
    ouvert.
    \begin{enumerate}
	\item f est croissante $\Longleftrightarrow f'(x) \geq 0,
	    \forall x \in I$
	\item $f'(x) >0 \forall x \in I$, alors f est strictement
	    croissante sur I
	\item f est décroissante $\Longleftrightarrow f'(x) \leq 0,
	    \forall x \in I$
	\item $f'(x) <0 \forall x \in I$, alors f est strictement
	    décroissante sur I
    \end{enumerate}
\end{theorem}

\begin{intuition}
    La preuve se fait avec le théorème des accroissements finis.
\end{intuition}

\begin{remark}
    f strictement croissante $\not\Longrightarrow f'(x)>0 \forall x$
\end{remark}

\subsection{Règle de l'Hospital}

\begin{theorem}[Règle de l'Hospital]
    Soient $f,g:I \to \mathbb{R}$, des fonctions dérivables. Soit
    $a \in I$ et supposons que f(a)=0 et g(a)=0. Soit $a \in I$ et
    supposons que $f(a)=0, g(a)=0$. Si
    $$ \lim_{x \to a} \frac{f'(x)}{g'(x)}=l, \text{ alors }
    \lim_{x \to a} \frac{f(x)}{g(x)} = l$$
\end{theorem}

\begin{remark}
    L'hypothèse que $ \lim_{x \to a} \frac{f'(x)}{g'(x)}$ existe suppose
    aussi que $g'(x) \neq 0 \forall x$ assez proche de a. La règle de
    l'hospital marche quand même s'il y a une discontinuité dans
    l'intervalle I, on ne veut juste pas que ce soit autour de a
\end{remark}

\begin{intuition}
    La règle de l'hospital nous permet d'évaluer la limite de forme
    indéterminée avec la dérivée
\end{intuition}

\begin{lemma}
    Soient $f,g:[a,b] \to \mathbb{R}$, deux fonctions continues sur [a,b]
    et dérivales sur (a,b). Alors, il existe $c \in (a,b)$ tel que
    $$ (f(b) - f(a)) g'(c) = (g(b)-g(a))f'(c) $$
\end{lemma}

\begin{remark}
    Le cas g(x)=x est le théorème des accroissements finis
\end{remark}

\begin{remark}[Lien entre théorème des accroissements finis et Règle
    de l'Hospital]

\end{remark}

\subsection{Approximation}

\subsection{Développements Limités}

\subsection{Théorème de Taylor}

\subsection{Méthode de Newton}


\pagebreak

\section{Séries}

\subsection{Overview}


\subsection{Convergence des séries}

\subsection{Séries célèbres}

\subsection{Critères de Convergence pour les séries à termes positifs}

    \begin{enumerate}
	\item Critère de Comparaison
	\item Critère du Quotient
	\item Critère de Condensation de Cauchy
	\item Critère d'Alembert (du rapport)
	\item Critère de la Racine de Cauchy
	\item Critère de Dirichlet
    \end{enumerate}

\subsection{Convergence des séries alternées et Critère de Leibniz}


\pagebreak

\end{document}
\end{article}
