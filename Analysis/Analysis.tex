\documentclass{article}
\begin{document}
\title{Lectures Notes for Analysis}
\author{Emulie Chhor}
\maketitle

\section{Introduction}

Le premier cours d'analyse porte sur la même matière que le calcul différentiel
et intégral, mais est plus rigoureux. La majorité du cours met l'emphase sur la
démonstration des preuves et théorèmes et très peu sur le calcul.

MAT1000 porte sur les chapitres suivants:

    \begin{enumerate}
	\item Inégalités et Valeur Absolue
	\item Suprémum et Infimum
	\item Axiomes de Complétude
	\item Dénombrabilité
	\item Suites
	\item Continuité et Continuité Uniforme
	\item Dérivabilité
	\item Séries
    \end{enumerate}

\section{Inégalités et Valeur Absolue}

\subsection{Théorie}

\subsubsection{Propriétés des Inégalités}

\subsubsection{Propriétés des Valeurs Absolues}

\subsection{Pratique}

\subsubsection{Stratégies pour montrer des Inegalités}

    \begin{enumerate}
	\item Ajouter et Enlever le même terme
	\item x^2 > 0
    \end{enumerate}

\subsubsection{Pratique - Résoudre l'inégalité}

\section{Suprémum et Infimum}

\subsection{Théorie}

\subsubsection{Majorant et Minorant}

\subsubsection{Suprémum et Infimum}

\subsection{Pratique}

\subsubsection{Montrer que la suite possède un suprémum/infimum}

\subsubsection{Montrer que la suite ne possède pas un suprémum/infimum}

\subsubsection{Montrer que le suprémum/infimum est atteint}

\section{Axiomes de Complétude}

\subsection{Théorie}

\subsubsection{Propriété Archimédienne}

\subsubsection{Densité des Rationnels}

\subsection{Pratique}

\section{Dénombrabilité}

\subsection{Théorie}

\subsubsection{Injection, Surjection, Bijection}

\subsubsection{Cardinalité}

\subsubsection{Théorème - Cardinalité et Injection-Bijection-Surjection}

\subsection{Pratique}

\section{Suites}

\subsection{Théorie}

\subsubsection{Convergence d'une suite}

\subsubsection{Lim Sup et Lim inf}

\subsubsection{Théorème de Bolzano-Weistrass}

\subsubsection{Suite de Cauchy}

\subsubsection{Suites définies par récurrence}

\subsection{Pratique}

\section{Continuité et Continuité Uniforme}

\subsection{Théorie}

\subsubsection{Definition de la Continuité}

\subsubsection{Théorème des Valeurs Intermédiaires}

\subsection{Pratique}

\section{Dérivabilité}

\subsection{Théorie}

\subsubsection{Définition de la dérivabilité}

\subsubsection{Propriété des fonctions dérivables}

\subsubsubsection{Théorème de Rolle}

\subsubsubsection{Théorème des Accroissements Finis}

\subsubsubsection{Corollaire du théorème des acroissements finis - Cauchy}

\subsubsection{Extremums Relatifs et Absolues}

\subsubsection{Approximation}

\subsubsubsection{Développements Limités}

\subsubsubsection{Théorème de Taylor}

\subsubsubsection{Méthode de Newton}

\subsection{Pratique}

\subsubsection{Déterminer les zéros d'une fonction}

\subsubsection{Problèmes de Points Fixes}

\section{Séries}

\subsection{Théorie}

\subsubsection{Convergence des séries}

\subsubsection{Séries célèbres}

\subsubsection{Critères de Convergence pour les séries à termes positifs}

    \begin{enumerate}
	\item Critère de Comparaison
	\item Critère du Quotient
	\item Critère de Condensation de Cauchy
	\item Critère d'Alembert (du rapport)
	\item Critère de la Racine de Cauchy
	\item Critère de Dirichlet
    \end{enumerate}

\subsubsection{Convergence des séries alternées et Critère de Leibniz}


\subsection{Pratique}

\end{document}
\end{article}
