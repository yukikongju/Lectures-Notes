\documentclass{article}
\usepackage{amsmath}
\usepackage{amsfonts}
\usepackage{amsthm}
\usepackage{parskip}
\usepackage{textgreek}
\begin{document}
\title{Lectures Notes for Analysis}
\author{Emulie Chhor}
\maketitle

\section*{Introduction}

Le premier cours d'analyse porte sur la même matière que le calcul différentiel
et intégral, mais est plus rigoureux. La majorité du cours met l'emphase sur la
démonstration des preuves et théorèmes et très peu sur le calcul.\\

MAT1000 porte sur les chapitres suivants:\\

    \begin{enumerate}
	\item Nombres Rationnels et Nombres Réels
	\item Inégalités et Valeur Absolue
	\item Suprémum et Infimum
	\item Axiomes de Complétude
	\item Dénombrabilité
	\item Suites
	\item Continuité et Continuité Uniforme
	\item Dérivabilité
	\item Séries
    \end{enumerate}

\newtheorem{definition}{Definition}[subsection]
\newtheorem{theorem}{Theorem}[subsection]
\newtheorem{corollary}{Corollary}[subsection]
\newtheorem{lemma}[theorem]{Lemma}
\newtheorem{proposition}{Proposition}[section]
\newtheorem{axiom}{Axiome}
\newtheorem{property}{Propriété}[subsection]
\newtheorem*{remark}{Remarque}
\newtheorem*{problem}{Problème}
\newtheorem*{intuition}{Intuition}

\pagebreak

\section{Nombres Rationnels et Nombres Réels}

\subsection{Overview}

Malgré que le cours ne met pas l'emphase sur la topologie des réels, il est
important de comprendre que les réels possèdent certaines propriétés qui nous
permettent de manipuler les termes algébriquement. On définit la notion de
corps ordonné qui est TODO

\subsection{Topologie des Réels}

\begin{proposition}
    $\forall a \in \mathbb{Q}, a^2 \neq 2$
\end{proposition}

\begin{lemma}
    Soit $b \in \mathbb{Z}$. Si $b^2$ est pair, alors b est pair
\end{lemma}

\begin{axiom}
    Il existe un corps totalement ordonné de $\mathbb{R}$ tel que:\\
    \begin{enumerate}
	\item $\mathbb{Q} \subseteq \mathbb{R} \text{ et } \mathbb{Q}$ est un sous-corps
	    de $\mathbb{R}$
	\item $\mathbb{R}$ possède la propriété de Complétude
    \end{enumerate}
\end{axiom}

\begin{proposition}
    Soit $a,b \in \mathbb{Q}$ avec a<b. Alors il existe $c \in \mathbb{Q}$ tel que
    $a<c<b$
\end{proposition}

\begin{theorem}[Propriétés d'un corps]
    Un corps possède les propriétés suivantes:\\
    \begin{enumerate}
	\item Commutativité
	\item Associativité
	\item Distributivité
	\item Existence d'éléments neutres
	\item Existence de l'inverse additif et multiplicatif
	\item Lien entre les relations d'ordre et opérations
    \end{enumerate}
\end{theorem}

\begin{problem}[Montrer qu'un nombre est irrationnel]
\end{problem}

\pagebreak

\section{Inégalités et Valeur Absolue}

\subsection{Overview}

TODO

\subsection{Définition de la Valeur Absolue}

\begin{definition}[Valeur Absolue]
\end{definition}

\subsection{Propriétés des Valeurs Absolues}

\begin{remark}
    On peut interpréter la valeur absolue comme la distance entre 2 termes
\end{remark}

\subsection{Propriétés des Inégalités}

\begin{problem}[Montrer l'inégalité]
    Lorsqu'on veut montrer une inégalité, on utilise les astuces suivantes:\\

    \begin{enumerate}
	\item Ajouter et Enlever le même terme
	\item $x^2 > 0$
	\item Utiliser la transitivité pour comparer avec un autre terme
    \end{enumerate}

\end{problem}

\begin{problem}[Résoudre des inégalités]
    Lorsqu'on nous demande de résoudre des inégalités, on doit trouver les
    valeurs qui satisfont l'inégalité. Pour ce faire, il est pratique d'utiliser
    un tableau pour tester les intervalles de cohérence.
\end{problem}

\subsection{Induction}

\begin{intuition}[Induction]
    L'induction est une technique de preuve qui se fait en 3 étapes:
    \begin{enumerate}
	\item Cas de base
	\item Hypothèse d'induction: on suppose que l'étape n est vraie
	\item Étape d'induction: on montre que l'hypothèse est vraie en
	    utilisant l'hypothèse d'induction
    \end{enumerate}
\end{intuition}

\subsection{Inégalité de Bernouilli}

\begin{theorem}[Inégalité de Bernouilli]
    $ \forall n \in \mathbb{N}, \forall t > -1, (1+t)^n \geq 1+nt$
\end{theorem}

\pagebreak

\section{Suprémum et Infimum}

\subsection{Overview}

\subsection{Majorant et Minorant}

\begin{definition}[Majorant]
    Soit $E \subseteq \mathbb{R}$. E est majoré si $\exists c \in \mathbb{R}, \forall
    x \in E, x \leq c$. On dit que c est un majorant de E
\end{definition}

\begin{definition}[Minorant]
    Soit $E \subseteq \mathbb{R}$. E est minoré si $\exists c \in \mathbb{R}, \forall
    x \in E, x \geq c$. On dit que c en un minorant de E
\end{definition}

\begin{definition}[Ensemble Borné]
    Soit $E \subseteq \mathbb{R}$. E est borné si E est majoré et minoré
\end{definition}

\begin{proposition}
    E est borné $ \to \exists c \in \mathbb{R}, \forall x \in E, |x|<c $
\end{proposition}

\begin{remark}
    Un majorant et un minorant sont des valeurs que l'ensemble de peut jamais
    atteindre. Il peut en exister plusieurs
\end{remark}

\subsection{Suprémum et Infimum}

\begin{definition}[Supremum]
    Soit $E \subseteq \mathbb{R}$. On dit que $c \in \mathbb{R}$ est le supremum
    de E si c'est le plus petit majorant de E
\end{definition}

\begin{definition}[Infimum]
    Soit $E \subseteq \mathbb{R}$. On dit que $c \in \mathbb{R}$ est l'infimum
    de E si c'est le plus grand minotant de E
\end{definition}

\begin{proposition}
    Soit $ A \subseteq \mathbb{R}, B \subseteq \mathbb{R}$ tel que Sup A et Sup B
    existent. Alors, $sup(A+B)= sup A + sup B$
\end{proposition}

\begin{proposition}
    Le suprémum et l'infimum sont uniques.
\end{proposition}

\begin{problem}[Montrer que la suite possède un suprémum/infimum]
    Pour montrer que la suite possède un supremum/infimum, il faut montrer que:
    \begin{enumerate}
	\item c est un majorant/minorant
	\item c est le plus petit majorant/plus grand minorant
    \end{enumerate}

    Pour montrer la première étape, il suffit d'utiliser la définition. Pour
    montrer la deuxième étape, on procède par contradiction. On suppose qu'il
    existe un plus petit majorant/plus grand minorant et on montre que cela
    contredit l'hypothèse que S' était un majorant/minorant

\end{problem}

\begin{problem}[Montrer que la suite ne possède pas un suprémum/infimum]
    Une suite de possède pas de suprémum ou d'infimum si elle n'est pas bornée
    (l'ensemble est infini). On peut supposer qu'elle est bornée et utiliser
    l'axiome de complétude pour montrer qu'elle est majorée/minorée, puis on
    montre qu'il s'agit d'une contradiction avec la propriété Archimédienne.
\end{problem}

\begin{problem}[Montrer que le suprémum/infimum est atteint]
\end{problem}

\pagebreak
\section{Axiomes de Complétude}
\subsection{Overview}

\subsection{Propriété Archimédienne}

\begin{theorem}[Propriété Archimédienne]
    Pour tout $x,y \in \mathbb{R} \text{ avec } x>0$, il existe $ n \in \mathbb{N}
    \text{ tel que } nx>y$
\end{theorem}

\begin{remark}
    La propriété Archimédienne nous dit que les réels ne sont pas bornés. Ainsi,
    si on prend un nombre arbitraire, on peut en trouver un autre qui est plus
    petit ou plus grand.
\end{remark}

\subsection{Axiome de Complétude}

\begin{proposition}[Axiome de Complétude]
    Si $E \in \mathbb{R}$ est non-vide et majorée, alors sup E existe dans
    $\mathbb{R}$
\end{proposition}

\begin{corollary}[Corollaire de l'Axiome de Complétude]
    \begin{enumerate}
	\item $ \forall y \in \mathbb{R}, \exists n \in \mathbb{N}, n>y$
	\item $ \forall x \in \mathbb{R}, \exists! n \in \mathbb{R}$ tel que
	    $n \leq x \neq n+1$ (fonction plancher)
	\item (Densité de $\mathbb{Q}$ dans $\mathbb{Q}$): Soit $x,y \in \mathbb{R}$,
	    alors $\exists r \in \mathbb{Q}, x<r<y$
	\item (Densité de $\mathbb{R} \ \mathbb{Q}$ dans $\mathbb{Q}$): Soit $x,y \in \mathbb{R}$,
	    alors $\exists r \in \mathbb{R} \ \mathbb{Q}, x<r<y$
    \end{enumerate}
\end{corollary}

\begin{theorem}
    L'axiome de complétude est faux dans $\mathbb{Q}$
\end{theorem}

\subsection{Densité des Rationnels}

\begin{theorem}[Densité de $\mathbb{Q}$ dans $\mathbb{R}$]
    Soit $x,y \in \mathbb{R}$, alors $\exists r \in \mathbb{Q}, x<r<y$
\end{theorem}

\begin{theorem}[Densité de $\mathbb{Q}'$ dans $\mathbb{R}$]
    Soit $x,y \in \mathbb{R}$, alors $\exists r \in \mathbb{Q} ', x<r<y$
\end{theorem}

\begin{remark}
    Le densité des rationnels et des irrationnels découle de la propriété
    d'Archimède. Puisque les réels ne sont pas bornés, alors il existe une
    infinité de nombre dans un intervalle, alors il existe nécessairement un
    rationnel et un irrationnel dans cet intervalle.
\end{remark}

\begin{intuition}
    La preuve de la densité se fait en construisant un rationnel à partir de la
    distance entre x et y
\end{intuition}

\subsection{Racine n-ième}

\begin{definition}[Definition Racine n-ième]
    % Soit $q \in \mathbb{Q}, m \in \mathbb{Z}, n \in \mathbb{N} \text{ tel que }
    % q= \fract{m}{n}$. Pour $x \geq 0$, on pose:
    \begin{enumerate}
	% \item $x^q := ^n\sqrt(x^m)$ si $q \geq 0$
	% \item $x^q := (\fract{1}{x}) ^ (-q)$ si $q<0, x \neq 0$
    \end{enumerate}
\end{definition}

\begin{theorem}[Racine n-ième]
    $ \forall n \in \mathbb{N} , \forall x \geq 0, \exists! y \geq 0, y^n =x \,
    (^n\sqrt{x} = y)$
\end{theorem}

\pagebreak

\section{Dénombrabilité}
\subsection{Overview}

\subsection{Injection, Surjection, Bijection}

\begin{definition}[Injection]
\end{definition}

\begin{definition}[Surjection]
\end{definition}

\begin{theorem}[Bijection]
\end{theorem}

\subsection{Cardinalité}

\begin{definition}[Cardinalité]
    Deux ensembles A et B ont le même cardinal s'il existe une bijection $\Phi:A \to B$
\end{definition}

\begin{corollary}
    Si $\Phi:A \to B$ est une bijection, alors $\Phi ^(-1): B \to A$ existe et est
    bijective
\end{corollary}

\begin{definition}[Dénombrabilité]
    Un ensemble A est dénombrable s'il a le même cardinal que $\mathbb{N}$
\end{definition}

\begin{remark}
\item $\mathbb{N}$ est dénombrable
\item Un ensemble dénombrable est infini
\item Un ensemble est dénombrable $\Longleftrightarrow$ on peut former une suite
    infinie $a_1, a_2, ...$ contenant une et une seule fois chaque élément de A
\end{remark}

\begin{theorem}
    Soit $ f:A \to B$, une fonction:
    \begin{enumerate}
	\item si f est surjective, alors $|A| \geq |B|$
	\item si f est injective, alors $|A| \leq |B|$
	\item si f est bijective, alors $|A| = |B|$
    \end{enumerate}
\end{theorem}

\begin{proposition}
    \begin{enumerate}
	\item $\mathbb{Z}$ est dénombrable
    	\item $\mathbb{Q}$ est dénombrable
    	\item $\mathbb{R}$ n'est pas dénombrable
    \end{enumerate}
\end{proposition}

\begin{remark}[Diagonale de Cantor]
\end{remark}

\pagebreak
\section{Suites}

\subsection{Overview}

Dans cette section, on essaie de déterminer si une suite converge et si oui,
on veut calculer sa limite

\subsection{Convergence d'une suite}

\begin{definition}
\end{definition}

\begin{intuition}
\end{intuition}

\begin{remark}
\end{remark}

\begin{definition}
\end{definition}
\begin{definition}
\begin{definition}
\begin{definition}
\end{definition}
\end{definition}
\end{definition}

\subsection{Théorème des suites monotones}

\subsection{Lim Sup et Lim inf}

\subsection{Théorème de Bolzano-Weistrass}

\subsection{Suite de Cauchy}

\subsection{Suites définies par récurrence}


\pagebreak
\section{Continuité et Continuité Uniforme}
\subsection{Overview}


\subsection{Definition de la Continuité}

\subsection{Théorème des Valeurs Intermédiaires}


\pagebreak
\section{Dérivabilité}
\subsection{Overview}


\subsection{Définition de la dérivabilité}

\subsection{Propriété des fonctions dérivables}

\subsection{Théorème de Rolle}

\subsection{Théorème des Accroissements Finis}

\subsection{Corollaire du théorème des acroissements finis - Cauchy}

\subsection{Extremums Relatifs et Absolues}

\subsection{Approximation}

\subsection{Développements Limités}

\subsection{Théorème de Taylor}

\subsection{Méthode de Newton}


\subsection{Déterminer les zéros d'une fonction}

\subsection{Problèmes de Points Fixes}

\pagebreak
\section{Séries}
\subsection{Overview}


\subsection{Convergence des séries}

\subsection{Séries célèbres}

\subsection{Critères de Convergence pour les séries à termes positifs}

    \begin{enumerate}
	\item Critère de Comparaison
	\item Critère du Quotient
	\item Critère de Condensation de Cauchy
	\item Critère d'Alembert (du rapport)
	\item Critère de la Racine de Cauchy
	\item Critère de Dirichlet
    \end{enumerate}

\subsection{Convergence des séries alternées et Critère de Leibniz}


\pagebreak

\end{document}
\end{article}
