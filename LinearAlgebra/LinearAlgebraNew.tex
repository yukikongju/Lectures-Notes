\documentclass{article}
\usepackage{amsmath}
\usepackage{amsfonts}
\usepackage{amsthm}
\usepackage{parskip}
\usepackage{textgreek}
\begin{document}
\title{Notes de Cours - Algèbre Linéaire}
\author{Emulie Chhor}
\maketitle


\section*{Introduction}

Le cours d'algèbre linéaire comporte plusieurs chapitres:

    \begin{enumerate}
	\item Systèmes d'Équations Linéaires
	\item Matrices
	\item Déterminants
	\item Espaces et Sous-Espaces Vectoriels
	\item Orthogonalité et Projections
	\item Diagonalisation
    \end{enumerate}

\newtheorem{definition}{Definition}[subsection]
\newtheorem{theorem}{Theorem}[subsection]
\newtheorem{corollary}{Corollary}[subsection]
\newtheorem{lemma}[theorem]{Lemma}
\newtheorem{proposition}{Proposition}[section]
\newtheorem{axiom}{Axiome}
\newtheorem{property}{Propriété}[subsection]
\newtheorem*{remark}{Remarque}
\newtheorem*{problem}{Problème}
\newtheorem*{intuition}{Intuition}

\subsection{Pourquoi étudier l'algèbre linéaire?}

\pagebreak

\section{Systèmes d'Équations Linéaires}

\subsection{Overview}

Le but de cette section est de pouvoir résoudre des systèmes d'équations linéaires
avec l'algorithme de Gauss et de Gauss-Jordan

\subsection{Définition d'un système d'équations linéaires}
\subsection{Opérations élémentaires sur les lignes}

Il existe 3 opérations élémentaires sur les matrices:
\begin{enumerate}
    \item Multiplier la ligne i par une constante $k \neq 0$ : $L_i \rightarrow
	k L_j , k \neq 0$
    \item Permuter les lignes i et j : $ L_i \leftrightarrow L_j$
    \item Ajouter à la linge i un multiple d de la ligne j : $ L_i \rightarrow
	L_i + d L_j$
\end{enumerate}

\subsection{Forme des matrices}

Lorsqu'on essait de résoudre un SEL, on veut transformer notre matrice en l'une
des deux formes:
\begin{enumerate}
    \item Forme échelonée : Gauss
    \item Forme échelonée réduite : Gauss-Jordan
\end{enumerate}

\subsection{Types de Solutions}

Il existe 3 types de solutions:
\begin{enumerate}
    \item Solution Unique: chaque variable est associée à un pivot
    \item Infinité de Solutions: il existe une variable libre ou plus
    \item Aucune Solutions: le système est incompatible et on a
	$ [0 \, 0 \, ... \, 0 | \, k], \, k \neq 0$
\end{enumerate}

\subsection{Méthode de Gauss}
\subsection{Méthode de Gauss-Jordan}

\begin{remark}[Méthode de Gauss vs Gauss-Jordan]
\end{remark}

\pagebreak

\section{Matrices}
\subsection{Overview}

Le but de cette section est de se familiariser algébriquement avec la notion de
matrice. Cependant, il faut toujours garder en tête qu'une matrice représente
une transformation linéaire. Essentiellement, une transformation linéaire est
une fonction qui prend un vecteur et en recrache un autre. On verra plus tard
les détails.

\subsection{Opérations Matricielles}
\subsubsection{Addition et Multiplication par un scalaire}
\subsubsection{Multiplication Matricielle}
\subsubsection{Transposition de Matrice}
\subsection{Équation Ax=b}
\subsubsection{Combinaison Linéaire}
\subsubsection{Indépendance Linéaire}
\subsubsection{Équation Ax=b}
\subsection{Inversion de Matrices}

\pagebreak
\section{Déterminants}
\subsection{Overview}

Géométriquement, le déterminant d'une matrice représente le ratio entre l'aire
formée par les vecteurs avant et après transormations linéaires. Intuitivement,
le déterminant nous dit si une matrice est diagonalisable ou non puisque si
$det(A)=0$, alors la transformation à changer de dimension et n'est pas inversible.

On peut aussi utiliser les déterminants pour évaluer un SEL. On compare le déterminant
de chaque vecteur avant et après transformation.

\subsection{Définition du Déterminant}
\subsection{Propriétés du Déterminant}
\subsection{Règle de Cramer}

\pagebreak

\section{Espaces et Sous-Espaces Vectoriels}
\subsection{Overview}

Un aspect important à comprendre dans l'étude de l'algèbre linéaire est la notion
d'espace. Malgré qu'on s'attarde peu à la notion d'anneau, il faut se souvenir
qu'on peut manipuler les vecteurs algébriquement, car l'espace dans lequel il se
trouve est un field, un ensemble qui comporte certaines propriétés qu'on nomme
axiomes.\\

Un deuxième aspect important est la notion de transformation linéaire. Comme mentionné
plus tôt, une matrice peut être interprétée comme une transformation, c-à-d qu'elle
est une fonction qui transforme un vecteur en un autre vecteur. Cependant, cette
transformation peut changer les propriétés de l'espace. On s'attardera donc aux
transformations qui gardent ces propriétés, transformations qu'on nomme linéaire.

Finalement, on veut être capable de changer de base. Décrire un vecteur dans
l'espace est assez arbitraire, car chaque personne peut tracer son propre quadrillage
et obtenir un vecteur différent. Ainsi, pour s'assurer de parler le même language,
on veut être capable de traduire un vecteur/transformation d'un espace à un autre.

\subsection{Espaces vectoriels sur $\mathbb{R}$}
\subsection{Transformations Linéaires}
\subsection{Base d'un Espace Vectoriel}
\subsection{Système de Coordonnées dans $\mathbb{R}^n$}
\subsection{Changement de Base}

\pagebreak
\section{Orthogonalité et Projections}
\subsection{Overview}

\subsection{Orthogonalité et Projection}
\subsection{Sous-Espace Orthogonal}

\pagebreak

\section{Diagonalisation}
\subsection{Overview}

La diagonalisation est un processus utilisé dans le domaine du Machine Learning,
car elle permet de "compresser" les données dans une plus petite dimension. La
diagonalisation est notamment utilisée pour le PCA et le SVD.\\

Géométriquement, lorsqu'on parle de valeurs propres et de vecteurs propres, c'est
qu'il existe un vecteur qui reste sur le même axe après transformation linéaire.
Ainsi, appliquer la transformation linéaire sur ce vecteur peut être considérer
comme une multiplication par un scalaire $\lambda$, qui étire la longueur du vecteur.
Notre but est donc de trouver le vecteur qui reste dans le même axe, qu'on appelle
vecteur propre, et de trouver le facteur de multiplication, qu'on appele valeur
propre.

\subsection{Valeurs et Vecteurs Propres}
\subsection{Espace Propre et Multiplicité Géométrique}
\subsection{Diagonalisation d'une matrice quelconque}
\subsection{Diagonalisation d'une matrice symétrique}

\pagebreak

\end{document}
\end{article}
