\documentclass{article}
\usepackage{amsmath}
\usepackage{amsfonts}
\usepackage{amsthm}
\usepackage{parskip}
\usepackage{textgreek}
\begin{document}
\title{Notes de Cours - Algèbre Linéaire}
\author{Emulie Chhor}
\maketitle

\section*{Introduction}

Le cours d'algèbre linéaire comporte plusieurs chapitres:

    \begin{enumerate}
	\item Systèmes d'Équations Linéaires
	\item Matrices
	\item Déterminants
	\item Espaces et Sous-Espaces Vectoriels
	\item Orthogonalité et Projections
	\item Diagonalisation
    \end{enumerate}

\newtheorem{definition}{Definition}[subsection]
\newtheorem{theorem}{Theorem}[subsection]
\newtheorem{corollary}{Corollary}[subsection]
\newtheorem{lemma}[theorem]{Lemma}
\newtheorem{proposition}{Proposition}[section]
\newtheorem{axiom}{Axiome}
\newtheorem{property}{Propriété}[subsection]
\newtheorem*{remark}{Remarque}
\newtheorem*{problem}{Problème}
\newtheorem*{intuition}{Intuition}

\subsection{Pourquoi étudier l'algèbre linéaire?}

\pagebreak

\section{Systèmes d'Équations Linéaires}

\subsection{Overview}

Le but de cette section est de pouvoir résoudre des systèmes d'équations linéaires
avec l'algorithme de Gauss et de Gauss-Jordan

\subsection{Définition d'un système d'équations linéaires}
\subsection{Opérations élémentaires sur les lignes}

Il existe 3 opérations élémentaires sur les matrices:
\begin{enumerate}
    \item Multiplier la ligne i par une constante $k \neq 0$ : $L_i \rightarrow
	k L_j , k \neq 0$
    \item Permuter les lignes i et j : $ L_i \leftrightarrow L_j$
    \item Ajouter à la linge i un multiple d de la ligne j : $ L_i \rightarrow
	L_i + d L_j$
\end{enumerate}

\subsection{Forme des matrices}

Lorsqu'on essait de résoudre un SEL, on veut transformer notre matrice en l'une
des deux formes:
\begin{enumerate}
    \item Forme échelonée : Gauss
    \item Forme échelonée réduite : Gauss-Jordan
\end{enumerate}

\subsection{Types de Solutions}

Il existe 3 types de solutions:
\begin{enumerate}
    \item Solution Unique: chaque variable est associée à un pivot
    \item Infinité de Solutions: il existe une variable libre ou plus
    \item Aucune Solutions: le système est incompatible et on a
	$ [0 \, 0 \, ... \, 0 | \, k], \, k \neq 0$
\end{enumerate}

\subsection{Méthode de Gauss}
\subsection{Méthode de Gauss-Jordan}

\begin{remark}[Méthode de Gauss vs Gauss-Jordan]
\end{remark}

\pagebreak

\section{Matrices}
\subsection{Overview}

Le but de cette section est de se familiariser algébriquement avec la notion de
matrice. Cependant, il faut toujours garder en tête qu'une matrice représente
une transformation linéaire. Essentiellement, une transformation linéaire est
une fonction qui prend un vecteur et en recrache un autre. On verra plus tard
les détails.

\subsection{Opérations Matricielles}

\subsubsection{Addition et Multiplication par un scalaire}

\begin{definition}[Égalité Matricielle]
    On dit que 2 matrices $A_{mxn}$ et $B_{pxq}$ sont égales si:
    \begin{enumerate}
	\item meme dimension: m=p et n=q
	\item composantes identiques: $^(A)_ij = (B)ij, \forall i,j$
    \end{enumerate}
\end{definition}

\begin{definition}[Addition Matricielle]
    Soit A et B, deux matrices de même dimension. La Matrice A+B est
    de même dimension et est caractérisée par $$(A+B)_ij = (A)_ij + (B)_ij$$
\end{definition}

\begin{remark}[Interprétation Géométrique de l'addition Matricielle]

\end{remark}

\begin{definition}[Multiplication par un scalaire]
    Soit A, une matrice, et r un scalaire. La matrice rA, de meme dimension,
    est donnée par $$ (rA)_ij = r(A)_ij$$
\end{definition}

\begin{remark}[Interprétation Géométrique de la multiplication Matricielle]

\end{remark}

\begin{theorem}[Propriétés de la somme et du PPS matriciel]
    Soit A,B,C des matrices et r,s des scalaires.
    \begin{enumerate}
	\item (Commutativité de l'addition) : A + B = B + A
	\item (Associativité de l'addition) : (A+B)+C=A+(B+C)
	\item (Identité de la somme) : A + 0 = A
	\item r(A+B) = rA + rB
	\item (r+s)A = rA + sA
	\item r(sA) = (rs)A
    \end{enumerate}
\end{theorem}

\begin{remark}
    Algébriquement, on sait que les matrices possèdent ces propriétés,
    car ses composantes sont des réels ou des complexes, qui sont
    des fields. Géométriquement, on sait que les matrices sont construites
    à partir de vecteurs, alors il s'agit encore de d'addition et de
    multiplication, mais sur des vecteurs au lieu de composantes.
\end{remark}

\subsubsection{Multiplication Matricielle}

\begin{definition}[Multiplication Matricielle]
    Lignes x Colonnes
\end{definition}

\begin{theorem}[Propriétés de la Multiplication Matricielle]
    Soit A,B,C, des matrices carrées et r un scalaire. Alors,
    \begin{enumerate}
	\item $ AB \neq BA$
	\item $ r(AB) = (rA)B = A(rB), r \in \mathbb{R}$
	\item $ IA = AI = A$
	\item $A(B+C) = AB + AC$
	\item $(B+C)A=BA + CA$
	\item $A(BC) = (AB)C$
    \end{enumerate}
\end{theorem}

\begin{remark}[Interprétation géométrique]
    Géométriquement, on peut interpréter la multiplication matricielle
    comme la composition de transformations linéaires, ce qui fait en
    sorte que l'ordre est important.\\
    Que peut-on dire à propos de A et B si AB=BA?
\end{remark}

\subsubsection{Transposition de Matrice}

\begin{definition}[Transposition]
    La transposition de la matrice A, notée $A^T$, est la matrice dont on
    a interchangé ses lignes et ses colonnes. $$ (A^T)_{ij} = (A)_{ji}$$
\end{definition}

\begin{theorem}[Propriétés de la Transposée matricielle]
    Soit A,B,C, des matrices carrées et r un scalaire. Alors,
    \begin{enumerate}
	\item $(A^T)^T = A$
	\item $ (A+B)^T = A^T + B^T$
	\item $ (rA)^T = r(A^T), r \in \mathbb{R}$
	\item $ (AB)^T = B^T A^T$
    \end{enumerate}
\end{theorem}

\subsection{Équation Ax=b}

\subsubsection{Combinaison Linéaire}

\begin{definition}[Combinaison Linéaire]
    On dit que le vecteur u est une combinaison linéaire des vecteurs
    $v_1, v_2, v_3$ dans l'espace vectoriel V s'il existe des scalaires
    $c_2, c_2, c_3$ tels que: $$ u = c_1 v_1 + c_2 v_2 + ... + c_n v_n$$
\end{definition}

\begin{intuition}[Interprétation Géométrique de la Combinaison Linéaire]
    On veut exprimer un vecteur quelconque comme une somme de plusieurs
    autres. C'est comme si on voulait se rendre à la destination X, mais
    qu'on s'arrêtait à pleins d'endroits.
\end{intuition}

\subsubsection{Indépendance Linéaire}

\begin{definition}[Indépendance Linéaire]
    On dit que la famille des vecteurs $(v_1, v_2, ..., v_n)$ de l'espace
    vectoriel sont linéairement indépendants $\Longleftrightarrow
    c_1 = c_2 = ... = c_n = 0$ est la seule solution pour la combinaison
    linéaire $ c_1 v_1 + c_2 v_2 + ... + c_n v_n = 0$
\end{definition}

\begin{intuition}
    On ne peut pas écrire les vecteurs de l'ensemble comme une combinaison
    linéaire des autres. Plus tard, on verra que l'indépendance linéaire
    fait en sorte qu'on a un minimum de vecteurs nécessaires pour générer
    un ensemble de vecteurs.
\end{intuition}

\subsubsection{Équation Ax=b}

\begin{theorem}[Compatibilité de Ax=b]
    Le système Ax=b est compatible si et seulement si b est une combinaison
    linéaire des colonnes de A
\end{theorem}

\begin{intuition}
    On sait que A est une matrice de transformation, ce qui veut dire que b
    est le vecteur x après avoir appliqué cette transformation. Ainsi,
    on peut s'imaginer que les colonnes de A ont été utilisées pour faire
    le doit product, qui nous donne l'étirement de chaque composantes de x.
    Cet étirement peut donc être vu comme un scalaire par lequel on
    multiplie x pour obtenir le vecteur b.
\end{intuition}

\begin{theorem}
    Le système Ax=0 possède une unique solution si et seulement si les
    colonnes de A sont linéairement indépendantes.
\end{theorem}

\begin{intuition}
    On veut multiplier les vecteurs colonnes de A par le vecteur nul pour
    obtenir zéro, ce qui est équivalent à l'indépendance linéaire.
\end{intuition}

\subsection{Inversion de Matrices}

\subsubsection{Définition et Propriétés de l'inverse}

\begin{definition}[Matrice Inversible]
    On dit qu'une matrice carrée est inversible s'il existe une matrice B
    telle que $AB=BA=I$
\end{definition}

\begin{theorem}[Unicité de l'inverse]
    Si A est une matrice inversible, alors elle est l'inverse de A, notée
    $A^(-1)$, est unique
\end{theorem}

\begin{remark}[Interprétation de l'inverse matricielle]
    L'inverse d'une matrice est l'opération qui annule la transformation
    faite par la matrice A. On sait que l'inverse existe si la
    transformation ne change pas la dimension du vecteur/matrice. On
    verra plus tard pourquoi les matrices ayant un détarminant nul ne
    sont pas inversibles.
\end{remark}

\begin{theorem}[Propriétés de l'inverse matriciel]
    Soit A et B, des matrices. Alors,
    \begin{enumerate}
	\item $ (A^(-1))^(-1) = A$
	\item $ (AB))^(-1) = B^-1 A^-1$
	\item $ (A^T)^-1 = (A^-1)^T$
    \end{enumerate}

\end{theorem}

\subsubsection{Calcul de l'inverse avec la méthode de Gauss-Jordan}

\begin{definition}[Algorithme de Gauss-Jordan]
    On peut calculer l'inverse d'une matrice si $$ (A \, | \, I) ~ ...
    ~ (\, I \, | \, A^(-1))$$
\end{definition}

\begin{intuition}
    On est capable de trouver l'inverse avec Gauss-Jordan puisque chaque
    opération matricielle peut être associée à une matrice élémentaire.
    C'est comme si on faisait une composition de matrices élémentaires.
    On verra plus tard pourquoi les matrices élémentaires sont inversibles.
\end{intuition}

\subsubsection{Résolution du système Ax=b avec l'inverse}

On peut trouver le vecteur x à l'aide de l'inverse de la façon suivante:
\begin{multline}
    % $ Ax = B$ \\ $ A^(-1) (Ax) = A^(-1)b$ \\ $ (A^(-1) A)x = A^(-1)b $ \\
    % $ Ix = A^(-1)b $ \\ $ x = A^(-1)b$
\end{multline}

\subsubsection{Matrices élémentaires}

\begin{definition}[Matrice élémentaire]
    Une matrice élémentaire est une matrice obtenue de la matrice en
    effectuant une seule opération élémentaire ligne
\end{definition}

\begin{proposition}[Multiplication à gauche par une matrice élémentaire]
    Soit $A_{mxn}$ une matrice quelconque et $E_{mxm}$, une matrice
    élémentaire. La matrice obtenue de A par une opération élémentaire
    ligne est égale au produit EA.
\end{proposition}

\begin{proposition}[Inverse des Matrices Élémentaires]
    Toute matrice élémentaire est inversible
\end{proposition}

\pagebreak
\section{Déterminants}
\subsection{Overview}

Géométriquement, le déterminant d'une matrice représente le ratio entre l'aire
formée par les vecteurs avant et après transormations linéaires. Intuitivement,
le déterminant nous dit si une matrice est diagonalisable ou non puisque si
$det(A)=0$, alors la transformation à changer de dimension et n'est pas inversible.

On peut aussi utiliser les déterminants pour évaluer un SEL. On compare le déterminant
de chaque vecteur avant et après transformation.

\subsection{Définition du Déterminant}

\begin{definition}[Definition du Déterminant]

\end{definition}

\begin{remark}[Interprétation Géométrique du Déterminant]
    Le déterminant mesure le ratio de l'aire (ou le volume) formé par les
    vecteurs colonnes de la matrice A avant et après la transformation.
    Si le déterminant est de zéro, alors il y a compression et on a perdu
    une dimension. La matrice est donc non inversible. Si le déterminant
    est de 1, alors la transformation n'a pas changé la longueur des
    vecteurs. On a donc une rotation ou une translation. Finalement, si
    le déterminant est négatif, on a flippé l'orientation des vecteurs.
    \begin{enumerate}
	\item si $det(A) = 0$ :
	\item si $det(A) = 1$ :
	\item si $0 < det(A) < 1$ :
	\item si $det(A) > 0$ :
    \end{enumerate}
\end{remark}

\subsection{Propriétés du Déterminant}

\subsection{Règle de Cramer}

\begin{intuition}
    La règle de Cramer est un ratio
\end{intuition}

\pagebreak

\section{Espaces et Sous-Espaces Vectoriels}
\subsection{Overview}

Un aspect important à comprendre dans l'étude de l'algèbre linéaire est la notion
d'espace. Malgré qu'on s'attarde peu à la notion d'anneau, il faut se souvenir
qu'on peut manipuler les vecteurs algébriquement, car l'espace dans lequel il se
trouve est un field, un ensemble qui comporte certaines propriétés qu'on nomme
axiomes.\\

Un deuxième aspect important est la notion de transformation linéaire. Comme mentionné
plus tôt, une matrice peut être interprétée comme une transformation, c-à-d qu'elle
est une fonction qui transforme un vecteur en un autre vecteur. Cependant, cette
transformation peut changer les propriétés de l'espace. On s'attardera donc aux
transformations qui gardent ces propriétés, transformations qu'on nomme linéaire.

Finalement, on veut être capable de changer de base. Décrire un vecteur dans
l'espace est assez arbitraire, car chaque personne peut tracer son propre quadrillage
et obtenir un vecteur différent. Ainsi, pour s'assurer de parler le même language,
on veut être capable de traduire un vecteur/transformation d'un espace à un autre.

\subsection{Espaces vectoriels sur $\mathbb{R}$}

\begin{definition}[Espaces Vectoriels sur $\mathbb{R}$]
    On appelle espace vectoriel tout ensemble non vide V constitué
    d'objets appelés vecteurs sur lequel sont définies deux opérations
    (l'addition et la multiplication par un scalaire), vérifiant les
    10 axiomes suivants:
    \begin{enumerate}
	\item $u+v \in V$
	\item $u+v = v+u$
	\item $(a+v)+w=u+(v+w)$
	\item $u+0=u$
	\item $u+(-u)=0$
	\item $ cu \in V$
	\item $ c(u+v)=cu+vu$
	\item $ (c+d)u=cu+du$
	\item $ (cd)u=c(du)$
	\item $1u=u$
    \end{enumerate}
\end{definition}

\begin{remark}
    Les réels forment un ring, c-à-d que:
    \begin{enumerate}
	\item L'opération binaire de l'addition est un groupe commutatif
	\item L'opération binaire de multiplication est un groupe commutatif
	\item Distributivité de la multiplication sur l'addition
    \end{enumerate}
\end{remark}

\begin{definition}[Sous-espaces vectoriels]
    Soit V, un espace vectoriel muni de l'addition et de la multiplication
    par un scalaire. Un sous-espace vectoriel H de V est un sous-ensemble de
    V qui vérifie les propriétés suivantes:
    \begin{enumerate}
	\item Le vecteur nul de V appartiet à H $(0 àin H)$
	\item pour tout $u,v çin H, u+v \in H$
	\item pour tout $u \in H$ et tout scalaire $c, cu \in H$
    \end{enumerate}
\end{definition}

\subsection{Transformations Linéaires}

\begin{definition}[Transformations Linéaires]
    On dit qu'une fonction $T: \mathbb{R}^n \rightarrow \mathbb{R}^n$
    est linéaire si pour tout $u,v \in \mathbb{R}^n$ et $k \in \mathbb{R}$:
    \begin{enumerate}
	\item T(ku)=kT(u)
	\item T(u+v) = T(u) + T(v)
    \end{enumerate}
\end{definition}

\begin{intuition}
    On dit qu'une transformation est linéaire si elle préserve les
    opérations dans les réels.
\end{intuition}

\begin{remark}[Pourquoi veut-on travailler avec des transformations
    linéaires]
\end{remark}

\begin{theorem}[Représentation Matricielle]
    Soit $T: \mathbb{R}^n \rightarrow \mathbb{R}^m$, une transformation
    linéaire. Alors il existe une unique matrice A, de dimension m x n
    telle que $$ T(u)=A u$$
\end{theorem}

\begin{intuition}
    La représentation matricielle nous dit que la transformation matricielle
    est équivalente à la multiplication matricielle, ce qu'on savait déjà.
\end{intuition}

\begin{proposition}[Transformations Linéaires de $\mathbb{R}^2$
    dans $\mathbb{R}^2$]
    On distingue 4 transformations linéaires principales dans
    $\mathbb{R}^2$:
    \begin{enumeration}
	\item Étirement:
	\item Homothétie:
	\item Réflexion:
	\item Rotation:
    \end{enumeration}
\end{proposition}

\begin{definition}[Composition de Transformations Linéaires]
\end{definition}

\subsection{Injectivité et Noyau}

\begin{definition}[Injectivité]
    Soit $f:X \rightarrow Y$ une fonction. On dit que f est injective si
    pour tout $x_1, x_2 \in X$ $$ x_1 \neq x_2 \Longrightarrow f(x_1)
    \neq f(x_2)$$
\end{definition}

\begin{theorem}[Injectivité d'une Transformation Linéaire]
    Soit $T: \mathbb{R}^n \rightarrow \mathbb{R}^m$ une transformation
    linéaire. $$ \text{T est injective } \Longleftrightarrow ker(T)={0}$$
\end{theorem}

\begin{definition}[Surjectivité]
    Soit $f:X \rightarrow Y$, une fonction. On dit que f est surjective
    si pour tout $y \in Y$, il existe $x \in X$ tel que $f(x)=y$
\end{definition}

\begin{theorem}[Surjectivité d'une Transformation Linéaire]
    Soit $f:X \rightarrow Y$, une fonction. $$ \text{ f est surjective }
    \Longleftrightarrow Im(f)=Y$$
\end{theorem}

\subsection{Base d'un Espace Vectoriel}

\begin{definition}[Ensemble Générateur]
    Un ensemble B de vecteurs de l'espace vectoriel V est un ensemble
    générateur de V si tout élément de V peut s'écrire comme une combinaison
    linéaire des éléments de B
\end{definition}

\begin{intuition}
    On peut former tous les vecteurs de l'espace vectoriel à l'aide des
    vecteurs de ce même ensemble. Notons qu'il peut y avoir des vecteurs
    redondants.
\end{intuition}

\begin{definition}[Base d'un espace vectoriel]
    Un ensemble B de vecteurs de l'espace vectoriel V est une base de V
    si:
    \begin{enumerate}
	\item les vecteurs de B sont linéairement indépendnats
	\item B est un ensemble générateur de V
    \end{enumerate}
\end{definition}

\begin{remark}
    Pour qu'un ensemble forme une base, il faut qu'on ait le nombre
    minimal de vecteurs pour générer l'ensemble, mais pas trop pour
    que les vecteurs soient redondants.
\end{remark}

\subsection{Système de Coordonnées dans $\mathbb{R}^n$}

\begin{definition}[Système de coordonnées]
    Soit $B=(b_1, ... , b_n)$, une base d'un espace vectoriel V. Alors
    pour tout vecteur x de V, il existe une unique famille $(c_1, ...,
    c_n)$ de scalaires telle que $$x= c_1 b_1 + ... + c_n b_n$$
    Le vecteur $[c_1 ... c_n]$ est appelé le vecteur coordonnées de x
    dans la base B et est noté $[x]_B$
\end{definition}

\begin{remark}
    Le vecteur des coordonées, aussi appelé vecteur des composantes,
    est un vecteur formé des scalaire de la combinaison linéaire.
    Intuitivement, on peut le voir comme le scalaire par lequel on
    doit multiplier chaque composantes pour obtenir x
\end{remark}

\begin{definition}[Matrice de Passage]
    Soit $B=(b_1, ... , b_n)$, une base de $\mathbb{R}^n$. La matrice
    définie par $$P_B = [b_1 ... b_n]$$ est appelée la matrice de passage
    de la base B. Pour tout vecteur $x \in \mathbb{R}^n$, on a
    l'équation $$ x = P_B [x]_B$$ De plus, la matrice $P_B$ est inversible,
    car ses colonnes forment une base, et on a l'équation
    $$ [x]_B = P_B ^(-1) x$$
\end{definition}

\begin{remark}
    L'équation $ x = P_B [x]_B$ est la même équation que $b=Ax$. On a que
    x=b, $P_B = A$ et $ [x]_B = x$. Ainsi, on peut interpréter la matrice
    de passage comme la matrice intermédiaire qu'on peut utiliser pour
    passer d'une base à une autre.
\end{remark}

\subsection{Changement de Base}

\begin{intuition}
    Caractériser un vecteur est assez arbitraire, car chaque personne
    peut choisir un quadrillage différent et obtenir un résultat
    différent. Ainsi, pour s'assurer de parler le même language, on
    veut être capable de traduire un vecteur dans un espace et dans un
    autre. On se sert de la matrice de changement de base pour passer
    d'une base à l'autre.
\end{intuition}

\begin{theorem}[Matrice de Changement de Base]
    Soit $B=(b_1, ... , b_n)$ et $C=(c_1, ... , c_n)$, deux bases de
    d'un espace vectoriel V. Il existe une unique matrice n x n, notée
    $ P_{C \leftarrow B}$ telle que pour tout $x \in V$
    $$ [x]_C = P_{C \leftarrow B}[x]_B, P_C = [[b_1]_C ... [b_n]_C] $$
    Cette matrice est appelée la matrice de changement de base de B à C
\end{theorem}

\begin{remark}
    On peut construire la matrice de transformation pour traduire un
    vecteur de la base B dans la base C en prenant la matrice construite
    par les vecteurs colonnes de la base originelle traduite dans la
    nouvelle base.\\
    En pratique, on trouve la matrice de changement de base en échelonnant
    $$ (c_1 c_2 c_3 | b_1 b_2 b_3) ~ (I | [b_1]_C [b_2]_C [b_3]_C)
    ~ (I | P_{C \leftarrow B})$$
\end{remark}

\begin{remark}[Lien entre matrice de changement de base et matrice de
    passage]

\end{remark}

\subsection{Représentation matricielle d'une transformation Linéaire de V
dans W}

\pagebreak

\section{Orthogonalité et Projections}

\subsection{Overview}

Le but du chapitre est de définir ce qu'est l'orthogonalité et déterminer
si une matrice est orthogonale, et si elle ne l'est pas, de la diagonaliser
avec le Procédé de Graham-Schmidt. Essentiellement, on veut travailler
avec une base orthonormale, c-à-d dont les vecteurs sont orthogonaux
entre eux, car les propriétés linéaires des vecteurs et de l'espace sont
gardées. De plus, travailler avec une base orthonormale est plus pratique
puisque le quadrillage forment des rectangles (et non des parallélogrammes).

\subsection{Produit Scalaire et Projection Orthogonale}

\subsubsection{Orthogonalié et Projections}

\begin{definition}[Produit Scalaire]
    Soit u et v, deux vecteur de $\mathbb{R}^n$. Le produit scalaire de u
    et v est le nombre réel $$ u \cdot v = u^T v = u_1 v_1 + u_2 v_2 +
    ... + u_n v_n$$
\end{definition}

\begin{theorem}[Propriétés du produit scalaire]
    Soit u,v et w, trois vecteurs de $\mathbb{R}^n$ et c, un nombre réel.
    Alors,
    \begin{enumerate}
	\item $ u \cdot v = v \cdot u$
	\item $ u \cdot (v+w) = u \cdot v + u \cdot w$
	\item $(cu) \cdot v = u \cdot (cv) = c(u \cdot v)$
	\item $ u \cdot u \geq 0$ et $ u \cdot u = 0 \Longleftrightarrow
	    u = 0$
    \end{enumerate}
\end{theorem}

\begin{remark}[Interprétation géométrique du produit scalaire]
    Comme le nom le dit, le produit scalaire de deux vecteurs nous donne
    un scalaire. Ce nombre nous dit à quel point le premier vecteur va
    dans la même direction que le deuxième. Une autre interprétation peut
    se faire avec la formule de l'angle, qu'on verra plus tard.
\end{remark}

\subsubsection{Norme d'un vecteur}

\begin{theorem}[Norme d'un vecteur]
    On appelle la longueur d'un vecteur (ou norme) $u \in \mathbb{R}^n$
    le nombre réel positif ou nul défini par
    $$ || u|| = \sqrt(u_1 ^2 + u_2 ^2 + ... + u_n^2) = \sqrt(u \cdot u)$$
\end{theorem}

\begin{definition}[Vecteur Unitaire]
    Un vecteur ayant une norme de 1 est appelé un vecteur unitaire.
\end{definition}

\begin{intuition}[D'ou vient la formule pour la norme d'un vecteur]
\end{intuition}

\begin{theorem}[Inégalité de Cauchy-Schwarz]
    Soit u et v, deux vecteurs de $\mathbb{R}^n$, alors
    $ | u \cdot v | \leq ||u|| ||v||$
\end{theorem}

\begin{theorem}[Propriétés de la norme]
    Soit u et v, deux vecteurs de $\mathbb{R}^n$, alors
    \begin{enumerate}
	\item $ ||u||=0 \Longleftrightarrow u=0$
	\item $ ||cu|| = |c| ||u||$
	\item $ || u+v || \leq ||u||+||v||$
    \end{enumerate}
\end{theorem}

\subsubsection{Angle entre deux vecteurs}

\begin{theorem}[Angle entre deux vecteurs]
    Soit u et v, deux vecteurs non nul de $\mathbb{R}^n$. L'angle entre
    u et v est le nombre réel $\Theta \in [0, \pi]$ tel que
    $$ cos(\Theta) = \frac{u \cdot v}{||u||||v||}$$
\end{theorem}

\begin{intuition}[D'ou vient la formule de l'angle entre 2 vecteurs]
\end{intuition}

\begin{theorem}[Vecteurs Orthogonaux]
    On dit que deux vecteurs u et v de $\mathbb{R}^n$ sont orthogonaux
    (ou perpendiculaires) si $u \cdot v = 0$
\end{theorem}

\begin{intuition}
    On sait que le produit scalaire peut être interprété comme la
    quantité du premier vecteur qui va dans la même direction que le
    deuxième. Ainsi, géométriquement, on sait que deux vecteurs orthogonaux
    ne vont pas dans la même direction, d'ou le produit vectoriel de 0
\end{intuition}

\subsubsection{Projection Orthogonale}

\begin{proposition}[Projection Orthogonale]
    Soit $ u \neq 0$ et y, deux vecteurs de $\mathbb{R}^n$. La projection
    orthogonale de y sur u est le vecteur donné par
    $$ ŷ = proj_u y = \frac{y \cdot u}{u \cdot u} u $$
\end{proposition}

\begin{intuition}[D'ou vient la projection orthogonale]
    Soit $ŷ=ku, y=ŷ+z, z \perp u$. On a que
    \begin{enumerate}
	\item $ŷ=ku$ :
	\item $y=ŷ+z = ku +z$
	\item $ z \perp u \Longleftrightarrow z \cdot u=0$
    \end{enumerate}
\end{intuition}

\subsection{Sous-Espace Orthogonal}

\begin{definition}[Sous-Ensemble Orthogonal]
    Soit W, un sous-ensemble de $\mathbb{R}^n$. L'ensemble des vecteurs
    orthogonaux à tous les vecteurs de W est appelé le sous-ensemble
    orthogonal de W. Il est noté
    $$ W^\perp = {v \in \mathbb{R}^n | \forall _{w \in W} v \perp w } $$
\end{definition}

\begin{intuition}
    C'est la famille des vecteurs orthogonaux à tous les vecteurs de la
    famille W
\end{intuition}

\begin{problem}[Trouver $W^\perp$]
    Si on veut déterminer $W^\perp$, on n'a qu'à résoudre le système
    d'équations linéaires formés de vecteurs de $W_i \cdot v = 0$
\end{problem}

\begin{definition}[Sous-Espace Orthogonal]
    Soit W, un sous-ensemble de $\mathbb{R}^n$. Alors $W^\perp$ est un
    sous-espace vectoriel
\end{definition}

\begin{intuition}
    On sait que $W^\perp$,car ses vecteurs sont linéairement indépendants
\end{intuition}

\begin{theorem}[Description d'un sous-espace orthogonal]
    Un vecteur appartient à $W^\perp \Longleftrightarrow$ si il est
    orthogonal à tous les vecteurs d'une famille génératrice de W
\end{theorem}

\subsubsection{Famille Orthogonale}

\begin{definition}[Famille Orthogonale]
    On dit qu'une famille de vecteurs $B =(u_1, u_2, ... , u_p)$ de
    $\mathbb{R}^n$ est orthogonale si pour tout $i \neq j, u_i$ est
    orthogonal à $u_j(u_i \cdot u_j =0)$
\end{definition}

\begin{intuition}
    Une famille orthogonale est une famille dont tous les vecteurs
    sont orthogonales entre eux c-à-d qu'ils sont à 90 degrés entre eux.
    Travailler avec une famille orthogonale est plutôt pratique, car
    notre quadrillage est rectangulaire (et non pas des parallélogrammes)
\end{intuition}

\begin{problem}[Montrer que la famille B est orthogonale]
    Pour montrer qu'une famille est orthogonale, on veut montrer
    que chaque paire de vecteurs de la famille est orthogonale
    entre elles, c-à-d que $u_i \cdot u_j =0$
\end{problem}

\begin{theorem}[Base Orthogonale]
    Une famille orthogonale de vecteurs non nuls est une base de
    l'espace qu'elle engendre
\end{theorem}

\begin{intuition}
    On sait que les vecteurs d'une famille orthogonale sont perpendiculaires
    entre eux, ce qui veut aussi dire qu'ils sont linéairement indépendants
    et forment donc une base
\end{intuition}

\begin{theorem}[Coordonnées dans une base orthogonale]
    Soit $B =(u_1, u_2, ... , u_p)$, une base orthogonale d'un sous-espace
    W de $\mathbb{R}^n$. Pour tout $y \in W$,
    $$ y = proj_{u_1}y + ... + proj_{u_p}y = \sum proj_{u_i}y =
    \sum (\frac{y \cdot u_i}{u_i \cdot u_i}) u_i$$
\end{theorem}

\begin{intuition}

\end{intuition}

\begin{remark}[Illustration du théorème]

\end{remark}


\begin{problem}[Trouver les coordonnées de y dans la base B]

\end{problem}

\pagebreak

\section{Diagonalisation}

\subsection{Overview}

La diagonalisation est un processus utilisé dans le domaine du Machine Learning,
car elle permet de "compresser" les données dans une plus petite dimension. La
diagonalisation est notamment utilisée pour le PCA et le SVD.\\

Géométriquement, lorsqu'on parle de valeurs propres et de vecteurs propres, c'est
qu'il existe un vecteur qui reste sur le même axe après transformation linéaire.
Ainsi, appliquer la transformation linéaire sur ce vecteur peut être considérer
comme une multiplication par un scalaire $\lambda$, qui étire la longueur du vecteur.
Notre but est donc de trouver le vecteur qui reste dans le même axe, qu'on appelle
vecteur propre, et de trouver le facteur de multiplication, qu'on appele valeur
propre.

\subsection{Valeurs et Vecteurs Propres}

\subsection{Espace Propre et Multiplicité Géométrique}

\subsection{Diagonalisation d'une matrice quelconque}

\subsection{Diagonalisation d'une matrice symétrique}

\pagebreak

\section{Ressources}%
\label{sec:Ressources}

\begin{enumerate}
    \item Nathaniel Johnston
    \item Kimberly Brehm
    \item Mathema-TICS
\end{enumerate}


\end{document}
\end{article}
