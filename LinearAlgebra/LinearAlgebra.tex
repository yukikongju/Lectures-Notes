\documentclass{article}
\usepackage{amsmath}
\usepackage{amsfonts}
\usepackage{amsthm}
\usepackage{parskip}
\usepackage{textgreek}
\begin{document}
\title{Lecture Notes for Linear Algebra}
\author{Emulie Chhor}
\maketitle

\section{Introduction}

Linear Algebra is the study of vectors and its transformation into space, and has
several applications in Engineering and Computer Science. For example, matrices
are utilize when applying masks in Machine Learning and Linear transformation are
used for computer graphics. Lecture Videos are from Mathema-TICS and
from Kimberly Brehm.\\

Linear Algebra is often taught following this sequence:

    \begin{enumerate}
	\item Linear Equations
	\item Matrix Algebra
	\item Determinants
	\item Vector Spaces
	\item Eigenvalues and Eigenvectors
	\item Orthogonality and Least Squares
    \end{enumerate}

Professor tend to emphasize on the algebraic component of linear algebra, but we
always have to keep in mind what the structures and objects represent graphically.

\newtheorem{definition}{Definition}[subsection]
\newtheorem{theorem}{Theorem}[subsection]
\newtheorem{corollary}{Corollary}[subsection]
\newtheorem{lemma}[theorem]{Lemma}
\newtheorem*{remark}{Remark}

\part{Linear Equations}

\subsection{Overview}

\subsection{Solving Linear Systems}

% \subsubsection{Overview}

The first chapter of linear algebra focusses on how we can use the Gaussian
method to solve linear systems. We want to determine if the system is compatible,
and if it is, we want to verify id the solution is unique or more.

\subsubsection{Matrix Notation}

A common representation of linear system is with augmented matrix.

\subsubsection{Types of Solutions}

When we solve linear systems, we often transform it into its matrix notation.
However, it can be useful to think about it in terms of vectors or plane that
crosses one another.\\

A SEL can have different types of solutions:

\begin{enumerate}
    \item No solution: Lines are parallel (geometrically); contradiction (algebraically)
    \item Exactly one solution: Lines crosses each other; all variables are associated to one pivot
    \item Infinitely many solutions: Lines are on top of each other; we have a free varaible
\end{enumerate}

\begin{definition}[Consistent and Inconsistent System]
    A system of linear equations is said to be consistent if it has one or more
    solutions; it is inconsistent if it has no solution.
\end{definition}

\subsubsection{Matrix Operations}

To solve a linear system, we want to use the elementary operations.

\begin{enumerate}
    \item (Replacement) Replace one row by the sum of itself with a multiple
	of another row
    \item (Interchange) Interchange two rows
    \item (Scaling) Multiply all entries in a row by a nonzero constant
\end{enumerate}

\begin{definition}[Pivot]
\end{definition}

\begin{definition}[Free Variables]
\end{definition}

\begin{definition}[Echelon Matrix]
    We say we a matrix is in echelon form if:
    \begin{enumerate}
	\item
    \end{enumerate}
\end{definition}

\begin{definition}[Reduced Echelon Matrix]

\end{definition}

\subsubsection{Row Reduction Algorithm}

\subsection{Vector Equations}

The second goal of this chapter is to understand vectors operations and properties.
We will define them geometrically and algebrically.

\subsubsection{Vector Addition}

Geometrically, we can visualize vector addition as if we were adding vectors
at the tip of each other.

\subsubsection{Vector Properties}

Since $\mathbb{R}^n$ is a field, our vector space follows the field axiom:

\begin{enumerate}
    \item
    \item
    \item
    \item
    \item
    \item
    \item
    \item
    \item
\end{enumerate}

\subsubsection{Linear Combination}

Linear Combination is a weighted sum of vector that allow us to generate other
vectors. Later, we will see that when vectors are independant, we can generate
new vector in another dimension.

\begin{definition}[Linear Combination]
\end{definition}

\begin{definition}[Span]
\end{definition}

\subsubsection{Matrix Equation Ax=b}

\begin{definition}[Computation of Ax]
    The product of a matrix and a vector can be interpreted as TODO
\end{definition}

\begin{theorem}[Properties of Matrix-Vector Product]

\end{theorem}

\begin{definition}[Homogeneous System]
\end{definition}

\subsubsection{Linear Independence}

Intuitively, two vectors are linearly independant if they can be expressed in
terms of the other one. We care about linear independance because it will allows
us to add another dimension to our basis when generating new vectors.

\begin{definition}[Linear Independence]
\end{definition}

\subsubsection{Linear Transfomations}

Geometrically, we can interpret a matrix as a transformation ie a function that
transform a vector into another one when we perform matrix multiplication. However,
we rather focus on linear transformation, which are transformations where the
transformed vectors keeps the same properties as it has before (addition and
multiplication)

\begin{definition}[Linear Transformation]
\end{definition}

\begin{definition}[Types of Linear Transformation]
    There exists a few notable linear transformation
    \begin{enumerate}
	\item
	\item
	\item Translation
	\item Rotation
    \end{enumerate}
\end{definition}

\subsubsection{Kernel and Image}

\begin{definition}[Kernel]
\end{definition}

\begin{definition}[Image]
\end{definition}

\part{Matrix Algebra}
\subsection{Overview}

\subsubsection{Matrix Addition}

Geometrically, we can interpret matrix addition as TODO

\begin{theorem}[Properties of Matrix Addition]
\end{theorem}

\subsubsection{Matrix Multiplication}

Matrix multiplication is often used for composition of linear transformations.
Instead of performing a bunch of transformation one after the other, we can
compute a single matrix that perform all the transformation in one shot.

\begin{definition}[Matrix Multiplication]
\end{definition}

\begin{theorem}[Properties of Matrix Multiplication]
\end{theorem}

\subsubsection{Transpose of a Matrix}

We can interpret the transpose of a matrix as TODO

\begin{definition}[Transpose of a Matrix]
\end{definition}

\begin{theorem}[Transpose Properties]
\end{theorem}

\subsubsection{Inverse of a Matrix}

Until now, we wanted to compute the matrix that represented the linear transformation
to be applied onto our vector. However, if we are given the initial vector and
the final vector, we can use the inverse of a matrix to find the linear transformation.

\begin{remark}
    We have the equation Ax=b, and we want to find the matrix A given x and b.
    With real, we would divide. However, with matrices, we cannot divide. The
    inverse matrix act similarly as a division in the real.
\end{remark}

\begin{remark}
    We know a matrix has an inverse if its determinant is not 0. Intuitively,
    the determinant measure how much the area between two vectors has
    increase/decrease. A determinant of zero would mean that the vector has
    collapse of dimension (as if we were to divide by zero)
\end{remark}

\begin{definition}[Inverse of a Matrix]
\end{definition}

\begin{theorem}[Inverse of 2x2 Matrix]
\end{theorem}

\begin{theorem}[Finding $A^-1$ with Gauss Elimination]
    $[A | I] ~ [I|A^-1]$
\end{theorem}

\begin{theorem}[The Invertible Matrix Theorem]
\end{theorem}

\subsubsection{Elementary Matrices}

Elementary matrices are important because they help us find the inverse of a
matrix. Since elementary matrices are invertible, we wil use this property
when diagonalizing our matrices (later)

\begin{definition}[Elementary Matrix]
\end{definition}

\subsubsection{LU Factorization}

The LU factorization is a technique used to solve linear systems by using the
fact that triangular matrices are easier to calculate their inverse.

\begin{theorem}[LU Factorization]
\end{theorem}

\subsubsection{Subspaces of $\mathbb{R}^n$}

Subspaces are subsets of a vector space that is also a vector space. We can think
of lines or planes. TODO

\begin{definition}[Subspace]
\end{definition}

\begin{definition}[Column Space]
\end{definition}

\begin{definition}[Null Space]
\end{definition}

\begin{theorem}[Basis]
\end{theorem}

\subsubsection{Dimension and Rank}

\begin{definition}[Coordinate System]
\end{definition}

\begin{definition}[Dimension]
\end{definition}

\begin{definition}[Rank]
\end{definition}

\begin{theorem}[Rank Theorem]
\end{theorem}

\begin{theorem}[Basis Theorem]
\end{theorem}

\begin{theorem}[The Invertible Matrix Theorem]
\end{theorem}

\part{Determinants}

\subsection{Overview}

Intuitively, the determinant of a matrix measure the ratio between the area before
and after a transformation. It helps us determine wether a matrix is inversible
and allow us to calculate the surface and volume of parallelograms.

\begin{definition}[Determinant]
\end{definition}

\begin{theorem}
    If A is a triangular matrix, the det A is the product of the entries on the
    main diagonal of A
\end{theorem}

\subsubsection{Properties of Determinant}

\begin{theorem}[Row Operations]
\end{theorem}

\begin{theorem}[Matrix Invertible]
    A square matrix A is invertible if and only if $det A \neq 0$
\end{theorem}

\begin{theorem}
    If A is an n x n matrix, then $A^T = det A$
\end{theorem}

\begin{remark}
    Intuitively, we know that the determinant of a matrix and its transpose
    is the same because the area is the same.
\end{remark}

\begin{theorem}[Multiplicative Property]
    If A and B are n x n matrices, then $det AB = (det A)(det B)$
\end{theorem}

\subsubsection{Cramer's Rule}

Cramer's Rule is another method to solve linear systems. Instead of using
Gaussian method with matrices, we use determinant instead.

\begin{theorem}[Cramer's Rule]
\end{theorem}

\begin{theorem}[Inverse Formula]
    Let A be an invertible n x n matrix. Then
    % $$ A^-1 = \fract{1}{det A} adj(A) $$
\end{theorem}

\subsubsection{Determinant as Area or Volume}

Determinant are also used to calculate the area of a parallelogram.


\part{Vector Spaces}
\subsection{Overview}

\subsubsection{Vector Space and Subspaces}

\begin{definition}[Vector Space]
\end{definition}

\begin{definition}[Subspace]
\end{definition}

\subsubsection{Null Subspaces, Column Spaces, and Linear Transfomations}

\begin{definition}[Null Space]
\end{definition}

\begin{theorem}
    The null space of an mx n matix A is a subspace of $\mathbb{R}^n$. Equivalenty,
    the set of all solutions to a system Ax=0 of m homogeneous linear equations
    in n unknowns is a subspace of $\mathbb{R}^n$.
\end{theorem}

\begin{definition}[Column Space]
\end{definition}

\begin{theorem}
    The column space of an mx n matrix A is a subspace of $\mathbb{R}^n$
\end{theorem}

\subsubsection{Kernel and Range of Linear Transfomation}

\begin{definition}[Linear Transfomation]
\end{definition}

\part{Eigenvalues and Eigenvectors}
\subsection{Overview}
\part{Orthogonality and Least Squares}
\subsection{Overview}

\end{document}
\end{article}
