\documentclass{article}
\usepackage{amsmath}
\usepackage{amsfonts}
\usepackage{amsthm}
\usepackage{parskip}
\usepackage{textgreek}
\begin{document}
\title{Lecture Notes for Complex Analysis - Richard Borcheds}
\author{Emulie Chhor}
\maketitle

\section*{Introduction}

This is my lecture notes from 1998 Field Winner Richard Borcheds' on
Complex Analysis

\newtheorem{definition}{Definition}[subsection]
\newtheorem{theorem}{Theorem}[subsection]
\newtheorem{corollary}{Corollary}[subsection]
\newtheorem{lemma}[theorem]{Lemma}
\newtheorem{proposition}{Proposition}[section]
\newtheorem{axiom}{Axiome}
\newtheorem{property}{Propriété}[subsection]
\newtheorem*{remark}{Remarque}
\newtheorem*{problem}{Problème}
\newtheorem*{intuition}{Intuition}

\section{Introduction}
\section{Arithmetic}
\section{Roots}
\section{Exp, log, sin, cos}
\section{Holomorphic functions}
\section{Harmonic functions}
\section{Integration}
\section{Cauchy's theorem}
\section{Cauchy's integral formula}
\section{Analytic continuation}
\section{Locally uniform convergence}
\section{Residue calculus}
\section{Summing series}
\section{Zeta function functional equation}
\section{Singularities}
\section{Gamma function}
\section{Maximum modulus principle}
\section{Elliptic functions}
\section{Weierstrass elliptic functions}
\section{Classification of elliptic functions}

\end{document}
\end{article}

