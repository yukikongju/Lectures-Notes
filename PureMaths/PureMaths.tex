\documentclass{article}
\usepackage{amsmath}
\usepackage{amsfonts}
\usepackage{amsthm}
\usepackage{parskip}
\usepackage{textgreek}
\begin{document}
\title{Lectures Notes from Pure Maths by Steve Warner}
\author{Emulie Chhor}
\maketitle

\section*{Introduction}

This book is an introduction to undergraduate pure mathematics. This books contains
16 chapters with exercices ranging from different levels:

\begin{enumerate}
    \item Lesson 1 - Logic: Statements and Truth
    \item Lesson 2 - Set Theory: Sets and Subsets
    \item Lesson 3 - Abstract Algebra: Semigroups, Monoids and Groups
    \item Lesson 4 - Number Theory: The Ting of Integers
    \item Lesson 5 - Real Analysis: The Complete Ordered Field of Reals
    \item Lesson 6 - Topology: The Topology of R
    \item Lesson 7 - Complex Analysis: The Field of Complex Numbers
    \item Lesson 8 - Linear Algebra: Vector Spaces
    \item Lesson 9 - Logic: Logical Arguments
    \item Lesson 10 - Set Theory: Reltions and Functions
    \item Lesson 11 - Abstract ALgebra: Strucutres and Homomorphisms
    \item Lesson 12 - Number Theory: Primes, GCD, and LCM
    \item Lesson 13 - Real Analysis: Limits and Continuity
    \item Lesson 14 - Topology: Spaces and Homeomorphisms
    \item Lesson 15 - Complex Analysis: Complex Valued Functions
    \item Lesson 16 - Linear Algebra: Linear Transformations
\end{enumerate}

\pagebreak

\newtheorem{definition}{Definition}[subsection]
\newtheorem{theorem}{Theorem}[subsection]
\newtheorem{corollary}{Corollary}[subsection]
\newtheorem{lemma}[theorem]{Lemma}
\newtheorem*{remark}{Remark}

\section{Lesson 1 - Logic: Statements and Truth}

\subsection{Overview}

This section introduces the notion of statements and truth tables, which is the
foundation of proofs.

\subsection{Statements with words}

\par
The goal of this section is to determine wether a sentence is a statement or not.\\

A statement or proposition is a sentence that can be true or false, but not both
simultaneously. If it expresses a single idea, we say it is an atomic statement.
If we want to create a statement with more than one idea, we need to connect the
atomic statement using logical connectives.

\subsection{Statements with symbols}

In mathematics, we want to express mathematical statement in symbols since they
actract away the unecessary clutter and help us focus on the form of the statement.
Consequently, we need to define a set of symbol used to define the common logical
connectives. The most common are:

\begin{enumerate}
    \item Conjunction
    \item Disjunction
    \item Negation
    \item Implication
    \item Biconditional
\end{enumerate}

\subsection{Truth Table}

\par
Each logical connective is associated to a truth table, which tell us the
truth value of a compound statement based on the truth value of the propositional
variables. Here is the truth table for the common logical connectives:\\

We can also use truth table to determine if two statement are equivalent. To do
so, we compare their truth table.

\subsection{Problem Set}
\subsubsection{Level 1}
\subsubsection{Level 2}
\subsubsection{Level 3}
\subsubsection{Level 4}
\subsubsection{Level 5}
\pagebreak

\section{Lesson 2 - Set Theory: Sets and Subsets}

\subsection{Overview}

This section introduce the notion of sets, subsets, union and intersection.

\subsection{Why do we care about sets}

Having a good intuition on sets allows us to regroup similar element and compare
them together. We can show that set contains some properties so that the elements
all share those properties.

\subsection{Describing Sets}

\begin{definition}[Set]
    A set is a collection of objects. The set can either be finite or infinite. We
    can describe the set based on a common characteristic among the elements of the
    set using the set builder notation. We write \(\{x|P(x)\}\), where P(x) is the common
    characteristic.
\end{definition}

\begin{definition}[Axiom of Extensionality]
    Two sets are equivalent if they contain the same element, we write:
    $$ \forall x(x \in A \leftrightarrow x \in B) $$
\end{definition}

\begin{definition}[Cardinality]
    The cardinality of an element is the number of different element in the set. For
    example, the set S={1,2,3} has the same cardinality as the set T={1,2,2,3}
\end{definition}

\begin{theorem}[Fence-Post Formula]
    To count the number of integers in a set, we use the fence-post formula
    $$ n - m + 1 $$
\end{theorem}

\subsection{Subsets}

\begin{definition}[Subset]
    We say that A is a subset of B if every element of A is an element of B. We write
    \( A \subseteq B\)
    $$ \forall x(x \in A \rightarrow x \in B) $$
\end{definition}

\begin{theorem}
    Every set A is a subset of itself
    $$ \forall x(x \in A \rightarrow x \in A) $$
\end{theorem}

\begin{theorem}
    The empty set is a subset of every set
    $$ \forall x(x \in \emptyset \rightarrow x \in A) $$
\end{theorem}

\begin{theorem}[Transitivity of sets]
    Let A, B, C be sets such that $$ A \subseteq B, B \subseteq C. Then, A \subseteq C $$
\end{theorem}

\begin{theorem}
    There are \(2^n\) subsets in a set
\end{theorem}

\subsection{Unions and Intersections}

\begin{definition}[Union]
    The union of the sets A and B, written \(A \cup B\), is the set of elements that
    are in A or B (or both).
    $$ \forall x(x | x \in A \lor x \in B) $$
\end{definition}

\begin{definition}[Intersection]
    The intersection of the sets A and B, written \(A \cap B\), is the set of
    elements that are in A and B simultaneously.
    $$ \forall x(x | x \in A \land x \in B) $$
\end{definition}

\begin{theorem}
    If A and B are sets, then \( A \subseteq A \cup B\)
\end{theorem}

\begin{theorem}
    \(B \subseteq A \iff A \cup B = A\)
\end{theorem}

\begin{theorem}
    \(B \subseteq A \iff A \cap B = B\)
\end{theorem}

\begin{definition}[Reflexive]
    A relation R is reflexive if  $ \forall x(x R x) $
\end{definition}

\begin{definition}[Symmetric]
    A relation R is symmetric if $$\forall x \forall y(x R y \rightarrow y R x)$$
\end{definition}

\subsection{Problem Set}
\subsubsection{Level 1}
\subsubsection{Level 2}
\subsubsection{Level 3}
\subsubsection{Level 4}
\subsubsection{Level 5}
\pagebreak

\section{Lesson 3 - Abstract Algebra: Semigroups, Monoids and Groups}
\subsection{Overview}

This section focus on the properties of semigroups, monoids and groups. It gives
us an intuition on why some set behave a certain way while other don't.

To determine wether a set has certain properties, we often use a multiplication
table.

\subsection{Binary Operations and Closure}

\begin{definition}[Binary Operation]
    A binary operation on a set is a rule that combines two elements of the set to
    produce another element of the set
\end{definition}

\begin{definition}[Closed]
    We say that the set S is closed under the partiel binary operation * if
    whenever \(a,b \in S\), we have \(a * b \in S \)
\end{definition}

\subsection{Semigroups and Associativity}

\begin{definition}[Associativity]
    Let * be a binary operation on a set. We say that * is associative in S if
    for all x, y, z in S, we have
    $$ x * (y * z) = (x * y) * z $$
\end{definition}


\begin{definition}[Semigroup]
    A semigroup is a pair (S,*), where S is a set and * is an associative binary
    opertaion on S
\end{definition}

\begin{corollary}
    If the binary operator * is not associative in S, then the pair (S,*) is not
    a semigroup
\end{corollary}

\begin{definition}[Abelian or Commutative]
    Let * be a binary operation on a set. We say that * is abelian (or commutative)
    in S if for all x, y, z in S, we have
    $$ x * y = y * x $$
\end{definition}

\begin{definition}[Abelian Semigroup]
    An abelian semigroup is a semigroup that is commutative. Therefore, it has the
    following properties:
    \begin{enumerate}
	\item Closed
	\item Associative
	\item Commutative
    \end{enumerate}
\end{definition}

\subsection{Monoids and Identity}

\begin{definition}[Identity]
    Let (S,*) be a semigroup. An element e of S is called an identity with respect
    of the binary operation * if for all \( a \in S \), we have \( a * e = e * a = a \)
\end{definition}

\begin{definition}[Monoid]
    A monoid is a semigroup with an identity. In other word, a monoid is
    \begin{enumerate}
	\item Closed
	\item Associative
	\item Identity
    \end{enumerate}
\end{definition}

\begin{theorem}[Unique Identity]
    Let (M, *) be a monoid with identity e. The identity element is unique
\end{theorem}

\subsection{Groups and Inverses}

\begin{definition}[Inverse]
    Let (M,*) be a monoid with identity e. An element a of M is called invertible
    if there is an element \( b \in M \) such that \( a * b = b * a = e \)
\end{definition}

\begin{definition}[Group]
    A group is a monoid in which every element is invertible. Therefore, a group
    follows the following properties
    \begin{enumerate}
	\item Closed
	\item Associative
	\item Identity
	\item Inversible
    \end{enumerate}
\end{definition}

\begin{theorem}[Unique Inverse]
    Let (G, *) be a group. Each element in G has a unique inverse
\end{theorem}

\subsection{Problem Set}
\subsubsection{Level 1}
\subsubsection{Level 2}
\subsubsection{Level 3}
\subsubsection{Level 4}
\subsubsection{Level 5}
\pagebreak

\section{Lesson 4 - Number Theory: The Ring of Integers}

\subsection{Overview}

The goal of this section is to familiarise ourselves with induction proofs. However,
the proofs we have to prove utilize integers properties, so we have to define
ring properties first.

The notion of ring utilize the concepts of closure, associativity, abelian,
identity and inverses which we saw in the previous section.

We want to understand the properties of the common set $\mathbb{N}$, $\mathbb{Z}$,
$\mathbb{Q}$, $\mathbb{R}$, as it will allow us to use those properties to work
with more complex objects such as vectors or complex numbers.

\subsection{Ring and Distributivity}

\begin{definition}[Commutative Group]
    A commutative group is a group that follows the following properties:
    \begin{enumerate}
	\item Closure
	\item Associative
	\item Commutative
	\item Identity
	\item Inverse
    \end{enumerate}
\end{definition}

\begin{lemma}
    $ (\mathbb{Z} , +) $ is a commutative group
\end{lemma}

\begin{definition}[Commutative Monoid]
    A commutative monoid is a monoid that follows the following properties:
    \begin{enumerate}
	\item Closure
	\item Associative
	\item Commutative
	\item Identity
    \end{enumerate}
\end{definition}

\begin{lemma}
    $ (\mathbb{Z} , \cdot) $ is a commutative monoid
\end{lemma}

\begin{definition}[Ring]
    A ring is a triple $ (R,+, \cdot) $ where R is a set, + and $ \cdot $ are binary
    operations on R that satisfies:
    \begin{enumerate}
	\item (R, +) is commutative group
	\item (R, $ \cdot $) is a commutative monoid
	\item Multiplication is distributive over addition in R. That is, for all
	    x,y,z $\in $ R, we have
	    $$ x \cdot (y+z) = x \cdot y + x \cdot z \text{ and }
	    (y+z) \cdot x = y \cdot x + z \cdot x $$
    \end{enumerate}

    It should be noted that the properties that define a ring are called the ring axioms
\end{definition}

\begin{lemma}
    $ (\mathbb{Z} , +, \cdot) $ is a commutative ring
\end{lemma}

\begin{lemma}
    $ (\mathbb{N} , +, \cdot) $ is a ring because $ (\mathbb{N} , +) $ is not a group.
    We say it is a semiring
\end{lemma}

\subsection{Divisibility}

\begin{definition}[Even]
    An integer a is called even if there is another integer b such that a = 2b
\end{definition}

\begin{definition}[Odd]
    An integer a is called odd if there is another integer b such that a = 2b + 1
\end{definition}

\begin{definition}[Sum]
    We define the sum of integers a and b to be a + b
\end{definition}

\begin{definition}[Product]
    We define the product of integers a and b to be a $ \cdot $ b
\end{definition}

\begin{theorem}
    The sum of two even integer is even
\end{theorem}

\begin{theorem}
    The product of two integers that are each divisible by k is also divisible by k
\end{theorem}

\subsection{Induction}

\begin{definition}[Well Ordering Principle]
    The Well Ordering Principle says that every nonempty subset of natural numbers
    has a least element
\end{definition}

\begin{theorem}[Principle of Mathematical Induction]
    Let S be a set of natural numbers such that
    \begin{enumerate}
	\item $ 0 \in S $
	\item $ \text{for all k } \in \mathbb{N} , k \in S \rightarrow k+1.
	    \text{ Then, } S= \mathbb{N} $
    \end{enumerate}
\end{theorem}

\begin{lemma}[Standard Advanced Calculus Trick]
    We can add and substract the same quantities without changing the result
\end{lemma}

\subsection{Problem Set}
\subsubsection{Level 1}
\subsubsection{Level 2}
\subsubsection{Level 3}
\subsubsection{Level 4}
\subsubsection{Level 5}
\pagebreak

\section{Lesson 5 - Real Analysis: The Complete Ordered Field of Reals}
\subsection{Overview}

The goal of this section is to define the set of numbers. We are introduced
(Q, +, $\cdot$) which is an ordered field, but we are shown that we can't
generate all the numbers with that set. We need the irrational numbers.

\subsection{Field}

\begin{definition}[Field]
    A field is a triple (F, +, $\cdot $), where F is a set and + and $ \cdot$
    are binary operations on F satisfying:
    \begin{enumerate}
	\item (F, +) is a commutative group
	\item (F, $\cdot $) is a commutative group
	\item Multiplication is distributive over addition in F. That is, for all
	    x,y,z $\in $ F, we have
	    $$ x \cdot (y+z) = x \cdot y + x \cdot z \text{ and }
	    (y+z) \cdot x = y \cdot x + z \cdot x $$
	\item 0 $\neq$ 1
    \end{enumerate}

    The properties that define a field are called the field axioms
\end{definition}

\begin{lemma}[Set of Natural Numbers]
    The set $ \mathbb{N} $ is the set of natural numbers and the structure
    ($ \mathbb{N} $ , +, $\cdot $) is a semiring
\end{lemma}

\begin{lemma}[Set of Integers]
    The set $ \mathbb{Z} $ is the set of integers and the structure
    ($ \mathbb{Z} $ , +, $\cdot $) is a ring
\end{lemma}

\begin{lemma}[Set of Rational Numbers]
    The set $ \mathbb{Q} $ is the set of rational numbers and the structure
    ($ \mathbb{Q} $, +, $\cdot $) is a field
\end{lemma}

\begin{definition}[Substraction]
    If a,b $ \in $ F, we define the substraction a-b= a+(-b)
\end{definition}

\begin{definition}[Division]
    If a,b $ \in $ F and b $\neq$0, we define the division $a/b= ab^-1$
\end{definition}

\subsection{Ordered Rings and Fields}

\begin{definition}[Positive and Negative Elements]
    If a $\in$ P, we say that a is positive and if -a $\in$ P, we say that a is
    negative
\end{definition}

\begin{definition}[Ordered Ring]
    We say that a ring (R,+, $\cdot$) is ordered if there is a nonempty subset
    P of R, called the set of positive elements of R satisfying the folowing
    properties
    \begin{enumerate}
	\item if a,b $\in$ P, then a + b $\in$ P
	\item if a,b $\in$ P, then ab $\in$ P
	\item if a $\in$ P, then exactly one of the following holds:
	    $$ a \in P, a=0, \text{or} -a \in P $$
    \end{enumerate}
\end{definition}

\begin{remark}
    Since P is the set of positive element, the following are equivalent:
    \begin{enumerate}
	\item $a \in P \Longleftrightarrow a \geq 0$
	\item $-a \in P \Longleftrightarrow a \leq 0$
	\item $ a \leq b \Longleftrightarrow a<b or a=b$
	\item $ a \geq b \Longleftrightarrow a>b or a=b$
    \end{enumerate}
\end{remark}

\begin{definition}[Nonnegative Number]
    Let x be a non negative number. Then, a is positive or zero
\end{definition}

\begin{theorem}
    $ (\mathbb{Q}, +, \cdot) $ is an ordered field
\end{theorem}

\begin{theorem}
    Let (F, $\leq$ ) be an ordered field. Then, for all x $\in$ F*, $ x \cdot x > 0 $
\end{theorem}

\begin{theorem}
    Every ordered field (F, $\leq$ ) contains a copy of the natural numbers.
\end{theorem}

\begin{theorem}
    Let (F, $\leq$ ) be an ordered field and let $ x \in F$ with $ x > 0$.
    Then, $ \frac{1}{x} > 0 $
\end{theorem}

\subsection{Why Isn't $\mathbb{Q}$ Enough?}

\begin{theorem}[Pythagorean Theorem]
    In a right triangle with legs of length a and b, and a hypotenuse of length c
    $$ c^2 = a^2 + b^2 $$
\end{theorem}

\begin{theorem}
    There does not exist a rational number a such that $a^2=2$
\end{theorem}

\begin{remark}
    Proving a number is irrational is done by assuming the rational can be
    expressed as a irreductible fraction gcd(p,q)=1 and showing that the
    smallest divisor is not 1
\end{remark}

\subsection{Completeness}

\begin{definition}[Upper Bound]
    Let (F, $\leq$ ) be an ordered field and let S be a nonempty subset of F.
    We say that S is bounded above if there is $ M \in F$ such that for all
    $s \in S, s \leq M$. Each number M is called an upper bound of S
\end{definition}

\begin{definition}[Lower Bound]
    Let (F, $\leq$ ) be an ordered field and let S be a nonempty subset of F.
    We say that S is bounded below if there is $ K \in F$ such that for all
    $s \in S, K \leq s$. Each number K is called an lower bound of S
\end{definition}

\begin{definition}[Bounded Set]
    We say that S is bounded if it is bounded above and bounded below. Otherwise,
    we say that S is unbounded.
\end{definition}

\begin{definition}[Supremum]
    A least upper bound of a set S is an upper bound that is smaller than any
    other upper bound of S
\end{definition}

\begin{definition}[Infimum]
    A greatest lower bound of S is a lower bound that is larger than any other
    other lower bound of S
\end{definition}

\begin{remark}
    To show that the number n is a supremum/infimum, we need to show that
    \begin{enumerate}
	\item The set S has a upper bound/lower bound
	\item There is no greater lower bound / lower upper bound by contradiction.
	    We use the fact that S < S' -> S' = S + $\varepsilon$, where
	    $\varepsilon$ > 0, which contradict the fact that S' is a upper/lower
	    bound
    \end{enumerate}
\end{remark}

\begin{definition}[Completeness Property]
    An ordered field (F, $\leq$ ) has the Completeness Property if every nonempty
    subset of F that is bounded above has a least upper upper bound in F. In this
    case, we cay that (F, $\leq$ ) is a complete ordered field.
\end{definition}

\begin{remark}
    The proof is done by contradiction. We assume that the naturals are bounded
    (we know that they are not because $\mathbb{N}$ is not an ordered field?) and
    we prove that there is a least upper bound that is not a upper bound using
    the compleness property.
\end{remark}

\begin{corollary}
    Every nonempty set of real numbers that is bounded below has a greatest
    lower bound (infimum)
\end{corollary}

\begin{theorem}
    There is exactly one complete ordered field
\end{theorem}

\begin{theorem}[Archimedian Property of $\mathbb{R}$]
    For every $x \in \mathbb{R}, there is n \in \mathbb{N} such that n > x $
\end{theorem}

\begin{corollary}
    Let $x<y$. There exist $n \in \mathbb{N}$ such that $nx>y$
\end{corollary}

\begin{remark}
    The Archimedian property tell us that the set of natural numbers is infinite.
    Therefore, if we choose any arbitrary natural, we can find another natural
    that can be smaller or greater than it.
\end{remark}

\begin{theorem}[Density Theorem]
    If $x, y \in \mathbb{R} \text{with} x<y$ then there is $q \in \mathbb{Q}
    \text{ with } x<q<y $
\end{theorem}

\begin{remark}
    The Density theorem tells us that we can find a rational number between any
    two real numbers. Intuitively, we can think that is because there is infintely
    many numbers in the interval, so there must be a rational number in it.
\end{remark}

\begin{remark}
    To prove the density theorem, we want to construct a rational number with
    the Archimedian and the Well Ordering Principle. TODO
\end{remark}

\begin{corollary}
    There is a real number that is not rational between any two real numbers.
\end{corollary}

\subsection{Problem Set}
\subsubsection{Level 1}
\subsubsection{Level 2}
\subsubsection{Level 3}
\subsubsection{Level 4}
\subsubsection{Level 5}
\pagebreak

\section{Lesson 6 - Topology: The Topology of R}

\subsection{Overview}

\subsection{Intervals of Real Numbers}

\begin{definition}[Interval]
    A set I of real numbers is called an interval id any real number that lies
    between two numbers in I is also in I. We write:
    $$ \forall x, y \in I, \forall z \in \mathbb{R}, \text{if x is less than z
    and z is less than y, the z is in I} $$

    Here is a list of the other types of intervals:
    \begin{enumerate}
	\item Open Interval
	\item Closed Interval
	\item Half-open Interval
	\item Infinite Open Interval
	\item Infinit Closed Interval
    \end{enumerate}

\end{definition}

\begin{theorem}
    If an interval I is bounded, the there are a,b $\in \mathbb{R} $ such that
    one of the following holds:
    $$ I=(a,b), I=(a,b], \text{ or } I=[a,b) $$
\end{theorem}

\subsection{Operations on Sets}

\begin{definition}[Union]
    The union of the sets A and B, written \(A \cup B\), is the set of elements that
    are in A or B (or both).
    $$ \forall x(x | x \in A \lor x \in B) $$
\end{definition}

\begin{definition}[Intersection]
    The intersection of the sets A and B, written \(A \cap B\), is the set of
    elements that are in A and B simultaneously.
    $$ \forall x(x | x \in A \land x \in B) $$
\end{definition}

\begin{definition}[Difference]
    The difference A \ B is the set of elements that are in A and not in B
    $$ A \ B = {x|x \in A \text{ and } x \notin B} $$
\end{definition}

\begin{definition}[Symmetric Difference]
    The symmetric difference A $ \triangle $ B is the set of elements that are
    in A or B, but not both
    $$ A \triangle B = (A \ B) \cup (B \ A) $$
\end{definition}

\begin{theorem}
    The operation of forming unions is associative
\end{theorem}

\subsection{Open and Closed Sets}

\begin{definition}[Open Set]
    A subset X of $\mathbb{R}$ is open if for every real number $ x \in \mathbb{R}$,
    there is an open interval (a,b) with $ x \in (a,b) \text{ and } (a,b) \subseteq X$
\end{definition}

\begin{definition}[Closed Set]
\end{definition}

\begin{theorem}
    Let $a \in \mathbb{R}$ The infinite interval (a, $\infty$) is an open set
\end{theorem}

\begin{theorem}
    $\emptyset $ and $\mathbb{R}$ are both open sets
\end{theorem}

\begin{theorem}
    A subset X of $\mathbb{R}$ is open if and only if for every real number
    $x \in X$, there is a positive real number c suche that $(x-c, x+c) \subseteq X$
\end{theorem}

\begin{theorem}
    The union of two open sets in $\mathbb{R}$ is an open set in $\mathbb{R}$
\end{theorem}

\begin{theorem}
    Let X be a set of open subsets of $\mathbb{R}$. Then UX is open
\end{theorem}

\begin{theorem}
    Every open set in $\mathbb{R}$ can be expressed as a union of bounded open
    intervals
\end{theorem}

\begin{theorem}
    The intersection of two open sets in $\mathbb{R}$ is an open set in $\mathbb{R}$
\end{theorem}

\begin{theorem}
    The intersection of two closed sets in $\mathbb{R}$ is a closed set in $\mathbb{R}$
\end{theorem}

\subsection{Problem Set}
\subsubsection{Level 1}
\subsubsection{Level 2}
\subsubsection{Level 3}
\subsubsection{Level 4}
\subsubsection{Level 5}
\pagebreak

\section{Lesson 7 - Complex Analysis: The Field of Complex Numbers}
\subsection{Overview}

We should alway keep in mind wether we are in a field or ring when working with
linear and quadratic equation.

\subsection{A Limitation of the Reals}

\begin{definition}[Linear Equation]
    A linear equation has the form ax+b=0.
\end{definition}

\begin{definition}[Quadratic Equation]
    A quadratic equation has the form $a^2 + bx + c =0 $, where $a \neq 0$
\end{definition}

\subsection{The Complex Field}

\begin{definition}[Standard From of a Complex Number]
    The standard form of a complex number is a + bi, where a and b are real numbers.
    The set of complex number is $\mathbb{C} = {a + bi | a,b \in \mathbb{R}}$
\end{definition}

\begin{definition}[The Complex Plane]
    We can visualize a complex number as a point in the Complex Plane, which has
    a real axis (in x) and an imaginary axis (in y). The point (0,0) is called
    the origin\\

    The Complex plane allow us to visualize a complex number as a vector. If
    z is a complex number such as z=a+bi, we call a the real part of z and b the
    imaginary part of z. We write a = Re z and b = Im z
\end{definition}

\begin{definition}[Equality]
    Two complex numbers are equal if and only if they have the same real and
    imaginary part.
    % $$ a + bi = c + di \iif a = c \text{ and } b = d $$
\end{definition}

\begin{definition}[Addition]
    We can add two complex numbers by adding their real and imaginary parts.
    $$ (a+bi) + (c+di) = (a+c) + (b+d) i$$
\end{definition}

\begin{definition}[Substraction]
    We can find the difference of two complex numbers by substracting their real
    and imaginary parts.
    $$ (a+bi) - (c+di) = (a-c) + (b-d) i$$
\end{definition}

\begin{definition}[Division]
    Let z and w be complex numbers such that $z \in \mathbb{C}$ and
    $w \in \mathbb{C}*$. We define the quotient $\frac{z}{w}$ by
    % $$ \frac{z}{w} = z w^-1 = (a+bi) (\frac{c}{c^2 + d^2} - \fract{d}{(c^2 + d^2)i}) $$
\end{definition}

\begin{definition}[Conjugate]
    The conjugate of the complex number z=a+bi is the complex number
    $\overline{z} = a - bi $
\end{definition}

\begin{definition}[Real Number]
    Let z be a complex number such that z=a+bi. If b=0, then we call z a real
    number.
\end{definition}

\begin{definition}[Pure Imaginary Number]
    Let z be a complex number such that z=a+bi. If a=0, then we call z a pure
    imaginary number
\end{definition}

\begin{theorem}
    $i^2 = -1$
\end{theorem}

\begin{theorem}
    $ (\mathbb{C}, +, \cdot)$ is a field
\end{theorem}

\begin{corollary}
    $ (\mathbb{R}, +, \cdot)$ is a subfield of $ (\mathbb{C}, +, \cdot)$
\end{corollary}

\begin{theorem}
    The field of complex numbers cannot be ordered
\end{theorem}

\subsection{Absolute Value and Distance}

\begin{definition}[Square Root]
    If x and y are real or complex numbers such that $y=x^2$, the we call x a
    sqaure root of y. If x is a positive real number, then we say that x is the
    positive square root of y and we write $x=\sqrt(y)$
\end{definition}

\begin{definition}[Modulus of a Complex Number]
    The absolute value or the modulus of the complex number z=a+bi is the nonnegative
    real number $$ |z | = \sqrt(a^2 + b^2) = \sqrt((Re z)^2 + (Im z)^2) $$
\end{definition}

\begin{definition}[Distance between Complex Numbers]
    The distance between the complex numbers z=a+bi and w=c+di is
    $$ d(z,w) = |z-w| = \sqrt((c-a)^2 + (d-b)^2)$$
\end{definition}

\begin{theorem}[The Triangle Inequality]
    For all $z,w \in \mathbb{C}, |z+w| \leq |z| + |w| $
\end{theorem}

\subsection{Basic Topology of $\mathbb{C}$}

\begin{definition}[Circle]
    A circle in the Complex Plane is the set of all points that are at a fixed
    distance from a fixed point. The fixed distance is called the radius of the
    circle and the fixed point is called the center of the circle\\

    If a circle has radius of r>0 and center c=a+bi, then any point z=x+yi on the
    circle must satisfy $|z-c| = r$, or equivalently, $(x-a)^2 + (y-b)^2 = r^2$
\end{definition}

\begin{definition}[Open Disk]
    An open disk in $\mathbb{C}$ consists of all the points in the interior of a
    circle. If a is the center of the open disk and r is the radius of the open
    disk, then any point z inside the disk satisfies $|z-a|<r$
\end{definition}

\begin{definition}[r-neighborhood of a]
    $ N_r(a) = {z \in \mathbb{C} | |z-a| < r} $ is also called the r-neighborhood of a.
\end{definition}

\begin{definition}[Diameter]
    In $\mathbb{R}$, an r-neighborhood of a is the open interval $N_r(a)=(a-r, a+r)$
    The diameter of this interval is 2r
\end{definition}

\begin{definition}[Closed Disk]
    A closed disk is the interior of a circle together with the circle itself
    (boundary included). If a is the center of the closed disk and r is the radius
    of the closed disk, the any point z inside the closed disk satisfies
    $ |z-a| \leq r $
\end{definition}

\begin{definition}[Punctured Open Disk]
    A punctured open sidk consists of all the points in the interior of a circle
    except for the center of the circle. If a is the center of the punctured open
    disk and r is the radius of the open disk, then any point z inside the
    punctured disk satifies $|z-a|<r$ and $z \neq a$\\

    Since $z \neq a$ is equivalent to $z-a \neq 0$, then it is also equivalent to
    $ |z-a| \neq 0$. Since |z-a| must be nonnegative, then $|z-a|>0$ or $ 0<|z-a|$.

    Therfore, a puncture open disk with center a and radius r consists of all
    points z that satisfy 0<|z-a|<r
\end{definition}

\begin{definition}[Deleted r-neighborhood of a]
    $N_{r}^{\odot}(a)=\{z|0<| z-a \mid<r\}$ is also called a deleted r-neighborhood
    of a
\end{definition}

\begin{definition}[Open Subset]
    A subset X of $\mathbb{C} $ is said to be open if for every complex number
    z $\in$ X, there is an open disk D with $z \in D$ and $D \subseteq X$
\end{definition}

\begin{theorem}
    A subset X of $\mathbb{C}$ is open if and only if for every complex number
    w $\in$ X, there is a positive real number d such that $N_d(w) \subseteq X$
\end{theorem}

\begin{theorem}[Closed Subset]
    A subset X of $\mathbb{C}$ is said to be closed if the complement of X
    $\in \mathbb{C}$, noted $\mathbb{C} \ X$, is open

    The complement concist of all complex numbers not in X
\end{theorem}

\subsection{Problem Set}
\subsubsection{Level 1}
\subsubsection{Level 2}
\subsubsection{Level 3}
\subsubsection{Level 4}
\subsubsection{Level 5}
\pagebreak

\section{Lesson 8 - Linear Algebra: Vector Spaces}
\subsection{Overview}

In the previous section, we looked at three structure called fields:

\begin{enumerate}
    \item $\mathbb{Q}$ : field of rational numbers
    \item $\mathbb{R}$ : field of real numbers
    \item $\mathbb{C}$ : field of complex numbers
\end{enumerate}

And we also saw that $\mathbb{Q}$ is a subfield of $\mathbb{R}$, which is also
a subset of $\mathbb{C}$. This means that every rational number is a real number
and every real number is a complex number.

Understanding that $\mathbb{Q}$, $\mathbb{R}$ and $\mathbb{C}$ are fields is pretty
neat, since they have two operations (addition and substraction) that satisifes
closure, associativity, commutativity, identity, inverse, and distributive,
which allows us to perform high school algebra on its elements.

Consequently, since vectors are also in these fields, we can apply the field
properties on vectors.

\subsection{Vector Spaces Over Fields}

\begin{definition}[Vector Space]
    A vector space over a field $\mathbb{F}$ is a set V with a binary operation +
    on V (called addition) and an operation called scalar multiplication satisfying:
    \begin{enumerate}
	\item (V,+) is a commutative group
	\item (Closure under scalar multiplication) for all $ k \in \mathbb{F},
	    kv \in V$
	\item (Scalar multiplication Identity) If 1 is the multiplicative identity
	    $\mathbb{F} \text{ and } v \in V$, then 1v=v
	\item (Associativity of scalar multiplication) For all j,k $\in \mathbb{F}
	    \text{ and } v \in V$, (jk)v=j(kv)
	\item (Distributivity of 1 scalar over 2 vectors) For all $ k \in \mathbb{F}
	    \text{ and } v,w \in V$, k(v+w)=kv+kw
	\item (Distributivity of 2 scalars over 1 vector) For all j,k $\in
	    \mathbb{F} \text{ and } v \in V$, (j+k)v = jv + kv
    \end{enumerate}
\end{definition}

\subsection{Subspaces}

\begin{definition}[Subspace]
    Let V be a vector space over a field $\mathbb{F}$. A subset U of V is called
    subspace of V, written $ U \leq V$, if it is also a vector space with respect
    to the same operations of addition and scalar multiplication as they were
    defined in V.
\end{definition}

\begin{theorem}
    Let V be a vector space over a field $\mathbb{F}$ and let $ U \subseteq V$.
    Then $ U \leq V$ if and only if:
    \begin{enumerate}
	\item $ 0 \in U$
	\item for all $v,w \in U, v+w \in U$
	\item for all $ v\in U \text{and} k \in \mathbb{F}, kv \in U$
    \end{enumerate}
\end{theorem}

\begin{theorem}
    Let V be a vector space over a field $\mathbb{F}$ and let U and W be
    subspaces of V. Then $ U \cap W$ is a subspace of V
\end{theorem}

\subsection{Bases}

\begin{definition}[Linear Combination]
    Let V be a vector space over a field $\mathbb{F}$, let $v,w \in V$ and
    $j,k \in \mathbb{F}$. The expression jv+kw is called a linear combination
    of vectors v and w. We call the scalars j and k weights
\end{definition}

\begin{definition}[Span]
    If $v,w \in V$, where V is a vector space over a field $\mathbb{F}$, then
    the set of all linear combinations of v and w is called the span of v and w.
    Symbolically, we have  $ span{v,w} = {jv+kw|j,k \in \mathbb{F}} $
\end{definition}

\begin{theorem}
    Let $V=\mathbb{R}^2 = {(a,b)|a,b \in \mathbb{R}}$ be the vector space over
    $\mathbb{R}$ with the usual definitions of addition and scalar multiplication.
    Then span{(1,0),(0,1)} = $V=\mathbb{R}^2$
\end{theorem}

\begin{definition}[Linear Independance]
    If $v,w \in V$, where V is avector space over a field $\mathbb{F}$, then we
    say that v and w are linearly independanent if neither vector is a scalar
    multiple of the other one. Otherwise, we say that v and w are linearly dependant.
\end{definition}

\begin{theorem}
    Let V be a vector space over a field $\mathbb{F}$ and let $v,w \in V$. Then
    v and w are linearly dependent if and only if there are $j,k \in \mathbb{F}$,
    not both 0, such that jv+kw=0
\end{theorem}

\begin{theorem}
    Let $V=\mathbb{R}^{n}=\left\{\left(k_{1}, k_{2}, \ldots, k_{n}\right) \mid k_{1}, k_{2}, \ldots, k_{n} \in \mathbb{R}\right\}$ be the vector space over $\mathbb{R}$ with the
    usual definitions of addition and scalar multiplication. Then
    $$
    \operatorname{span}\{(1,0,0, \ldots, 0),(0,1,0, \ldots, 0), \ldots,(0,0,0, \ldots, 1)\}=\mathbb{R}^{n}
    $$
\end{theorem}
\subsection{Problem Set}
\subsubsection{Level 1}
\subsubsection{Level 2}
\subsubsection{Level 3}
\subsubsection{Level 4}
\subsubsection{Level 5}
\pagebreak

\section{Lesson 9 - Logic: Logical Arguments}
\subsection{Overview}

In lesson 1, we introduce the principle of statement and logical
connector to express a proposition.

\subsection{Statements and Substatements}

\begin{definition}[Substatement]
    A substatement is a statement where we dropped unecessary
    parentheses (parenthesese that don't add clarity)
\end{definition}

\subsection{Logical Equivalence}

\begin{definition}[Logical Statement]
    Let $\phi$ and $\psi$ be statements. We say that $\phi$ and $\psi$
    are logically equivalent if every truth assignment of the propositional
    variables lead to the same truth table. We write $\phi \equiv \psi$
\end{definition}

\begin{proposition}[Logical Equivalence]
    Let p,q,r be propositional variables. Here is a list of propositional
    equivalence
    \begin{enumerate}
	\item Law of double negation: $p \equiv \neg(\neg p)$
	\item De Morgan's laws: \neg(p \wedge q) \equiv \neg p \vee \neg q & \neg(p \vee q) \equiv \neg p \wedge \neg q
	\item Commutative laws:  p \wedge q \equiv q \wedge p & p \vee q \equiv q \vee p
	\item Associative laws:  (p \wedge q) \wedge r \equiv p \wedge(q \wedge r) & (p \vee q) \vee r \equiv p \vee(q \vee r)
	\item Distributive laws:  p \wedge(q \vee r) \equiv(p \wedge q) \vee(p \wedge r) & p \vee(q \wedge r) \equiv(p \vee q) \wedge(p \vee r)
	\item Identity laws: & p \wedge \mathrm{T} \equiv p & p \wedge \mathrm{F} \equiv \mathrm{F} & p \vee \mathrm{T} \equiv \mathrm{T} \quad p \vee \mathrm{F} \equiv p
	\item Negation laws: p \wedge \neg p \equiv \mathrm{F} & p \vee \neg p \equiv \mathrm{T}
	\item Redundancy laws: p \wedge p \equiv p & p \vee p \equiv p
	\item Absorption laws: (p \vee q) \wedge p \equiv p & (p \wedge q) \vee p \equiv p
	\item Law of the conditional:  p \rightarrow q \equiv \neg p \vee q
	\item Law of contrapositive: $p \rightarrow q=\neg q \rightarrow \neg p$
	\item Contrapositive:
	    $p \rightarrow q \equiv \neg q \rightarrow \neg p$
	\item Biconditional: $p \leftrightarrow q \equiv(p \rightarrow q) \wedge(q \rightarrow p)$

    \end{enumerate}
\end{proposition}

\begin{definition}[Tautology]
    A statement that has truth value T for all truth assignements is
    called a Tautology
\end{definition}

\begin{definition}[Contradiction]
    A statement that has truth value F for all truth assigments is called
    a contradiction
\end{definition}

\begin{proposition}[Law of Logical Equivalence]
    \begin{enumerate}
	\item Law of transitivity of logical equivalence: Let $\phi, \psi$,
	    and $\tau$ be statements such that $\phi \equiv \psi$ and
	    $\psi \equiv \tau$. Then $\phi \equiv \tau$
	\item Law of substitution of logical equivalents: Let $\phi, \psi$, and $\tau$ be statements such that $\phi \equiv \psi$ and $\phi$ is a substatement of $\tau$. Let $\tau^{*}$ be the sentence formed by replacing $\phi$ by $\psi$ inside of $\tau$. Then $\tau^{*} \equiv \tau$
	\item Law of substitution of sentences: Let $\phi$ and $\psi$ be statements such that $\phi \equiv \psi$, let $p$ be a propositional variable, and let $\tau$ be a statement. Let $\phi^{*}$ and $\psi^{*}$ be the sentences formed by replacing every instance of $p$ with $\tau$ in $\phi$ and $\psi$, respectively. Then $\phi^{*} \equiv \psi^{*}$
    \end{enumerate}
\end{proposition}

\subsection{Validity in Sentential Logic}
\subsection{Problem Set}
\subsubsection{Level 1}
\subsubsection{Level 2}
\subsubsection{Level 3}
\subsubsection{Level 4}
\subsubsection{Level 5}
\pagebreak

\section{Lesson 10 - Set Theory: Relations and Functions}
\subsection{Overview}
\subsection{Relations}
\subsection{Equivalence Relations and Partitions}
\subsection{Orderings}
\subsection{Functions}
\subsection{Equinumerosity}
\subsection{Problem Set}
\subsubsection{Level 1}
\subsubsection{Level 2}
\subsubsection{Level 3}
\subsubsection{Level 4}
\subsubsection{Level 5}
\pagebreak

\section{Lesson 11 - Abstract Algebra: Strucutres and Homomorphisms}
\subsection{Overview}
\subsection{Problem Set}
\subsection{Structures and Substructures}
\subsection{Homomorphisms}
\subsection{Images and Kernels}
\subsection{Normal Subgroups and Ring Ideals}
\subsubsection{Level 1}
\subsubsection{Level 2}
\subsubsection{Level 3}
\subsubsection{Level 4}
\subsubsection{Level 5}
\pagebreak

\section{Lesson 12 - Number Theory: Primes, GCD, and LCM}
\subsection{Overview}
\subsection{Prime Numbers}
\subsection{The Division Algorithm}
\subsection{GCD and LCM}
\subsection{Problem Set}
\subsubsection{Level 1}
\subsubsection{Level 2}
\subsubsection{Level 3}
\subsubsection{Level 4}
\subsubsection{Level 5}
\pagebreak

\section{Lesson 13 - Real Analysis: Limits and Continuity}
\subsection{Overview}
\subsection{Strips and Rectangles}
\subsection{Limits and Continuity}
\subsection{Equivalent Definitions of Limits and Continuity}
\subsection{Basic Examples}
\subsection{Limit and Continuity Theorems}
\subsection{Limits Involving Infinity}
\subsection{One-Sided Limits}
\subsection{Problem Set}
\subsubsection{Level 1}
\subsubsection{Level 2}
\subsubsection{Level 3}
\subsubsection{Level 4}
\subsubsection{Level 5}
\pagebreak

\section{Lesson 14 - Topology: Spaces and Homeomorphisms}
\subsection{Overview}
\subsection{Topological Spaces}
\subsection{Bases}
\subsection{Types of Topological Spaces}
\subsection{Continuous Functions and Homeomorphisms}
\subsection{Problem Set}
\subsubsection{Level 1}
\subsubsection{Level 2}
\subsubsection{Level 3}
\subsubsection{Level 4}
\subsubsection{Level 5}
\pagebreak

\section{Lesson 15 - Complex Analysis: Complex Valued Functions}
\subsection{Overview}
\subsection{The Unit Circle}
\subsection{Exponential Form of a Complex Number}
\subsection{Functions of a Complex Variable}
\subsection{Limits and Continuity}
\subsection{The Reimann Sphere}
\subsection{Problem Set}
\subsubsection{Level 1}
\subsubsection{Level 2}
\subsubsection{Level 3}
\subsubsection{Level 4}
\subsubsection{Level 5}
\pagebreak

\section{Lesson 16 - Linear Algebra: Linear Transformations}
\subsection{Overview}
\subsection{Linear Transformations}
\subsection{Matrices}
\subsection{The Matric of a Linear Transformation}
\subsection{Images and Kernels}
\subsection{Eigenvalues and Eigenvectors}
\subsection{Problem Set}
\subsubsection{Level 1}
\subsubsection{Level 2}
\subsubsection{Level 3}
\subsubsection{Level 4}
\subsubsection{Level 5}
\pagebreak

\end{document}
\end{article}
