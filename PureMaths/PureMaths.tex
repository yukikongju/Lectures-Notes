\documentclass{article}
\usepackage{amsmath}
\usepackage{amsfonts}
\usepackage{parskip}
\usepackage{textgreek}
\begin{document}
\title{Lectures Notes from Pure Maths by Steve Warner}
\author{Emulie Chhor}
\maketitle

\section*{Introduction}

This book is an introduction to undergraduate pure mathematics. This books contains
16 chapters with exercices ranging from different levels:

\begin{enumerate}
    \item Lesson 1 - Logic: Statements and Truth
    \item Lesson 2 - Set Theory: Sets and Subsets
    \item Lesson 3 - Abstract Algebra: Semigroups, Monoids and Groups
    \item Lesson 4 - Number Theory: The Ting of Integers
    \item Lesson 5 - Real Analysis: The Complete Ordered Field of Reals
    \item Lesson 6 - Topology: The Topology of R
    \item Lesson 7 - Complex Analysis: The Field of Complex Numbers
    \item Lesson 8 - Linear Algebra: Vector Spaces
    \item Lesson 9 - Logic: Logical Arguments
    \item Lesson 10 - Set Theory: Reltions and Functions
    \item Lesson 11 - Abstract ALgebra: Strucutres and Homomorphisms
    \item Lesson 12 - Number Theory: Primes, GCD, and LCM
    \item Lesson 13 - Real Analysis: Limits and Continuity
    \item Lesson 14 - Topology: Spaces and Homeomorphisms
    \item Lesson 15 - Complex Analysis: Complex Valued Functions
    \item Lesson 16 - Linear Algebra: Linear Transformations
\end{enumerate}

\pagebreak

\newtheorem{definition}{Definition}[subsection]
\newtheorem{theorem}{Theorem}[subsection]
\newtheorem{corollary}{Corollary}[subsection]
\newtheorem{lemma}[theorem]{Lemma}

\section{Lesson 1 - Logic: Statements and Truth}

\subsection{Overview}

This section introduces the notion of statements and truth tables.

\subsection{Statements with words}

\par
The goal of this section is to determine wether a sentence is a statement or not.\\

A statement or proposition is a sentence that can be true or false, but not both
simultaneously. If it expresses a single idea, we say it is an atomic statement.
If we want to create a statement with more than one idea, we need to connect the
atomic statement using logical connectives.

\subsection{Statements with symbols}

In mathematics, we want to express mathematical statement in symbols since they
actract away the unecessary clutter and help us focus on the form of the statement.
Consequently, we need to define a set of symbol used to define the common logical
connectives. The most common are:

\begin{enumerate}
    \item Conjunction
    \item Disjunction
    \item Negation
    \item Implication
    \item Biconditional
\end{enumerate}

\subsection{Truth Table}

\par
Each logical connective is associated to a truth table, which tell us the
truth value of a compound statement based on the truth value of the propositional
variables. Here is the truth table for the common logical connectives:\\

We can also use truth table to determine if two statement are equivalent. To do
so, we compare their truth table.

\subsection{Problem Set}
\subsubsection{Level 1}
\subsubsection{Level 2}
\subsubsection{Level 3}
\subsubsection{Level 4}
\subsubsection{Level 5}
\pagebreak

\section{Lesson 2 - Set Theory: Sets and Subsets}

\subsection{Overview}

This section introduce the notion of sets, subsets, union and intersection.


\subsection{Describing Sets}


\begin{definition}[Set]
A set is a collection of objects. The set can either be finite or infinite. We
can describe the set based on a common characteristic among the elements of the
set using the set builder notation. We write \(\{x|P(x)\}\), where P(x) is the common
characteristic.
\end{definition}

\begin{definition}[Axiom of Extensionality]
Two sets are equivalent if they contain the same element, we write:
$$ \forall x(x \in A \leftrightarrow x \in B) $$
\end{definition}

\begin{definition}[Cardinality]
The cardinality of an element is the number of different element in the set. For
example, the set S={1,2,3} has the same cardinality as the set T={1,2,2,3}
\end{definition}

\begin{theorem}[Fence-Post Formula]
To count the number of integers in a set, we use the fence-post formula
$$ n - m + 1 $$
\end{theorem}

\subsection{Subsets}

\begin{definition}[Subset]
We say that A is a subset of B if every element of A is an element of B. We write
\( A \subseteq B\)
$$ \forall x(x \in A \rightarrow x \in B) $$
\end{definition}

\begin{theorem}
Every set A is a subset of itself
$$ \forall x(x \in A \rightarrow x \in A) $$
\end{theorem}

\begin{theorem}
    The empty set is a subset of every set
    $$ \forall x(x \in \emptyset \rightarrow x \in A) $$
\end{theorem}

\begin{theorem}[Transitivity of sets]
Let A, B, C be sets such that $$ A \subseteq B, B \subseteq C. Then, A \subseteq C $$
\end{theorem}

\begin{theorem}
    There are \(2^n\) subsets in a set
\end{theorem}

\subsection{Unions and Intersections}

\begin{definition}[Union]
    The union of the sets A and B, written \(A \cup B\), is the set of elements that
    are in A or B (or both).
    $$ \forall x(x | x \in A \lor x \in B) $$
\end{definition}

\begin{definition}[Intersection]
    The intersection of the sets A and B, written \(A \cup B\), is the set of
    elements that are in A and B simultaneously.
    $$ \forall x(x | x \in A \land x \in B) $$
\end{definition}

\begin{theorem}
    If A and B are sets, then \( A \subseteq A \cup B\)
\end{theorem}

\begin{theorem}
    \(B \subseteq A \iff A \cup B = A\)
\end{theorem}

\begin{theorem}
    \(B \subseteq A \iff A \cap B = B\)
\end{theorem}

\begin{definition}[Reflexive]
    A relation R is reflexive if  $ \forall x(x R x) $
\end{definition}

\begin{definition}[Symmetric]
    A relation R is symmetric if $$\forall x \forall y(x R y \rightarrow y R x)$$
\end{definition}

\subsection{Problem Set}
\subsubsection{Level 1}
\subsubsection{Level 2}
\subsubsection{Level 3}
\subsubsection{Level 4}
\subsubsection{Level 5}
\pagebreak

\section{Lesson 3 - Abstract Algebra: Semigroups, Monoids and Groups}
\subsection{Overview}

This section focus on the properties of semigroups, monoids and groups. It gives
us an intuition on why some set behave a certain way while other don't.

To determine wether a set has certain properties, we often use a multiplication
table.

\subsection{Binary Operations and Closure}

\begin{definition}[Binary Operation]
    A binary operation on a set is a rule that combines two elements of the set to
    produce another element of the set
\end{definition}

\begin{definition}[Closed]
    We say that the set S is closed under the partiel binary operation * if
    whenever \(a,b \in S\), we have \(a * b \in S \)
\end{definition}

\subsection{Semigroups and Associativity}

\begin{definition}[Associativity]
    Let * be a binary operation on a set. We say that * is associative in S if
    for all x, y, z in S, we have
    $$ x * (y * z) = (x * y) * z $$
\end{definition}


\begin{definition}[Semigroup]
    A semigroup is a pair (S,*), where S is a set and * is an associative binary
    opertaion on S
\end{definition}

\begin{corollary}
    If the binary operator * is not associative in S, then the pair (S,*) is not
    a semigroup
\end{corollary}

\begin{definition}[Abelian or Commutative]
    Let * be a binary operation on a set. We say that * is abelian (or commutative)
    in S if for all x, y, z in S, we have
    $$ x * y = y * x $$
\end{definition}

\begin{definition}[Abelian Semigroup]
    An abelian semigroup is a semigroup that is commutative. Therefore, it has the
    following properties:
    \begin{enumerate}
	\item Closed
	\item Associative
	\item Commutative
    \end{enumerate}
\end{definition}

\subsection{Monoids and Identity}

\begin{definition}[Identity]
    Let (S,*) be a semigroup. An element e of S is called an identity with respect
    of the binary operation * if for all \( a \in S \), we have \( a * e = e * a = a \)
\end{definition}

\begin{definition}[Monoid]
    A monoid is a semigroup with an identity. In other word, a monoid is
    \begin{enumerate}
	\item Closed
	\item Associative
	\item Identity
    \end{enumerate}
\end{definition}

\begin{theorem}[Unique Identity]
    Let (M, *) be a monoid with identity e. The identity element is unique
\end{theorem}

\subsection{Groups and Inverses}

\begin{definition}[Inverse]
    Let (M,*) be a monoid with identity e. An element a of M is called invertible
    if there is an element \( b \in M \) such that \( a * b = b * a = e \)
\end{definition}

\begin{definition}[Group]
    A group is a monoid in which every element is invertible. Therefore, a group
    follows the following properties
    \begin{enumerate}
	\item Closed
	\item Associative
	\item Identity
	\item Inversible
    \end{enumerate}
\end{definition}

\begin{theorem}[Unique Inverse]
    Let (G, *) be a group. Each element in G has a unique inverse
\end{theorem}

\subsection{Problem Set}
\subsubsection{Level 1}
\subsubsection{Level 2}
\subsubsection{Level 3}
\subsubsection{Level 4}
\subsubsection{Level 5}
\pagebreak

\section{Lesson 4 - Number Theory: The Ring of Integers}

\subsection{Overview}

The goal of this section is to familiarise ourselves with induction proofs. However,
the proofs we have to prove utilize integers properties, so we have to define
ring properties first.

The notion of ring utilize the concepts of closure, associativity, abelian,
identity and inverses which we saw in the previous section.

\subsection{Ring and Distributivity}

\begin{definition}[Commutative Group]
    A commutative group is a group that follows the following properties:
    \begin{enumerate}
	\item Closure
	\item Associative
	\item Commutative
	\item Identity
	\item Inverse
    \end{enumerate}
\end{definition}

\begin{lemma}
    $ (\mathbb{Z} , +) $ is a commutative group
\end{lemma}

\begin{definition}[Commutative Monoid]
    A commutative monoid is a monoid that follows the following properties:
    \begin{enumerate}
	\item Closure
	\item Associative
	\item Commutative
	\item Identity
    \end{enumerate}
\end{definition}

\begin{lemma}
    $ (\mathbb{Z} , \cdot) $ is a commutative monoid
\end{lemma}

\begin{definition}[Ring]
    A ring is a triple $ (R,+, \cdot) $ where R is a set, + and $ \cdot $ are binary
    operations on R that satisfies:
    \begin{enumerate}
	\item (R, +) is commutative group
	\item (R, $ \cdot $) is a commutative monoid
	\item Multiplication is distributive over addition in R. That is, for all
	    x,y,z $\in $ R, we have
	    $$ x \cdot (y+z) = x \cdot y + x \cdot z \text{ and }
	    (y+z) \cdot x = y \cdot x + z \cdot x $$
    \end{enumerate}

    It should be noted that the properties that define a ring are called the ring axioms
\end{definition}

\begin{lemma}
    $ (\mathbb{Z} , +, \cdot) $ is a commutative ring
\end{lemma}

\begin{lemma}
    $ (\mathbb{N} , +, \cdot) $ is a ring because $ (\mathbb{N} , +) $ is not a group.
    We say it is a semiring
\end{lemma}

\subsection{Divisibility}

\begin{definition}[Even]
    An integer a is called even if there is another integer b such that a = 2b
\end{definition}

\begin{definition}[Odd]
    An integer a is called odd if there is another integer b such that a = 2b + 1
\end{definition}

\begin{definition}[Sum]
    We define the sum of integers a and b to be a + b
\end{definition}

\begin{definition}[Product]
    We define the product of integers a and b to be a $ \cdot $ b
\end{definition}

\begin{theorem}
    The sum of two even integer is even
\end{theorem}

\begin{theorem}
    The product of two integers that are each divisible by k is also divisible by k
\end{theorem}

\subsection{Induction}

\begin{definition}[Well Ordering Principle]
    The Well Ordering Principle says that every nonempty subset of natural numbers
    has a least element
\end{definition}

\begin{theorem}[Principle of Mathematical Induction]
    Let S be a set of natural numbers such that
    \begin{enumerate}
	\item $ 0 \in S $
	\item $ \text{for all k } \in \mathbb{N} , k \in S \rightarrow k+1.
	    \text{ Then, } S= \mathbb{N} $
    \end{enumerate}
\end{theorem}

\begin{lemma}[Standard Advanced Calculus Trick]
    We can add and substract the same quantities without changing the result
\end{lemma}

\subsection{Problem Set}
\subsubsection{Level 1}
\subsubsection{Level 2}
\subsubsection{Level 3}
\subsubsection{Level 4}
\subsubsection{Level 5}
\pagebreak

\section{Lesson 5 - Real Analysis: The Complete Ordered Field of Reals}
\subsection{Overview}

The goal of this section is TODO

\subsection{Field}

\begin{definition}[Field]
    A field is a triple (F, +, $\cdot $), where F is a set and + and $ \cdot$
    are binary operations on F satisfying:
    \begin{enumerate}
	\item (F, +) is a commutative group
	\item (F, $\cdot $) is a commutative group
	\item Multiplication is distributive over addition in F. That is, for all
	    x,y,z $\in $ F, we have
	    $$ x \cdot (y+z) = x \cdot y + x \cdot z \text{ and }
	    (y+z) \cdot x = y \cdot x + z \cdot x $$
	\item 0 $\neq$ 1
    \end{enumerate}

    The properties that define a field are called the field axioms
\end{definition}

\begin{lemma}[Set of Natural Numbers]
    The set $ \mathbb{N} $ is the set of natural numbers and the structure
    ($ \mathbb{N} $ , +, $\cdot $) is a semiring
\end{lemma}

\begin{lemma}[Set of Integers]
    The set $ \mathbb{Z} $ is the set of integers and the structure
    ($ \mathbb{Z} $ , +, $\cdot $) is a ring
\end{lemma}

\begin{lemma}[Set of Rational Numbers]
    The set $ \mathbb{Q} $ is the set of rational numbers and the structure
    ($ \mathbb{Q} $, +, $\cdot $) is a field
\end{lemma}

\begin{definition}[Substraction]
    If a,b $ \in $ F, we define the substraction a-b= a+(-b)
\end{definition}

\begin{definition}[Division]
    If a,b $ \in $ F and b $\neq$0, we define the division $a/b= ab^-1$
\end{definition}

\subsection{Ordered Rings and Fields}

\begin{definition}[Positive and Negative Elements]
    If a $\in$ P, we say that a is positive and if -a $\in$ P, we say that a is
    negative
\end{definition}

\begin{definition}[Ordered Ring]
    We say that a ring (R,+, $\cdot$) is ordered if there is a nonempty subset
    P of R, called the set of positive elements of R satisfying the folowing
    properties
    \begin{enumerate}
	\item if a,b $\in$ P, then a + b $\in$ P
	\item if a,b $\in$ P, then ab $\in$ P
	\item if a $\in$ P, then exactly one of the following holds:
	    $$ a \in P, a=0, \text{or} -a \in P $$
    \end{enumerate}
\end{definition}

\begin{theorem}
    $ (\mathbb{Q}, +, \cdot) $ is an ordered field
\end{theorem}

\begin{theorem}
    Let (F, $\leq$ ) be an ordered field. Then, for all x $\in$ F*, $ x \cdot x > 0 $
\end{theorem}

\begin{theorem}
    Every ordered field (F, $\leq$ ) contains a copu of the natural numbers.
\end{theorem}

\begin{theorem}
    Let (F, $\leq$ ) be an ordered field and let $ x \in F$ with $ x > 0$.
    Then, $ \frac{1}{x} > 0 $
\end{theorem}

\subsection{Why Isn't $\mathbb{Q}$ Enough?}

\begin{theorem}[Pythagorean Theorem]
    In a right triangle with legs of length a and b, and a hypotenuse of length c
    $$ c^2 = a^2 + b^2 $$
\end{theorem}

\begin{theorem}
    There does not exist a rational number a such that $a^2=2$
\end{theorem}

\subsection{Completeness}

\begin{definition}[Upper Bound]
    Let (F, $\leq$ ) be an ordered field and let S be a nonempty subset of F.
    We say that S is bounded above if there is $ M \in F$ such that for all
    $s \in S, s \leq M$. Each number M is called an upper bound of S
\end{definition}

\begin{definition}[Lower Bound]
    Let (F, $\leq$ ) be an ordered field and let S be a nonempty subset of F.
    We say that S is bounded below if there is $ K \in F$ such that for all
    $s \in S, K \leq s$. Each number K is called an lower bound of S
\end{definition}

\begin{definition}[Bounded Set]
    We say that S is bounded if it is bounded above and bounded below. Otherwise,
    we say that S is unbounded.
\end{definition}

\begin{definition}[Supremum]
    A least upper bound of a set S is an upper bound that is smaller than any
    other upper bound of S
\end{definition}

\begin{definition}[Infimum]
    A greatest lower bound of S is a lower bound that is larger than any other
    other lower bound of S
\end{definition}

\begin{definition}[Completeness Property]
    An ordered field (F, $\leq$ ) has the Completeness Property if every nonempty
    subset of F that is bounded above has a least upper upper bound in F. In this
    case, we cay that (F, $\leq$ ) is a complete ordered field.
\end{definition}

\begin{theorem}
    There is exaclty one complete ordred field
\end{theorem}

\begin{theorem}[Archimedian Property of $\mathbb{R}$]
    For every $x \in \mathbb{R}, there is n \in \mathbb{N} such that n > x $
\end{theorem}

\begin{theorem}[Density Theorem]
    If $x, y \in \mathbb{R} \text{with} x<y$ then there is $q \in \mathbb{Q}
    \text{ with } x<q<y $
\end{theorem}

\subsection{Problem Set}
\subsubsection{Level 1}
\subsubsection{Level 2}
\subsubsection{Level 3}
\subsubsection{Level 4}
\subsubsection{Level 5}
\pagebreak

\section{Lesson 6 - Topology: The Topology of R}
\subsection{Overview}
\subsection{Problem Set}
\subsubsection{Level 1}
\subsubsection{Level 2}
\subsubsection{Level 3}
\subsubsection{Level 4}
\subsubsection{Level 5}
\pagebreak

\section{Lesson 7 - Complex Analysis: The Field of Complex Numbers}
\subsection{Overview}
\subsection{Problem Set}
\subsubsection{Level 1}
\subsubsection{Level 2}
\subsubsection{Level 3}
\subsubsection{Level 4}
\subsubsection{Level 5}
\pagebreak

\section{Lesson 8 - Linear Algebra: Vector Spaces}
\subsection{Overview}
\subsection{Problem Set}
\subsubsection{Level 1}
\subsubsection{Level 2}
\subsubsection{Level 3}
\subsubsection{Level 4}
\subsubsection{Level 5}
\pagebreak

\section{Lesson 9 - Logic: Logical Arguments}
\subsection{Overview}
\subsection{Problem Set}
\subsubsection{Level 1}
\subsubsection{Level 2}
\subsubsection{Level 3}
\subsubsection{Level 4}
\subsubsection{Level 5}
\pagebreak

\section{Lesson 10 - Set Theory: Reltions and Functions}
\subsection{Overview}
\subsection{Problem Set}
\subsubsection{Level 1}
\subsubsection{Level 2}
\subsubsection{Level 3}
\subsubsection{Level 4}
\subsubsection{Level 5}
\pagebreak

\section{Lesson 11 - Abstract ALgebra: Strucutres and Homomorphisms}
\subsection{Overview}
\subsection{Problem Set}
\subsubsection{Level 1}
\subsubsection{Level 2}
\subsubsection{Level 3}
\subsubsection{Level 4}
\subsubsection{Level 5}
\pagebreak

\section{Lesson 12 - Number Theory: Primes, GCD, and LCM}
\subsection{Overview}
\subsection{Problem Set}
\subsubsection{Level 1}
\subsubsection{Level 2}
\subsubsection{Level 3}
\subsubsection{Level 4}
\subsubsection{Level 5}
\pagebreak

\section{Lesson 13 - Real Analysis: Limits and Continuity}
\subsection{Overview}
\subsection{Problem Set}
\subsubsection{Level 1}
\subsubsection{Level 2}
\subsubsection{Level 3}
\subsubsection{Level 4}
\subsubsection{Level 5}
\pagebreak

\section{Lesson 14 - Topology: Spaces and Homeomorphisms}
\subsection{Overview}
\subsection{Problem Set}
\subsubsection{Level 1}
\subsubsection{Level 2}
\subsubsection{Level 3}
\subsubsection{Level 4}
\subsubsection{Level 5}
\pagebreak

\section{Lesson 15 - Complex Analysis: Complex Valued Functions}
\subsection{Overview}
\subsection{Problem Set}
\subsubsection{Level 1}
\subsubsection{Level 2}
\subsubsection{Level 3}
\subsubsection{Level 4}
\subsubsection{Level 5}
\pagebreak

\section{Lesson 16 - Linear Algebra: Linear Transformations}
\subsection{Overview}
\subsection{Problem Set}
\subsubsection{Level 1}
\subsubsection{Level 2}
\subsubsection{Level 3}
\subsubsection{Level 4}
\subsubsection{Level 5}
\pagebreak

\end{document}
\end{article}
