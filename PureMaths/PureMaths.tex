\documentclass{article}
\begin{document}
\title{Lectures Notes from Pure Maths by Steve Warner}
\author{Emulie Chhor}
\maketitle

\section*{Introduction}

This book is an introduction to undergraduate pure mathematics. This books contains
16 chapters with exercices ranging from different levels:

\begin{enumerate}
    \item Lesson 1 - Logic: Statements and Truth
    \item Lesson 2 - Set Theory: Sets and Subsets
    \item Lesson 3 - Abstract Algebra: Semigroups, Monoids and Groups
    \item Lesson 4 - Number Theory: The Ting of Integers
    \item Lesson 5 - Real Analysis: The Complete Ordered Field of Reals
    \item Lesson 6 - Topology: The Topology of R
    \item Lesson 7 - Complex Analysis: The Field of Complex Numbers
    \item Lesson 8 - Linear Algebra: Vector Spaces
    \item Lesson 9 - Logic: Logical Arguments
    \item Lesson 10 - Set Theory: Reltions and Functions
    \item Lesson 11 - Abstract ALgebra: Strucutres and Homomorphisms
    \item Lesson 12 - Number Theory: Primes, GCD, and LCM
    \item Lesson 13 - Real Analysis: Limits and Continuity
    \item Lesson 14 - Topology: Spaces and Homeomorphisms
    \item Lesson 15 - Complex Analysis: Complex Valued Functions
    \item Lesson 16 - Linear Algebra: Linear Transformations
\end{enumerate}

\pagebreak

\section{Lesson 1 - Logic: Statements and Truth}

\subsection{Overview}

This section introduces the notion of statements and truth tables.

\subsection{Statements with words}

\par
The goal of this section is to determine wether a sentence is a statement or not.\\

A statement or proposition is a sentence that can be true or false, but not both
simultaneously. If it expresses a single idea, we say it is an atomic statement.
If we want to create a statement with more than one idea, we need to connect the
atomic statement using logical connectives.

\subsection{Statements with symbols}

In mathematics, we want to express mathematical statement in symbols since they
actract away the unecessary clutter and help us focus on the form of the statement.
Consequently, we need to define a set of symbol used to define the common logical
connectives. The most common are:

\begin{enumerate}
    \item Conjunction
    \item Disjunction
    \item Negation
    \item Implication
    \item Biconditional
\end{enumerate}

\subsection{Truth Table}

\par
Each logical connective is associated to a truth table, which tell us the
truth value of a compound statement based on the truth value of the propositional
variables. Here is the truth table for the common logical connectives:\\

We can also use truth table to determine if two statement are equivalent. To do
so, we compare their truth table.


\subsection{Problem Set}
\subsubsection{Level 1}
\subsubsection{Level 2}
\subsubsection{Level 3}
\subsubsection{Level 4}
\subsubsection{Level 5}
\pagebreak

\section{Lesson 2 - Set Theory: Sets and Subsets}
\subsection{Overview}
\subsection{Problem Set}
\subsubsection{Level 1}
\subsubsection{Level 2}
\subsubsection{Level 3}
\subsubsection{Level 4}
\subsubsection{Level 5}
\pagebreak

\section{Lesson 3 - Abstract Algebra: Semigroups, Monoids and Groups}
\subsection{Overview}
\subsection{Problem Set}
\subsubsection{Level 1}
\subsubsection{Level 2}
\subsubsection{Level 3}
\subsubsection{Level 4}
\subsubsection{Level 5}
\pagebreak

\section{Lesson 4 - Number Theory: The Ting of Integers}
\subsection{Overview}
\subsection{Problem Set}
\subsubsection{Level 1}
\subsubsection{Level 2}
\subsubsection{Level 3}
\subsubsection{Level 4}
\subsubsection{Level 5}
\pagebreak

\section{Lesson 5 - Real Analysis: The Complete Ordered Field of Reals}
\subsection{Overview}
\subsection{Problem Set}
\subsubsection{Level 1}
\subsubsection{Level 2}
\subsubsection{Level 3}
\subsubsection{Level 4}
\subsubsection{Level 5}
\pagebreak

\section{Lesson 6 - Topology: The Topology of R}
\subsection{Overview}
\subsection{Problem Set}
\subsubsection{Level 1}
\subsubsection{Level 2}
\subsubsection{Level 3}
\subsubsection{Level 4}
\subsubsection{Level 5}
\pagebreak

\section{Lesson 7 - Complex Analysis: The Field of Complex Numbers}
\subsection{Overview}
\subsection{Problem Set}
\subsubsection{Level 1}
\subsubsection{Level 2}
\subsubsection{Level 3}
\subsubsection{Level 4}
\subsubsection{Level 5}
\pagebreak

\section{Lesson 8 - Linear Algebra: Vector Spaces}
\subsection{Overview}
\subsection{Problem Set}
\subsubsection{Level 1}
\subsubsection{Level 2}
\subsubsection{Level 3}
\subsubsection{Level 4}
\subsubsection{Level 5}
\pagebreak

\section{Lesson 9 - Logic: Logical Arguments}
\subsection{Overview}
\subsection{Problem Set}
\subsubsection{Level 1}
\subsubsection{Level 2}
\subsubsection{Level 3}
\subsubsection{Level 4}
\subsubsection{Level 5}
\pagebreak

\section{Lesson 10 - Set Theory: Reltions and Functions}
\subsection{Overview}
\subsection{Problem Set}
\subsubsection{Level 1}
\subsubsection{Level 2}
\subsubsection{Level 3}
\subsubsection{Level 4}
\subsubsection{Level 5}
\pagebreak

\section{Lesson 11 - Abstract ALgebra: Strucutres and Homomorphisms}
\subsection{Overview}
\subsection{Problem Set}
\subsubsection{Level 1}
\subsubsection{Level 2}
\subsubsection{Level 3}
\subsubsection{Level 4}
\subsubsection{Level 5}
\pagebreak

\section{Lesson 12 - Number Theory: Primes, GCD, and LCM}
\subsection{Overview}
\subsection{Problem Set}
\subsubsection{Level 1}
\subsubsection{Level 2}
\subsubsection{Level 3}
\subsubsection{Level 4}
\subsubsection{Level 5}
\pagebreak

\section{Lesson 13 - Real Analysis: Limits and Continuity}
\subsection{Overview}
\subsection{Problem Set}
\subsubsection{Level 1}
\subsubsection{Level 2}
\subsubsection{Level 3}
\subsubsection{Level 4}
\subsubsection{Level 5}
\pagebreak

\section{Lesson 14 - Topology: Spaces and Homeomorphisms}
\subsection{Overview}
\subsection{Problem Set}
\subsubsection{Level 1}
\subsubsection{Level 2}
\subsubsection{Level 3}
\subsubsection{Level 4}
\subsubsection{Level 5}
\pagebreak

\section{Lesson 15 - Complex Analysis: Complex Valued Functions}
\subsection{Overview}
\subsection{Problem Set}
\subsubsection{Level 1}
\subsubsection{Level 2}
\subsubsection{Level 3}
\subsubsection{Level 4}
\subsubsection{Level 5}
\pagebreak

\section{Lesson 16 - Linear Algebra: Linear Transformations}
\subsection{Overview}
\subsection{Problem Set}
\subsubsection{Level 1}
\subsubsection{Level 2}
\subsubsection{Level 3}
\subsubsection{Level 4}
\subsubsection{Level 5}
\pagebreak

\end{document}
\end{article}
