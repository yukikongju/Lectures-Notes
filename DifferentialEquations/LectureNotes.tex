\documentclass{article}
\usepackage{amsmath}
\usepackage{amsfonts}
\usepackage{amsthm}
\usepackage{parskip}
\usepackage{textgreek}
\begin{document}
\title{Lecture Notes - Differential Equations by Professor Macauley}
\author{Emulie Chhor}
\maketitle

\section*{Introduction}

This document is a summary of concepts I have learned from Professor
Macauley's Differential Equations Course.

His course is separated into the following chapter:
\begin{enumerate}
    \item Introduction to ODE
    \item First Order Differential Equations
    \item Second Order Differential Equations
    \item Systems of Differential Equations
    \item Laplace Transforms
    \item Fourier Series and Boundary Value Problems
    \item Partiel Differential Equations
\end{enumerate}

\newtheorem{definition}{Definition}[subsection]
\newtheorem{theorem}{Theorem}[subsection]
\newtheorem{corollary}{Corollary}[subsection]
\newtheorem{lemma}[theorem]{Lemma}
\newtheorem{proposition}{Proposition}[section]
\newtheorem{axiom}{Axiome}
\newtheorem{property}{Propriété}[subsection]
\newtheorem*{remark}{Remarque}
\newtheorem*{problem}{Problème}
\newtheorem*{intuition}{Intuition}

\pagebreak

\section{Introduction to ODE}
\subsection*{Overview}

Le premier chapitre introduit la notion d'Ordinary Differential Equations,
qui sont des equations differentielles à une seule variable. On
n'apprend pas encore comment les résoudre, mais on désire les dessiner
puisque ça nous donne une bonne idée de l'allure de la famille de solutions.

Notons qu'il existe d'autres tyes d'équations différentielles (ex:
PDEs, qui sont des équa diff avec 2 variables), mais on se penche sur les
ODEs en premier.

On verra 4 méthodes pour résoudre les ODEs:
\begin{enumerate}
    \item Separating Variables: isolate dy/y and integrate both sides
    \item Integrating Factor: multiplier par une constante d'intégration
	pour pouvoir intégrer
    \item Variables Parameters
    \item indetermined coefficient: $ y(t) = y_h(t) + y_p(t) $

\end{enumerate}

\subsection{What is a differential equation? }

\begin{definition}[Linear Differential Equation]
    On dit qu'une transformation est linéaire si les coefficients de
    l'équation différentielle sont de degré 0 et que y et ses dérivées
    sont exposant 1
\end{definition}

\begin{definition}[Order of Derivative]
    L'ordre d'une équation différentielle est la plus grande dérivée.
\end{definition}

\begin{definition}[Solution générale et particulière]
    \begin{enumerate}
        \item Particular Solutions: no arbitrary parameters $ y = 3x$
	\item One-Parameter family of solutions $ y=3x+c$
	\item Trivial Solution: y=0
    \end{enumerate}
    Une solution explicite est une équation qui n'isole pas y. Une solution
    implicite est une équation avec y qui n'est pas isolée.
    Plus généralement, si on a une constante, on parle de solutions
    générale, sinon, on parle de solution particulière.
\end{definition}

Une équation différentielle est une équation définie par ses fonctions
et ses dérivées. On l'utilise pour modéliser:
\begin{enumerate}
    \item Growth : $ P'(t) = r P(t)$
    \item Decay : $ P'(t) = r (1- \frac{P(t)}{M})$
    \item Logistic Equation: $ P'(t) = r (1- \frac{P(t)}{M})P(t)$
\end{enumerate}

Éventuellement, le graphe devrait converger vers une valeur quelconque.\\

Il est à noter qu'on cherche la famille de fonctions qui vérifie l'équation.
 On pourra choisir la solution parmi cette famille de solutions plus tard.

\subsection{Plotting solutions to differential equations. }

Présentement, on n'est pas encore capable de résoudre une équation
différentielle, mais on peut utiliser les outils du calcul pour tracer
le graphe pour avoir une idée de l'allure de la famille de solutions.

On distingue 2 types de solutions:
\begin{enumerate}
    \item Isocline: y' est une constante: $ y' = 0$, with c
    \item Autonomous ODEs: y' est une fonction quelconque: $ y' = f(y)$
\end{enumerate}

\subsubsection{Comment tracer les solutions}

TODO

\subsection{Approximating solutions to differential equations. }

Présentement, on n'a pas enore les outils pour résoudre une équa diff.
Par contre, on peut toujours se servir des concepts vu en calcul pour
approximer les solutions à une équations différentielles. Il existe
plusieurs méthodes:
\begin{enumerate}
    \item Euler's method: Approximating using stepwise Linear Approximations
    \item Runge-Kutta's Method: skipped
\end{enumerate}

\subsubsection{Euler's Method}

La méthode d'Euler conciste à tracer la fonction à l'aide d'approximation
linéaire. On doit d'abord choisir un "step", qui représente la distance
sur lequel on trace notre droite. Plus le step est petit, plus notre graphe
aura l'air d'une courbe.

Plus généralement, si on a $ y' = f(t,y)$ et $ y(t_0)$ avec un stepsize
de h, on a:
$$ (t_{k+1}, y_{k+1}) = (t_k + h, y_k + f(t_k, y_k) \cdot h) $$
Il s'agit donc d'une méthode récursive pour tracer des graphes:
on détermine la pente entre deux steps en considérant le triangle
rectangle qui forme le point précédent et le point suivant.

\pagebreak
\section{First Order Differential Equations}
\subsection*{Overview}

\subsection{Separation of variables}

Cette setion met l'emphase sur la séparation de variables. En gros, on
veut réécrire notre équation différentielle sous la notation de Leibniz
afin d'isoler dy/y et dx/x et d'avoir une équation p/r au temps. On voit
qu'en intégrant, on obtient les équations suivantes:
\begin{enumerate}
    \item Exponential Growth: $ y' = ky \Longleftrightarrow y(t) = C e^{kt} $
    \item Exponential Decay: $ y' = -ky \Longleftrightarrow y(t) = C e^{-kt} $
    \item Decay to Initial Value: $ y' = -k (y-M) \Longleftrightarrow
	y(t) = C e^{-kt} + M $
\end{enumerate}

Il faut aussi se souvenir que le k, représentant le rate of change, nous
dit à quel point le decay/growth se fait rapidement.

\begin{remark}[Stratégies de résolution]
    Pour résoudre des intégrales, il est parfois utile d'utiliser
    \begin{enumerate}
        \item intégration par fraction partielle
	\item Rational zero theorem: trouver les facteurs de p et q, et
	    résoudre par division euclidienne une foit qu'on a trouvé le
	    premier zéro
    \end{enumerate}
\end{remark}

\subsection{Initial value problems}

Cette section focussait sur la résolution de ODEs. Tout d'abord, on
introduisait la notion de valeur initiale. Auparavant, on voulait
trouver la solution générale du ODEs. Ici, on nous donne des valeurs
intiales $ y(t_0) = y_0 $. On veut trouver une solution particulière.

\begin{intuition}
    Graphiquement, c'est comme si on calculait la famille de solutions
    générale, et on choisissait la courbe qui passait par le point initial
\end{intuition}

Pour ce faire, on doit trouver:
\begin{enumerate}
    \item C: initial rate of change that can be found by solving
	initial value (t=0)
    \item k: constant rate of change that can be found by using
	initial and end time (t=0 and t=5 for exemple)
    \item t: time an which we measure value
    \item P(t): value of function given using for all parameters
\end{enumerate}

Dépendemment du problème, on doit solve for une des 4 variables
\begin{enumerate}
    \item Find Initial Rate of Change : solve for C
    \item Find how much x is worth at time t: use C, k, t to find P(t)
    \item Find half live: use P(t), P(0), C, k to find t, the time when
	P has value of P(t)
\end{enumerate}

\begin{remark}
    On utilise souvent les logs pour abaisser l'exposant. On préfère
    travailler avec des fractions positives
\end{remark}

\subsection{Falling objects with air resistance: Newton's 2nd Law}

Dans cette section, on voit que la 2e loi de Newton F=ma qui considère
la résistance de l'air, peut être modéliser avec une ODEs: exponential
decay to value. ON voit aussi pourquoi on voudrait intégrer une ODEs

\textbf{Comment trouver modéliser v(t) avec une ODE}

La deuxième loi de Newton nous dit que
\begin{enumerate}
    \item Sans résistance de l'air: $ F=ma=-mg $
    \item Avec résistance de l'air: $ F=-mg - R(v) $, R(v): résistance
	de l'air
\end{enumerate}

De plus, on peut considérer que la résistance à l'air est proportionnelle
à la vitesse (dans le sens inverse): $$ R(v) = -rv $$

Aussi, il faut se souvenir que
\begin{enumerate}
    \item Vitesse: $ v(t) = d'(t) $, d: distance
    \item Accélération: $ a(t) = v'(t) $, v: vitesse
\end{enumerate}

Ainsi, on a que la 2nd loi de Newton qui considère la résistance de l'air
est $$ v' = -g - \frac{r}{m} v $$, qui peut être remodelée pour obtenir
un exponential decay to a value
$$ v' = -g - \frac{r}{m} = \frac{r}{m} (\frac{-mg}{r} - v),
k = \frac{r}{m}, A = \frac{-mg}{r}$$, avec v: vitesse limitante,
A: terminal velocity

Ainsi, on a que $$ V(t) = -\frac{mg}{r} + C e^{\frac{-r}{m} t} $$

\textbf{Problèmes}

Dépendemment du problème, on nous demande de trouver:
\begin{enumerate}
    \item terminal velocity: $ A = \frac{-mg}{r} $
    \item Limiting velocity: $ v' = 0 $
    \item C: initial value
    \item m: mass
    \item g: constante de gravité
    \item v(t): velocity after t time
    \item rate of change r: use terminal velocity and solve for r
    \item distance d(t): on n'a qu'à intégrer la fonction de vitesse
	$$ \int_{{a}}^{{b}} {v(t)} \: d{t} $$
\end{enumerate}

Bref
\begin{enumerate}
    \item Comment on a trouver la ODE pour modéliser l'accélération
	en considérant le frottement de l'air
    \item Résoudre des problèmes en utilisant la modélisation de F=ma
	en ODE
\end{enumerate}

\subsection{Solving 1st order inhomogeneous ODEs.}

Depuis le début du cours, on a vu comment résoudre des équations homogènes
à l'aide de la méthode par séparation. Cependant, cette stratégie ne peut pas
être utilisée pour résoudre des équations inhomogènes

Rappel: la différence entre une équation homogène et inhomogène est que f(t) = 0
\begin{enumerate}
    \item Homogenous Equation: $ y' + a(t)y(t) = 0 $
    \item Inhomogenous Equation: $ y' + a(t)y(t) = f(t)$
\end{enumerate}

On discerne deux méthodes pour résoudre des ODEs inhomogènes:
\begin{enumerate}
    \item Integrating Factor
    \item Variation of Parameters
\end{enumerate}

Notons que les 2 méthodes sont équivalentes, mais il est préférable d'utiliser
la deuxième, puisqu'elle possède un "built-in correction" nous permettant de
voir si on a fait des erreur

\subsubsection{Integrating Factor}

La première méthode consiste à multiplier par un facteur d'intégration afin de
pouvoir intégrer. Pour choisir ce facteur d'intégration, on doit regarder a(t),
le coefficient qui multiplie y(t).

Plus généralement, les étapes sont les suivantes:
\begin{enumerate}
    \item Identifier a(t) afin de trouver le facteur d'intégration
    \item Calculer le facteur d'intégration: $ e^( \int_{{}}^{{}} {A(t)} \:
	d{t} {}) $
    \item Multiplier l'équation inhomogène par le facteur d'intégration des
	deux côtés: on obtient "l'inverse du produit de la dérivée"
    \item Écrire le produit de la dérivée comme une dérivée
    \item Intégrer des deux bords, et isoler y pour trouver la solution
	générale
\end{enumerate}

Notons pour que cette stratégie fonctionne, on doit savoir intégrer la partie
de droite

\subsubsection{Variation of Parameters}

La deuxième méthode consiste à trouver l'équation homogène en ignorant la
constante f(t), puis à plugger le guess dans l'équation inhomogène

Les étapes sont les suivantes:
\begin{enumerate}
    \item Trouver la solution à l'équation homogène en ignorant f(t)
    \item Guesser la solution générale pour trouver
	y et y': $ y(t) = v(t) y_h (t) = v e^t $
    \item Solve for v by isolating v' and integrating both sides
    \item Plugger y et y' dans l'équation inhomogène: il devrait y avoir des
	termes qui s'annulent
    \item Substitutionner v dans l'équation générale
	$ y(t) = v(t) y_h (t) = v e^t $
\end{enumerate}

En d'autres mots, on devrait retrouver la constante f(t) dans le v

TO REVIEW

\subsection{Linear differential equations}

Cette section nous présente 2 résultas cachés des équations différentielles:
\begin{enumerate}
    \item Superposition: Les solutions des équations différentielles homogènes
	sont linéairement indépendants
    \item Inhomogenous ODEs: on peut trouver la solution générale d'une
	ODEs inhomogènes en additionnant son équation homogène et une
	équation particulière
\end{enumerate}

\subsubsection{Superposition}

La superposition nous dit que si une ODE homogène $y'+a(t)y(t) = 0$ a comme
solution $y_1(t)$ et $y_2(t)$, alors $ C_1 y_1(t) + C_2 y_2(t) $ est une
solution $ \forall c_1, c_2$

\textbf{Pourquoi c'est vrai?}

Si on plug $ C_1 y_1(t) + C_2 y_2(t) $ dans l'équation homogène initiale
$y'+a(t)y(t) = 0$, on obtient que $ C_1 \cdot 0 + C_2 \cdot 0$

\subsubsection{Quick Trick to solve Inhomogenous ODEs}

On peut trouver la solution générale d'une ODE inhomogène en additionnant
son équation homogène et une équation particulière:
$$ y(t) = y_h (t) + y_p (t) $$

Les étapes sont les suivantes:
\begin{enumerate}
    \item Trouver la solution Homogène $y_h$
    \item Trouver la solution particulière $y_p$ (choisir 0)
    \item Trouver la solution générale: $$ y(t) = y_h (t) + y_p (t) $$
\end{enumerate}

\textbf{Pourquoi ça marche}

Si y est la solution générale $y'+a(t)y(t) = f(t)$ et la solution particulière
$y_p'+a(t)y_p(t) = f(t)$, alors en les soustrayant, on obtient
$$ (y-y_p)' +a(t) (y-y_p) = 0$$, et $ (y - y_p) $ est une solution à
l'équation homogène

\subsection{Basic mixing problems.}

Dans un problème de mixing problem, on veut déterminer la concentration
de soluté dans un solvant à un temps donné. Le problème le plus simple
est le cas suivant: on ajoute de l'eau à la même concentration qu'il
n'en sort. On caractérise le volume à un temps donné par
$$ Concentration(t) = \frac{x(t)}{Vol(t)} $$

De plus, on sait que le rate of change à un temps t est donné par
$$ x'(t) = (rate in) - (rate out) $$

Ainsi, on doit calculer
\begin{enumerate}
    \item Rate In := (volume rate) x (concentration) := (volume qui sort
	dans un intervalle de temps) x (concentration du solvant)
    \item Rate Out := (volume rate) x (concentration) :=
	(volume qui sort) x (concentration du solvant dans le tank) :=
	(volume) x $(\frac{x(t)}{Vol(t)} )$
    \item Solve for x and for C using one of the 4 methods: on cherche
	x pour pouvoir trouver C
\end{enumerate}

\subsection{Advanced mixing problems.}
TODO
\subsection{The logistic equation.}

\pagebreak

\section{Second Order Differential Equations}
\subsection*{Overview}

\subsection{Second order linear ODEs.}
\subsection{Equations with constant coefficients.}
\subsection{The method of undetermined coefficients.}
\subsection{Simple harmonic motion.}
\subsection{Damped and driven harmonic motion.}
\subsection{Variation of parameters.}
\subsection{Cauchy-Euler equations.}
\subsection{Power series solutions.}
\subsection{The method of Frobenius.}

\pagebreak

\section{Systems of Differential Equations}
\subsection*{Overview}

\subsection{Basic matrix algebra.}
\subsection{Eigenvalues and eigenvectors.}
\subsection{Mixing with two tanks.}
\subsection{Solving a 2x2 system of ODEs.}
\subsection{Phase portraits with real eigenvalues.}
\subsection{Phase portraits with complex eigenvalues.}
\subsection{Phase portraits with repeated eigenvalues.}
\subsection{Stability of phase portraits.}
\subsection{of parameters for systems.}

\pagebreak
\section{Laplace Transforms}
\subsection*{Overview}

\subsection{What is a Laplace transform?}
\subsection{Properties and applications of Laplace transforms}
\subsection{Discontinuous forcing terms}
\subsection{Periodic forcing terms}
\subsection{Impulse functions}
\subsection{Convolution}

\pagebreak
\section{Fourier Series and Boundary Value Problems}

\subsection{Introduction to Fourier series}
\subsection{Computing Fourier series}
\subsection{Fourier sine and cosine series}
\subsection{Complex Fourier series}
\subsection{Applications of Fourier series}
\subsection{Boundary value problems}

\subsection*{Overview}


\pagebreak
\section{Partial Differential Equations}
\subsection*{Overview}

\subsection{The heat equation}
\subsection{Different boundary conditions}
\subsection{The transport equation}
\subsection{The wave equation}
\subsection{Harmonic functions}
\subsection{Laplace's equation}
\subsection{The 2D heat equation}
\subsection{2D wave equation}

\pagebreak

\section{Systems of Nonlinear Differential Equations}

\subsection*{Overview}

\subsection{Modeling with nonlinear systems}
\subsection{Linearization and steady-state analysis}
\subsection{Predator-prey models}


\pagebreak
\end{document}
\end{article}

