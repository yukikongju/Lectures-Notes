\documentclass{article}
\usepackage{amsmath}
\usepackage{amsfonts}
\usepackage{amsthm}
\usepackage{parskip}
\usepackage{textgreek}
\begin{document}
\title{Lecture Notes - Differential Equations by Professor Macauley}
\author{Emulie Chhor}
\maketitle

\section*{Introduction}

This document is a summary of concepts I have learned from Professor
Macauley's Differential Equations Course.

His course is separated into the following chapter:
\begin{enumerate}
    \item Introduction to ODE
    \item First Order Differential Equations
    \item Second Order Differential Equations
    \item Systems of Differential Equations
    \item Laplace Transforms
    \item Fourier Series and Boundary Value Problems
    \item Partiel Differential Equations
\end{enumerate}

\newtheorem{definition}{Definition}[subsection]
\newtheorem{theorem}{Theorem}[subsection]
\newtheorem{corollary}{Corollary}[subsection]
\newtheorem{lemma}[theorem]{Lemma}
\newtheorem{proposition}{Proposition}[section]
\newtheorem{axiom}{Axiome}
\newtheorem{property}{Propriété}[subsection]
\newtheorem*{remark}{Remarque}
\newtheorem*{problem}{Problème}
\newtheorem*{intuition}{Intuition}

\pagebreak

\section{Introduction to ODE}
\subsection*{Overview}

Le premier chapitre introduit la notion d'Ordinary Differential Equations,
qui sont des equations differentielles à une seule variable. On
n'apprend pas encore comment les résoudre, mais on désire les dessiner
puisque ça nous donne une bonne idée de l'allure de la famille de solutions.

Notons qu'il existe d'autres tyes d'équations différentielles (ex:
PDEs, qui sont des équa diff avec 2 variables), mais on se penche sur les
ODEs en premier.

\subsection{What is a differential equation? }

Une équation différentielle est une équation définie par ses fonctions
et ses dérivées. On l'utilise pour modéliser:
\begin{enumerate}
    \item Growth : $ P'(t) = r P(t)$
    \item Decay : $ P'(t) = r (1- \frac{P(t)}{M})P(t)$
\end{enumerate}

Éventuellement, le graphe devrait converger vers une valeur quelconque.\\

Il est à noter qu'on cherche la famille de fonctions qui vérifie l'équation.
 On pourra choisir la solution parmi cette famille de solutions plus tard.

\subsection{Plotting solutions to differential equations. }

Présentement, on n'est pas encore capable de résoudre une équation
différentielle, mais on peut utiliser les outils du calcul pour tracer
le graphe pour avoir une idée de l'allure de la famille de solutions.

On distingue 2 types de solutions:
\begin{enumerate}
    \item Isocine: y' est une constante
    \item Autonomous ODEs: y' est une fonction quelconque
\end{enumerate}

\subsubsection{Comment tracer les solutions}

TODO

\subsection{Approximating solutions to differential equations. }

Présentement, on n'a pas enore les outils pour résoudre une équa diff.
Par contre, on peut toujours se servir des concepts vu en calcul pour
approximer les solutions à une équations différentielles. Il existe
plusieurs méthodes:
\begin{enumerate}
    \item Euler's method: Approximating using stepwise Linear Approximations
    \item Runge-Kutta's Method: skipped
\end{enumerate}

\subsubsection{Euler's Method}

La méthode d'Euler conciste à tracer la fonction à l'aide d'approximation
linéaire. On doit d'abord choisir un "step", qui représente la distance
sur lequel on trace notre droite. Plus le step est petit, plus notre graphe
aura l'air d'une courbe.

Plus généralement, si on a $ y' = f(t,y)$ et $ y(t_0)$ avec un stepsize
de h, on a:
$$ (t_{k+1}, y_{k+1}) = (t_k + h, y_k + f(t_k, y_k) \cdot h) $$
Il s'agit donc d'une méthode récursive pour tracer des graphes:
on détermine la pente entre deux steps en considérant le triangle
rectangle qui forme le point précédent et le point suivant.

\pagebreak
\section{First Order Differential Equations}
\subsection*{Overview}

\subsection{Separation of variables. }
\subsection{Initial value problems.}
\subsection{Falling objects with air resistance.}
\subsection{Solving 1st order inhomogeneous ODEs.}
\subsection{Linear differential equations.}
\subsection{Basic mixing problems.}
\subsection{Advanced mixing problems.}
\subsection{The logistic equation.}

\pagebreak
\section{Second Order Differential Equations}
\subsection*{Overview}

\subsection{Second order linear ODEs.}
\subsection{Equations with constant coefficients.}
\subsection{The method of undetermined coefficients.}
\subsection{Simple harmonic motion.}
\subsection{Damped and driven harmonic motion.}
\subsection{Variation of parameters.}
\subsection{Cauchy-Euler equations.}
\subsection{Power series solutions.}
\subsection{The method of Frobenius.}

\pagebreak
\section{Systems of Differential Equations}
\subsection*{Overview}

\subsection{Basic matrix algebra.}
\subsection{Eigenvalues and eigenvectors.}
\subsection{Mixing with two tanks.}
\subsection{Solving a 2x2 system of ODEs.}
\subsection{Phase portraits with real eigenvalues.}
\subsection{Phase portraits with complex eigenvalues.}
\subsection{Phase portraits with repeated eigenvalues.}
\subsection{Stability of phase portraits.}
\subsection{of parameters for systems.}

\pagebreak
\section{Laplace Transforms}
\subsection*{Overview}

\subsection{What is a Laplace transform?}
\subsection{Properties & applications of Laplace transforms.}
\subsection{Discontinuous forcing terms.}
\subsection{Periodic forcing terms.}
\subsection{Impulse functions.}
\subsection{Convolution.}

\pagebreak
\section{Fourier Series and Boundary Value Problems}

\subsection{Introduction to Fourier series.}
\subsection{Computing Fourier series.}
\subsection{Fourier sine and cosine series.}
\subsection{Complex Fourier series.}
\subsection{Applications of Fourier series.}
\subsection{Boundary value problems.}

\subsection*{Overview}


\pagebreak
\section{Partial Differential Equations}
\subsection*{Overview}

\subsection{The heat equation.}
\subsection{Different boundary conditions.}
\subsection{The transport equation.}
\subsection{The wave equation.}
\subsection{Harmonic functions.}
\subsection{Laplace's equation.}
\subsection{The 2D heat equation.}
\subsection{2D wave equation.}

\pagebreak

\section{Systems of Nonlinear Differential Equations}

\subsection*{Overview}

\subsection{Modeling with nonlinear systems.}
\subsection{Linearization and steady-state analysis}
\subsection{Predator-prey models}


\pagebreak
\end{document}
\end{article}

