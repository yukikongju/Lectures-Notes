\documentclass{article}
\usepackage{amsmath}
\usepackage{amsfonts}
\usepackage{amsthm}
\usepackage{hyperref}
\usepackage{parskip}
\usepackage{textgreek}
\begin{document}
\title{Lecture Notes for Graph Theory}
\author{Emulie Chhor}
\maketitle

\section*{Introduction}

\begin{enumerate}
    \item Fundamentals Concepts: Graphs, Digraphs, Degrees
    \item Connectivity
    \item Optimization
    \item Shortest Path
    \item Planar Graphs
    \item Coloring
    \item Flow
\end{enumerate}

\newtheorem{definition}{Definition}[subsection]
\newtheorem{theorem}{Theorem}[subsection]
\newtheorem{corollary}{Corollary}[subsection]
\newtheorem{lemma}[theorem]{Lemma}
\newtheorem{proposition}{Proposition}[section]
\newtheorem{axiom}{Axiome}
\newtheorem{property}{Propriété}[subsection]
\newtheorem*{remark}{Remarque}
\newtheorem*{problem}{Problème}
\newtheorem*{intuition}{Intuition}

\section{Fundamentals Concepts: Graphs, Digraphs, Degrees}

\subsection{Why study Graph Theory}%
\label{sub:Why study Graph Theory}

TODO

\subsection{Overview}%
\label{sub:Overview}

\begin{enumerate}
    \item What is a graph
    \item Terminology: walk, trail, path, circuit, cycle
    \item Graph Cycle
    \item Connected Vertices and Connected Graphs
    \item Types of Graphs: Path Graph, Cycle Graph, Complete Graph,
	Complement of a graph, Bipartite Graph, Complete Bipartite Graph
    \item Directed Graphs
\end{enumerate}

\subsection{What is a Graph}%
\label{sub:What is a Graph}

Un graphe est un "ordered pair" composé de deux éléments:
\begin{enumerate}
    \item Vertex: ensemble des "noeuds" composants le graphe
    \item Edges: ensemble de sous-ensembles qui nous dit quels "noeuds"
	sont reliés
\end{enumerate}

Notre but est de différentier les différents types de graphes et de
définir la terminlogie pour parler d'un graphe
\begin{enumerate}
    \item Undirected Graph vs Directed Graph
    \item Simple Graph
    \item Order, Size
    \item Adjacence
\end{enumerate}

\begin{definition}[Graph]
    A graph G is an ordered pair G=(V,E) where V is a finite set of
    elements and E is a set of 2 subsets of V
\end{definition}

\begin{definition}[Directed and Undirected Graph]
    A directed graph, also called digraph, is a graph that has a
    direction associated with its edges. In other words, the subsets
    in the Edge set are ordered.
    An undirected graph is a graph whose edge subsets are not ordered.
    In other word, if two nodes are connected, then we can reach a to b
    and b from a.
\end{definition}

\begin{definition}[Order and Size]
    \begin{enumerate}
        \item Order |V| : number of vertex in the graph
	\item Size |E|: number of edges in the graph
    \end{enumerate}
\end{definition}

\begin{definition}[Simple Graph]
    \begin{enumerate}
        \item No loop
	\item No multiples edges
    \end{enumerate}
\end{definition}

\begin{definition}[Adjacence]
    On peut parler d'adjacence pour les vertex et les edges.
    \begin{enumerate}
        \item Vertex Adjacence: 2 vertex are adjacents if they are
	    connected by an edge
        \item Edge Adjacence: 2 edges are adjacent if they have a
	    vertex in between them
    \end{enumerate}
\end{definition}

\subsection{Terminology}%
\label{ssub:Terminology}

\begin{definition}[Walk]
    \begin{enumerate}
        \item Walk: Sequence of adjacent vertices. We can go back on our
	    steps: we can traverse edges and vertices several times.
	    We say the vertices lie on the walk.
	\item Length: Number of "steps" we make (even though we may go
	    back and forth).
	\item Open walk: the final vertex is not the same as where we
	    started
	\item Closed Walk: the end vertex is the same where we started
    \end{enumerate}
\end{definition}

\begin{remark}
    On peut utiliser les définitions suivantes pour les trail et autres
    aussi:
    \begin{enumerate}
        \item open/closed
	\item endpoints
	\item length
    \end{enumerate}
\end{remark}

\begin{definition}[Trail]
    A sequence of adjacent vertices without traversing the same edge
    more than once
\end{definition}

\begin{definition}[Path]
    A path is a sequence of adjactent vertices, but we cannot traverse
    the same vertices more than once (which also means we can't
    traverse the same edge). Can be defined as
    \begin{enumerate}
	\item List of vertices: $ P=(v_1, v_2, ..., v_8)$
	\item List of alternating vertices and edges: $ P=(v_1, v_1v_2,
	    ..., v_8) $
    \end{enumerate}
    Habituellement, on préfère définir un chemin par une liste de vertices
\end{definition}

\begin{definition}[Circuit]
    TODO
\end{definition}

\begin{definition}[Path and Cycle]
    \begin{enumerate}
        \item A Path $P_n$ is a graph whose vertices can be arranged
	    in a sequence such that the edge set is
	    $ E = {v_i v_{i+1} | i = 1,2,...,n-1} $
	\item A Cycle $C_n$ is a graph whose vertices can be arranged in
	    a cyclic sequence such that the edge set is
	    $ E = {v_i v_{i+1} | i = 1,2,...,n-1} \cup{v_1v_n}$
    \end{enumerate}
\end{definition}

\begin{definition}[Degree of Path and Cycle]
    The degree of a path and a cycle is the number of vertex it has.
\end{definition}

\begin{definition}[Girth]
    Smallest Cycle in the graph
\end{definition}

\begin{definition}[Distance and Diameter between vertices]
    Soit deux noeud u et v.
    \begin{enumerate}
        \item Distance entre u et v: plus court chemin entre u et v
	\item Diameter entre u et v: plus long chemin entre u et v
    \end{enumerate}
\end{definition}

\begin{theorem}[Properties of Degrees in Path and Cycle]
    \begin{enumerate}
	\item A path of degree n has n nodes and (n-1) edges
	\item A cycle of degree n has n nodes and n edges
    \end{enumerate}
\end{theorem}

\begin{proposition}
    Every graph G contains a path of length n and a cycle of length
    at least n+1
\end{proposition}

\subsection{Connected and Disconnected Graphs}%
\label{sub:Connected Graphs}

\begin{definition}[Connected Graph]
    A graph is connected if for every pair of disinct vertices $ u,v \in
    V(G)$, there is a path from u to v in G. Otherwise, we say the
    graph is disconnected
\end{definition}

\begin{definition}[Connected Vertices]

\end{definition}


\begin{definition}[Open and Closed Neighborhood]
    TODO
\end{definition}

\subsection{Families of Graph and Special Graph}%
\label{sub:Families of Graph and Special Graph}

\begin{enumerate}
    \item Complete Graph $K_n$: simple graph with an edge between every
	pair of vertices
    \item Empty graph: Graph with no edges
    \item Bipartite Graph: a graph whose vertex can be partitionned into
	two sets $V_1$ and $V_2$ such that every edges $ u,v \in E$ has
	$u \in V_1$ and $v in V_2$
    \item Complete Bipartite Graph: every node can reach all nodes in
	the other subset (end)
    \item Star
    \item Path:
    \item Cycle: l'ensemble de noeud allant d'un noeud à lui-même
\end{enumerate}

\subsection{Bipartite Graphs}%
\label{sub:Bipartite Graphs}

\begin{remark}[Importance des graphes bipartites]
    Intuitivement, les graphes bipartites peuvent être séparés en 2
    sous-ensembles dont l'image de chaque élément est mappé à l'autre
    set et pas sur un élément du même ensemble.
\end{remark}


\section{Connectivity}
\section{Optimization}
\section{Shortest Path}
\section{Planar Graphs}
\section{Coloring}
\section{Flow}

\section{Ressources}%
\label{sec:Ressources}

\subsection{Books}%
\label{sub:Books}

- Reinhard Diestel: Graph Theory

\subsection{Courses}%
\label{sub:Courses}

- Wrath of Math: Graph Theory Playlist
- Sarada Herke: Graph Theory

\subsection{Exercices}%
\label{sub:Exercices}

- Introduction to Graph Theory by Douglas B. West: Proofs-based book on
graph theory

\end{document}
\end{article}
