\documentclass{article}
\usepackage{amsmath}
\usepackage{amsfonts}
\usepackage{amsthm}
\usepackage{parskip}
\usepackage{hyperref}
\usepackage{textgreek}
\begin{document}
\title{Lecture Notes for LeetCode}
\author{Emulie Chhor}
\maketitle

\section*{Introduction}

Le but de ce document est de compiler les stratégie utilisées pour
résoudre les questions de LeetCode. Les stratégies sont séparées de la
façon suivante:

    \begin{enumerate}
	\item Arrays
	\item Linked List
	\item String
	\item Hash Table
	\item Sorting
	\item Recursion
	\item Trees
	\item Dynamic Programming
	\item Graph Problems
    \end{enumerate}

\newtheorem{solution}{Solution}[subsection]
\newtheorem*{remark}{Remarque}
\newtheorem*{problem}{Problème}
\newtheorem*{intuition}{Intuition}

\section{Arrays}

\subsection{Two Sums (#1)}%
\label{sub:Two Sums (#1)}

\textbf{Solution: Hash Table}

On veut utiliser un hashtable pour storer chacune des valeurs, et regarder
si la différence entre l'élément et le target est dans le hastable.

\begin{enumerate}
    \item First Pass: Store Array in HashTable
    \item Second Pass: Compute Difference:
	$$ maxProfit = max(maxProfit, hashtable[j] - minValue) $$
\end{enumerate}


\section{Linked List}
\section{String}
\section{Hash Table}
\section{Sorting}
\section{Recursion}
\section{Trees}
\section{Dynamic Programming}
\section{Graph Problems}

\end{document}
\end{article}

