\documentclass{article}
\usepackage{amsmath}
\usepackage{amsfonts}
\usepackage{amsthm}
\usepackage{parskip}
\usepackage{hyperref}
\usepackage{textgreek}
\begin{document}
\title{Lecture Notes for LeetCode}
\author{Emulie Chhor}
\maketitle

\section*{Introduction}

Le but de ce document est de compiler les stratégie utilisées pour
résoudre les questions de LeetCode. Les stratégies sont séparées de la
façon suivante:

    \begin{enumerate}
	\item Arrays
	\item Linked List
	\item String
	\item Hash Table
	\item Sorting
	\item Recursion
	\item Trees
	\item Dynamic Programming
	\item Backtracking
	\item Greedy
	\item Graph Problems
    \end{enumerate}

Il est à noter que les problèmes les plus durs peuvent toujours être
séparés en problème plus simple. De plus, certains problèmes peuvent
être considérés comme des problèmes récurrents, c-à-d qu'ils sont
souvent utilisées non pas comme une fin en soit, mais comme un stratégie
intermédiaire pour résoudre un plus gros problème.

\newtheorem{strategy}{Strategy}[subsection]
\newtheorem*{remark}{Remarque}
\newtheorem*{problem}{Problème}
\newtheorem*{intuition}{Intuition}

\section{Arrays}%
\label{sec:Arrays}

\subsection*{Overview}%
\label{sub:Overview}

\subsection{Two Pointers}%
\label{sub:Two Pointers}

\subsection{Sliding Window Technique}%
\label{sub:Sliding Window Technique}

\subsubsection{In-place Sliding Window}%
\label{ssub:In-place Sliding Window}

\subsubsection{Dynamic Sliding Window}%
\label{ssub:Dynamic Sliding Window}

\subsection{Kadane's Algorithm}%
\label{sub:Kadane's Algorithm}

\section{Linked List}%
\label{sec:Linked List}

\subsection{Fast-and-Slow Pointers}%
\label{sub:Fast-and-Slow Pointers}

\section{Trees}%
\label{sec:Trees}

\subsection{Tree Traversal}%
\label{sub:Tree Traversal}

\textbf{Motivation}

Malgré qu'on nous demande pas explicitement de traverser un arbre, certains
problèmes utilisent la structure des traversals. Il existe 4 façons de
traverser les arbres:
\begin{enumerate}
    \item preorder (depth)
    \item in-order(depth)
    \item post-order(depth)
    \item level-order (breadth)
\end{enumerate}

Les traversals depth-first peuvent être implémentées récursivement ou
itérativement avec un stack/queue.

\subsubsection{Pre-order}%
\label{ssub:Pre-order}

\textbf{Implémentation Récursive}

\begin{enumerate}
    \item visit node
    \item traverse left
    \item traverse right
\end{enumerate}

\textbf{Implémentation Itérative}

\subsubsection{In-order}%
\label{ssub:In-order}

\textbf{Implémentation Récursive}

\begin{enumerate}
    \item traverse left
    \item visit node
    \item traverse right
\end{enumerate}

\textbf{Implémentation Itérative}

\subsubsection{Post-Order}%
\label{ssub:Post-Order}

\textbf{Implémentation Récursive}

\begin{enumerate}
    \item traverse left
    \item traverse right
    \item visit node
\end{enumerate}

\textbf{Implémentation Itérative}

\subsubsection{Level-Order}%
\label{ssub:Level-Order}

\textbf{Implémentation Itérative}

\subsection{Find Smallest/Biggest Element}%
\label{sub:Find Smallest/Biggest Element}

\textbf{Motivation}

Un sous-problème qui revient souvent est de trouver le plus gros/plus
petit élément dans un subtree. Malgré que l'algorithme est simple, il
nous permet de trouver le successeur/prédécesseur lors de la suppression
dans un arbre.

\textbf{Intuition}

Puisque le BST possède le BST invariant, qui nous dit que $ node.left \leq
node \leq node.right $, on trouve le plus petit enfant en traversant la
gauche jusqu'à ce qu'on trouve un noeud null. De la même façon, pour trouver
le noeud le plus grand, on doit traverser le BST en choisissant toujours
le noeud de droite, jusqu'à avoir un null.

\subsection{Find Depth/Height of Tree}%
\label{sub:Find Depth/Height of Tree}

\section{Graph Problems}%
\label{sec:Graph Problems}

\subsection{Shortest Path}%
\label{sub:Shortest Path}

\subsection{Topological Sort}%
\label{sub:Topological Sort}

\subsection{Conectivity}%
\label{sub:Conectivity}

\subsection{Minimum Spanning Tree}%
\label{sub:Minimum Spanning Tree}

\subsection{Cycle Detection}%
\label{sub:Cycle Detection}

\section*{Ressources}%
\label{sec:Ressources}

\begin{enumerate}
    \item Currated Top 75 LeetCode Questions: \url{https://www.teamblind.com/post/New-Year-Gift---Curated-List-of-Top-100-LeetCode-Questions-to-Save-Your-Time-OaM1orEU}
\end{enumerate}

\end{document}
\end{article}

