\documentclass{article}
\begin{document}
\title{Lectures Notes for Calculus}
\author{Emulie Chhor}
\maketitle

\section{Introduction}

\subsection{Why Study Calculus?}

\subsection{Overview of Calculus}

Le Calcul se découle en 4 cours:

    \begin{enumerate}
	\item Calcul Différentiel
	\item Calcul Intégral
	\item Calcul à Plusieurs Variables
	\item Vector Calculus
    \end{enumerate}

Le premier cours de calcul porte sur les limites, la continuité et la
différentiabilité. Puisqu'il s'agit d'un premier cours de calcul, une grande
partie est axée sur le calcul des limites et la dérivations et ses applications.
C'est dans le cours d'analyse qu'on prend une approche plus rigoureusement les
notions de limites, continuité et différentiabilité.

Le deuxième cours de calcul porte sur les techniques d'intégrations et les suites
et séries.

Le troisième cours de calcul porte encore sur les limites, la convergence,
la continuité et la différentiabilité, mais on ajoute une 3e dimension. On verra
comment utiliser les coordonnées polaires et cartésiennes pour changer les bornes
d'intégrations et visualiser les intégrales à dessiner.

Finalement, le quatrième cours de calcul porte sur le Vector Calculus.

\subsection{Calcul Différentiel}

\subsection{Calcul Intégral}

\subsection{Calcul à plusieurs variables}

\subsection{Vector Calculus}

\end{document}
\end{article}
