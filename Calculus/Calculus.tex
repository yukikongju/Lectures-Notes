\documentclass{article}
\usepackage{amsmath}
\usepackage{amsfonts}
\usepackage{amsthm}
\usepackage{parskip}
\usepackage{textgreek}
\usepackage{chngcntr}
\counterwithin*{section}{part}
\begin{document}
\title{Lectures Notes for Calculus}
\author{Emulie Chhor}
\maketitle

\section*{Introduction}

\section{Why Study Calculus?}

\section{Overview of Calculus}

Le Calcul se découle en 4 cours:

    \begin{enumerate}
	\item Calcul Différentiel
	\item Calcul Intégral
	\item Calcul à Plusieurs Variables
	\item Vector Calculus
    \end{enumerate}

Le premier cours de calcul porte sur les limites, la continuité et la
différentiabilité. Puisqu'il s'agit d'un premier cours de calcul, une grande
partie est axée sur le calcul des limites et la dérivations et ses applications.
C'est dans le cours d'analyse qu'on prend une approche plus rigoureusement les
notions de limites, continuité et différentiabilité.

Le deuxième cours de calcul porte sur les techniques d'intégrations et les suites
et séries.

Le troisième cours de calcul porte encore sur les limites, la convergence,
la continuité et la différentiabilité, mais on ajoute une 3e dimension. On verra
comment utiliser les coordonnées polaires et cartésiennes pour changer les bornes
d'intégrations et visualiser les intégrales à dessiner.

Finalement, le quatrième cours de calcul porte sur le Vector Calculus.

\pagebreak

\newtheorem{definition}{Definition}[subsection]
\newtheorem{theorem}{Theorem}[subsection]
\newtheorem{corollary}{Corollary}[subsection]
\newtheorem{lemma}[theorem]{Lemma}
\newtheorem{proposition}{Proposition}[section]
\newtheorem{axiom}{Axiome}
\newtheorem{property}{Propriété}[subsection]
\newtheorem*{remark}{Remarque}
\newtheorem*{problem}{Problème}
\newtheorem*{intuition}{Intuition}

\part{Pre-Calculus}
\section{Overview}
\pagebreak

\part{Calcul Différentiel}

\section{Overview}

Le calcul différentiel se découle en plusieurs chapitres:

\begin{enumerate}
    \item Fonctions
    \item Limites et Continuité
    \item Dérivées
    \item Applications des dérivées
\end{enumerate}

\section{Fonctions}
\subsection{Détermination du domaine, des zéros et du graphe d’une fonction}
\subsection{Caractéristiques des fonction algébriques et transcendantes usuelles}
\section{Limites et continuité}
\subsection{Notion informelle de limite}
\subsection{Calcul des limites}
\subsection{Formes indéterminées}
\subsection{Continuité d’une fonction}

\section{Dérivées}
\subsection{Définition en terme de limite}
\subsection{Calcul de la dérivée à l’aide de limites}
\subsection{Propriétés des dérivées}
\subsection{Formules de dérivation}
\subsection{Calcul de dérivées à l’aide des formules}
\subsection{Dérivée des fonctions transcendantes}
\subsubsection{Fonctions Trigonométriques}
\subsubsection{Fonctions Trigonométriques Inverses}
\subsubsection{Fonctions Exponentielles}
\subsubsection{Fonctions Logarithmique}
\subsection{Dérivation implicite}

\section{Applications des dérivées}
\subsection{Croissance et décroissance}
\subsection{Maximums et minimums}
\subsection{Concavité et points d’inflexion}
\subsection{Tableau de variation et graphes de fonctions}
\subsection{Asymptotes verticales et horizontales}
\subsection{Optimisation}

\pagebreak

\part{Calcul Intégral}

\section{Overview}

Le calcul intégral se découle en plusieurs chapitres:
\begin{enumerate}
    \item Introduction aux Intégrales
    \item Fonctions exponentielles, logarithmiques, trigos et inverses trigos
    \item Techniques d'intégration
    \item Applications d'Intégration
    \item Équations différentielles
    \item Suites et Séries
\end{enumerate}

\section{Limites, continuité et dérivées}
\subsection{Dérivation logarithmique}
\subsection{Règle de l’Hospital}
\subsection{Différentielles}
\section{Intégrale indéfinie}
\subsection{Équations différentielles à variables séparables}
\subsection{Formules d’intégration}
\subsection{Changement de variable}
\section{Intégrale définie}
\subsection{Notation sigma}
\subsection{Sommes de Riemann}
\subsection{Théorème fondamental du calcul}
\subsection{Changement de variable}
\section{Techniques d’intégration}
\subsection{Intégration par parties}
\subsection{Substitution trigonométrique}
\subsection{Décomposition en fractions partielles}
\section{Applications de l’intégrale définie}
\subsection{Aire entre deux courbes}
\subsection{Volume de solides de section connue}
\subsection{Surfaces et volumes de révolution}
\subsection{Longueur d’une courbe}
\section{Suites et séries}
\subsection{Suites — définition et notion de convergence}
\subsection{Séries — définition}
\subsection{Séries Notables}
\subsection{Critères de convergence}
\subsubsection{Séries à termes positifs}
\subsubsection{Séries alternées}
\subsection{Séries de puissance}
\subsubsection{Séries de Taylor et de MacLaurin}

\pagebreak

\part{Calcul à plusieurs variables}
\section{Overview}

Le calcul à plusieurs varaibles se découle en plusieurs chapitres:

\begin{enumerate}
    \item Suites et Séries
    \item Vecteurs et Matrices
    \item Équations des droites et des plans
    \item Fonctions de plusieurs variables
    \item Equations Paramétriques et Coordonnées Polaires
    \item Dérivées Partielles
    \item Optimisation
    \item Intégrales Multiples
\end{enumerate}
\pagebreak

\part{Vector Calculus}
\section{Overview}
\pagebreak

\end{document}
\end{article}
